% !TeX root = tcolorbox.tex
% include file of tcolorbox.tex (manual of the LaTeX package tcolorbox)
\clearpage
\section{Option Keys}\label{sec:optkeys}%
\tcbset{external/prefix=external/coreoptions_}%
For the \meta{options} in \refEnv{tcolorbox} respectively \refCom{tcbset}
the following |pgf| keys can be applied%
\footnote{Strictly speaking, they are \texttt{pgfkeys} keys.
  The \refPkg{pgfkeys} package is auto-loaded by \refPkg{pgf} and is
  documented in \cite[Part~VII]{tantau:tikz_and_pgf}.}.
The key tree path |/tcb/| is not to
be used inside these macros. It is easy to add your own style keys using
the syntax for |pgf| keys, see \cite{tantau:tikz_and_pgf,sturm:latex} or the examples
starting from page~\pageref{sec:latextutorial}.


\subsection{Title}
\begin{docTcbKey}{title}{=\meta{text}}{no default, initially empty}
  Creates a heading line with \meta{text} as content.
\begin{exdispExample*}{title}{sbs,lefthand ratio=0.6}
\begin{tcolorbox}[title=My heading line]
This is a \textbf{tcolorbox}.
\end{tcolorbox}
\end{exdispExample*}
\end{docTcbKey}

\begin{docTcbKey}{notitle}{}{no value, initially set}
  Removes the title line if set before.
\end{docTcbKey}


\begin{docTcbKey}{adjusted title}{=\meta{text}}{style, no default, initially unset}
  Creates a heading line with \meta{text} as content. The minimal height of
  this line is adjusted to fit the text given by \refKey{/tcb/adjust text}.
  This option makes sense
  for single line headings if boxes are set side by side with equal height.
  Note that it is very easy to trick this adjustment.
\begin{exdispExample}[runs=2]{adjusted_title}
\tcbset{colback=White,arc=0mm,width=(\linewidth-4pt)/4,
equal height group=AT,before=,after=\hfill,fonttitle=\bfseries}

The following titles are not adjusted:\\
\foreach \n in {xxx,ggg,AAA,\"Agypten}
{\begin{tcolorbox}[title=\n,colframe=red!75!black]
  Some content.\end{tcolorbox}}
Now, we try again with adjusted titles:\\
\foreach \n in {xxx,ggg,AAA,\"Agypten}
{\begin{tcolorbox}[adjusted title=\n,colframe=blue!75!black]
  Some content.\end{tcolorbox}}
\end{exdispExample}
\end{docTcbKey}


\begin{docTcbKey}{adjust text}{=\meta{text}}{no default, initially \texttt{\"Apgjy}}
  This sets the reference text for \refKey{/tcb/adjusted title}. If your texts
  never exceed \enquote{\"Apgjy} in depth and height you don't need to care about this option.
\end{docTcbKey}

\clearpage

\begin{docTcbKey}[][doc new=2014-11-24]{squeezed title}{=\meta{text}}{style, no default, initially unset}
  Creates a single heading line with \meta{text} as content.
  If the \meta{text} is longer than the available space, the text is
  squeezed to fit into the available space.
\begin{exdispExample}{squeezed_title}
% \tcbuselibrary{raster}
\begin{tcbitemize}[raster columns=3,raster equal height,
  colframe=red!75!black,colback=red!5!white,fonttitle=\bfseries]
\tcbitem[squeezed title={Short title}]
  First box
\tcbitem[squeezed title={This is a very very long title}]
  Second box
\tcbitem[squeezed title={This title is clearly to long for this application}]
  Third box
\end{tcbitemize}
\end{exdispExample}
\end{docTcbKey}

\begin{docTcbKey}[][doc new=2014-11-24]{squeezed title*}{=\meta{text}}{style, no default, initially unset}
  This is a combination of \refKey{/tcb/adjusted title} and  \refKey{/tcb/squeezed title}.
\begin{exdispExample}{squeezed_title_2}
% \tcbuselibrary{raster}
\begin{tcbitemize}[raster columns=3,raster equal height,
  colframe=red!75!black,colback=red!5!white,fonttitle=\bfseries]
\tcbitem[squeezed title*={Short title}]
  First box
\tcbitem[squeezed title*={This is a very very long title}]
  Second box
\tcbitem[squeezed title*={This title is clearly to long for this application}]
  Third box
\end{tcbitemize}
\end{exdispExample}
\end{docTcbKey}


\begin{docTcbKey}[][doc new=2019-03-01]{titlebox}{=\meta{mode}}{no default, initially \texttt{visible}}
  Controls the treatment of the title part of the box.
  Feasible values for \meta{mode} are:
  \begin{itemize}
  \item\docValue{visible}: usual type setting of the title box,
  \item\docValue{invisible}: empty space instead of the title contents.
  \end{itemize}
\begin{exdispExample}{titlebox}
\begin{tcolorbox}[title=My invisible title,
  titlebox=invisible]
This is a \textbf{tcolorbox}.
\end{tcolorbox}
\end{exdispExample}
\end{docTcbKey}



\clearpage
\begin{docTcbKey}{detach title}{}{no value}
Detaches the title from its normal position. The text of the title is
stored into \docAuxCommand{tcbtitletext} and the formatted title is
available by \docAuxCommand{tcbtitle}.
The main application is to move the title from its usual place to another one.
\begin{exdispExample}{detach_title}
\newtcolorbox{mybox}[2][]{colbacktitle=red!10!white,
  colback=blue!10!white,coltitle=red!70!black,
  title={#2},fonttitle=\bfseries,#1}

\begin{mybox}{My title}
  This is a \textbf{tcolorbox}.
\end{mybox}
\begin{mybox}[detach title,before upper={\tcbtitle\quad}]{My title}
  This is a \textbf{tcolorbox}.
\end{mybox}
\begin{mybox}[detach title,after upper={\par\hfill\tcbtitle}]{My title}
  This is a \textbf{tcolorbox}.
\end{mybox}
\end{exdispExample}
\end{docTcbKey}

\begin{docTcbKey}{attach title}{}{no value}
Attaches the title to its normal position. This option is used to reverse
\refKey{/tcb/detach title}.
\end{docTcbKey}


\begin{docTcbKey}[][doc updated=2015-07-08]{attach title to upper}{\colOpt{=\meta{text}}}{style, default empty, initially unset}
Attaches the title to the begin of the upper part of the box content.
The optional \meta{text} is set between the formatted title and the box content.
\begin{exdispExample}{attach_title_to_upper}
\newtcolorbox{mybox}[2][]{colbacktitle=red!10!white,
  colback=blue!10!white,coltitle=red!70!black,
  title={#2},fonttitle=\bfseries,#1}

\begin{mybox}[attach title to upper={\ ---\ }]{My title}
  This is a \textbf{tcolorbox}.
\end{mybox}
\begin{mybox}[attach title to upper,after title={:\ }]{My title}
  This is a \textbf{tcolorbox}.
\end{mybox}
\end{exdispExample}
\end{docTcbKey}


\bigskip
\begin{marker}
More title options are documented in \zvref{subsec:contentadditions}
and \zvref{subsec:skinboxedtitle}.
\end{marker}



\clearpage
\subsection{Subtitle}

Inside the box content, one or more subtitles can be added.
In general, a subtitle is a further \refEnv{tcolorbox} which inherits some
color and geometry options from the enclosing box. It may be customized
just like any other \refEnv{tcolorbox}.

\begin{docCommand}[doc new=2014-10-10]{tcbsubtitle}{\oarg{options}\marg{text}}
  Used inside a \refEnv{tcolorbox} to add a subtitle box with the given \meta{text}.
  This is an independent \refEnv{tcolorbox} which is formatted by several
  inherited properties of the enclosing box, by further settings from
  \refKey{/tcb/subtitle style}, and by the given \meta{options}.
\begin{exdispExample*}{tcbsubtitle_1}{sbs,lefthand ratio=0.6}
\begin{tcolorbox}[title=My title,
    colback=red!5!white,
    colframe=red!75!black,
    fonttitle=\bfseries]
  This is a \textbf{tcolorbox}.
\tcbsubtitle[before skip=\baselineskip]%
  {My subtitle}
  Further text.
\end{tcolorbox}
\end{exdispExample*}
\begin{exdispExample*}{tcbsubtitle_2}{sbs,lefthand ratio=0.6}
\begin{tcolorbox}[title=My title,
    colback=red!5!white,
    colframe=red!75!black,
    colbacktitle=yellow!50!red,
    coltitle=red!25!black,
    fonttitle=\bfseries]
  This is a \textbf{tcolorbox}.
\tcbsubtitle[before skip=\baselineskip]%
  {My subtitle}
  Further text.
\end{tcolorbox}
\end{exdispExample*}
\end{docCommand}

\begin{docTcbKey}[][doc new=2014-10-10]{subtitle style}{=\meta{options}}{no default, initially empty}
  Adds |tcolorbox| \meta{options} to the settings for \refCom{tcbsubtitle}.
\begin{exdispExample*}{subtitle_style}{sbs,lefthand ratio=0.6}
\begin{tcolorbox}[title=My title,
  colback=red!5!white,
  colframe=red!75!black,
  colbacktitle=yellow!50!red,
  coltitle=red!25!black,
  fonttitle=\bfseries,
  subtitle style={boxrule=0.4pt,
    colback=yellow!50!red!25!white,
    colupper=red!75!gray} ]
  This is a \textbf{tcolorbox}.
\tcbsubtitle{My subtitle}
  Further text.
\tcbsubtitle{Second subtitle}
  Further text.
\end{tcolorbox}
\end{exdispExample*}
\end{docTcbKey}

\clearpage
\subsection{Upper Part}
The text content of a \refEnv{tcolorbox} may be parted into a mandatory \emph{upper part}
and an optional \emph{lower part}. These parts are separated by
\refCom{tcblower}. If there is no \refCom{tcblower} present, there is no
\emph{lower part} and the \emph{upper part} forms the complete text content.

\begin{docTcbKey}[][doc new=2015-01-06]{upperbox}{=\meta{mode}}{no default, initially \texttt{visible}}
  Controls the treatment of the upper part of the box. If there is no lower part,
  this is the complete text content.
  Feasible values for \meta{mode} are:
  \begin{itemize}
  \item\docValue{visible}: usual type setting of the upper part,
  \item\docValue{invisible}: empty space instead of the upper part contents.
  \end{itemize}
\begin{exdispExample}{upperbox}
\begin{tcolorbox}[upperbox=invisible,colback=white]
This is a \textbf{tcolorbox} (but invisible).
\end{tcolorbox}

\begin{tcolorbox}[upperbox=invisible,colback=white]
This is a \textbf{tcolorbox} (but invisible).
\tcblower
This is the lower part.
\end{tcolorbox}
\end{exdispExample}
\end{docTcbKey}


\begin{docTcbKey}[][doc new and updated={2015-01-06}{2019-03-01}]{visible}{}{style, no value}
  Shortcut for setting \refKey{/tcb/upperbox}, \refKey{/tcb/lowerbox}, and \refKey{/tcb/titlebox}
  to be \docValue{visible}.
\end{docTcbKey}

\begin{docTcbKey}[][doc new and updated={2015-01-06}{2019-03-01}]{invisible}{}{style, no value}
  Shortcut for setting \refKey{/tcb/upperbox}, \refKey{/tcb/lowerbox}, and \refKey{/tcb/titlebox}
  to be \docValue{invisible}.
\begin{exdispExample}{invisible}
\begin{tcolorbox}[invisible]
This is a \textbf{tcolorbox} (but invisible).
\end{tcolorbox}
\end{exdispExample}
\end{docTcbKey}


\clearpage
\begin{docTcbKey}[][doc new=2015-05-04]{saveto}{=\meta{file name}}{no default, initially empty}
  Saves the content of the box into a file for an optional later usage.
  This is the counterpart of \refKey{/tcb/savelowerto}, but it saves not
  only the upper part but the whole content. If a lower part is present,
  it is also saved including \refCom{tcblower}.
  An empty \meta{file name} deactivates \refKey{/tcb/saveto}.
  \begin{marker}
  This option cannot be combined with \refKey{/tcb/savelowerto} or
  \refKey{/tcb/redirectlowerto}.
  \end{marker}

\begin{exdispExample}{saveto_1}
\begin{tcolorbox}[invisible,saveto=\jobname_mysave1.tex,colback=white]
This is a \textbf{tcolorbox} which seems to be empty.
The content is saved for later usage.
\end{tcolorbox}

Now, we load the saved text:\\
\input{\jobname_mysave1.tex}
\end{exdispExample}

\begin{exdispExample}{saveto_2}
\begin{tcolorbox}[saveto=\jobname_mysave2.tex]
This is a \textbf{tcolorbox}.
\tcblower
This is the lower part.
\end{tcolorbox}

Now, we load the saved text:
\begin{tcolorbox}[colframe=red,colback=red!10,
  coltitle=black,colbacktitle=red!20,sidebyside,
  title=Here we see the saved content including the lower part]
\input{\jobname_mysave2.tex}
\end{tcolorbox}
\end{exdispExample}
\end{docTcbKey}



\clearpage
\subsection{Lower Part}
\begin{docTcbKey}{lowerbox}{=\meta{mode}}{no default, initially \texttt{visible}}
  Controls the treatment of the lower part of the box.
  Feasible values for \meta{mode} are:
  \begin{itemize}
  \item\docValue{visible}: usual type setting of the lower part,
  \item\docValue{invisible}: empty space instead of the lower part contents,
  \item\docValue{ignored}: the lower part is not used (here).
  \end{itemize}
  The last two values are usually applied in connection with |savelowerto|.

  \begin{marker}
  The lower part is always typeset into a box which is used or dropped
  accordingly. The only exception is applying \refKey{/tcb/redirectlowerto}.
  \end{marker}

\begin{exdispExample}{lowerbox}
\begin{tcolorbox}[lowerbox=invisible,colback=white]
This is a \textbf{tcolorbox}.
\tcblower
This is the lower part (but invisible).
\end{tcolorbox}

\begin{tcolorbox}[lowerbox=ignored,colback=white]
This is a \textbf{tcolorbox}.
\tcblower
This is the lower part (but ignored).
\end{tcolorbox}
\end{exdispExample}
\end{docTcbKey}


\clearpage
\begin{docTcbKey}[][doc updated=2014-11-28]{savelowerto}{=\meta{file name}}{no default, initially empty}
  Saves the content of the lower part into a file for an optional later usage.
  An empty \meta{file name} deactivates \refKey{/tcb/savelowerto}.
  \begin{marker}
  This option cannot be combined with \refKey{/tcb/saveto}.
  \end{marker}

\begin{exdispExample}{savelowerto}
\begin{tcolorbox}[lowerbox=invisible,savelowerto=\jobname_bspsave.tex,colback=white]
This is a \textbf{tcolorbox}.
\tcblower
This is the lower part which may be quite complex:
$\displaystyle f(x)=\frac{1+x^2}{1-x^2}$.
\end{tcolorbox}

Now, we load the saved text:\\
\input{\jobname_bspsave.tex}
\end{exdispExample}
\end{docTcbKey}

\medskip

\begin{docTcbKey}[][doc new=2023-07-17]{redirectlowerto}{=\meta{file name}}{no default, initially empty}
  This is a combination of \refKey{/tcb/savelowerto} and \refKey{/tcb/lowerbox}|=|\docValue{ignored}
  \emph{and additionally} avoids typesetting the content of the lower part.
  This can be useful, if the lower part contains counters or similar.
  An empty \meta{file name} deactivates the saving, but the \refKey{/tcb/lowerbox}
  setting will still be \docValue{ignored}.
  \begin{marker}
  This option cannot be combined with \refKey{/tcb/saveto}.
  \end{marker}
\begin{exdispExample}{redirectlowerto}

\setcounter{enumi}{1}
Test counter: \theenumi

\begin{tcolorbox}[redirectlowerto=\jobname_bspsave2.tex,colback=white]
This is a \textbf{tcolorbox}.
\tcblower
This is the lower part.\stepcounter{enumi}
New value of test counter: \theenumi.
\end{tcolorbox}

Now, we load the saved text:\\
\input{\jobname_bspsave2.tex}
\end{exdispExample}
\end{docTcbKey}



\clearpage
\begin{docTcbKey}{lower separated}{\colOpt{=true\textbar false}}{default |true|, initially |true|}
If set to |true|, the lower part is visually separated from the upper part.
It depends on the chosen skin how the visualization of the separation is done.
\enlargethispage*{1cm}
\begin{exdispExample}{lower_separated}
% \tcbuselibrary{skins,raster}
\begin{tcbraster}[colback=red!5!white,colframe=red!75!black,
  fonttitle=\bfseries,fontlower=\itshape]
%
\begin{tcolorbox}[title=Lower separated]
This is the upper part.
\tcblower
This is the lower part.
\end{tcolorbox}
%
\begin{tcolorbox}[title=Lower not separated,lower separated=false]
This is the upper part.
\tcblower
This is the lower part.
\end{tcolorbox}
%
\begin{tcolorbox}[sidebyside,title=Lower separated]
This is the upper part.
\tcblower
This is the lower part.
\end{tcolorbox}
%
\begin{tcolorbox}[sidebyside,title=Lower not separated,lower separated=false]
This is the upper part.
\tcblower
This is the lower part.
\end{tcolorbox}
%
\begin{tcolorbox}[beamer,title=Lower separated]
This is the upper part.
\tcblower
This is the lower part.
\end{tcolorbox}
%
\begin{tcolorbox}[beamer,title=Lower not separated,lower separated=false]
This is the upper part.
\tcblower
This is the lower part.
\end{tcolorbox}
%
\end{tcbraster}
\end{exdispExample}
\end{docTcbKey}


\clearpage
\begin{docTcbKey}{savedelimiter}{=\meta{name}}{no default, initially \texttt{tcolorbox}}
  Used in connection with new environment definitions which extend
  |tcolorbox| and use or allow the option \refKey{/tcb/savelowerto} or \refKey{/tcb/redirectlowerto}.
  To catch the end of the new box environment \meta{name} has to be the name of
  this environment. Additionally, the environment definition has to use
  |\tcolorbox| instead of
  |\begin{tcolorbox}| and |\endtcolorbox| instead of |\end{tcolorbox}|.
\begin{exdispExample}{savedelimiter1}
\newenvironment{mybox}[1]{%
  \tcolorbox[savedelimiter=mybox,
             savelowerto=\jobname_bspsave3.tex,lowerbox=ignored,
             colback=red!5!white,colframe=red!75!black,fonttitle=\bfseries,
             title={#1}]}%
  {\endtcolorbox}

\begin{mybox}{My Example}
Upper part.
\tcblower
Saved lower part!
\end{mybox}

Now, the saved part is used:
\begin{tcolorbox}[colback=green!5]
\input{\jobname_bspsave3.tex}
\end{tcolorbox}
\end{exdispExample}

\enlargethispage*{1cm}

The |savedelimiter| is used implicitly with \refCom{newtcolorbox} which
allows a more convenient usage:
\begin{exdispExample}{savedelimiter2}
\newtcolorbox{mybox}[1]{%
             savelowerto=\jobname_bspsave4.tex,lowerbox=ignored,
             colback=red!5!white,colframe=red!75!black,fonttitle=\bfseries,
             title={#1}}%

\begin{mybox}{My Example}
Upper part.
\tcblower
Saved lower part!
\end{mybox}

Now, the saved part is used:
\begin{tcolorbox}[colback=green!5]
\input{\jobname_bspsave4.tex}
\end{tcolorbox}
\end{exdispExample}
\end{docTcbKey}



\clearpage
\subsection{Colors and Fonts}
\begin{docTcbKey}{colframe}{=\meta{color}}{no default, initially \texttt{black!75!white}}
  Sets the frame \meta{color} of the box.
\begin{exdispExample*}{colframe}{sbs,lefthand ratio=0.6}
\begin{tcolorbox}[colframe=red!50!white]
This is a \textbf{tcolorbox}.
\end{tcolorbox}
\end{exdispExample*}
\end{docTcbKey}

\begin{docTcbKey}{colback}{=\meta{color}}{no default, initially \texttt{black!5!white}}
  Sets the background \meta{color} of the box.
\begin{exdispExample*}{colback}{sbs,lefthand ratio=0.6}
\begin{tcolorbox}[colback=red!50!white]
This is a \textbf{tcolorbox}.
\end{tcolorbox}
\end{exdispExample*}
\end{docTcbKey}

Also see \refKey{/tcb/colbacklower} of the \mylib{skins} library.

\begin{docTcbKey}{title filled}{\colOpt{=true\textbar false}}{default |true|, initially |false|}
  Switches the drawing of the title background according to the given value.
  This option is set to |true| automatically by \refKey{/tcb/colbacktitle},
  \refKey{/tcb/opacitybacktitle}, and \refKey{/tcb/title style},
  and \refKey{/tcb/title code}.

\begin{exdispExample*}{title_filled}{sbs,lefthand ratio=0.6}
\begin{tcolorbox}[title=My title,title filled]
This is a \textbf{tcolorbox}.
\end{tcolorbox}
\begin{tcolorbox}[title=My title,
  title filled=false]
This is a \textbf{tcolorbox}.
\end{tcolorbox}
\end{exdispExample*}
\end{docTcbKey}


\begin{docTcbKey}{colbacktitle}{=\meta{color}}{no default, initially \texttt{black!50!white}}
  Sets the background \meta{color} of the title area of the box.
\begin{exdispExample*}{colbacktitle}{sbs,lefthand ratio=0.6}
\begin{tcolorbox}[colbacktitle=red!50!white,
  title=My title,coltitle=black,
  fonttitle=\bfseries]
This is a \textbf{tcolorbox}.
\end{tcolorbox}
\end{exdispExample*}
\end{docTcbKey}



\clearpage

\begin{docTcbKey}{colupper}{=\meta{color}}{no default, initially \texttt{black}}
  Sets the text \meta{color} of the upper part.
\begin{exdispExample*}{colupper}{sbs,lefthand ratio=0.6}
\begin{tcolorbox}[colupper=red!75!black]
This is a \textbf{tcolorbox}.
\tcblower
This is the lower part.
\end{tcolorbox}
\end{exdispExample*}
\end{docTcbKey}


\begin{docTcbKey}{collower}{=\meta{color}}{no default, initially \texttt{black}}
  Sets the text \meta{color} of the lower part.
\begin{exdispExample*}{collower}{sbs,lefthand ratio=0.6}
\begin{tcolorbox}[collower=red!75!black]
This is a \textbf{tcolorbox}.
\tcblower
This is the lower part.
\end{tcolorbox}
\end{exdispExample*}
\end{docTcbKey}


\begin{docTcbKey}{coltext}{=\meta{color}}{style, no default, initially \texttt{black}}
  Sets the text \meta{color} of the box. This is an abbreviation for setting
  |colupper| and |collower| to the same value.
\begin{exdispExample*}{coltext}{sbs,lefthand ratio=0.6}
\begin{tcolorbox}[coltext=red!75!black]
This is a \textbf{tcolorbox}.
\tcblower
This is the lower part.
\end{tcolorbox}
\end{exdispExample*}
\end{docTcbKey}


\begin{docTcbKey}{coltitle}{=\meta{color}}{no default, initially \texttt{white}}
  Sets the title text \meta{color} of the box.
\begin{exdispExample*}{coltitle}{sbs,lefthand ratio=0.6}
\begin{tcolorbox}[coltitle=red!75!black,
  colbacktitle=black!10!white,title=Test]
This is a \textbf{tcolorbox}.
\end{tcolorbox}
\end{exdispExample*}
\end{docTcbKey}

\clearpage

\begin{docTcbKey}{fontupper}{=\meta{text}}{no default, initially empty}
  Sets \meta{text} before the content of the upper part (e.\,g.\ font settings).
\begin{exdispExample}{fontupper}
\begin{tcolorbox}[fontupper=Hello!~\sffamily]
This is a \textbf{tcolorbox}.
\end{tcolorbox}
\end{exdispExample}
\end{docTcbKey}


\begin{docTcbKey}{fontlower}{=\meta{text}}{no default, initially empty}
  Sets \meta{text} before the content of the lower part (e.\,g.\ font settings).
\begin{exdispExample}{fontlower}
\begin{tcolorbox}[fontlower=\sffamily\bfseries]
This is a \textbf{tcolorbox}.
\tcblower
This is the lower part.
\end{tcolorbox}
\end{exdispExample}
\end{docTcbKey}


\begin{docTcbKey}{fonttitle}{=\meta{text}}{no default, initially empty}
  Sets \meta{text} before the content of the title text (e.\,g.\ font settings).
\begin{exdispExample}{fonttitle}
\begin{tcolorbox}[fonttitle=\sffamily\bfseries\large,title=Hello]
This is a \textbf{tcolorbox}.
\end{tcolorbox}
\end{exdispExample}
\end{docTcbKey}

\bigskip
\begin{marker}
  More color options are provided by using skins documented in
  Section \ref{sec:skins} from page \pageref{sec:skins}.
\end{marker}


\clearpage
\subsection{Text Alignment}

\begin{docTcbKeys}[
    doc parameter   = {=\meta{alignment}},
    doc description = {no default, initially \texttt{justify}},
    doc new         = 2015-05-07,
  ]
  {
    {
      doc name = halign,
    },
    {
      doc name = halign upper,
    }
  }
%\begin{docTcbKey}[][doc new=2015-05-07]{halign}{=\meta{alignment}}{no default, initially \texttt{justify}}
  If there is no lower part, |halign| determines the horizontal \meta{alignment}
  of the text content.
  Otherwise, |halign| determines the horizontal \meta{alignment}
  of the upper part of the box only.
  The feasible values for \meta{alignment} are more or less identical to
  the corresponding |/tikz/align| settings, even if the implementation differs.
  \begin{itemize}
  \item\docValue{justify}: usual left and right justified type setting.
  \item\docValue{left}: left border justification in analogy to plain \TeX.
  \item\docValue{flush left}: left border justification with |\raggedright| of \LaTeX.
  \item\docValue{right}: right border justification in analogy to plain \TeX.
  \item\docValue{flush right}: right border justification with |\raggedleft| of \LaTeX.
  \item\docValue{center}: centering in analogy to plain \TeX.
  \item\docValue{flush center}: centering with |\centering| of \LaTeX.
  \end{itemize}
  The differences between the flush and non-flush version are explained in
  detail in the \tikzname\ manual \cite{tantau:tikz_and_pgf}. The short story is that
  the non-flush versions will often look more balanced but with more
  hyphenations.
  \refKey{/tcb/halign upper} is an alias for \refKey{/tcb/halign}.

\begin{exdispExample}{halign}
\tcbset{colback=red!5!white,colframe=red!75!black,size=small,
  fonttitle=\bfseries,width=3.5cm,box align=top,
  nobeforeafter}

\begin{tcolorbox}[adjusted title=flush center,halign=flush center]
This is a demonstration text for showing how line breaking works.
\end{tcolorbox}
\begin{tcolorbox}[adjusted title=flush left,halign=flush left]
This is a demonstration text for showing how line breaking works.
\end{tcolorbox}
\begin{tcolorbox}[adjusted title=flush right,halign=flush right]
This is a demonstration text for showing how line breaking works.
\end{tcolorbox}

\begin{tcolorbox}[adjusted title=center,halign=center]
This is a demonstration text for showing how line breaking works.
\end{tcolorbox}
\begin{tcolorbox}[adjusted title=left,halign=left]
This is a demonstration text for showing how line breaking works.
\end{tcolorbox}
\begin{tcolorbox}[adjusted title=right,halign upper=right]
This is a demonstration text for showing how line breaking works.
\end{tcolorbox}
\end{exdispExample}

\end{docTcbKeys}

\clearpage

\begin{docTcbKeys}[
    doc parameter   = {=\marg{code}},
    doc description = {no default},
    doc new         = 2022-07-07,
  ]
  {
    {
      doc name = halign code,
    },
    {
      doc name = halign upper code,
    }
  }
  If \refKey{/tcb/halign} resp. \refKey{/tcb/halign upper} is not flexible enough,
  \refKey{/tcb/halign code} resp. \refKey{/tcb/halign upper code} allows to set
  arbitrary \meta{code} to configure the horizontal alignment of the text content,
  e.\,g.\ using alignments from specialized packages.
\end{docTcbKeys}


\begin{docTcbKey}[][doc new=2015-05-07]{halign lower}{=\meta{alignment}}{no default, initially \texttt{justify}}
  |halign lower| determines the horizontal \meta{alignment} of the lower part of the box.
  The feasible values for \meta{alignment} are the same as for \refKey{/tcb/halign}.

\begin{exdispExample}{halign_lower}
\begin{tcbraster}[raster columns=3,fonttitle=\bfseries,
  colback=red!5!white,colframe=red!75!black]

  \begin{tcolorbox}[adjusted title=flush center,halign lower=flush center]
    Upper part. \tcblower Lower part.
  \end{tcolorbox}
  \begin{tcolorbox}[adjusted title=flush left,halign lower=flush left]
    Upper part. \tcblower Lower part.
  \end{tcolorbox}
  \begin{tcolorbox}[adjusted title=flush right,halign lower=flush right]
    Upper part. \tcblower Lower part.
  \end{tcolorbox}
  \begin{tcolorbox}[adjusted title=center,halign lower=center]
    Upper part. \tcblower Lower part.
  \end{tcolorbox}
  \begin{tcolorbox}[adjusted title=left,halign lower=left]
    Upper part. \tcblower Lower part.
  \end{tcolorbox}
  \begin{tcolorbox}[adjusted title=right,halign lower=right]
    Upper part. \tcblower Lower part.
  \end{tcolorbox}

\end{tcbraster}
\end{exdispExample}
\end{docTcbKey}


\begin{docTcbKeys}[
    doc parameter   = {=\marg{code}},
    doc description = {no default},
    doc new         = 2022-07-07,
    doc name        = halign lower code,
  ]{}
  If \refKey{/tcb/halign lower} is not flexible enough,
  \refKey{/tcb/halign lower code} allows to set
  arbitrary \meta{code} to configure the horizontal alignment of the lower
  part of the box.
\end{docTcbKeys}


\clearpage
\begin{docTcbKey}[][doc new=2015-05-07]{halign title}{=\meta{alignment}}{no default, initially \texttt{justify}}
  |halign title| determines the horizontal \meta{alignment} of the title of the box.
  The feasible values for \meta{alignment} are the same as for \refKey{/tcb/halign}.

\begin{exdispExample}{halign_title}
\begin{tcbraster}[raster columns=3,fonttitle=\bfseries,
  colback=red!5!white,colframe=red!75!black]

  \begin{tcolorbox}[adjusted title=flush center,halign title=flush center]
    This is a \textbf{tcolorbox}.
  \end{tcolorbox}
  \begin{tcolorbox}[adjusted title=flush left,halign title=flush left]
    This is a \textbf{tcolorbox}.
  \end{tcolorbox}
  \begin{tcolorbox}[adjusted title=flush right,halign title=flush right]
    This is a \textbf{tcolorbox}.
  \end{tcolorbox}
  \begin{tcolorbox}[adjusted title=center,halign title=center]
    This is a \textbf{tcolorbox}.
  \end{tcolorbox}
  \begin{tcolorbox}[adjusted title=left,halign title=left]
    This is a \textbf{tcolorbox}.
  \end{tcolorbox}
  \begin{tcolorbox}[adjusted title=right,halign title=right]
    This is a \textbf{tcolorbox}.
  \end{tcolorbox}

\end{tcbraster}
\end{exdispExample}
\end{docTcbKey}


\begin{docTcbKeys}[
    doc parameter   = {=\marg{code}},
    doc description = {no default},
    doc new         = 2022-07-07,
    doc name        = halign title code,
  ]{}
  If \refKey{/tcb/halign title} is not flexible enough,
  \refKey{/tcb/halign title code} allows to set
  arbitrary \meta{code} to configure the horizontal alignment of the title
  of the box.
\end{docTcbKeys}



\begin{docTcbKey}[][doc updated=2015-05-07]{flushleft upper}{}{style, no value}
  Shortcut for setting \refKey{/tcb/halign} to \docValue{flush left}.
\end{docTcbKey}

\begin{docTcbKey}[][doc updated=2015-05-07]{center upper}{}{style, no value}
  Shortcut for setting \refKey{/tcb/halign} to \docValue{flush center}.
\end{docTcbKey}

\begin{docTcbKey}[][doc updated=2015-05-07]{flushright upper}{}{style, no value}
  Shortcut for setting \refKey{/tcb/halign} to \docValue{flush right}.
\end{docTcbKey}

\begin{docTcbKey}[][doc updated=2015-05-07]{flushleft lower}{}{style, no value}
  Shortcut for setting \refKey{/tcb/halign lower} to \docValue{flush left}.
\end{docTcbKey}

\begin{docTcbKey}[][doc updated=2015-05-07]{center lower}{}{style, no value}
  Shortcut for setting \refKey{/tcb/halign lower} to \docValue{flush center}.
\end{docTcbKey}

\begin{docTcbKey}[][doc updated=2015-05-07]{flushright lower}{}{style, no value}
  Shortcut for setting \refKey{/tcb/halign lower} to \docValue{flush right}.
\end{docTcbKey}

\clearpage

\begin{docTcbKey}[][doc updated=2015-05-07]{flushleft title}{}{style, no value}
  Shortcut for setting \refKey{/tcb/halign title} to \docValue{flush left}.
\end{docTcbKey}

\begin{docTcbKey}[][doc updated=2015-05-07]{center title}{}{style, no value}
  Shortcut for setting \refKey{/tcb/halign title} to \docValue{flush center}.
\end{docTcbKey}

\begin{docTcbKey}[][doc updated=2015-05-07]{flushright title}{}{style, no value}
  Shortcut for setting \refKey{/tcb/halign title} to \docValue{flush right}.
\end{docTcbKey}


\begin{marker}
The vertical alignment settings are only relevant for boxes which are larger
than their natural height, see \zvref{sec:heightcontrol}.
\end{marker}

\begin{docTcbKey}[][doc updated=2015-07-16]{valign}{=\meta{alignment}}{no default, initially |top|}
  If the height of a |tcolorbox| is not the natural height, |valign|
  determines the vertical \meta{alignment} of the upper part.
  Feasible values are
  \begin{itemize}
  \item\docValue{top}: Anchor text at top.
  \item\docValue{center}: Anchor text at center.
  \item\docValue{bottom}: Anchor text at bottom.
  \item\docValue{scale}: Scale text vertically to fit into the available space.
    This is brutal and may not look very good. Consider \zvref{sec:fitting}
    alternatively.
  \item\docValue{scale*}: Like \docValue{scale}, but scaling is bounded by
    \refKey{/tcb/valign scale limit}.
  \end{itemize}
  For a box with natural height, these settings are meaningless.
\begin{exdispExample}{valign}
\tcbset{width=(\linewidth-2mm)/4,before=,after=\hfill,
colframe=blue!75!black,colback=white,height=2cm}

\foreach \myalign in {top,center,bottom,scale}
{\begin{tcolorbox}[valign=\myalign]
  This is a \textbf{tcolorbox}.
\end{tcolorbox}}
\end{exdispExample}
\end{docTcbKey}


\begin{docTcbKey}[][doc new=2015-05-07]{valign upper}{=\meta{alignment}}{no default, initially \texttt{top}}
  Alias for \refKey{/tcb/valign}.
\end{docTcbKey}

\begin{docTcbKey}{valign lower}{=\meta{alignment}}{no default, initially |top|}
  This key has the same meaning for the lower part as \refKey{/tcb/valign}
  for the upper part, i.\,e., it determines
  the vertical \meta{alignment} of the lower part with feasible values
  |top|, |center|, |bottom|, |scale|, and |scale*|.
\end{docTcbKey}

\begin{docTcbKey}[][doc new=2015-07-16]{valign scale limit}{=\meta{real number}}{no default, initially \texttt{1.1}}
  Sets an upper scale limit for the \docValue{scale*} setting in
  \refKey{/tcb/valign} and \refKey{/tcb/valign lower}.
  Note that this value is not reset by \refKey{/tcb/reset}. So, changes
  also apply to embedded boxes.
\end{docTcbKey}


Also see \refKey{/tcb/sidebyside align} for alignment settings when
upper part and lower part are set side-by-side.

\clearpage
\subsection{Geometry}
\subsubsection{Width}

\begin{docTcbKey}{width}{=\meta{length}}{no default, initially \cs{linewidth}}
  Sets the total width of the colored box to \meta{length}.
  See also \refKey{/tcb/height}.
\begin{exdispExample}{width}
\tcbset{colback=red!5!white,colframe=red!75!black}

\begin{tcolorbox}[width=\linewidth/2]
This is a \textbf{tcolorbox}.
\end{tcolorbox}
\end{exdispExample}
\end{docTcbKey}


\begin{docTcbKey}[][doc new=2014-10-31]{text width}{=\meta{length}}{style, no default}
  Sets the text width of the upper part to \meta{length}.
  See also \refKey{/tcb/text height}.
\begin{exdispExample}{text_width}
\tcbset{colback=red!5!white,colframe=red!75!black}

\begin{tcolorbox}[text width=4cm]
This is a \textbf{tcolorbox} where the text has a width of 4cm.
\end{tcolorbox}
\end{exdispExample}
\end{docTcbKey}

\begin{docTcbKey}[][doc new=2014-11-07]{add to width}{=\meta{length}}{style, no default}
  Adds \meta{length} to the current total width of the colored box.
\begin{exdispExample*}{add_to_width}{sbs,lefthand ratio=0.6}
\tcbset{width=4cm,colback=red!5!white,
  colframe=red!75!black}

\begin{tcolorbox}
This is a \textbf{tcolorbox}.
\end{tcolorbox}

\begin{tcolorbox}[add to width=1cm]
This is a \textbf{tcolorbox}.
\end{tcolorbox}
\end{exdispExample*}
\end{docTcbKey}

See \zvref{sec:heightcontrol} for setting fixed height values.


\clearpage
\subsubsection{Rules}
\begin{docTcbKey}{toprule}{=\meta{length}}{no default, initially \texttt{0.5mm}}
  Sets the line width of the top rule to \meta{length}.
\begin{exdispExample}{toprule}
\tcbset{colback=red!5!white,colframe=red!75!black}

\begin{tcolorbox}[toprule=3mm]
This is a \textbf{tcolorbox}.
\end{tcolorbox}
\end{exdispExample}
\end{docTcbKey}


\begin{docTcbKey}{bottomrule}{=\meta{length}}{no default, initially \texttt{0.5mm}}
  Sets the line width of the bottom rule to \meta{length}.
\begin{exdispExample}{bottomrule}
\tcbset{colback=red!5!white,colframe=red!75!black}

\begin{tcolorbox}[bottomrule=3mm]
This is a \textbf{tcolorbox}.
\end{tcolorbox}
\end{exdispExample}
\end{docTcbKey}

\begin{docTcbKey}{leftrule}{=\meta{length}}{no default, initially \texttt{0.5mm}}
  Sets the line width of the left rule to \meta{length}.
\begin{exdispExample}{leftrule}
\tcbset{colback=red!5!white,colframe=red!75!black}

\begin{tcolorbox}[leftrule=3mm]
This is a \textbf{tcolorbox}.
\end{tcolorbox}
\end{exdispExample}
\end{docTcbKey}


\begin{docTcbKey}{rightrule}{=\meta{length}}{no default, initially \texttt{0.5mm}}
  Sets the line width of the right rule to \meta{length}.
\begin{exdispExample}{rightrule}
\tcbset{colback=red!5!white,colframe=red!75!black}

\begin{tcolorbox}[rightrule=3mm]
This is a \textbf{tcolorbox}.
\end{tcolorbox}
\end{exdispExample}
\end{docTcbKey}

\clearpage
\begin{docTcbKey}{titlerule}{=\meta{length}}{no default, initially \texttt{0.5mm}}
  Sets the line width of the rule below the title to \meta{length}.
\begin{exdispExample}{titlerule}
\tcbset{enhanced,colback=red!5!white,colframe=red!75!black,
  colbacktitle=red!90!black}

\begin{tcolorbox}[titlerule=3mm,title=This is the title]
This is a \textbf{tcolorbox}.
\end{tcolorbox}
\end{exdispExample}
\end{docTcbKey}


\begin{docTcbKey}{boxrule}{=\meta{length}}{style, no default, initially \texttt{0.5mm}}
  Sets all rules of the frame to \meta{length}, i.\,e., 
  \refKey{/tcb/toprule}, \refKey{/tcb/bottomrule}, \refKey{/tcb/leftrule},
  \refKey{/tcb/rightrule}, and \refKey{/tcb/titlerule}.
\begin{exdispExample}{boxrule}
\tcbset{colback=red!5!white,colframe=red!75!black}

\begin{tcolorbox}[boxrule=3mm]
This is a \textbf{tcolorbox}.
\end{tcolorbox}
\end{exdispExample}
\end{docTcbKey}

\bigskip
\begin{marker}
  More options for drawing a \refKey{/tcb/borderline} are provided by using skins documented in
  Section \ref{sec:skins} from page \pageref{sec:skins}.
\end{marker}


\subsubsection{Arcs}
\begin{docTcbKey}{arc}{=\meta{length}}{no default, initially \texttt{1mm}}
  Sets the inner radius of the four frame arcs to \meta{length}.
\begin{exdispExample}{arc}
\tcbset{colback=red!5!white,colframe=red!75!black}

\begin{tcolorbox}[arc=0mm]
This is a \textbf{tcolorbox}.
\end{tcolorbox}
\begin{tcolorbox}[arc=3mm]
This is a \textbf{tcolorbox}.
\end{tcolorbox}
\end{exdispExample}
\end{docTcbKey}


\clearpage
\begin{docTcbKey}[][doc new=2015-05-05]{circular arc}{}{style, no value}
  Sets \refKey{/tcb/arc} to match the half of the inner width of the colored box.
  If width and height of the box are identical, this gives a circle.
  \begin{marker}
  If the height of the box is smaller than the width, the result will look
  quite ugly.
  \end{marker}
\begin{exdispExample*}{circular_arc}{sbs,lefthand ratio=0.6}
\begin{tcolorbox}[width=3cm,
  colback=red!5!white,
  colframe=red!75!black,
  halign=center,valign=center,
  square,circular arc]
This is a \textbf{tcolorbox}.
\end{tcolorbox}
\end{exdispExample*}
\end{docTcbKey}


\begin{docTcbKey}[][doc new=2015-05-05]{bean arc}{}{style, no value}
  Sets \refKey{/tcb/arc} to match the smaller value of the
  half of the inner width and of the inner height of the colored box.
  \begin{marker}
  This only works for a fixed \refKey{/tcb/height}. Also, \refKey{/tcb/bean arc}
  must be used \emph{after} width and height are set by option keys.
  \end{marker}
\begin{exdispExample*}{bean_arc}{sbs,lefthand ratio=0.6}
\tcbset{size=fbox,boxrule=0.5mm,
  colback=red!5!white,
  colframe=red!75!black,
  halign=center,valign=center}

\begin{tcolorbox}[width=3cm,height=2cm,
  bean arc]
Box A
\end{tcolorbox}

\begin{tcolorbox}[width=2cm,height=3cm,
  bean arc]
Box B
\end{tcolorbox}
\end{exdispExample*}
\end{docTcbKey}

\begin{docTcbKey}[][doc new=2015-05-05]{octogon arc}{}{style, no value}
  Sets \refKey{/tcb/arc} to match $\frac{1}{2+\sqrt{2}}$ of the inner width
  of the colored box. If width and height of the box are identical,
  the interior is a regular octogon.
\begin{exdispExample*}{octogon_arc}{sbs,lefthand ratio=0.8}
\begin{tcolorbox}[enhanced,
  size=minimal,auto outer arc,
  width=2.1cm,octogon arc,
  colback=red,colframe=white,colupper=white,
  fontupper=\fontsize{7mm}{7mm}\selectfont\bfseries\sffamily,
  halign=center,valign=center,
  square,arc is angular,
  borderline={0.2mm}{-1mm}{red}  ]
STOP
\end{tcolorbox}
\end{exdispExample*}
\end{docTcbKey}

\clearpage
\begin{docTcbKey}[][doc new=2015-05-05]{arc is angular}{}{no value, initially unset}
  Using this options applies a patch which straightens the corners arcs of
  the boxes. The little arcs are replaced by little straight lines.
  \begin{marker}
  This patch is considered as an experimental feature.
  It changes some of the original \tikzname\ code. This change may break
  with future updates of \tikzname.
  \end{marker}

\begin{exdispExample*}{arc_is_angular}{sbs,lefthand ratio=0.6}
\tcbset{colback=red!5!white,colframe=red!75!black,
  arc=3mm}

\begin{tcolorbox}[arc is angular]
This is a \textbf{tcolorbox}.
\end{tcolorbox}
\begin{tcolorbox}[arc is curved]
This is a \textbf{tcolorbox}.
\end{tcolorbox}
\end{exdispExample*}

\end{docTcbKey}


\begin{docTcbKey}[][doc new=2015-05-05]{arc is curved}{}{no value, initially set}
  This option resets the patch from \refKey{/tcb/arc is angular}. The
  original \tikzname\ code is activated.
\end{docTcbKey}


\begin{docTcbKey}{outer arc}{=\meta{length}}{no default, initially unset}
  Sets the outer radius of the four frame arcs to \meta{length}.
\begin{exdispExample}{outer_arc}
\tcbset{colback=red!5!white,colframe=red!75!black}

\begin{tcolorbox}[arc=4mm,outer arc=1mm]
This is a \textbf{tcolorbox}.
\end{tcolorbox}
\end{exdispExample}
\end{docTcbKey}


\begin{docTcbKey}{auto outer arc}{}{no value, initially set}
  Sets the outer radius of the four frame arcs automatically in
  dependency of the inner radius given by \refKey{/tcb/arc}.
\end{docTcbKey}


\clearpage
\subsubsection{Spacing}
\begin{docTcbKey}{boxsep}{=\meta{length}}{no default, initially \texttt{1mm}}
  Sets a common padding of \meta{length} between the text content and the
  frame of the box. This value is added to the key values of
  |left|, |right|, |top|, |bottom|, and |middle| at the appropriate places.
\begin{exdispExample}{boxsep}
\tcbset{colback=red!5!white,colframe=red!75!black,width=(\linewidth-4mm)/2,
  before=,after=\hfill}

\begin{tcolorbox}[boxsep=5mm]
This is a \textbf{tcolorbox}.
\end{tcolorbox}
\begin{tcolorbox}[boxsep=5mm,draft]
This is a \textbf{tcolorbox}.
\end{tcolorbox}
\end{exdispExample}
\end{docTcbKey}


\begin{docTcbKey}{left}{=\meta{length}}{style, no default, initially \texttt{4mm}}
  Sets the left space between all text parts and frame (additional to |boxsep|).
  This is an abbreviation for setting
  |lefttitle|, |leftupper|, and |leftlower| to the same value.
\begin{exdispExample}{left}
\tcbset{colback=red!5!white,colframe=red!75!black}

\begin{tcolorbox}[left=0mm]
This is a \textbf{tcolorbox}.
\end{tcolorbox}
\end{exdispExample}
\end{docTcbKey}


\begin{docTcbKey}[][doc new=2017-02-16]{left*}{=\meta{length}}{style, no default}
  Sets \refKey{/tcb/left} such that \meta{length} is the distance between
  the left bounding box and the text parts.
\begin{exdispExample}{left_star}
\tcbset{colback=red!5!white,colframe=red!75!black}

This is some text.
\begin{tcolorbox}[grow to left by=5mm,left*=0mm,
  enhanced,show bounding box]
This is a \textbf{tcolorbox}.
\end{tcolorbox}
\end{exdispExample}
\end{docTcbKey}

\clearpage
\begin{docTcbKey}{lefttitle}{=\meta{length}}{no default, initially \texttt{4mm}}
  Sets the left space between title text and frame (additional to |boxsep|).
\begin{exdispExample}{lefttitle}
\tcbset{colback=red!5!white,colframe=red!75!black}

\begin{tcolorbox}[lefttitle=3cm,title=My Title]
This is a \textbf{tcolorbox}.
\end{tcolorbox}
\end{exdispExample}
\end{docTcbKey}


\begin{docTcbKey}{leftupper}{=\meta{length}}{no default, initially \texttt{4mm}}
  Sets the left space between upper text and frame (additional to |boxsep|).
\begin{exdispExample}{leftupper}
\tcbset{colback=red!5!white,colframe=red!75!black}

\begin{tcolorbox}[leftupper=3cm,title=My Title]
This is a \textbf{tcolorbox}.
\end{tcolorbox}
\end{exdispExample}
\end{docTcbKey}

\begin{docTcbKey}{leftlower}{=\meta{length}}{no default, initially \texttt{4mm}}
  Sets the left space between lower text and frame (additional to |boxsep|).
\begin{exdispExample}{leftlower}
\tcbset{colback=red!5!white,colframe=red!75!black}

\begin{tcolorbox}[leftlower=3cm]
This is a \textbf{tcolorbox}.
\tcblower
This is the lower part.
\end{tcolorbox}
\end{exdispExample}
\end{docTcbKey}

\enlargethispage*{1cm}

\begin{docTcbKey}{right}{=\meta{length}}{style, no default, initially \texttt{4mm}}
  Sets the right space between all text parts and frame (additional to |boxsep|).
  This is an abbreviation for setting
  |righttitle|, |rightupper|, and |rightlower| to the same value.
\begin{exdispExample}{right}
\tcbset{colback=red!5!white,colframe=red!75!black}

\begin{tcolorbox}[width=5cm,right=2cm]
This is a \textbf{tcolorbox}.
\end{tcolorbox}
\end{exdispExample}
\end{docTcbKey}

\clearpage

\begin{docTcbKey}[][doc new=2017-02-16]{right*}{=\meta{length}}{style, no default}
  Sets \refKey{/tcb/right} such that \meta{length} is the distance between
  the right bounding box and the text parts.
\begin{exdispExample}{right_star}
\tcbset{colback=red!5!white,colframe=red!75!black}

\flushright This is some text.
\begin{tcolorbox}[grow to right by=5mm,right*=0mm,
  halign=right,enhanced,show bounding box]
This is a \textbf{tcolorbox}.
\end{tcolorbox}
\end{exdispExample}
\end{docTcbKey}



\begin{docTcbKey}{righttitle}{=\meta{length}}{no default, initially \texttt{4mm}}
  Sets the right space between title text and frame (additional to |boxsep|).
\begin{exdispExample}{righttitle}
\tcbset{colback=red!5!white,colframe=red!75!black}

\begin{tcolorbox}[width=5cm,righttitle=2cm,title=My very long title text]
This is a \textbf{tcolorbox} with standard upper box dimensions.
\end{tcolorbox}
\end{exdispExample}
\end{docTcbKey}


\begin{docTcbKey}{rightupper}{=\meta{length}}{no default, initially \texttt{4mm}}
  Sets the right space between upper text and frame (additional to |boxsep|).
\begin{exdispExample}{rightupper}
\tcbset{colback=red!5!white,colframe=red!75!black}

\begin{tcolorbox}[width=5cm,rightupper=2cm,title=My very long title text]
This is a \textbf{tcolorbox} with compressed upper box dimensions.
\end{tcolorbox}
\end{exdispExample}
\end{docTcbKey}


\clearpage
\begin{docTcbKey}{rightlower}{=\meta{length}}{no default, initially \texttt{4mm}}
  Sets the right space between lower text and frame (additional to |boxsep|).
\begin{exdispExample}{rightlower}
\tcbset{colback=red!5!white,colframe=red!75!black}

\begin{tcolorbox}[width=5cm,rightlower=2cm]
This is a \textbf{tcolorbox} with standard upper box dimensions.
\tcblower
This is the lower part with large space at right.
\end{tcolorbox}
\end{exdispExample}
\end{docTcbKey}



\begin{docTcbKey}{top}{=\meta{length}}{no default, initially \texttt{2mm}}
  Sets the top space between text and frame (additional to |boxsep|).
\begin{exdispExample}{top}
\tcbset{colback=red!5!white,colframe=red!75!black}

\begin{tcolorbox}[top=0mm]
This is a \textbf{tcolorbox}.
\tcblower
This is the lower part.
\end{tcolorbox}
\end{exdispExample}
\end{docTcbKey}


\begin{docTcbKey}{toptitle}{=\meta{length}}{no default, initially \texttt{0mm}}
  Sets the top space between title and frame (additional to |boxsep|).
\begin{exdispExample}{toptitle}
\tcbset{colback=red!5!white,colframe=red!75!black}

\begin{tcolorbox}[toptitle=3mm,title=My title]
This is a \textbf{tcolorbox}.
\end{tcolorbox}
\end{exdispExample}
\end{docTcbKey}

\clearpage
\begin{docTcbKey}{bottom}{=\meta{length}}{no default, initially \texttt{2mm}}
  Sets the bottom space between text and frame (additional to |boxsep|).
\begin{exdispExample}{bottom}
\tcbset{colback=red!5!white,colframe=red!75!black}

\begin{tcolorbox}[bottom=0mm]
This is a \textbf{tcolorbox}.
\tcblower
This is the lower part.
\end{tcolorbox}
\end{exdispExample}
\end{docTcbKey}

\begin{docTcbKey}{bottomtitle}{=\meta{length}}{no default, initially \texttt{0mm}}
  Sets the bottom space between title and frame (additional to |boxsep|).
\begin{exdispExample}{bottomtitle}
\tcbset{colback=red!5!white,colframe=red!75!black}

\begin{tcolorbox}[bottomtitle=3mm,title=My title]
This is a \textbf{tcolorbox}.
\end{tcolorbox}
\end{exdispExample}
\end{docTcbKey}


\begin{docTcbKey}{middle}{=\meta{length}}{no default, initially \texttt{2mm}}
  Sets the space between upper and lower text to the separation line
  (additional to |boxsep|).
\begin{exdispExample}{middle}
\tcbset{colback=red!5!white,colframe=red!75!black}

\begin{tcolorbox}[middle=0mm,boxsep=0mm]
This is a \textbf{tcolorbox}.
\tcblower
This is the lower part.
\end{tcolorbox}
\end{exdispExample}
\end{docTcbKey}


\clearpage
\subsubsection{Size Shortcuts}
\begin{docTcbKey}{size}{=\meta{name}}{no default, initially \texttt{normal}}
  Sets all geometry keys with exception of \refKey{/tcb/width} to
  predefined length values.
  For \meta{name}, the following values are feasible:
  \begin{itemize}
  \item\docValue{normal}: normal sized boxes e.\,g.\ of width |\linewidth|.
  \item\docValue{title}: title line sized boxes.
  \item\docValue{small}: small boxes e.\,g.\ for keyword highlighting.
  \item\docValue{fbox}: identical to the standard |\fbox|.
  \item\docValue{tight}: no padding space at all.
  \item\docValue{minimal}: no padding space, no box rules.
  \end{itemize}

\begin{exdispExample}{size_1}
\tcbset{colback=red!5!white,colframe=red!75!black}

\foreach \s in {normal,title,small,fbox,tight,minimal} {
  \tcbox[size=\s,on line]{\s} }

\foreach \s in {normal,title,small,fbox,tight,minimal} {
  \tcbox[size=\s,on line,title=Test]{\s} }

\foreach \s in {normal,title,small,fbox,tight,minimal} {
  \begin{tcolorbox}[size=\s,on line,title=Test,width=2.2cm]
    \s \tcblower lower\end{tcolorbox} }
\end{exdispExample}

\bigskip

\begin{tcolorbox}[tabularx={l|XXXXXX},title=Predefined values,
enhanced,fonttitle=\small\bfseries,fontupper=\small\ttfamily,
colback=yellow!10!white,colframe=red!50!black,colbacktitle=red!15!white,
coltitle=black,center title
]
            & normal & title  & small & fbox  & tight & minimal\\\hline
boxrule     & 0.5mm  & 0.4mm  & 0.3mm & 0.4pt & 0.4pt & 0.0pt \\
boxsep      & 1.0mm  & 1.0mm  & 1.0mm & 3.0pt & 0.0pt & 0.0pt \\
left        & 4.0mm  & 2.0mm  & 1.0mm & 0.0pt & 0.0pt & 0.0pt \\
right       & 4.0mm  & 2.0mm  & 1.0mm & 0.0pt & 0.0pt & 0.0pt \\
top         & 2.0mm  & 0.25mm & 0.0mm & 0.0pt & 0.0pt & 0.0pt \\
bottom      & 2.0mm  & 0.25mm & 0.0mm & 0.0pt & 0.0pt & 0.0pt \\
toptitle    & 0.0mm  & 0.0mm  & 0.0mm & 0.0pt & 0.0pt & 0.0pt \\
bottomtitle & 0.0mm  & 0.0mm  & 0.0mm & 0.0pt & 0.0pt & 0.0pt \\
middle      & 2.0mm  & 0.75mm & 0.5mm & 1.0pt & 0.2pt & 0.0pt \\
arc         & 1.0mm  & 0.75mm & 0.5mm & 1.0pt & 0.0pt & 0.0pt \\
outer arc   & auto   & auto   & auto  & auto  & 0.0pt & 0.0pt \\
\end{tcolorbox}
\end{docTcbKey}


\clearpage
\begin{docTcbKey}{oversize}{\colOpt{=\meta{length}}}{style, default |0pt|}
  Sets the text width of the upper part to the current line width plus an
  optional \meta{length}.
  This is achieved by changing the keys \refKey{/tcb/width}
  \refKey{/tcb/enlarge left by}, and
  \refKey{/tcb/enlarge right by} appropriately.
  The resulting box is overlapping into the left and right margin of
  the page.
  Note that this style option has to be given \emph{after} all other
  geometry keys!
  Also see \refKey{/tcb/grow sidewards by} and \refKey{/tcb/spread sidewards}.
\begin{dispListing}
\tcbset{colback=red!5!white,colframe=red!75!black,fonttitle=\bfseries}

\textit{Normal text for comparison:}\\
\lipsum[2]

\begin{tcolorbox}[oversize,title=Oversized box]
\lipsum[2]
\end{tcolorbox}

\begin{tcolorbox}[title=Normal box]
\lipsum[2]
\end{tcolorbox}
\end{dispListing}
\end{docTcbKey}

{\tcbusetemp}


\clearpage
\subsubsection{Toggle Left and Right}
\begin{docTcbKey}[][doc updated=2017-02-16]{toggle left and right}{=\meta{toggle preset}}{default |evenpage|, initially |none|}
  According to the \meta{toggle preset}, the left and the right settings
  of the |tcolorbox| are switched or not. Feasible values are:
  \begin{itemize}
  \item\docValue{none}: no switching.
  \item\docValue{forced}: the values of the left and right rules, spaces, and corners are switched.
  \item\docValue{evenpage}: if the page is an even page, the values of the left and
    right rules, spaces, and corners are switched. This value also sets
    \refKey{/tcb/check odd page} to |true|.
  \end{itemize}
\begin{marker}
Horizontal bounding box enlargements are not toggled by this option.
They can be toggled independently by \refKey{/tcb/toggle enlargement}.
For example, \refKey{/tcb/oversize} changes the bounding box.
\end{marker}
\begin{dispListing}
% \usepackage{lipsum}
% \usetikzlibrary{patterns}
% \tcbuselibrary{skins,breakable}
\begin{tcolorbox}[enhanced,breakable,
  toggle left and right,sharp corners,
  boxrule=0mm,top=0mm,bottom=0mm,left=1mm,right=1mm,
  rightrule=1cm,colupper=blue!25!black,
  interior style={fill overzoom image=lichtspiel.jpg,fill image opacity=0.25},
  frame style={pattern=crosshatch dots light steel blue},
  overlay={%
    \begin{tcbclipframe}
    \tcbifoddpage{\coordinate (X) at ([xshift=-5mm]frame.east);}
                 {\coordinate (X) at ([xshift=5mm]frame.west);}
    \fill[shading=ball,ball color=blue!50!white,opacity=0.5] (X) circle (4mm);
    \end{tcbclipframe}}]
\lipsum[1-6]
\end{tcolorbox}
\end{dispListing}
\medskip

This example switches a |1cm| thick rule from the left to the right side
depending on the page number. Thereby, the rule is always on the outer side
of the double-sided paper. Additionally, a ball is drawn on the outer side
with help of an overlay.
\bigskip

\tcbusetemp
\end{docTcbKey}

\clearpage
\subsection{Corners}\label{subsec:corners}

The four corners of any |tcolorbox| can be set individually as
\refKey{/tcb/sharp corners} or as \refKey{/tcb/rounded corners}.
These settings are also reflected in the behavior of \refKey{/tcb/borderline}
and \refKey{/tcb/shadow} as one would expect.

By default, all four corners are \emph{rounded}. So, only the
\refKey{/tcb/sharp corners} option will be necessary for most use cases.
The \refKey{/tcb/rounded corners} option can be used to revert a \refKey{/tcb/sharp corners}
setting.

\begin{docTcbKey}{sharp corners}{=\meta{position}}{default |all|, initially unset}
The \meta{position} denotes one or more of the four box corners to be set as
\emph{sharp} corners. The not assigned corners will retain their mode.
Feasible values for \meta{position} are:
\begin{itemize}
\foreach \p in {northwest,northeast,southwest,southeast,north,south,east,west,downhill,uphill,all}
{
\item\tcbox[on line,size=title,arc=2mm,colframe=red!75!black,colback=red!5!white,
  enlarge top by=0.5mm,enlarge bottom by=0.5mm,sharp corners=\p]{\docValue{\p}}
}
\end{itemize}
\begin{exdispExample*}{sharp_corners_1}{sbs,lefthand ratio=0.6}
\begin{tcolorbox}[colback=red!5!white,
  colframe=red!75!black,
  sharp corners=northwest ]
This is a \textbf{tcolorbox}.
\end{tcolorbox}
\end{exdispExample*}
\begin{exdispExample*}{sharp_corners_2}{sbs,lefthand ratio=0.6}
\begin{tcolorbox}[colback=red!5!white,
  colframe=red!75!black,
  sharp corners ]
This is a \textbf{tcolorbox}.
\end{tcolorbox}
\end{exdispExample*}
\end{docTcbKey}

\clearpage
\begin{docTcbKey}{rounded corners}{=\meta{position}}{default |all|, initially |all|}
The \refKey{/tcb/rounded corners} can be used to revert a \refKey{/tcb/sharp corners}
setting. The \meta{position} denotes one or more of the four box corners to be set as
\emph{rounded} corners. The not assigned corners will retain their mode.
Feasible values for \meta{position} are\footnote{The graphical examples assume
  that the boxes where set to have sharp corners before.}:
\begin{itemize}
\foreach \p in {northwest,northeast,southwest,southeast,north,south,east,west,downhill,uphill,all}
{
\item\tcbox[on line,size=title,arc=2mm,colframe=red!75!black,colback=red!5!white,
  enlarge top by=0.5mm,enlarge bottom by=0.5mm,sharp corners,rounded corners=\p]{\docValue{\p}}
}
\end{itemize}
\begin{exdispExample*}{rounded_corners}{sbs,lefthand ratio=0.6}
\begin{tcolorbox}[colback=red!5!white,
  colframe=red!75!black,sharp corners,
  rounded corners=northwest ]
This is a \textbf{tcolorbox}.
\end{tcolorbox}
\end{exdispExample*}
\end{docTcbKey}


\begin{docTcbKey}{sharpish corners}{}{style, no value}
  Shortcut for setting \refKey{/tcb/arc} and \refKey{/tcb/outer arc}
  to |0pt|. With this setting, rounded corners will appear as quasi-sharp,
  but e.\,g.\ the shadow will be somewhat rounder than the shadow
  of really sharp corners.
  \begin{marker}
  Corners are still of type \emph{rounded} with this option, but appear
  \emph{sharp}. To switch back to rounded corners, one has to adapt
  \refKey{/tcb/arc} and \refKey{/tcb/outer arc}.
  \end{marker}
\begin{exdispExample*}{sharpish_corners}{sbs,lefthand ratio=0.6}
\begin{tcolorbox}[colback=red!5!white,
  colframe=red!75!black,
  sharpish corners ]
This is a \textbf{tcolorbox}.
\end{tcolorbox}
\end{exdispExample*}
\end{docTcbKey}

\clearpage

The following examples will show the differences between
\refKey{/tcb/rounded corners}, \refKey{/tcb/sharpish corners}, and \refKey{/tcb/sharp corners}.
The later two give the same core box, but \refKey{/tcb/borderline}
and \refKey{/tcb/shadow} settings are slightly different.
The following examples use \refKey{/tcb/drop fuzzy shadow}.

\begin{extcolorbox}[minipage]{corners_comparison}[blankest]
\foreach \n in {rounded corners,sharpish corners,sharp corners}{
\begin{tcolorbox}[enhanced jigsaw,frame empty,interior empty,fuzzy halo,halign=center,beforeafter skip=4mm]
\begin{tcolorbox}[enhanced,drop fuzzy shadow,width=\linewidth-1cm,
  colback=red!5!white, colframe=red!75!black, fonttitle=\bfseries,
  title=My title,\n,
  tikz={spy using outlines={circle, magnification=8, size=2cm, connect spies}},
  overlay={\spy [blue, size=4cm] on (frame.south east)
      in node at ([xshift=-2.5cm,yshift=-2.5cm]frame.south east);
  \node[right] at ([xshift=2cm,yshift=-1cm]frame.south west) {\textbf{\Large\ttfamily\n}};
  }]
This is a \textbf{tcolorbox}.
\end{tcolorbox}
\end{tcolorbox}}
\end{extcolorbox}

\clearpage
\subsection{Transparency}

\begin{marker}
Transparency effects are likely to be used in conjunction with \emph{jigsaw}
skin variants, see \zvref{subsec:skinjigsaw}.
\end{marker}

\begin{docTcbKey}{opacityframe}{=\meta{fraction}}{no default, initially \texttt{1.0}}
  Sets the frame opacity of the box to the given \meta{fraction}.
\begin{exdispExample*}{opacityframe}{sbs,lefthand ratio=0.6,
  segmentation style={preaction={fill=white},pattern=checkerboard,pattern color=gray!40}
  }
\begin{tcolorbox}[opacityframe=0.25,
  colframe=red]
This is a \textbf{tcolorbox}.
\end{tcolorbox}
\end{exdispExample*}
\end{docTcbKey}

\begin{docTcbKey}{opacityback}{=\meta{fraction}}{no default, initially \texttt{1.0}}
  Sets the background opacity of the box to the given \meta{fraction}.
\begin{exdispExample*}{opacityback}{sbs,lefthand ratio=0.6,segmentation style={preaction={fill=white},pattern=checkerboard,pattern color=gray!40}}
\begin{tcolorbox}[standard jigsaw,colframe=red,
  opacityframe=0.5, opacityback=0.5]
This is a \textbf{tcolorbox}.
\end{tcolorbox}
\end{exdispExample*}
\end{docTcbKey}

Also see \refKey{/tcb/opacitybacklower} of the \mylib{skins} library.

\begin{docTcbKey}{opacitybacktitle}{=\meta{fraction}}{no default, initially \texttt{1.0}}
  Sets the title background opacity of the box to the given \meta{fraction}.
\begin{exdispExample*}{opacitybacktitle}{sbs,lefthand ratio=0.6,segmentation style={preaction={fill=white},pattern=checkerboard,pattern color=gray!40}}
\begin{tcolorbox}[standard jigsaw,colframe=red,
  opacityframe=0.5, opacitybacktitle=0.5,
  title filled, title=This is a title]
This is a \textbf{tcolorbox}.
\end{tcolorbox}
\end{exdispExample*}
\end{docTcbKey}


\begin{docTcbKey}{opacityfill}{=\meta{fraction}}{style, no default, initially \texttt{1.0}}
  Sets the fill opacity for frame, interior and optionally the title background
  to the given \meta{fraction}.
\begin{exdispExample*}{opacityfill}{sbs,lefthand ratio=0.6,segmentation style={preaction={fill=white},pattern=checkerboard,pattern color=gray!40}}
\begin{tcolorbox}[standard jigsaw,colframe=red,
  opacityfill=0.7, title=This is a title]
This is a \textbf{tcolorbox}.
\end{tcolorbox}
\end{exdispExample*}
\end{docTcbKey}

\clearpage
\begin{docTcbKey}{opacityupper}{=\meta{fraction}}{no default, initially \texttt{1.0}}
  Sets the text opacity of the upper box part to the given \meta{fraction}.
\begin{exdispExample*}{opacityupper}{sbs,lefthand ratio=0.6}
\begin{tcolorbox}[enhanced,opacityupper=0.5,
  interior style={preaction={fill=white},pattern=checkerboard,pattern color=gray!40}]
This is a \textbf{tcolorbox}.
\end{tcolorbox}
\end{exdispExample*}
\end{docTcbKey}


\begin{docTcbKey}{opacitylower}{=\meta{fraction}}{no default, initially \texttt{1.0}}
  Sets the text opacity of the lower box part to the given \meta{fraction}.
\begin{exdispExample*}{opacitylower}{sbs,lefthand ratio=0.6}
\begin{tcolorbox}[enhanced,opacitylower=0.5,
  interior style={preaction={fill=white},pattern=checkerboard,pattern color=gray!40}]
This is a \textbf{tcolorbox}.
\tcblower
This is the lower part.
\end{tcolorbox}
\end{exdispExample*}
\end{docTcbKey}

\begin{docTcbKey}{opacitytext}{=\meta{fraction}}{no default, initially \texttt{1.0}}
  Sets the text opacity of the upper and the lower box part to the given \meta{fraction}.
\begin{exdispExample*}{opacitytext}{sbs,lefthand ratio=0.6}
\begin{tcolorbox}[enhanced,opacitytext=0.5,
  interior style={preaction={fill=white},pattern=checkerboard,pattern color=gray!40}]
This is a \textbf{tcolorbox}.
\tcblower
This is the lower part.
\end{tcolorbox}
\end{exdispExample*}
\end{docTcbKey}


\begin{docTcbKey}{opacitytitle}{=\meta{fraction}}{no default, initially \texttt{1.0}}
  Sets the text opacity of the box title to the given \meta{fraction}.
\begin{exdispExample*}{opacitytitle}{sbs,lefthand ratio=0.6}
\begin{tcolorbox}[enhanced,opacitytitle=0.7,
  coltitle=black,
  fonttitle=\bfseries,title=This is a title,
  title style={preaction={fill=white},pattern=checkerboard,pattern color=gray!40}]
This is a \textbf{tcolorbox}.
\end{tcolorbox}
\end{exdispExample*}
\end{docTcbKey}


\begin{exdispExample*}{opacity_general}{segmentation style={preaction={fill=white},pattern=checkerboard,pattern color=gray!40}}
\begin{tcolorbox}[enhanced jigsaw,fonttitle=\bfseries,title=This is a title,
  opacityframe=0.5,opacityback=0.25,opacitybacktitle=0.25,opacitytext=0.8,
  colback=red!5!white,colframe=red!75!black,colbacktitle=yellow!20!red]
This is a \textbf{tcolorbox}.
\end{tcolorbox}
\end{exdispExample*}



\clearpage
\subsection{Height Control}\label{sec:heightcontrol}
In a typical usage scenario, the height of a |tcolorbox| is computed automatically
to fit the content. Nevertheless, the height can be set to a fixed value
or to fit commonly for several boxes, e.\,g.\ if boxes are set side by side.

\bigskip
\begin{marker}
  The height control keys are only applicable to unbreakable boxes.
  If a box is set to be \refKey{/tcb/breakable}, the height is always
  computed according to the \emph{natural height}.
\end{marker}
\bigskip


\begin{docTcbKey}{natural height}{}{no value, initially set}
  Sets the total height of the colored box to its natural height depending
  on the box content.
\end{docTcbKey}

\begin{docTcbKey}{height}{=\meta{length}}{no default}
  Sets the total height of the colored box to \meta{length} independent
  of the box content. \meta{length} is the minimum height of the box, if
  \refKey{/tcb/height plus} is larger than zero.
\begin{exdispExample}{height}
\tcbset{width=(\linewidth-2mm)/3,before=,after=\hfill,
colframe=blue!75!black,colback=white}

\begin{tcolorbox}[height=1cm,valign=center]
  This box has a height of 1cm.
\end{tcolorbox}
\begin{tcolorbox}[height=2cm,valign=center]
  This box has a height of 2cm.
\end{tcolorbox}
\begin{tcolorbox}[height=3cm,split=0.5,valign=center,valign lower=center]
  This box has a height of 3cm.
  \tcblower
  Lower part.
\end{tcolorbox}
\end{exdispExample}
\end{docTcbKey}

\enlargethispage*{10mm}
\begin{docTcbKey}{height plus}{=\meta{length}}{no default, initially |0pt|}
  The box may extend a given fixed \refKey{/tcb/height} up to the given \meta{length}.
\begin{exdispExample}{height_plus}
\tcbset{colback=red!5!white,colframe=red!75!black,left=1mm,top=1mm,bottom=1mm,
  right=1mm,boxsep=0mm,width=3cm,nobeforeafter}

\begin{tcolorbox}[height=1cm]
This is a tcolorbox.
\end{tcolorbox}
\begin{tcolorbox}[height=1cm,height plus=1cm]
This is a tcolorbox.
\end{tcolorbox}
\begin{tcolorbox}[height=1cm,height plus=1cm]
This is a tcolorbox. This is a tcolorbox. This is a tcolorbox.
\end{tcolorbox}
\end{exdispExample}
\end{docTcbKey}


\begin{docTcbKey}{height from}{=\meta{min} \texttt{to} \meta{max}}{style, no default}
  Sets the box height to a dimension between \meta{min} and \meta{max}.
\begin{exdispExample}{height_from}
% \usepackage{lipsum}
\newtcolorbox{mybox}{colback=red!5!white,colframe=red!75!black,left=1mm,top=1mm,
  bottom=1mm,right=1mm,boxsep=0mm,width=4.5cm,nobeforeafter,
  height from=2cm to 8cm}

\begin{mybox}
This is a tcolorbox.
\end{mybox}
\begin{mybox}
This is a tcolorbox. This is a tcolorbox. This is a tcolorbox.
\end{mybox}
\begin{mybox}
\lipsum[2]
\end{mybox}
\end{exdispExample}
\end{docTcbKey}


\begin{docTcbKey}[][doc new=2014-10-31]{text height}{=\meta{length}}{style, no default}
  Sets the text height to \meta{length}. This is the length from the top
  of the upper part to the bottom of the optional lower part.
  See also \refKey{/tcb/text width}.
\begin{exdispExample}{text_height}
\tcbset{colback=red!5!white,colframe=red!75!black}

\begin{tcolorbox}[text height=2cm]
This is a \textbf{tcolorbox} where the text area has a height of 2cm.
\end{tcolorbox}
\end{exdispExample}
\end{docTcbKey}

\clearpage

\begin{docTcbKey}[][doc new=2014-11-07]{add to height}{=\meta{length}}{style, no default}
  Adds \meta{length} to the current height of the colored box.
  \refKey{/tcb/height} has to be set before this key is used!
  If this option is used several times, then the \refKey{/tcb/height} is
  also increased several times.
\begin{exdispExample}{add_to_height}
\tcbset{height=2cm,
  valign=center,width=(\linewidth-2mm)/2,
  before=,after=\hfill,colframe=blue!75!black,colback=white}

\begin{tcolorbox}
  This box has a height of 2cm.
\end{tcolorbox}
\begin{tcolorbox}[add to height=1cm]
  This box has a height of 3cm.
\end{tcolorbox}
\end{exdispExample}
\end{docTcbKey}


\begin{docTcbKey}[][doc new=2016-02-16]{add to natural height}{=\meta{length}}{style, no default}
  The application of this option generates a box with natural height plus
  the given \meta{length}. If this option is used several times, then the
  last setting of \meta{length} wins. The resulting box is not considered
  a fixed height box and the implementation is quite different to
  \refKey{/tcb/add to height}.
\begin{exdispExample}{add_to_natural_height}
\tcbset{valign=center,width=(\linewidth-2mm)/2,
  before=,after=\hfill,colframe=blue!75!black,colback=white}

\begin{tcolorbox}
  This box has natural height.
\end{tcolorbox}
\begin{tcolorbox}[add to natural height=1cm]
  This box has natural height plus 1 cm.
\end{tcolorbox}
\end{exdispExample}
\end{docTcbKey}


\clearpage
\begin{docTcbKey}[][doc new and updated={2014-09-22}{2016-02-17}]{height fill}{\colOpt{=true\textbar false\textbar maximum}}{default |true|, initially |false|}
If set to \docValue*{true}, the height of the |tcolorbox| is set to the rest of the
available vertical space of the current page.
If set to \docValue{maximum}, the page is compressed as much as possible.
Note that the |tcolorbox|
is always set as its own paragraph using this option.
Also see \refKey{/tcb/text fill}.
\begin{marker}
Note that the library \mylib{breakable} has to be loaded to use this key!
\end{marker}
This height control key is only applicable to unbreakable boxes, but it
uses code from the library \mylib{breakable}.
The counterpart for breakable boxes is \refKey{/tcb/height fixed for}.

This option can and should not be used for boxes in boxes, but it can be
used for boxes inside a \refEnv{tcbraster}.

\begin{dispListing}
% \usepackage{lipsum}
% \tcbuselibrary{breakable}
\begin{tcolorbox}[height fill,
  colback=red!5!white,colframe=red!75!black,fonttitle=\bfseries,
  title=Box which fills the rest of the page]
\lipsum[1]
\end{tcolorbox}
\end{dispListing}
\end{docTcbKey}
{\tcbusetemp}


\clearpage
\begin{docTcbKey}[][doc new={2017-06-28}]{inherit height}{\colOpt{=\meta{fraction}}}{default |1|, initially unset}
  If this option is used for a |tcolorbox| which is embedded inside
  another (outer) |tcolorbox| \emph{and} if this outer |tcolorbox| has
  a fixed height, then the given \meta{fraction} of the available text height
  of the outer |tcolorbox| is used as \refKey{/tcb/height} for the current
  |tcolorbox|.
  Otherwise, \refKey{/tcb/natural height} is applied for the current
  |tcolorbox|.

\begin{exdispExample}{inherit_height}
\tcbset{colframe=blue!75!black,colback=white,fonttitle=\bfseries}

\begin{tcolorbox}[title=Outer box with fixed height 3.5cm,height=3.5cm]
  \begin{tcolorbox}[title=Inner box,nobeforeafter,inherit height]
    This inner box matches the available space.
  \end{tcolorbox}
\end{tcolorbox}

\begin{tcolorbox}[title=Outer box with natural height]
  \begin{tcolorbox}[title=Inner box,nobeforeafter,inherit height]
    This inner box has its natural height.
  \end{tcolorbox}
\end{tcolorbox}

\begin{tcolorbox}[title=Outer box with fixed height 5cm,height=5cm]
  \begin{tcolorbox}[title=Inner box,nobeforeafter,inherit height]
    \begin{tcolorbox}[colframe=red,beforeafter skip=0pt,inherit height=0.6]
      Deeply nested box using 60 percent of the available space.
    \end{tcolorbox}
    \begin{tcolorbox}[colframe=red,beforeafter skip=0pt,inherit height=0.4]
      Deeply nested box using 40 percent of the available space.
    \end{tcolorbox}
  \end{tcolorbox}
\end{tcolorbox}
\end{exdispExample}
\end{docTcbKey}


\clearpage

\begin{docTcbKey}[][doc new=2015-05-05]{square}{}{style, no value}
  Sets \refKey{/tcb/height} to match the width of the colored box.
\begin{exdispExample*}{square}{sbs,lefthand ratio=0.6}
\begin{tcolorbox}[width=3cm,
  colback=red!5!white,
  colframe=red!75!black,
  halign=center,valign=center,
  square]
This is a \textbf{tcolorbox}.
\end{tcolorbox}
\end{exdispExample*}
\end{docTcbKey}



\begin{docTcbKey}{space}{=\meta{fraction}}{no default, initially 0}
  If the height of a |tcolorbox| is not the natural height, the space
  difference between the forced and the natural size is distributed
  between the upper and the lower part of the box. This space could also
  be negative.
  \meta{fraction} with a value between 0 and 1 is the amount of space
  which is added to the upper part, the rest is added to the lower part.
  If there is no lower part, then all of the space is added to
  the upper part always.
\begin{exdispExample}{fraction}
\tcbset{width=(\linewidth-2mm)/3,before=,after=\hfill,
colframe=blue!75!black,colback=white,height=3cm}

\foreach \f in {0.2,0.4,0.7}
{\begin{tcolorbox}[space=\f]
    This is the upper part.
    \tcblower
    This is the lower part.
\end{tcolorbox}}
\end{exdispExample}
\end{docTcbKey}

\begin{docTcbKey}{space to upper}{}{style}
  This is an abbreviation for |space=1|, i.\,e., all extra space is added
  to the upper part.
\end{docTcbKey}

\begin{docTcbKey}{space to lower}{}{style, initially set}
  This is an abbreviation for |space=0|, i.\,e., all extra space is added
  to the lower part (if there is any).
\end{docTcbKey}

\clearpage
\begin{docTcbKey}{space to both}{}{style}
  This is an abbreviation for |space=0.5|, i.\,e., the extra space
  equally distributed between the upper and the lower part.
\begin{exdispExample}{space_to_both}
\tcbset{width=(\linewidth-2mm)/3,before=,after=\hfill,
colframe=blue!75!black,colback=white,height=3cm}

\foreach \myspace in {space to upper,space to both,space to lower}
{\begin{tcolorbox}[\myspace]
    This is the upper part.
    \tcblower
    This is the lower part.
\end{tcolorbox}}
\end{exdispExample}
\end{docTcbKey}



\begin{docTcbKey}[][doc new and updated={2015-02-15}{2020-07-30}]{space to}{=\meta{macro}}{no default, initially unset}
  If the height of a |tcolorbox| is not the natural height, the space
  difference between the forced and the natural size is saved into the
  given local \meta{macro}. This \meta{macro} can and should be used inside
  the box content to add content which is vertically sized to match \meta{macro}.
  \begin{marker}
    \begin{itemize}
    \item The actual length saved into \meta{macro} is adapted dynamically
      during several compilations -- at least two, but maybe more.
    \item Due to the adaption algorithm, objects can be sized with
      \meta{macro} plus any offset length.
    \item Never ever use \meta{macro} multiplied with a factor. The only
      exception to this rule is that the space can be split into parts which
      sum to \meta{macro}.
    \item Never use this in combination with \refKey{/tcb/fit}.
    \end{itemize}
  \end{marker}
\begin{exdispExample}[runs=3]{space_to_1}
\begin{tcolorbox}[colframe=blue!75!black,colback=white,height=3cm,
    space to=\myspace]
  This is my box of height 3cm. The space is filled with a picture:\\[2mm]
  \includegraphics[width=\linewidth,height=\myspace]{goldshade.png}\\[1mm]
  This is some other text.
\end{tcolorbox}
\end{exdispExample}

\begin{exdispExample}[runs=3]{space_to_2}
\begin{tcolorbox}[colframe=blue!75!black,colback=white,height=3cm,
    space to=\myspace]
  \includegraphics[width=\linewidth,
    height=0.33\dimexpr\myspace]{blueshade.png}\\[1mm]
  This is my box of height 3cm.\\[2mm]
  \includegraphics[width=\linewidth,
    height=0.67\dimexpr\myspace]{goldshade.png}
\end{tcolorbox}
\end{exdispExample}
\end{docTcbKey}




\begin{docTcbKey}{split}{=\meta{fraction}}{no default}
  If the height of a |tcolorbox| is not the natural height, the
  \meta{fraction} with a value between 0 and 1 determines the positioning
  of the segmentation between the upper and the lower part. Here, 0 stands
  for top and 1 for bottom. Note that the box is split regardless of
  the actual dimensions of the text parts!
\begin{exdispExample}{split}
\tcbset{width=(\linewidth-2mm)/3,before=,after=\hfill,height=3cm,
colback=white,colframe=blue!75!black,valign=center,valign lower=center}

\foreach \f in {0.1,0.5,0.8}
{\begin{tcolorbox}[split=\f]
This is the upper part.
\tcblower
This is the lower part with a lot of text in several lines.
\end{tcolorbox}}
\end{exdispExample}
\end{docTcbKey}




\clearpage
\begin{docTcbKey}[][doc updated=2014-11-07]{equal height group}{=\meta{id}}{no default}
  Boxes which are members of an |equal height group| will all get the
  same height, i.\,e., the maximum of all their natural heights. The
  \meta{id} serves to distinguish between different height groups.
  %This \meta{id} should contain only characters which are feasible
  %for \TeX\ macro names, typically alphabetic characters but no numerals
  %and spaces.
  Note that you have to compile twice to see changes and
  that height groups are global definitions.


\begin{exdispExample}[runs=2]{equal_height_group}
\tcbset{width=(\linewidth-2mm)/3,before=,after=\hfill,arc=0mm,
colframe=blue!75!black,colback=white,fonttitle=\bfseries}

\begin{tcolorbox}[equal height group=A,adjusted title={One}]
  My smallest box.
\end{tcolorbox}%
\begin{tcolorbox}[equal height group=A,adjusted title={Two}]
  This box is also small.
  \tcblower
  But with a lower part.
\end{tcolorbox}%
\begin{tcolorbox}[equal height group=A,adjusted title={Three}]
  This box contains a lot of text just to fill the space
  with word flowing and flowing and flowing until the box
  is filled with all of it.
\end{tcolorbox}\linebreak
%
\tcbset{width=(\linewidth-1mm)/2,before=,after=\hfill,arc=0mm,
colframe=red!75!black,colback=white}
%
\begin{tcolorbox}[equal height group=B]
  Now, we use another equal height group.
\end{tcolorbox}%
\begin{tcolorbox}[equal height group=B,after=]
  \begin{equation*}
    \int\limits_{0}^{1} x^2 = \frac13.
  \end{equation*}
\end{tcolorbox}
\end{exdispExample}
\end{docTcbKey}

\medskip
\begin{marker}
See \zvref{sec:raster} for more equal height options.
\end{marker}

\clearpage
\begin{docTcbKey}{minimum for equal height group}{=\meta{id}:\meta{length}}{no default, initially unset}
  Plants a \meta{length} into the equal height group with
  the given \meta{id}. This ensures that the height will not drop below
  \meta{length}. Note that you cannot reduce a computed height value
  by using this key with a small value.
  The difference to applying \refKey{/tcb/height} directly is that the boxes
  are never too small for their content.

\begin{dispExample}
\tcbset{colframe=blue!75!black,colback=white,arc=0mm,
  before=,after=\hfill,fonttitle=\bfseries,left=2mm,right=2mm,
  width=3.5cm,
  equal height group=C,
  minimum for equal height group=C:3.5cm}

\begin{tcolorbox}
  My first box. All boxes will get 3.5cm times 3.5cm
  if the content height is not too large.
\end{tcolorbox}%
\begin{tcolorbox}
  My second box.
  \tcblower
  This is the lower part.
\end{tcolorbox}%
\begin{tcblisting}{}
\textbf{Mixed}
with a listing.
\end{tcblisting}
\begin{tcolorbox}[title={Fourth box}]
  My final box.
\end{tcolorbox}%
\end{dispExample}
\end{docTcbKey}


\begin{docTcbKey}[][doc new=2016-03-24]{minimum for current equal height group}{=\meta{length}}{no default, initially unset}
  Sets \refKey{/tcb/minimum for equal height group} for the current equal height
  group. Apparently, this only works for an already known equal height group, i.\,e.,
  \refKey{/tcb/equal height group} has to be set \emph{before} this option is used.
  This option is likely to be used in combination with \refKey{/tcb/raster equal height}

\begin{exdispExample}[runs=2]{minimum_for_current_equal_height_group}
% \tcbuselibrary{raster}
\begin{tcbitemize}[raster equal height,colframe=blue!75!black,colback=white,
  raster every box/.style={minimum for current equal height group=2cm}]
  \tcbitem A
  \tcbitem B
\end{tcbitemize}
\end{exdispExample}

\end{docTcbKey}


\clearpage
\begin{docTcbKey}[][doc new and updated={2015-11-27}{2016-02-22}]{use height from group}{\colOpt{=\meta{id}}}{style, default current group}
  Sets the current box to a fixed \refKey{/tcb/height} which is copied from
  an equal height group with the given \meta{id}. If this height is not
  available during the current compilation, no fixed height setting is used.
  If \meta{id} is omitted, the current equal height group is used which has
  to be set before by \refKey{/tcb/equal height group}.\par
  Note that the natural height of the current box is not considered for
  computation of the group height. The main application for
  \refKey{/tcb/use height from group} is that the height can be adapted
  further by \refKey{/tcb/add to height}.

\begin{dispExample}
\begin{tcolorbox}[use height from group=C,add to height=-2cm,
  colframe=blue!75!black,colback=white]
Height from group \enquote{C} of the previous example, but reduced by 2cm.
\end{tcolorbox}%
\end{dispExample}

\begin{exdispExample}[runs=2]{use_height_from_group}
% \tcbuselibrary{raster}
Every line is inside an equal height group:
\begin{tcbraster}[raster equal height=rows,
    title=Box \thetcbrasternum,
    enhanced,size=small,colframe=red!50!black,colback=red!10!white]
  \begin{tcolorbox}First line\\second line\\
    The height of this box rules.\end{tcolorbox}
  \begin{tcolorbox}[use height from group]Test\end{tcolorbox}
  \begin{tcolorbox}[use height from group]
    First line\\second line\end{tcolorbox}
  \begin{tcolorbox}The height of this box rules.\end{tcolorbox}
\end{tcbraster}
\end{exdispExample}
\end{docTcbKey}



\begin{docCommand}[doc new=2015-11-27]{tcbheightfromgroup}{\marg{macro}\marg{id}}
  Saves the height from an equal height group with the given \meta{id}
  to a \meta{macro}. If this height is not available during the current compilation,
  \meta{macro} is set to |0pt|.
\end{docCommand}


\clearpage
\subsection{Box Content Additions}\label{subsec:contentadditions}
The following options introduce some arbitrary \meta{code} to the content
of a |tcolorbox|. These additions can be given at the beginning or at the ending
of the title, the upper part, or the lower part.

\begin{docTcbKey}[][doc updated=2023-01-27]{before title}{=\meta{code}}{no default, initially unset}
  The given \meta{code} is placed \emph{after} the color and font settings
  and \emph{before} the content of the title.
  The \meta{code} is appended by a final |\ignorespaces|.
\begin{exdispExample}{before_title}
\tcbset{before title={\textcolor{yellow}{\large Important:}~},
  colback=red!5!white,colframe=red!75!black,fonttitle=\bfseries}

\begin{tcolorbox}[title=My title]
This is a \textbf{tcolorbox}.
\end{tcolorbox}
\end{exdispExample}
\end{docTcbKey}


\begin{docTcbKey}[][doc new=2023-01-27]{before title*}{=\meta{code}}{no default, initially unset}
  The given \meta{code} is placed \emph{after} the color and font settings
  and \emph{before} the content of the title.
  In contrast to \refKey{/tcb/before title}, no |\ignorespaces| is appended.
\end{docTcbKey}


\begin{docTcbKey}[][doc updated=2023-01-27]{after title}{=\meta{code}}{no default, initially unset}
  The given \meta{code} is placed \emph{after} the content of the title.
  The \meta{code} is prepended by a leading |\unskip|.
\begin{exdispExample}{after_title}
\tcbset{after title={\hfill\colorbox{Navy}{approved}},
  colback=red!5!white,colframe=red!75!black,fonttitle=\bfseries}

\begin{tcolorbox}[title=My title]
This is a \textbf{tcolorbox}.
\end{tcolorbox}
\end{exdispExample}
\end{docTcbKey}


\begin{docTcbKey}[][doc new=2023-01-27]{after title*}{=\meta{code}}{no default, initially unset}
  The given \meta{code} is placed \emph{after} the content of the title.
  In contrast to \refKey{/tcb/after title}, no |\unskip| is prepended.
\end{docTcbKey}


\clearpage
\begin{docTcbKey}{before upper}{=\meta{code}}{no default, initially empty}
  The given \meta{code} is placed \emph{after} the color and font settings
  and \emph{before} the content of the upper part.
  The \meta{code} is appended by a final |\ignorespaces|.
\begin{exdispExample}{before_upper}
\tcbset{before upper={\textit{The story:}\par},
  colback=red!5!white,colframe=red!75!black,fonttitle=\bfseries}

\begin{tcolorbox}[title=My title]
This is a \textbf{tcolorbox}.
\end{tcolorbox}
\end{exdispExample}
\end{docTcbKey}


\begin{docTcbKey}[][doc new=2019-02-26]{before upper*}{=\meta{code}}{no default, initially unset}
  The given \meta{code} is placed \emph{after} the color and font settings
  and \emph{before} the content of the upper part.
  In contrast to \refKey{/tcb/before upper}, no |\ignorespaces| is appended.
  Use this for situations where |\ignorespaces| is not needed or causes harm.
\begin{exdispExample}{before_upper_star}
\begin{tcolorbox}[size=small,tile,
  colback=yellow!20,colbacktitle=yellow!70!black,
  title=My table,hbox,center,center title,
  before upper*=\begin{tabular}{cc},
  after upper*=\end{tabular},
]
  \multicolumn{2}{c}{Title}\\
  one & two \\
  three & four\\
\end{tcolorbox}
\end{exdispExample}
\end{docTcbKey}

\clearpage

\begin{docTcbKey}[][doc updated=2016-10-21]{after upper}{=\meta{code}}{no default, initially empty}
  The given \meta{code} is placed \emph{after} the content of the upper part.
  The \meta{code} is prepended by a leading |\unskip|.

\begin{exdispExample}{after_upper_1}
\tcbset{after upper={\par\hfill\textit{Read more next week}},
  colback=red!5!white,colframe=red!75!black,fonttitle=\bfseries}

\begin{tcolorbox}[title=My title]
This is a \textbf{tcolorbox}.
\end{tcolorbox}
\end{exdispExample}

\begin{exdispExample}{after_upper_2}
\begin{tcolorbox}[before upper=\flqq,after upper=\frqq,
  colback=red!5!white,colframe=red!75!black]
This is a \textbf{tcolorbox}.
\end{tcolorbox}
\end{exdispExample}
\end{docTcbKey}


\begin{docTcbKey}[][doc new and updated={2016-10-21}{2019-02-28}]{after upper*}{=\meta{code}}{no default, initially unset}
  The given \meta{code} is placed \emph{after} the content of the upper part.
  In contrast to \refKey{/tcb/after upper}, no |\unskip| is prepended.
  Use this for situations where |\unskip| is not needed or causes harm.
  See \refKey{/tcb/before upper*} for an example.

\begin{marker}
  From version 3.80 to 3.94, this option prepended an |\unskip| to the given \meta{code}.\\
  From version 3.95 to 4.15, this option was deprecated.\\
  From version 4.20, this option is re-established with changed semantic (no |\unskip|!)
\end{marker}
\end{docTcbKey}


\clearpage
\begin{docTcbKey}{before lower}{=\meta{code}}{no default, initially empty}
  The given \meta{code} is placed \emph{after} the color and font settings
  and \emph{before} the content of the lower part.
  The \meta{code} is appended by a final |\ignorespaces|.
\begin{exdispExample}{before_lower}
\tcbset{before lower=\textit{Behold:~},colback=red!5!white,colframe=red!75!black}

\begin{tcolorbox}
This is a \textbf{tcolorbox}.
\tcblower
This is the lower part.
\end{tcolorbox}
\end{exdispExample}
\end{docTcbKey}


\begin{docTcbKey}[][doc new=2019-02-26]{before lower*}{=\meta{code}}{no default, initially unset}
  The given \meta{code} is placed \emph{after} the color and font settings
  and \emph{before} the content of the lower part.
  In contrast to \refKey{/tcb/before lower}, no |\ignorespaces| is appended.
  Use this for situations where |\ignorespaces| is not needed or causes harm.
\begin{exdispExample}{before_lower_star}
\begin{tcolorbox}[size=small,bicolor,sidebyside,center lower,
  colback=yellow!30,colbacklower=yellow!20,colframe=yellow!80!black,
  before lower*=\begin{tabular}{cc},
  after lower*=\end{tabular},
]
My table
\tcblower
  \multicolumn{2}{c}{Title}\\
  one & two \\
  three & four\\
\end{tcolorbox}
\end{exdispExample}
\end{docTcbKey}


\clearpage

\begin{docTcbKey}[][doc updated=2016-10-21]{after lower}{=\meta{code}}{no default, initially empty}
  The given \meta{code} is placed \emph{after} the content of the lower part.
  The \meta{code} is prepended by a leading |\unskip|.


\begin{exdispExample}{after_lower_2}
\begin{tcolorbox}[after lower=\ \textit{This is the end.},
  colback=red!5!white,colframe=red!75!black]
This is a \textbf{tcolorbox}.
\tcblower
This is the lower part.
\end{tcolorbox}
\end{exdispExample}
\end{docTcbKey}


\begin{docTcbKey}[][doc new and updated={2016-10-21}{2019-02-28}]{after lower*}{=\meta{code}}{no default, initially unset}
  The given \meta{code} is placed \emph{after} the content of the lower part.
  In contrast to \refKey{/tcb/after upper}, no |\unskip| is prepended.
  Use this for situations where |\unskip| is not needed or causes harm.

\begin{exdispExample}{after_lower_1}
\begin{tcolorbox}[before lower*=$,after lower*=$,
  colback=red!5!white,colframe=red!75!black]
This is a \textbf{tcolorbox}.
\tcblower
\sin^2(x)+\cos^2(x)=1.
\end{tcolorbox}
\end{exdispExample}

\begin{marker}
  From version 3.80 to 3.94, this option prepended an |\unskip| to the given \meta{code}.\\
  From version 3.95 to 4.15, this option was deprecated.\\
  From version 4.20, this option is re-established with changed semantic (no |\unskip|!)
\end{marker}
\end{docTcbKey}


\clearpage
\begin{marker}
If \refKey{/tcb/text fill} is used, one cannot have a lower part
and the box is unbreakable.
\end{marker}

\begin{docTcbKey}[][doc new=2015-07-15]{text fill}{}{style, no value}
  This style sets \refKey{/tcb/before upper} and \refKey{/tcb/after upper}
  to embed the upper part with a minipage. If a fixed height was applied
  e.\,g.\ by \refKey{/tcb/height} or \refKey{/tcb/height fill}, this minipage
  gets a matching height. This allows to use vertical glue macros like
  |\vfill| to act like expected. If the box has no fixed height,
  setting \refKey{/tcb/text fill} has no other effect as making the box
  unbreakable.
\begin{exdispExample}{text_fill}
\begin{tcolorbox}[colback=red!5!white,colframe=red!75!black,fonttitle=\bfseries,
  height=8cm,text fill,
  title=My filled box]
This is a \textbf{tcolorbox}.
\par\vfill
\begin{center}
  My middle text.
\end{center}
\par\vfill
This is the end of my box.
\end{tcolorbox}
\end{exdispExample}
\end{docTcbKey}


\clearpage

\begin{marker}
If \refKey{/tcb/tabulars} or \refKey{/tcb/tabulars*} are used, one cannot
have a lower part.
\end{marker}

\begin{docTcbKey}[][doc new={2019-09-19}]{tabulars}{=\marg{preamble}}{style}
  This style sets \refKey{/tcb/before upper} and \refKey{/tcb/after upper}
  and several geometry keys to support a |tabular*| with the
  given \meta{preamble}.
  The packages \refPkg{array} and \refPkg{colortbl} have to be loaded separately.
\begin{exdispExample}{tabulars_1}
% \usepackage{array}
% \usepackage{colortbl} - or - \usepackage[table]{xcolor}
\tcbset{enhanced,fonttitle=\bfseries\large,fontupper=\normalsize\sffamily,
  colback=yellow!10!white,colframe=red!50!black,colbacktitle=red!15!white,
  coltitle=black,center title}

\begin{tcolorbox}[tabulars={@{\extracolsep{\fill}\hspace{5mm}}lrrrrr@{\hspace{5mm}}},
  boxrule=0.5pt,title=My table]
Group & One     & Two     & Three    & Four     & Sum\\\hline\hline
Red   & 1000.00 & 2000.00 &  3000.00 &  4000.00 & 10000.00\\\hline
Green & 2000.00 & 3000.00 &  4000.00 &  5000.00 & 14000.00\\\hline
Blue  & 3000.00 & 4000.00 &  5000.00 &  6000.00 & 18000.00\\\hline\hline
Sum   & 6000.00 & 9000.00 & 12000.00 & 15000.00 & 42000.00
\end{tcolorbox}
\end{exdispExample}
\end{docTcbKey}


\begin{docTcbKey}[][doc new={2019-09-19}]{tabulars*}{=\marg{code}\marg{preamble}}{style}
  This is a variant of \refKey{/tcb/tabulars} which adds some \meta{code}
  before the table starts.
\begin{exdispExample}{tabulars_2}
% \usepackage{array}
% \usepackage{colortbl} - or - \usepackage[table]{xcolor}
\tcbset{enhanced,fonttitle=\bfseries\large,fontupper=\normalsize\sffamily,
  colback=yellow!10!white,colframe=red!50!black,colbacktitle=red!15!white,
  coltitle=black,center title}

\begin{tcolorbox}[tabulars*={\arrayrulewidth0.5mm\renewcommand\arraystretch{1.4}}%
    {@{\extracolsep{\fill}\hspace{20mm}}lll@{\hspace{20mm}}},
  title=My table]
One     & Two     & Three \\\hline\hline
1000.00 & 2000.00 &  3000.00\\\hline
2000.00 & 3000.00 &  4000.00
\end{tcolorbox}
\end{exdispExample}
\end{docTcbKey}



\clearpage
\begin{marker}
If \refKey{/tcb/tabularx} or \refKey{/tcb/tabularx*} are used, one cannot
have a lower part.
\end{marker}

\begin{docTcbKey}{tabularx}{=\marg{preamble}}{style}
  This style sets \refKey{/tcb/before upper} and \refKey{/tcb/after upper}
  and several geometry keys to support a |tabularx| with the
  given \meta{preamble}.
  The packages \refPkg{tabularx} \cite {carlisle:tabularx}, \refPkg{array}, and \refPkg{colortbl}
  have to be loaded separately.
\begin{exdispExample}{tabularx_1}
% \usepackage{array,tabularx}
% \usepackage{colortbl} - or - \usepackage[table]{xcolor}
\newcolumntype{Y}{>{\raggedleft\arraybackslash}X}% see tabularx
\tcbset{enhanced,fonttitle=\bfseries\large,fontupper=\normalsize\sffamily,
  colback=yellow!10!white,colframe=red!50!black,colbacktitle=red!15!white,
  coltitle=black,center title}

\begin{tcolorbox}[tabularx={X||Y|Y|Y|Y||Y},title=My table]
Group & One     & Two     & Three    & Four     & Sum\\\hline\hline
Red   & 1000.00 & 2000.00 &  3000.00 &  4000.00 & 10000.00\\\hline
Green & 2000.00 & 3000.00 &  4000.00 &  5000.00 & 14000.00\\\hline
Blue  & 3000.00 & 4000.00 &  5000.00 &  6000.00 & 18000.00\\\hline\hline
Sum   & 6000.00 & 9000.00 & 12000.00 & 15000.00 & 42000.00
\end{tcolorbox}
\end{exdispExample}
\end{docTcbKey}


\begin{docTcbKey}{tabularx*}{=\marg{code}\marg{preamble}}{style}
  This is a variant of \refKey{/tcb/tabularx} which adds some \meta{code}
  before the table starts.
\begin{exdispExample}{tabularx_2}
% \usepackage{array,tabularx}
% \usepackage{colortbl} - or - \usepackage[table]{xcolor}
\tcbset{enhanced,fonttitle=\bfseries\large,fontupper=\normalsize\sffamily,
  colback=yellow!10!white,colframe=red!50!black,colbacktitle=red!15!white,
  coltitle=black,center title}

\begin{tcolorbox}[tabularx*={\arrayrulewidth0.5mm}{X|X|X},title=My table]
One     & Two     & Three \\\hline\hline
1000.00 & 2000.00 &  3000.00\\\hline
2000.00 & 3000.00 &  4000.00
\end{tcolorbox}
\end{exdispExample}
\end{docTcbKey}


\clearpage
\begin{marker}
If \refKey{/tcb/tabularray} or \refKey{/tcb/tabularray*} are used, one cannot have a lower part.
Breakable and floating |tabularray| variants (called long and tall) are not
supported inside a (tcolor-)box.
\end{marker}

\begin{docTcbKey}[][doc new={2025-09-09}]{tabularray}{=\marg{inner specs}}{style}
  This style sets \refKey{/tcb/before upper} and \refKey{/tcb/after upper}
  and several geometry keys to support a |tabularray| (actually, the |tblr| environment)
  with the given \meta{inner specs}.
  The package \refPkg{tabularray} \cite{lvjr:tabularray} has to be loaded separately.
\begin{exdispExample}{tabularray_1}
% \usepackage{tabularray}
% Note option 'clip upper' to avoid glitches for the colored last line
\begin{tcolorbox}[enhanced,
  fonttitle=\bfseries\large, fontupper=\normalsize\sffamily,
  colback=yellow9!20, colframe=blue2, colbacktitle=blue5, coltitle=white,
  center title, clip upper,
  title=My table,
  tabularray={
    colspec={X|[0.5mm]X|X|X|X|[0.5mm]X},
    column{2-Z}={halign=r},
    row{1,Z}={font=\bfseries,bg=red9},
    }
  ]
Group & One     & Two     & Three    & Four     & Sum\\\hline[0.5mm]
Red   & 1000.00 & 2000.00 &  3000.00 &  4000.00 & 10000.00\\\hline
Green & 2000.00 & 3000.00 &  4000.00 &  5000.00 & 14000.00\\\hline
Blue  & 3000.00 & 4000.00 &  5000.00 &  6000.00 & 18000.00\\\hline[0.5mm]
Sum   & 6000.00 & 9000.00 & 12000.00 & 15000.00 & 42000.00
\end{tcolorbox}
\end{exdispExample}
\end{docTcbKey}

\begin{docTcbKey}[][doc new={2025-09-09}]{tabularray*}{=\marg{code}\marg{inner specs}}{style}
  This is a variant of \refKey{/tcb/tabularray} which adds some \meta{code}
  before the table starts.
  For example, \meta{code} may contain |\SetTblrInner| or |\SetTblrOuter| commands
  or renew commands like |\lTblrDefaultHruleColorTl|, etc.
\end{docTcbKey}


\clearpage
\begin{docTcbKey}{tikz upper}{\colOpt{=\meta{options}}}{style}
  This style adds a centered |tikzpicture| environment to the start and end
  of the upper part. The \meta{options} may be given as \tikzname\  picture options.
\begin{exdispExample}{tikz_upper}
\begin{tcolorbox}[tikz upper,fonttitle=\bfseries,colback=white,colframe=black,
                  title=\tikzname\ drawing]
  \path[fill=yellow,draw=yellow!75!red] (0,0) circle (1cm);
  \fill[red] (45:5mm) circle (1mm);
  \fill[red] (135:5mm) circle (1mm);
  \draw[line width=1mm,red] (215:5mm) arc (215:325:5mm);
\end{tcolorbox}
\end{exdispExample}
\end{docTcbKey}


\begin{docTcbKey}{tikz lower}{\colOpt{=\meta{options}}}{style}
  This style adds a centered |tikzpicture| environment to the start and end
  of the lower part. The \meta{options} may be given as \tikzname\  picture options.
\begin{exdispExample}{tikz_lower}
% \tcbuselibrary{skins,listings}

\begin{tcblisting}{tikz lower,listing side text,fonttitle=\bfseries,
  bicolor,colback=LightBlue!50!white,colbacklower=white,colframe=black,
  righthand width=3cm,title=\tikzname\ drawing}
\path[fill=yellow,draw=yellow!75!red]
    (0,0) circle (1cm);
\fill[red] (45:5mm) circle (1mm);
\fill[red] (135:5mm) circle (1mm);
\draw[line width=1mm,red]
    (215:5mm) arc (215:325:5mm);
\end{tcblisting}
\end{exdispExample}
\end{docTcbKey}


\clearpage
\begin{docTcbKey}{tikznode upper}{\colOpt{=\meta{options}}}{style}
  This style places the upper part content into a centered
  \tikzname\  node. The \meta{options} may be given as \tikzname\  node options.
  This style is especially useful for boxes with multiline texts which are
  fitted to the text width.
\begin{exdispExample}{tikznode_upper}
% \tcbuselibrary{skins}
\newtcbox{\headline}[1][]{enhanced,center,
  ignore nobreak,fontupper=\Large\bfseries,
  colframe=red!50!black,colback=red!10!white,
  drop fuzzy shadow=yellow,tikznode upper,#1}

\headline{Important\\Headline}
\end{exdispExample}
\end{docTcbKey}

\begin{docTcbKey}{tikznode lower}{\colOpt{=\meta{options}}}{style}
  This style places the lower part content into a centered
  \tikzname\ node. The \meta{options} may be given as \tikzname\  node options.
\begin{exdispExample}{tikznode_lower}
\begin{tcolorbox}[bicolor,colback=LightBlue!50!white,colbacklower=white,
  colframe=black,tikznode lower={inner sep=2pt,draw=red,fill=yellow}]
Upper part.
\tcblower
Lower part.
\end{tcolorbox}
\end{exdispExample}
\end{docTcbKey}

\begin{docTcbKey}{tikznode}{\colOpt{=\meta{options}}}{style}
  Shortcut for setting \refKey{/tcb/tikznode upper} and \refKey{/tcb/tikznode lower}
  the same time.
\end{docTcbKey}


\begin{docTcbKey}{varwidth upper}{\colOpt{=\meta{length}}}{style, default \refKey{/tcb/width}}
  This style places the upper part content into a |varwidth| environment.
  This style needs the \refPkg{varwidth} package \cite{arseneau:2011a} to be loaded manually.
  The resulting box has a maximal width of \meta{length}.
  This option is only senseful for a \refCom{tcbox}.
\begin{exdispExample*}{varwidth_upper}{sbs,lefthand ratio=0.6}
% \usepackage{varwidth}
\newtcbox{\varbox}{colframe=red!50!black,
  colback=red!10!white,varwidth upper}

\varbox{Short text.}
\varbox{This box contains is a longer text
  which is broken.}
\end{exdispExample*}
\end{docTcbKey}

\clearpage

\begin{docTcbKeys}[
    doc new = 2023-10-16,
    doc description = style,
  ]
  {
    { doc name = environment upper, doc parameter={=\meta{name}} },
    { doc name = environment upper args, doc parameter={=\marg{name}\marg{code}} }
  }
  This shortcut style sets \refKey{/tcb/before upper*}
  to |\begin|\marg{name} and \refKey{/tcb/after upper*}
  to |\end|\marg{name}, i.\,e., the upper part is enclosed by the environment \meta{name}.
  The second variant also places \meta{code} after |\begin|\marg{name}.
\begin{exdispExample}{environment_upper_1}
\begin{tcolorbox}[colback=yellow!30, colframe=yellow!80!black,
  environment upper=itemize
]
  \item one
  \item two
\end{tcolorbox}
\end{exdispExample}
  Note that you may need \emph{additional} brackets around \meta{code}, if
  \meta{code} contains some para\-meter(s) for the environment \meta{name}.
\begin{exdispExample}{environment_upper_2}
\begin{tcolorbox}[center upper,
  colback=yellow!30, colframe=yellow!80!black,
  environment upper args={tabular}{{cc}}
]
  one & two \\
  three & four\\
\end{tcolorbox}
\end{exdispExample}
\begin{exdispExample}{environment_upper_3}
\begin{tcolorbox}[colback=yellow!30, colframe=yellow!80!black,
  environment upper args = {list}{{\bfseries Fact:}
    {\setlength\itemsep{0pt}\setlength\parsep{0pt}}},
]
  \item Blue is red
  \item Red is green
\end{tcolorbox}
\end{exdispExample}
\end{docTcbKeys}


\begin{marker}
\refKey{/tcb/environment upper} and friends (following) cannot be used to
wrap another |tcolorbox|-based environment, i.\,e., never use
|/tcb/environment upper=tcolorbox| or
|/tcb/environment upper=etc-derived-from-tcolorbox|.
\end{marker}


\clearpage

\begin{docTcbKeys}[
    doc new = 2023-10-16,
    doc description = style,
  ]
  {
    { doc name = environment lower, doc parameter={=\meta{name}} },
    { doc name = environment lower args, doc parameter={=\marg{name}\marg{code}} }
  }
  This shortcut style sets \refKey{/tcb/before lower*}
  to |\begin|\marg{name} and \refKey{/tcb/after lower*}
  to |\end|\marg{name}, i.\,e., the lower part is enclosed by the environment \meta{name}.
  The second variant also places \meta{code} after |\begin|\marg{name}.
  Note that you may need \emph{additional} brackets around \meta{code}, if
  \meta{code} contains some para\-meter(s) for the environment \meta{name}.

  See \refKey{/tcb/environment upper} for analog examples.
\end{docTcbKeys}


\begin{docTcbKeys}[
    doc new = 2023-10-16,
    doc description = style,
  ]
  {
    { doc name = environment title, doc parameter={=\meta{name}} },
    { doc name = environment title args, doc parameter={=\marg{name}\marg{code}} }
  }
  This shortcut style sets \refKey{/tcb/before title*}
  to |\begin|\marg{name} and \refKey{/tcb/after title*}
  to |\end|\marg{name}, i.\,e., the title text is enclosed by the environment \meta{name}.
  The second variant also places \meta{code} after |\begin|\marg{name}.
  Note that you may need \emph{additional} brackets around \meta{code}, if
  \meta{code} contains some para\-meter(s) for the environment \meta{name}.

  See \refKey{/tcb/environment upper} for analog examples.
\end{docTcbKeys}



\clearpage
\subsection{Overlays}\label{subsec:overlays}
With an overlay, arbitrary \meta{graphical code} can be added to a
|tcolorbox|. This code is executed \emph{after} the frame and interior are
drawn and \emph{before} the text content is drawn. Therefore, you can
decorate the |tcolorbox| with your own extensions.
Common special cases are \emph{watermarks} which are implemented using overlays.
See Subsection \ref{subsec:watermarks} from page \pageref{subsec:watermarks} if
you want to add \emph{watermarks}.
\begin{marker}
If you use the core package only, the \meta{graphical code} has to be
\refPkg{pgf} code and there is not much assistance for positioning.
Therefore, the usage of the \refKey{/tcb/enhanced} mode from the library \mylib{skins}
is recommended which allows \tikzname\ code and gives access to
\refKey{/tcb/geometry nodes} for positioning.
\end{marker}

\begin{docTcbKey}{overlay}{=\meta{graphical code}}{no default, initially unset}
  Adds \meta{graphical code} to the box drawing process. This \meta{graphical code}
  is drawn \emph{after} the frame and interior and \emph{before} the text content.

\begin{exdispExample}{overlay_1}
% \tcbuselibrary{skins} % preamble
\tcbset{frogbox/.style={enhanced,colback=green!10,colframe=green!65!black,
  enlarge top by=5.5mm,
  overlay={\foreach \x in {2cm,3.5cm} {
    \begin{scope}[shift={([xshift=\x]frame.north west)}]
      \path[draw=green!65!black,fill=green!10,line width=1mm] (0,0) arc (0:180:5mm);
      \path[fill=black] (-0.2,0) arc (0:180:1mm);
    \end{scope}}}}}

\begin{tcolorbox}[frogbox,title=My title]
This is a \textbf{tcolorbox}.
\end{tcolorbox}
\end{exdispExample}

\enlargethispage*{5mm}
\begin{exdispExample}{overlay_2}
% \usetikzlibrary{patterns} % preamble
% \tcbuselibrary{skins}     % preamble
\tcbset{ribbonbox/.style={enhanced,colback=red!5!white,colframe=red!75!black,
  fonttitle=\bfseries,
  overlay={\path[fill=blue!75!white,draw=blue,double=white!85!blue,
    preaction={opacity=0.6,fill=blue!75!white},
    line width=0.1mm,double distance=0.2mm,
    pattern=fivepointed stars,pattern color=white!75!blue]
    ([xshift=-0.2mm,yshift=-1.02cm]frame.north east)
    -- ++(-1,1) -- ++(-0.5,0) -- ++(1.5,-1.5) -- cycle;}}}

\begin{tcolorbox}[ribbonbox,title=My title]
This is a \textbf{tcolorbox}.
\tcblower
This is the lower part.
\end{tcolorbox}
\end{exdispExample}
\end{docTcbKey}

\clearpage
\begin{docTcbKey}{no overlay}{}{style, no default, initially set}
  Removes the overlay if set before.
\end{docTcbKey}

\begin{docTcbKey}{overlay broken}{=\meta{graphical code}}{no default, initially unset}
  If the box is set to be \refKey{/tcb/breakable} and \emph{is} broken actually,
  then the \meta{graphical code} is added to the box drawing process.
  \refKey{/tcb/overlay} overwrites this key.
\end{docTcbKey}

\begin{docTcbKey}{overlay unbroken}{=\meta{graphical code}}{no default, initially unset}
  If the box is set to be \refKey{/tcb/breakable} but \emph{is not} broken actually
  or if the box is set to be \refKey{/tcb/unbreakable},
  then the \meta{graphical code} is added to the box drawing process.
  \refKey{/tcb/overlay} overwrites this key.
\end{docTcbKey}

\begin{docTcbKey}{overlay first}{=\meta{graphical code}}{no default, initially unset}
  If the box is set to be \refKey{/tcb/breakable} and \emph{is} broken actually,
  then the \meta{graphical code} is added to the box drawing process for
  the \emph{first} part of the break sequence.
  \refKey{/tcb/overlay} overwrites this key.
\end{docTcbKey}

\begin{docTcbKey}{overlay middle}{=\meta{graphical code}}{no default, initially unset}
  If the box is set to be \refKey{/tcb/breakable} and \emph{is} broken actually,
  then the \meta{graphical code} is added to the box drawing process for
  the \emph{middle} parts (if any) of the break sequence.
  \refKey{/tcb/overlay} overwrites this key.
\end{docTcbKey}

\begin{docTcbKey}{overlay last}{=\meta{graphical code}}{no default, initially unset}
  If the box is set to be \refKey{/tcb/breakable} and \emph{is} broken actually,
  then the \meta{graphical code} is added to the box drawing process for
  the \emph{last} part of the break sequence.
  \refKey{/tcb/overlay} overwrites this key.
\end{docTcbKey}

\begin{docTcbKey}{overlay unbroken and first}{=\meta{graphical code}}{no default, initially unset}
  This is an optimized abbreviation for setting
  \refKey{/tcb/overlay unbroken} and
  \refKey{/tcb/overlay first} together.
  \refKey{/tcb/overlay} overwrites this key.
\end{docTcbKey}

\begin{docTcbKey}{overlay middle and last}{=\meta{graphical code}}{no default, initially unset}
  This is an optimized abbreviation for setting
  \refKey{/tcb/overlay middle} and
  \refKey{/tcb/overlay last} together.
  \refKey{/tcb/overlay} overwrites this key.
\end{docTcbKey}

\begin{docTcbKey}{overlay unbroken and last}{=\meta{graphical code}}{no default, initially unset}
  This is an optimized abbreviation for setting
  \refKey{/tcb/overlay unbroken} and
  \refKey{/tcb/overlay last} together.
  \refKey{/tcb/overlay} overwrites this key.
\end{docTcbKey}

\begin{docTcbKey}[][doc new=2014-09-19]{overlay first and middle}{=\meta{graphical code}}{no default, initially unset}
  This is an optimized abbreviation for setting
  \refKey{/tcb/overlay first} and
  \refKey{/tcb/overlay middle} together.
  \refKey{/tcb/overlay} overwrites this key.
\end{docTcbKey}


\begin{dispListing*}{breakable,vfill before first,before upper={This example demonstrates
the application of break sequence specific overlay options.
Here, we define an environment |myexample| based
on |tcolorbox| where the visible drawing is done totally by overlay keys.\par
Here, the first application of |myexample| produces an unbroken |tcolorbox|.
The frame is drawn by the code given with \refKey{/tcb/overlay unbroken}.\par
The second application of |myexample| is broken into several parts which
are drawn by the codes given with
\refKey{/tcb/overlay first}, \refKey{/tcb/overlay middle}, and
\refKey{/tcb/overlay last}.
\par\bigskip
}}
% Preamble:
%\usepackage{lipsum}
%\tcbuselibrary{skins,breakable}
%\newcounter{example}
\colorlet{colexam}{red!75!black}
\newtcolorbox[use counter=example]{myexample}{%
  empty,title={Example \thetcbcounter},attach boxed title to top left,
  boxed title style={empty,size=minimal,toprule=2pt,top=4pt,
    overlay={\draw[colexam,line width=2pt]
      ([yshift=-1pt]frame.north west)--([yshift=-1pt]frame.north east);}},
  coltitle=colexam,fonttitle=\Large\bfseries,
  before=\par\medskip\noindent,parbox=false,boxsep=0pt,left=0pt,right=3mm,top=4pt,
  breakable,pad at break*=0mm,vfill before first,
  overlay unbroken={\draw[colexam,line width=1pt]
    ([yshift=-1pt]title.north east)--([xshift=-0.5pt,yshift=-1pt]title.north-|frame.east)
    --([xshift=-0.5pt]frame.south east)--(frame.south west); },
  overlay first={\draw[colexam,line width=1pt]
    ([yshift=-1pt]title.north east)--([xshift=-0.5pt,yshift=-1pt]title.north-|frame.east)
    --([xshift=-0.5pt]frame.south east); },
  overlay middle={\draw[colexam,line width=1pt] ([xshift=-0.5pt]frame.north east)
    --([xshift=-0.5pt]frame.south east); },
  overlay last={\draw[colexam,line width=1pt] ([xshift=-0.5pt]frame.north east)
    --([xshift=-0.5pt]frame.south east)--(frame.south west);},%
}

\begin{myexample}
\lipsum[1]
\end{myexample}

\begin{myexample}
\lipsum[2-11]
\end{myexample}

\lipsum[12]% following text
\end{dispListing*}
{\tcbusetemp}


%\begin{dispExample}
%% \tcbuselibrary{skins}
%% \newcounter{example}
%\newtcolorbox[use counter=example]{FancyTitle}[3][]{%
%  enhanced,colback=blue!10!white,colframe=orange,top=4mm,
%  enlarge top by=\baselineskip/2+1mm,
%  enlarge top at break by=0mm,pad at break=2mm,
%  fontupper=\normalsize,label={#3},
%  overlay unbroken and first={%
%    \node[rectangle,rounded corners,draw=black,fill=blue!20!white,
%      inner sep=1mm,anchor=west,font=\small]
%      at ([xshift=4.5mm]frame.north west)
%         {\strut\textbf{Example \thetcbcounter: #2}};},
%  #1}%

%\begin{FancyTitle}{My fancy title}{fancy:title}
%  \lipsum[1]
%\end{FancyTitle}
%\end{dispExample}

\clearpage
\subsection{Floating Objects}
\begin{docTcbKey}{floatplacement}{=\meta{values}}{no default, initially \texttt{htb}}
  Sets \meta{values} as default values for the usage of \refKey{/tcb/float}
  and \refKey{/tcb/float*}.
  Feasible are the usual parameters for floating objects.
\begin{dispListing}
\tcbset{enhanced,colback=red!5!white,colframe=red!75!black,
   watermark color=red!15!white}

\begin{tcolorbox}[floatplacement=t,float,
                  title=Floating box from |floatplacement|,
                  watermark text={I am floating}]
  This floating box is placed at the top of a page.
\end{tcolorbox}
\end{dispListing}
\end{docTcbKey}
{\tcbusetemp}


\begin{docTcbKey}{float}{\colOpt{=\meta{values}}}{default from \texttt{floatplacement}}
  Turns the box to a floating object where \meta{values} are the
  usual parameters for such floating objects.
  If they are not used, the placement uses the default values given by
 |floatplacement|.
\begin{dispListing}
\begin{tcolorbox}[float, title=Floating box from |float|,
    enhanced,watermark text={I'm also floating}]
  This box floats to a feasible place automatically. You do not have to
  use a numbering for this floating object.
\end{tcolorbox}
\end{dispListing}
\end{docTcbKey}
{\tcbusetemp}


\begin{docTcbKey}{float*}{\colOpt{=\meta{values}}}{default from \texttt{floatplacement}}
  Identical to \refKey{/tcb/float}, but for wide boxes spanning the whole page
  width of two column documents or in conjunction with the packages
  \refPkg{multicol} or \refPkg{paracol}. Note that you have to set
  |width=\textwidth| additionally, if the box should span the whole page width
  in these cases!
\begin{dispListing}
\begin{tcolorbox}[float*=b, title=Floating box from |float*|,width=\textwidth,
    enhanced,watermark text={I'm also floating}]
  In this single column document, you will see no difference to |float|.
\end{tcolorbox}
\end{dispListing}
\end{docTcbKey}
{\tcbusetemp}


\begin{docTcbKey}{nofloat}{}{style, initially set}
  Turns the floating behavior off.
\end{docTcbKey}


\begin{docTcbKey}[][doc new=2014-09-19]{every float}{=\marg{code}}{no default, initially empty}
  For floating objects, the \refKey{/tcb/before} and \refKey{/tcb/after}
  settings are ignored. Instead, \refKey{/tcb/before float} and \refKey{/tcb/after float}
  can be used. Further, with \refKey{/tcb/every float}, the given \meta{code} is inserted before
  a floating box. If the box is \refKey{/tcb/breakable}, the given \meta{code} is
  inserted before every part of the break sequence.
  The most common use case is |every float=\centering|.

\begin{dispListing}
\tcbox[float=htb,title={Floating box},every float=\centering,
  colback=blue!50!black,colframe=blue!50!white,colbacktitle=blue!10!white,
  coltitle=black,center title]
  {\includegraphics[height=6cm]{lichtspiel.jpg}}
\end{dispListing}
{\tcbset{reset}\tcbusetemp}

\end{docTcbKey}


\begin{docTcbKeys}[
    doc parameter   = {=\marg{code}},
    doc description = {no default, initially empty},
    doc new         = 2022-06-21,
  ]
  {
    {
      doc name        = before float,
    },
    {
      doc name        = after float,
    }
  }
  For floating objects, the \refKey{/tcb/before} and \refKey{/tcb/after}
  settings are ignored. Code can be inserted after the begin of the float
  environment and before \refKey{/tcb/every float} with
  \refKey{/tcb/before float} and between the end of the box and the end
  of the float environment with \refKey{/tcb/after float}.
  \begin{marker}
  These options are not compatible with \refKey{/tcb/breakable} floating objects
  and are ignored, if the box is set to be breakable.
  \end{marker}
\end{docTcbKeys}




\clearpage
\subsection{Embedding into the Surroundings}\label{subsec:surroundings}
Typically, but not necessarily, a |tcolorbox| is put inside a separate paragraph
and has some vertical space before and after it.
This behavior is controlled by the keys \refKey{/tcb/before} and \refKey{/tcb/after}.

\begin{marker}
Before version 4.40, the default setting for \refKey{/tcb/before}
and \refKey{/tcb/after} was given by \refKey{/tcb/autoparskip}.
Starting with version 4.40, the default setting is given by
\refKey{/tcb/before skip balanced} and \refKey{/tcb/after skip balanced}.\par
Note that old documents may need adaptions of page breaks.\par
Alternatively, the old default setting can be restored by using
\begin{dispListing}
\tcbsetforeverylayer{autoparskip}
\end{dispListing}
inside the document preamble.
\end{marker}


\begin{docTcbKey}{before}{=\meta{code}}{no default, initially see \refKey{/tcb/before skip balanced}}
  Sets the \meta{code} which is executed before the colored box.
  It is not used for floating boxes.
  Also, it is not used, if the box follows a heading immediately
  and \refKey{/tcb/ignore nobreak} is set to \docValue{false}.
\end{docTcbKey}

\begin{docTcbKey}{after}{=\meta{code}}{no default, initially see \refKey{/tcb/after skip balanced}}
  Sets the \meta{code} which is executed after the colored box.
  It is not used for floating boxes.
\end{docTcbKey}


\begin{docTcbKey}{nobeforeafter}{}{style, no value}
  Abbreviation for clearing the keys |before| and |after|. The colored box
  is not put into a paragraph and there is no space before or after the box.
\begin{exdispExample}{nobeforeafter}
\tcbset{myone/.style={colback=LightGreen,colframe=DarkGreen,
  equal height group=nobefaf,width=\linewidth/4,nobeforeafter}}
\begin{tcolorbox}[myone,title=Box 1]Box 1\end{tcolorbox}%
\begin{tcolorbox}[myone,title=Box 2]Box 2\end{tcolorbox}%
\begin{tcolorbox}[myone,title=Box 3]Box 3\end{tcolorbox}%
\begin{tcolorbox}[myone,title=Box 4]Box 4\end{tcolorbox}
\end{exdispExample}
\end{docTcbKey}

\begin{docTcbKey}{force nobeforeafter}{}{style, no value}
  Forces the setting of \refKey{/tcb/nobeforeafter} even if
  \refKey{/tcb/before} and \refKey{/tcb/after} are set to other values
  later. Do not use this option globally unless you \emph{really} know what you do.
  Note that embedded boxes do not inherit this forced clearance.
\end{docTcbKey}


\clearpage

\begin{docTcbKey}[][doc new={2020-09-25}]{before skip balanced}{=\meta{glue}}{no default, initially |0.5\textbackslash baselineskip plus 2pt|}
  Inserts some vertical space before the colored box. This style sets \refKey{/tcb/before}.\par
  If the depth of the
  preceding \TeX\ box is between |0pt| and |0.3\baselineskip|,
  the distance between the \emph{baseline} of the preceding \TeX\ box and the tcolorbox
  is set to \meta{glue}$+$|0.3\baselineskip|.\par
  If the depth is larger, the distance of the preceding \TeX\ box and the tcolorbox
  is set to \meta{glue}.\par
  Alternatively, see \refKey{/tcb/before skip} which ignores the \emph{baseline}.

\begin{exdispExample*}{before_skip_balanced}{sbs,lefthand ratio=0.6}
Some text.
\begin{tcolorbox}[before skip balanced=1cm,
    colframe=red!50!white]
  This is a \textbf{tcolorbox}.
\end{tcolorbox}
\end{exdispExample*}
\end{docTcbKey}

\begin{docTcbKey}[][doc new={2020-09-25}]{after skip balanced}{=\meta{glue}}{no default, initially |0.5\textbackslash baselineskip plus 2pt|}
  Inserts some vertical space of the given \meta{glue} after the colored box.
  This style sets \refKey{/tcb/after}.
  Additionally, |\prevdepth| is set to |0.3\baselineskip|. The following
  \TeX\ box may enlarge the space by further glue to adjust its \emph{baseline}.
  Alternatively, see \refKey{/tcb/after skip} which ignores the \emph{baseline}.

\begin{exdispExample*}{after_skip_balanced}{sbs,lefthand ratio=0.6}
\begin{tcolorbox}[after skip balanced=1cm,
    colframe=red!50!white]
  This is a \textbf{tcolorbox}.
\end{tcolorbox}
Some text.
\end{exdispExample*}
\end{docTcbKey}


\begin{docTcbKey}[][doc new={2020-09-25}]{beforeafter skip balanced}{=\meta{glue}}{no default, initially |0.5\textbackslash baselineskip plus 2pt|}
  Inserts some vertical space of the given \meta{glue} before \emph{and} after the colored box.
  This style sets \refKey{/tcb/before skip balanced} and \refKey{/tcb/after skip balanced}.
\begin{exdispExample*}{beforeafter_skip_balanced}{sbs,lefthand ratio=0.6}
\newtcolorbox{doubleline}[1][]{
  beforeafter skip balanced=0pt,
  height=1.8\baselineskip,
  enlarge top by=.1\baselineskip,
  enlarge bottom by=.1\baselineskip,
  colframe=blue!20,colback=blue!5,
  size=small,valign upper=center,#1 }

\noindent\begin{tikzpicture}
\path[use as bounding box] (0,0)
  rectangle (0.1,0.1);
\foreach \y in {0,1,...,9}  {
  \draw[very thin,red]
    (-0.2,-\y*\baselineskip) --
    (\linewidth+0.2cm,-\y*\baselineskip); }
\end{tikzpicture}
line 1\par
\begin{doubleline}  Abc  \end{doubleline}
\begin{doubleline}  Def  \end{doubleline}
line 2g\par
\begin{doubleline}  Ghi  \end{doubleline}
line 3\par
line 4 g
\end{exdispExample*}
\end{docTcbKey}


\clearpage

\begin{docTcbKey}[][doc new and updated={2020-09-25}{2015-03-16}]{before skip}{=\meta{glue}}{style, no default}
  Inserts some vertical space of the given \meta{glue} before the colored box.
  This style sets \refKey{/tcb/before}.
  In contrast to \refKey{/tcb/before skip balanced}, this \meta{glue} is
  relative to the lower edge of the preceding box and not to the baseline.
\begin{exdispExample*}{before_skip}{sbs,lefthand ratio=0.6}
Some text.
\begin{tcolorbox}[before skip=1cm,
    colframe=red!50!white]
  This is a \textbf{tcolorbox}.
\end{tcolorbox}
\end{exdispExample*}
\end{docTcbKey}

\begin{docTcbKey}[][doc new and updated={2020-09-25}{2017-02-01}]{after skip}{=\meta{glue}}{style, no default}
  Inserts some vertical space of the given \meta{glue} after the colored box.
  This style sets \refKey{/tcb/after}.
  In contrast to \refKey{/tcb/after skip balanced}, this \meta{glue} is
  relative to the upper edge of the following box and not to the baseline.
\begin{exdispExample*}{after_skip}{sbs,lefthand ratio=0.6}
\begin{tcolorbox}[after skip=1cm,
    colframe=red!50!white]
  This is a \textbf{tcolorbox}.
\end{tcolorbox}
Some text.
\end{exdispExample*}
\end{docTcbKey}

\begin{docTcbKey}[][doc new=2014-10-10]{beforeafter skip}{=\meta{glue}}{style, no default}
  Inserts some vertical space of the given \meta{glue} before \emph{and} after the colored box.
  This style sets \refKey{/tcb/before skip} and \refKey{/tcb/after skip}.
\begin{exdispExample*}{beforeafter_skip}{sbs,lefthand ratio=0.6}
\tcbset{beforeafter skip=0pt,
  colframe=red!50!white}

text before
\begin{tcolorbox}
  This is a \textbf{tcolorbox}.
\end{tcolorbox}
\begin{tcolorbox}
  Second box.
\end{tcolorbox}
text after
\end{exdispExample*}
\end{docTcbKey}


\clearpage

\begin{docTcbKey}[][doc new=2014-11-07]{left skip}{=\meta{length}}{style, no default, initially |0mm|}
  Inserts some horizontal space of the given \meta{length} before the colored box.
  This style sets \refKey{/tcb/grow to left by} with the negated \meta{length},
  i.\,e., the bounding box and box width are changed.
\begin{exdispExample*}{left_skip}{sbs,lefthand ratio=0.6}
\noindent\rule{\linewidth}{2pt}

\begin{tcolorbox}[left skip=1cm,
    colframe=red!50!white]
  This is a \textbf{tcolorbox}.
\end{tcolorbox}
\end{exdispExample*}
\end{docTcbKey}

\begin{docTcbKey}[][doc new=2014-11-07]{right skip}{=\meta{length}}{style, no default, initially |0mm|}
  Inserts some horizontal space of the given \meta{length} after the colored box.
  This style sets \refKey{/tcb/grow to right by} with the negated \meta{length},
  i.\,e., the bounding box and box width are changed.
\begin{exdispExample*}{right_skip}{sbs,lefthand ratio=0.6}
\noindent\rule{\linewidth}{2pt}

\begin{tcolorbox}[right skip=1cm,
    colframe=red!50!white]
  This is a \textbf{tcolorbox}.
\end{tcolorbox}
\end{exdispExample*}
\end{docTcbKey}

\begin{docTcbKey}[][doc new=2014-10-10]{leftright skip}{=\meta{length}}{style, no default}
  Inserts some horizontal space of the given \meta{length} before \emph{and} after the colored box.
  This style changes the bounding box and the box width.
\begin{exdispExample*}{leftright_skip}{sbs,lefthand ratio=0.6}
\noindent\rule{\linewidth}{2pt}

\begin{tcolorbox}[leftright skip=1cm,
    colframe=red!50!white]
  This is a \textbf{tcolorbox}.
\end{tcolorbox}
\end{exdispExample*}
\end{docTcbKey}


\clearpage

\begin{docTcbKey}[][doc updated=2017-02-01]{parskip}{}{style, no value}
  This options is considered to be superseded by
  \refKey{/tcb/before skip balanced} and \refKey{/tcb/after skip balanced}
  (see note on page~\pageref{subsec:surroundings}).\par
  Sets the keys |before| and |after| to values which are
  recommended, if the package \refPkg{parskip} \emph{is} used and there is no better
  idea for |before| and |after|. This is similar to:
\begin{dispListing}
\tcbset{parskip/.style={before={\par\pagebreak[0]\parindent=0pt},
                        after={\par}}}
\end{dispListing}
\end{docTcbKey}

\begin{docTcbKey}[][doc updated=2017-02-01]{noparskip}{}{style, no value}
  This options is considered to be superseded by
  \refKey{/tcb/before skip balanced} and \refKey{/tcb/after skip balanced}
  (see note on page~\pageref{subsec:surroundings}).\par
  Sets the keys |before| and |after| to values which are
  recommended, if the package \refPkg{parskip} is \emph{not} used and there is no better
  idea for |before| and |after|. This is similar to:
\begin{dispListing}
\tcbset{noparskip/.style={before={\par\pagebreak[0]\smallskip\parindent=0pt},
                          after={\par\smallskip}}}
\end{dispListing}
\end{docTcbKey}

\begin{docTcbKey}{autoparskip}{}{style, no value}
  This options is considered to be superseded by
  \refKey{/tcb/before skip balanced} and \refKey{/tcb/after skip balanced}
  (see note on page~\pageref{subsec:surroundings}).\par
  Tries to detect the usage of the package \refPkg{parskip} and sets
  the keys |before| and |after| accordingly. Actually, the following is done:
  \begin{itemize}
  \item If the length of |\parskip| is greater than |0pt| at the beginning of the document,
    \refKey{/tcb/parskip} is executed. Here, the usage of package \refPkg{parskip} is \emph{assumed}.
  \item Otherwise, if the length of |\parskip| is not greater than |0pt| at the beginning of the document,
    \refKey{/tcb/noparskip} is executed. Here, the absence of package \refPkg{parskip} is \emph{assumed}.
  \end{itemize}
\end{docTcbKey}


\clearpage

\begin{docTcbKey}{baseline}{=\meta{length}}{no default, initially |0pt|}
  Used to set the |\pgfsetbaseline| value of the resulting |tcolorbox|.
\begin{exdispExample}{baseline}
\tcbset{colframe=red!50!white,width=4cm,nobeforeafter}
Some text\dotfill
\begin{tcolorbox}[baseline=3mm]
One line.
\end{tcolorbox}
\begin{tcolorbox}[baseline=3mm]
First line.\\Second line.
\end{tcolorbox}
\end{exdispExample}
\end{docTcbKey}


\begin{docTcbKey}[][doc new=2014-10-10]{box align}{=\meta{alignment}}{style, no default, initially |bottom|}
  Used to set the \refKey{/tcb/baseline} value of the resulting |tcolorbox|.
  Feasible values for \meta{alignment} are:
  \begin{itemize}
  \item\docValue{bottom}: alignment with the box bottom,
  \item\docValue{top}: alignment with the box top,
  \item\docValue{center}: alignment with the box center,
  \item\docValue{base}: alignment with the box content base. This option
    is not applicable for a \refEnv{tcolorbox} but for a \refCom{tcbox} only.
    It is an alias for \refKey{/tcb/tcbox raise base}.
  \end{itemize}

\begin{exdispExample}{box_align_1}
\tcbset{colframe=red!50!white,width=4cm,nobeforeafter}
Some text\dotfill
\begin{tcolorbox}[box align=bottom]
One line.
\end{tcolorbox}
\begin{tcolorbox}[box align=bottom]
First line.\\Second line.
\end{tcolorbox}
\end{exdispExample}

\begin{exdispExample}{box_align_2}
\tcbset{colframe=red!50!white,width=4cm,nobeforeafter}
Some text\dotfill
\begin{tcolorbox}[box align=top]
One line.
\end{tcolorbox}
\begin{tcolorbox}[box align=top]
First line.\\Second line.
\end{tcolorbox}
\end{exdispExample}

\begin{exdispExample}{box_align_3}
\tcbset{colframe=red!50!white,width=4cm,nobeforeafter}
Some text\dotfill
\begin{tcolorbox}[box align=center]
One line.
\end{tcolorbox}
\begin{tcolorbox}[box align=center]
First line.\\Second line.
\end{tcolorbox}
\end{exdispExample}

\begin{exdispExample}{box_align_4}
\tcbset{colframe=red!50!white,nobeforeafter}
Some text\dotfill
\tcbox[nobeforeafter,box align=base]{One line}
\tcbox[nobeforeafter,box align=base,size=fbox]{Another line}
\end{exdispExample}
\end{docTcbKey}




\begin{docTcbKey}[][doc new=2014-12-11]{ignore nobreak}{\colOpt{=true\textbar false}}{default |true|, initially |false|}
After a heading, \LaTeX\ tries to avoid a break by setting a |nobreak| boolean value.
Starting from version |3.33|, the \refKey{/tcb/before} respectively \refKey{/tcb/before skip}
settings are not used after a heading if \refKey{/tcb/ignore nobreak} is
set to \docValue{false}. For an unbreakable box, \refKey{/tcb/before nobreak} is used instead.
Further, a \refKey{/tcb/breakable} box will also try to
avoid a break between a heading and a directly following first part of a
break sequence.

Set \refKey{/tcb/ignore nobreak} to \docValue{true}, if |nobreak| should be
ignored as prior to version |3.33|. Also, such a setting may be used locally to
enforce the \refKey{/tcb/before} setting.
\end{docTcbKey}

\begin{docTcbKey}[][doc new=2014-12-16]{before nobreak}{=\meta{code}}{no default, initially \cs{noindent}}
  Sets the \meta{code} which is executed before the colored box if it
  is unbreakable, if \refKey{/tcb/ignore nobreak} is not set, and if
  the box follows a heading.
\end{docTcbKey}


\begin{docTcbKey}[][doc new=2017-02-23]{parfillskip restore}{\colOpt{=true\textbar false}}{default |true|, initially |true|}
  If this option is set to be |true|, the minimum value of |\parfillskip| is
  tested at specific spots, if it is greater than |0pt|.
  If so, |\parfillskip| is restored to |\@flushglue| which happens to be
  the default value.

  These tests are executed for
  \refKey{/tcb/parskip},
  \refKey{/tcb/noparskip},
  \refKey{/tcb/after skip},
  \refKey{/tcb/breakable}, and
  \refEnv{tcbraster}.

  This option was created to automatically
  avoid overfull box warnings with |\parfillskip| changing packages.
\end{docTcbKey}



\clearpage
\subsection{Bounding Box}
Normally, every |tcolorbox| has a bounding box which fits exactly to the
dimensions of the outer frame. Therefore, \LaTeX\ reserves exactly the space
needed for the box.
This behavior can be changed by enlarging (or shrinking) the bounding box.
If the bounding box is enlarged, the |tcolorbox| will get some clearance
around it. If the bounding box is shrunk, i.\,e., enlarged with negative
values, the |tcolorbox| will overlap to other parts of the page.
For example, the |tcolorbox| could be stretched into the page margin.

\begin{marker}
The following examples use \refKey{/tcb/show bounding box} to display the
actual bounding box. For this, the library \mylib{skins} has to be included and
\refKey{/tcb/enhanced} has to be set.
\end{marker}

\subsubsection{Shifting Bounding Box Borders}

\begin{docTcbKey}{enlarge top initially by}{=\meta{length}}{no default, initially |0mm|}
  Enlarges the bounding box distance to the top of the box by \meta{length}.
  If the box is \emph{breakable}, only the first box of the break sequence
  gets enlarged. \refKey{/tcb/enlarge top by} overwrites this key.
\begin{exdispExample}{enlarge_top_initially_by}
\tcbset{colframe=blue!75!black,colback=white}

\begin{tcolorbox}[enlarge top initially by=-5mm]
This is a \textbf{tcolorbox}.
\end{tcolorbox}
\begin{tcolorbox}[enlarge top initially by=5mm,enhanced,show bounding box]
This is a \textbf{tcolorbox}.
\end{tcolorbox}
\end{exdispExample}
\end{docTcbKey}



\begin{docTcbKey}{enlarge bottom finally by}{=\meta{length}}{no default, initially |0mm|}
  Enlarges the bounding box distance to the bottom of the box by \meta{length}.
  If the box is \emph{breakable}, only the last box of the break sequence
  gets enlarged. \refKey{/tcb/enlarge bottom by} overwrites this key.
\begin{exdispExample}{enlarge_bottom_finally_by}
\tcbset{colframe=blue!75!black,colback=white}

\begin{tcolorbox}[enlarge bottom finally by=5mm]
This is a \textbf{tcolorbox}.
\end{tcolorbox}
\begin{tcolorbox}[enlarge bottom finally by=-5mm,enhanced,show bounding box]
This is a \textbf{tcolorbox}.
\end{tcolorbox}
\end{exdispExample}
\end{docTcbKey}

\clearpage


\begin{docTcbKey}{enlarge top at break by}{=\meta{length}}{no default, initially \texttt{0mm}}
  Enlarges the bounding box distance to the top of the box by \meta{length},
  \emph{if} the box is \refKey{/tcb/breakable}.
  In this case, it is applied to \emph{middle} and \emph{last} parts in a
  break sequence.
  \refKey{/tcb/enlarge top by} overwrites this key.
\end{docTcbKey}


\begin{docTcbKey}{enlarge bottom at break by}{=\meta{length}}{no default, initially \texttt{0mm}}
  Enlarges the bounding box distance to the bottom of the box by \meta{length},
  \emph{if} the box is \refKey{/tcb/breakable}.
  In this case, it is applied to \emph{first} and \emph{middle} parts in a
  break sequence. \refKey{/tcb/enlarge bottom by} overwrites this key.
\end{docTcbKey}


\begin{docTcbKey}{enlarge top by}{=\meta{length}}{no default, initially |0mm|}
  Enlarges the bounding box distance to the top of the box by \meta{length}.
  \refKey{/tcb/enlarge top initially by} and
  \refKey{/tcb/enlarge top at break by} are set to \meta{length}.
\end{docTcbKey}


\begin{docTcbKey}{enlarge bottom by}{=\meta{length}}{no default, initially |0mm|}
  Enlarges the bounding box distance to the bottom of the box by \meta{length}.
  \refKey{/tcb/enlarge bottom finally by} and
  \refKey{/tcb/enlarge bottom at break by} are set to \meta{length}.
\end{docTcbKey}


\begin{docTcbKey}{enlarge left by}{=\meta{length}}{no default, initially |0mm|}
  Enlarges the bounding box distance to the left side of the box by \meta{length}.
\begin{exdispExample}[safety=2cm]{enlarge_left_by}
\tcbset{colframe=blue!75!black,colback=white}

\begin{tcolorbox}[enlarge left by=2cm,width=5cm,enhanced,show bounding box]
This is a \textbf{tcolorbox}.
\end{tcolorbox}
\begin{tcolorbox}[enlarge left by=-2cm,width=\linewidth+2cm]
This is a \textbf{tcolorbox}.
\end{tcolorbox}
\end{exdispExample}
\end{docTcbKey}

\begin{docTcbKey}{enlarge right by}{=\meta{length}}{no default, initially |0mm|}
  Enlarges the bounding box distance to the right side of the box by \meta{length}.
\begin{exdispExample}[safety=2cm]{enlarge_right_by}
\tcbset{colframe=blue!75!black,colback=white}

\begin{tcolorbox}[enlarge right by=-2cm,width=\linewidth+2cm,
  enhanced,show bounding box]
This is a \textbf{tcolorbox}.
\end{tcolorbox}
\begin{tcolorbox}[enlarge right by=2cm,width=\linewidth-2cm]
This is a \textbf{tcolorbox}.
\end{tcolorbox}
\end{exdispExample}
\end{docTcbKey}

\clearpage
\begin{docTcbKey}{enlarge by}{=\meta{length}}{no default, initially |0mm|}
  Enlarges the bounding box distance to all sides of the box by \meta{length}.
\begin{exdispExample}{enlarge_by}
\tcbset{colframe=blue!75!black,colback=white,width=5cm,nobeforeafter}

\begin{tcolorbox}
This is a \textbf{tcolorbox}.
\end{tcolorbox}
\begin{tcolorbox}[enlarge by=5mm,enhanced,show bounding box]
This is a \textbf{tcolorbox}.
\end{tcolorbox}
\end{exdispExample}
\end{docTcbKey}



\begin{docTcbKey}{grow to left by}{=\meta{length}}{no default, initially |0mm|}
  Enlarges the current box width by \meta{length} and
  enlarges (shrinks) the bounding box distance to the left side of the box by
  $-$\meta{length}. Also see \refKey{/tcb/left skip}.
\begin{exdispExample}[safety=2cm]{grow_to_left_by}
\tcbset{colframe=blue!75!black,colback=white}

\begin{tcolorbox}[width=5cm,grow to left by=2cm,enhanced,show bounding box]
This is a \textbf{tcolorbox} with a width of 7cm.
\end{tcolorbox}
\end{exdispExample}
\end{docTcbKey}

\begin{docTcbKey}{grow to right by}{=\meta{length}}{no default, initially |0mm|}
  Enlarges the current box width by \meta{length} and
  enlarges (shrinks) the bounding box distance to the right side of the box by
  $-$\meta{length}. Also see \refKey{/tcb/right skip}.
\begin{exdispExample}[safety=2cm]{grow_to_right_by}
\tcbset{colframe=blue!75!black,colback=white}

\begin{tcolorbox}[grow to right by=2cm,enhanced,show bounding box]
This is a \textbf{tcolorbox}.
\end{tcolorbox}

\bigskip

\begin{tcolorbox}[grow to right by=2cm,grow to left by=1cm,
  enhanced,show bounding box]
This is a \textbf{tcolorbox}.
\end{tcolorbox}
\end{exdispExample}
\end{docTcbKey}

\clearpage

\begin{docTcbKey}[][doc new=2018-03-22]{grow sidewards by}{=\meta{length}}{no default, initially |0mm|}
  Shortcut for setting \refKey{/tcb/grow to left by} and \refKey{/tcb/grow to right by}
  to \meta{length}. Also see \refKey{/tcb/oversize} and \refKey{/tcb/spread sidewards}.
\begin{exdispExample}[safety=2cm]{grow_sidewards_by}
\tcbset{colframe=blue!75!black,colback=white}

\begin{tcolorbox}[grow sidewards by=2cm,enhanced,show bounding box]
This is a \textbf{tcolorbox}.
\end{tcolorbox}
\end{exdispExample}
\end{docTcbKey}


\subsubsection{Box Alignment}

\begin{docTcbKey}[][doc new=2015-11-20]{flush left}{}{style, no value}
  Enlarges the bounding box to the right side to fill the line completely.
\begin{exdispExample}{flush_left}
\tcbset{colframe=blue!75!black,colback=white}

\begin{tcolorbox}[flush left,width=5cm,enhanced,show bounding box]
This is a \textbf{tcolorbox}.
\end{tcolorbox}
\end{exdispExample}
\end{docTcbKey}


\begin{docTcbKey}[][doc new=2015-11-20]{flush right}{}{style, no value}
  Enlarges the bounding box to the left side to fill the line completely.
\begin{exdispExample}{flush_right}
\tcbset{colframe=blue!75!black,colback=white}

\begin{tcolorbox}[flush right,width=5cm,enhanced,show bounding box]
This is a \textbf{tcolorbox}.
\end{tcolorbox}
\end{exdispExample}
\end{docTcbKey}


\begin{docTcbKey}[][doc new=2015-11-20]{center}{}{style, no value}
  Enlarges the bounding box equally to both sides to fill the line completely.
\begin{exdispExample}{center}
\tcbset{colframe=blue!75!black,colback=white}

\begin{tcolorbox}[center,width=5cm,enhanced,show bounding box]
This is a \textbf{tcolorbox}.
\end{tcolorbox}
\end{exdispExample}
\end{docTcbKey}


\clearpage
\subsubsection{Toggle Enlargements}

\begin{docTcbKey}[][doc updated=2015-11-13]{toggle enlargement}{=\meta{toggle preset}}{default |evenpage|, initially |none|}
  According to the \meta{toggle preset}, the left and the right enlargements of
  the bounding box are switched or not. Feasible values are:
  \begin{itemize}
  \item\docValue{none}: no switching.
  \item\docValue{forced}: the values of the left and right enlargement are switched.
  \item\docValue{evenpage}: if the page is an even page, the values of the left and
    right enlargement are switched. This value also sets
    \refKey{/tcb/check odd page} to |true|.
  \end{itemize}
\begin{marker}
See \refKey{/tcb/toggle left and right} to toggle geometry settings.
\end{marker}

\begin{dispExample}
\tcbset{colframe=blue!75!black,colback=white,
  grow to left by=20mm,grow to right by=-5mm}

\begin{tcolorbox}[toggle enlargement=none,enhanced,show bounding box]
This is a \textbf{tcolorbox}.
\end{tcolorbox}
\begin{tcolorbox}[toggle enlargement=forced]
This is a \textbf{tcolorbox}.
\end{tcolorbox}
\begin{tcolorbox}[toggle enlargement=evenpage]
  This page is an \tcbifoddpage{odd}{even} page.
  Therefore, the left and right enlargements
  \tcbifoddpage{are not}{are} toggled.
\end{tcolorbox}
\end{dispExample}

\begin{dispListing}
\begin{tcolorbox}[colframe=red!60!black,colback=red!15!white,
  fonttitle=\bfseries,title=Floating box from \texttt{toggle enlargement},
  width=\textwidth,grow to right by=2cm,toggle enlargement,float=t]
  This page is an \tcbifoddpage{odd}{even} page.
  Therefore, the left and right enlargements \tcbifoddpage{are not}{are} toggled.
  This box stretches to the right margin on odd pages and to the left
  margin on even pages. The current document is one-sided -- this feature makes
  sense for two-sided documents only.
\end{tcolorbox}
\end{dispListing}
\tcbusetemp
\end{docTcbKey}



\clearpage
\subsubsection{Spread Box to Page Borders}

\begin{marker}
The following border options are \emph{not} applicable to nested boxes, boxes insides
tables, etc.
For boxes inside lists, the options \emph{may} work, but not necessarily.
Also, boxes should be set with |\noindent| and full width.
\end{marker}

\begin{docTcbKey}[][doc new=2017-02-13]{spread inwards}{\colOpt{=\meta{length}}}{default |0pt|, initially unset}
Enlarges the current box width to match the inner page border (left-handed side for one-sided
documents). If the optional \meta{length} is greater than |0pt|, the box
grows over the border, if \meta{length} is lower than |0pt|, there is a
margin between box and page border.
\refKey{/tcb/toggle enlargement} is set automatically for twosided documents.
\begin{dispListing}
\begin{tcolorbox}[enhanced,spread inwards,
  colframe=blue!75!black,colback=white,show bounding box]
This is a \textbf{tcolorbox}.
\end{tcolorbox}
\end{dispListing}
{\tcbusetemp}
\end{docTcbKey}

\begin{docTcbKey}[][doc new=2017-02-13]{spread outwards}{\colOpt{=\meta{length}}}{default |0pt|, initially unset}
Enlarges the current box width to match the outer page border (right-handed side for one-sided
documents). If the optional \meta{length} is greater than |0pt|, the box
grows over the border, if \meta{length} is lower than |0pt|, there is a
margin between box and page border.
\refKey{/tcb/toggle enlargement} is set automatically for twosided documents.
\begin{dispListing}
\begin{tcolorbox}[enhanced,spread outwards,
  colframe=blue!75!black,colback=white,show bounding box]
This is a \textbf{tcolorbox}.
\end{tcolorbox}
\end{dispListing}
{\tcbusetemp}
\end{docTcbKey}


\begin{docTcbKey}[][doc new=2017-02-13]{move upwards}{\colOpt{=\meta{length}}}{default |0pt|, initially unset}
Starts a new page with the box at the very top page border.
If the optional \meta{length} is greater than |0pt|, the box
moves over the border, if \meta{length} is lower than |0pt|, there is a
margin between box and page border.
\end{docTcbKey}


\begin{docTcbKey}[][doc new=2017-02-13]{move upwards*}{\colOpt{=\meta{length}}}{default |0pt|, initially unset}
Identical to \refKey{/tcb/move upwards}, but without starting a new page.
\end{docTcbKey}


\begin{docTcbKey}[][doc new=2017-02-13]{fill downwards}{\colOpt{=\meta{length}}}{default |0pt|, initially unset}
Enlarges the height of the box until the very bottom page border.
The library \mylib{breakable} has to be loaded, and
\refKey{/tcb/height fill} is set automatically.
If the optional \meta{length} is greater than |0pt|, the box
moves over the border, if \meta{length} is lower than |0pt|, there is a
margin between box and page border.
\begin{dispListing}
\begin{tcolorbox}[enhanced,fill downwards,
  colframe=blue!75!black,colback=white,show bounding box]
This is a \textbf{tcolorbox}.
\end{tcolorbox}
\end{dispListing}
{\tcbusetemp}
\end{docTcbKey}


\begin{tcolorbox}[enhanced,spread upwards,sharp corners=north,height=3cm,
  colframe=blue!75!black,interior style={top color=blue!50,bottom color=white}]
This is an example for \enquote{spread upwards}.
\end{tcolorbox}
\begin{docTcbKey}[][doc new=2017-02-13]{spread upwards}{\colOpt{=\meta{length}}}{default |0pt|, initially unset}
Combination of
\refKey{/tcb/move upwards}, \refKey{/tcb/spread inwards}, and \refKey{/tcb/spread outwards}.
The optional \meta{length} is used for all these keys.
\begin{dispListing}
\begin{tcolorbox}[enhanced,spread upwards,sharp corners=north,height=3cm,
  colframe=blue!75!black,interior style={top color=blue!50,bottom color=white}]
This is an example for \enquote{spread upwards}.
\end{tcolorbox}
\end{dispListing}
\end{docTcbKey}


\begin{docTcbKey}[][doc new=2017-02-13]{spread upwards*}{\colOpt{=\meta{length}}}{default |0pt|, initially unset}
Identical to \refKey{/tcb/move upwards}, but without starting a new page.
\end{docTcbKey}


\begin{docTcbKey}[][doc new=2017-02-13]{spread sidewards}{\colOpt{=\meta{length}}}{default |0pt|, initially unset}
Combination of \refKey{/tcb/spread inwards} and \refKey{/tcb/spread outwards}.
The optional \meta{length} is used for all these keys.
Also see \refKey{/tcb/oversize} and \refKey{/tcb/grow sidewards by}.
\begin{dispListing}
\begin{tcolorbox}[enhanced,spread sidewards,
  colframe=blue!75!black,colback=white,show bounding box]
This is a \textbf{tcolorbox}.
\end{tcolorbox}
\end{dispListing}
{\tcbusetemp}
\end{docTcbKey}


\begin{docTcbKey}[][doc new=2017-02-13]{spread}{\colOpt{=\meta{length}}}{default |0pt|, initially unset}
Combination of
\refKey{/tcb/move upwards}, \refKey{/tcb/fill downwards}, \refKey{/tcb/spread inwards},
and \refKey{/tcb/spread outwards}.
Such, the box fills the whole page.
The optional \meta{length} is used for all these keys.
\end{docTcbKey}


\begin{docTcbKey}[][doc new=2017-02-13]{spread downwards}{\colOpt{=\meta{length}}}{default |0pt|, initially unset}
Combination of
\refKey{/tcb/fill downwards}, \refKey{/tcb/spread inwards}, and \refKey{/tcb/spread outwards}.
The optional \meta{length} is used for all these keys.
\begin{dispListing}
\begin{tcolorbox}[enhanced,spread downwards,sharp corners=south,
  colframe=red!75!black,interior style={top color=white,bottom color=red!50}]
This is an example for \enquote{spread downwards}.
\end{tcolorbox}
\end{dispListing}
\end{docTcbKey}
\begin{tcolorbox}[enhanced,spread downwards,sharp corners=south,
  colframe=red!75!black,interior style={top color=white,bottom color=red!50}]
This is an example for \enquote{spread downwards}.
\end{tcolorbox}




\clearpage
\subsubsection{Box Extrusion}

\begin{marker}
The following keys should not be used with breakable boxes.
\end{marker}

\begin{docTcbKey}{shrink tight}{}{style, no value, initially unset}
  The total colored box is shrunk to the dimensions of the upper
  part. There should be no lower part and no title.
  This style sets the \refKey{/tcb/boxsep} to |0pt| and other geometry keys
  to fitting values. This option is likely to be used with the following
  extrusion keys.
\begin{exdispExample}{shrink_tight}
\tcbset{colframe=blue!75!black,colback=white,arc=0mm,boxrule=0.4pt,
        nobeforeafter,tcbox raise base,shrink tight}

\begin{tcolorbox}
This is a \textbf{tcolorbox}.
\end{tcolorbox}

Lorem \tcbox{ipsum} dolor sit amet, consectetuer adipiscing elit.
\end{exdispExample}
\end{docTcbKey}

\begin{docTcbKey}[][doc updated=2014-09-19]{extrude left by}{=\meta{length}}{style, no default, initially unset}
  The colored box is extruded by the given \meta{length} to the left side.
  The inner width and the bounding box is kept unchanged and the operation
  is additive!
\begin{exdispExample}{extrude_left_by}
\tcbset{enhanced,colframe=red,colback=yellow!25!white,
  frame style={opacity=0.25},interior style={opacity=0.5},
  nobeforeafter,tcbox raise base,shrink tight,extrude by=2mm}

Lorem ipsum dolor sit amet, consectetuer adipiscing elit. Ut purus elit,
vestibulum ut, placerat ac, adipiscing vitae, felis.
\tcbox[extrude left by=1cm]{Curabitur} dictum gravida mauris.
Nam arcu libero, nonummy eget, consectetuer id, vulputate a, magna.
\end{exdispExample}
\end{docTcbKey}

\begin{docTcbKey}[][doc updated=2014-09-19]{extrude right by}{=\meta{length}}{style, no default, initially unset}
  The colored box is extruded by the given \meta{length} to the right side.
  The inner width and the bounding box is kept unchanged and the operation
  is additive!
\begin{exdispExample}{extrude_right_by}
\tcbset{enhanced,colframe=red,colback=yellow!25!white,
  frame style={opacity=0.25},interior style={opacity=0.5},
  nobeforeafter,tcbox raise base,shrink tight,extrude by=2mm}

Lorem ipsum dolor sit amet, consectetuer adipiscing elit. Ut purus elit,
vestibulum ut, placerat ac, adipiscing vitae, felis.
\tcbox[extrude right by=1cm]{Curabitur} dictum gravida mauris.
Nam arcu libero, nonummy eget, consectetuer id, vulputate a, magna.
\end{exdispExample}
\end{docTcbKey}

\clearpage
\begin{docTcbKey}{extrude top by}{=\meta{length}}{style, no default, initially unset}
  The colored box is extruded by the given \meta{length} to the top side.
  The inner width and the bounding box is kept unchanged and the operation
  is additive!
\begin{exdispExample}{extrude_top_by}
\tcbset{enhanced,colframe=red,colback=yellow!25!white,
  frame style={opacity=0.25},interior style={opacity=0.5},
  nobeforeafter,tcbox raise base,shrink tight,extrude by=2mm}

Lorem ipsum dolor sit amet, consectetuer adipiscing elit. Ut purus elit,
vestibulum ut, placerat ac, adipiscing vitae, felis.
\tcbox[extrude top by=1cm]{Curabitur} dictum gravida mauris.
Nam arcu libero, nonummy eget, consectetuer id, vulputate a, magna.
\end{exdispExample}
\end{docTcbKey}

\begin{docTcbKey}{extrude bottom by}{=\meta{length}}{style, no default, initially unset}
  The colored box is extruded by the given \meta{length} to the bottom side.
  The inner width and the bounding box is kept unchanged and the operation
  is additive!
\begin{exdispExample}[safety=1cm]{extrude_bottom_by}
\tcbset{enhanced,colframe=red,colback=yellow!25!white,
  frame style={opacity=0.25},interior style={opacity=0.5},
  nobeforeafter,tcbox raise base,shrink tight,extrude by=2mm}

Lorem ipsum dolor sit amet, consectetuer adipiscing elit. Ut purus elit,
vestibulum ut, placerat ac, adipiscing vitae, felis.
\tcbox[extrude bottom by=1cm]{Curabitur} dictum gravida mauris.
Nam arcu libero, nonummy eget, consectetuer id, vulputate a, magna.
\end{exdispExample}
\end{docTcbKey}

\begin{docTcbKey}{extrude by}{=\meta{length}}{style, no default, initially unset}
  The colored box is extruded by the given \meta{length} to all sides.
  The inner width and the bounding box is kept unchanged and the operation
  is additive!
\begin{exdispExample}{extrude_by}
\tcbset{enhanced,colframe=red,colback=yellow!25!white,
  frame style={opacity=0.25},interior style={opacity=0.5},
  nobeforeafter,tcbox raise base,shrink tight,extrude by=2mm}

Lorem ipsum dolor sit amet, consectetuer adipiscing elit. Ut purus elit,
vestibulum ut, placerat ac, adipiscing vitae, felis. \tcbox{Curabitur} dictum
gravida mauris. \tcbox[colframe=Green,interior style={opacity=0.0}]{Nam}
arcu libero, nonummy eget, consectetuer id, \tcbox{vulputate} a, magna. Donec
vehicula augue eu neque. Pellentesque habitant morbi tristique senectus et netus
et malesuada fames ac turpis egestas. \tcbox{Mauris ut leo.}
\end{exdispExample}
\end{docTcbKey}

\clearpage
\subsection{Layered Boxes and Every Box Settings}\label{subsec:everybox}
A |tcolorbox| may contain another |tcolorbox| and so on. The package
takes track of the nesting level using a counter \docCounter{tcblayer}. Counter values
may be used for doing some fancy things, but you should never change
the counter value yourself.

The package takes special care for the first four layers or nesting levels,
called managed layers.
Here, footnote texts are administrated to find their intended place
and specific layer dependent options may be set by changing
\refKey{/tcb/every box on layer n}.
If needed, the number of managed layers can be increased by setting
\refCom{tcbsetmanagedlayers} to a higher value than~4.

The following styles have a considerable influence on how layered boxes
are processed. Note especially that nested boxes are getting a
\refKey{/tcb/reset} by default. You can change this, but be prepared for
surprises if you do.

If the defaults are \emph{not changed}, a |tcolorbox| gets its options
in the following order. Following options overwrite preceding options.
\begin{enumerate}
\item On package load, all options are set to default values.
\item Every \refCom{tcbset} command adds or changes options for the following boxes
  inside the current \TeX\ group.
\item While entering a |tcolorbox|, a \refKey{/tcb/every box on layer n} or
  \refKey{/tcb/every box on higher layers} option list is applied.
  With default settings this means:
  \begin{itemize}
  \item For layer 1 (lowest layer), the \refKey{/tcb/every box} option list is applied.
    Not overwritten options given by a preceding \refCom{tcbset} survive.
  \item For layer 2 and above (nested boxes), a \refKey{/tcb/reset} followed by
    \refKey{/tcb/every box} option list is applied.
    Every resettable options given by a preceding \refCom{tcbset}
    and by the surrounding box(es) are reset.
  \end{itemize}
\item The \meta{options} given to the |tcolorbox| are applied.
  Or, if the box was generated by \refCom{newtcolorbox} or friends,
  the \meta{options} given there are applied.
\item If the box was generated by \refCom{newtcolorbox} or friends,
  some automated options are applied.
\end{enumerate}


\begin{docTcbKey}{every box}{}{style}
By default, this style is empty.
\begin{dispListing}
% default setting:
\tcbset{every box/.style={}}
\end{dispListing}
It may be changed by redefining this style.
\begin{dispListing}
% setting all boxes to be enhanced:
\tcbset{every box/.style={enhanced}}
\end{dispListing}

\medskip
\begin{marker}
The alternative for setting something for every box (on every layer) is\\
\refCom{tcbsetforeverylayer}:
\begin{dispListing}
% setting all boxes to be enhanced:
\tcbsetforeverylayer{enhanced}
\end{dispListing}
\end{marker}
\end{docTcbKey}

\clearpage
\begin{docTcbKey}{every box on layer n}{}{style}
Here, |n| has to be replaced by a number ranging from 1 to the highest
managed layer number (4 by default).
\begin{dispListing}
% default settings:
\tcbset{
  every box on layer 1/.style={every box},
  every box on layer 2/.style={reset,every box},
  every box on layer 3/.style={reset,every box},
  every box on layer 4/.style={reset,every box},
  }
\end{dispListing}
\end{docTcbKey}


\begin{docTcbKey}{every box on higher layers}{}{style}
Higher layers are layers above the highest
managed layer number (4 by default).
\begin{dispListing}
% default setting:
\tcbset{every box on higher layers/.style={reset,every box}}
\end{dispListing}
\end{docTcbKey}


\begin{docCommand}{tcbsetmanagedlayers}{\marg{number}}
Replaces the highest managed layer number by \meta{number} where 4 is
the default. This macro can only be used inside the preamble.
Using a \meta{number} lower than 4 typically makes no sense, but is
not forbidden.
\end{docCommand}

\begin{tcboutputlisting}
% \usepackage{lipsum}   \tcbuselibrary{skins,breakable}
\tcbset{colframe=red!75!black,fonttitle=\bfseries,
  colback=red!5!white,
  every box/.style={enhanced,watermark text=\thetcblayer,
    before=\par\smallskip,after=\par\smallskip},
  every box on layer 2/.append style={colback=yellow!10!white,drop fuzzy shadow}}
\begin{tcolorbox}[enhanced jigsaw,breakable,title=Layer 1 Box]
Here comes a footnote\footnote{Footnote from layer 1 box}.
\lipsum[2]
  \begin{tcolorbox}[title=Layer 2 Box]
  abc\footnote{The footnote of abc}
  \end{tcolorbox}
  \begin{tcolorbox}[title=Another Box,ams equation]
    \tcbhighmath{\sum\limits_{n=1}^{\infty} \frac{1}{n}} = \infty.
  \end{tcolorbox}
Some text\footnote{Footnote from some text}.
  \begin{tcolorbox}[title=Yet Another Box]
    \tcboxfit[height=2cm]{\lipsum[1]}
    My text.
    \begin{tcolorbox}
      Another lipsum text\footnote{A lipsum text}. \lipsum[3]
      \begin{tcolorbox}[title=Layer 4,colframe=blue,colback=white]
        Layer 4\footnote{Layer 4 footnote}
      \end{tcolorbox}
      The End\footnote{Last footnote}.
    \end{tcolorbox}
  \end{tcolorbox}
\end{tcolorbox}
\end{tcboutputlisting}

\tcbinputlisting{base example,listing only,listing style=mydocumentation}

{\tcbuselistingtext}


\clearpage
\subsection{Capture Mode}\label{subsec:capture}

\begin{docTcbKey}{capture}{=\meta{mode}}{no default, initially |minipage|}
  The capture \meta{mode} defines how the box content is processed.

Feasible values for \meta{mode} are:
\begin{itemize}
\item\docValue{minipage}:\\
  This is the default \meta{mode} for \refEnv{tcolorbox}.
  The content may have an upper and a lower part. Optionally, the box
  can be \refKey{/tcb/breakable}. The box content is put into a
  minipage or into something similar to a minipage.
\item\docValue{hbox}:\\
  This is the default \meta{mode} for \refCom{tcbox}. The content cannot have
  a lower part and cannot be broken. The colored box is sized according
  to the dimensions of the content.
  A shortcut to set this mode is \refKey{/tcb/hbox}.
\item\docValue{fitbox}: (needs the \mylib{fitting} library)\\
  This is the default \meta{mode} for \refCom{tcboxfit}. The content cannot have
  a lower part and cannot be broken.
  The content is sized according to the dimensions of the colored box.
  A shortcut to set this mode is \refKey{/tcb/fit}.
\end{itemize}

\begin{exdispExample}{capture}
\tcbset{colframe=blue!75!black,colback=white}

\begin{tcolorbox}[capture=minipage]
This is a tcolorbox.
\end{tcolorbox}

\begin{tcolorbox}[capture=hbox]
This is a tcolorbox.
\end{tcolorbox}

\begin{tcolorbox}[capture=fitbox,height=9mm]% needs the `fitting' library
This is a tcolorbox.
\end{tcolorbox}
\end{exdispExample}
\end{docTcbKey}



\begin{docTcbKey}{hbox}{}{style, no default}
  Shortcut for |capture=hbox|.
\begin{exdispExample}{hbox}
\tcbset{colframe=blue!75!black,colback=white}

\begin{tcolorbox}[hbox]
This is a tcolorbox.
\end{tcolorbox}
\end{exdispExample}
\end{docTcbKey}


\begin{docTcbKey}{minipage}{}{style, no default}
  Shortcut for |capture=minipage|.
\end{docTcbKey}


\clearpage
\subsection{Text Characteristics}
\begin{docTcbKey}[][doc updated=2024-01-10]{parbox}{\colOpt{=true\textbar false}}{default |true|, initially |true|}
  The text inside a |tcolorbox| is formatted using a \LaTeX\ |minipage|
  if the box is unbreakable. If breakable, the box tries a mimicry of
  a |minipage|. In a |minipage| or |parbox|, paragraphs are formatted slightly different
  as the main text. If the key value is set to |false|, the normal main text
  behavior is restored. In some situations, this has some unwanted side
  effects. It is recommended that you use this \textbf{experimental setting} only
  where you really want to have this feature.
\end{docTcbKey}

\begin{dispListing}
% \usepackage{lipsum}  % preamble
\tcbset{width=(\linewidth-2mm)/2,nobeforeafter,arc=1mm,
  colframe=blue!75!black,colback=white,fonttitle=\bfseries,fontupper=\small,
  left=2mm,right=2mm,top=1mm,bottom=1mm,equal height group=parbox}

\begin{tcolorbox}[parbox,adjusted title={parbox=true (normal)}]
  \lipsum[1-2]
\end{tcolorbox}\hfill%
\begin{tcolorbox}[parbox=false,adjusted title={parbox=false}]
  \lipsum[1-2]
\end{tcolorbox}%
\end{dispListing}
{\tcbusetemp}


\clearpage
\begin{docTcbKey}{hyphenationfix}{\colOpt{=true\textbar false}}{default |true|, initially |false|}
  Long words at the beginning of paragraphs in very narrow boxes
  will not be hyphenated using |pdflatex|. This problem is circumvented by
  applying the |hyphenationfix| option.

\begin{exdispExample*}{hyphenationfix}{sbs,lefthand ratio=0.6}
\tcbset{colframe=blue!75!black,
  fontupper=\normalsize,
  colback=blue!5!white,width=4cm}

\begin{tcolorbox}
Rechnungsadjunktentochter.\par
Statthaltereikonzipist.
\end{tcolorbox}

\begin{tcolorbox}[hyphenationfix]
Rechnungsadjunktentochter.\par
Statthaltereikonzipist.
\end{tcolorbox}
\end{exdispExample*}

\smallskip
\begin{marker}
|parbox=false| and |hyphenationfix| should not be used together. They are
targeting different box types and they do not blend very well.
\end{marker}
\end{docTcbKey}


\subsection{Files}
\begin{docTcbKey}{tempfile}{=\meta{file name}}{no default, initially \cs{jobname.tcbtemp}}
  Sets \meta{file name} as name for the temporary file which is used inside
  \refEnv{tcbwritetemp} and \refCom{tcbusetemp} implicitly.
\end{docTcbKey}

\clearpage
\subsection{\texttt{\textbackslash tcbox} Specials}
The following options are applicable for \refCom{tcbox} and \refCom{tcboxmath}
only.

\begin{docTcbKey}{tcbox raise}{=\meta{length}}{no default, initially \texttt{0pt}}
  Raises the \refCom{tcbox} by the given \meta{length}.
 % Sets the line width of the right rule to \meta{length}.
\begin{exdispExample}{tcbox_raise}
\tcbset{colframe=blue!50!black,colback=white,colupper=red!50!black,
        fonttitle=\bfseries,nobeforeafter,center title}

Test\dotfill
\tcbox[tcbox raise base]{Hello World 1}\dotfill
\tcbox{Hello World 2}\dotfill
\tcbox[tcbox raise=5mm]{Hello World 3}
\end{exdispExample}
\end{docTcbKey}

\begin{docTcbKey}{tcbox raise base}{}{style, no value, initially unset}
  Raises the \refCom{tcbox} such that the base of its content matches
  the base of the environmental line; see example above.
\end{docTcbKey}

\begin{docTcbKey}{on line}{}{style, no value, initially unset}
  Combines \refKey{/tcb/tcbox raise base} with \refKey{/tcb/nobeforeafter}.
  The resulting box behaves analogue to |\fbox|.
\end{docTcbKey}

\begin{docTcbKey}{verbatim}{}{style, no value}
  Sets options for a \textit{verbatim} style \refCom{tcbox}.
  Since the indented boxes may contain only very few words, the
  dimensions are made smaller and \refKey{/tcb/nobeforeafter}
  and \refKey{/tcb/tcbox raise base} are set.
\begin{dispExample*}{sbs,lefthand ratio=0.6}
\DeclareTotalTCBox{\myverb}{ v }{verbatim,
  colframe=red!75!black,colupper=blue}{#1}

\myverb{\textbf} is a \myverb{\LaTeX} command.
\end{dispExample*}
\end{docTcbKey}



\clearpage
\begin{docTcbKey}[][doc new=2015-03-23]{tcbox width}{=\meta{mode}}{no default, initially \texttt{auto}}
Controls how \refCom{tcbox} respects a \refKey{/tcb/width} setting.
Feasible values for \meta{mode} are:
\begin{itemize}
\item\docValue{auto} (initial setting):
  ignore \refKey{/tcb/width} and set box width according to its content.
\item\docValue{auto limited}:
  Set box width according to its content, if it is smaller than \refKey{/tcb/width}.
  Otherwise, the content is set like in a \refEnv{tcolorbox} with line breaks.
\item\docValue{forced center}:
  Set box width according to \refKey{/tcb/width}.
  The content is centered and may overlap the box borders.
\item\docValue{forced left}:
  Set box width according to \refKey{/tcb/width}.
  The content is left aligned and may overlap the box borders.
\item\docValue{forced right}:
  Set box width according to \refKey{/tcb/width}.
  The content is right aligned and may overlap the box borders.
\item\docValue{minimum center}:
  Set box width according to \refKey{/tcb/width}, if the content fits into.
  The content is centered and the box width may grow beyond \refKey{/tcb/width}.
\item\docValue{minimum left}:
  Set box width according to \refKey{/tcb/width}, if the content fits into.
  The content is left aligned and the box width may grow beyond \refKey{/tcb/width}.
\item\docValue{minimum right}:
  Set box width according to \refKey{/tcb/width}, if the content fits into.
  The content is right aligned and the box width may grow beyond \refKey{/tcb/width}.
\end{itemize}

\enlargethispage*{1cm}

\begin{exdispExample}{tcbox_width}
\tcbset{size=small,on line,before upper=\strut,
  colframe=blue!75!black,colback=blue!5!white,
  fontupper=\normalsize,width=4cm}

\tcbox[tcbox width=auto]{auto}\qquad
\tcbox[tcbox width=auto limited]{auto limited}\qquad
\tcbox[tcbox width=auto limited]{auto limited with long text}\\
\tcbox[tcbox width=forced center]{forced center}\qquad
\tcbox[tcbox width=forced center]{forced center with long text}\\
\tcbox[tcbox width=forced left]{forced left}\qquad
\tcbox[tcbox width=forced left]{forced left with long text}\\
\tcbox[tcbox width=forced right]{forced right}\qquad
\tcbox[tcbox width=forced right]{forced right with long text}\\
\tcbox[tcbox width=minimum center]{minimum center}\qquad
\tcbox[tcbox width=minimum center]{minimum center with long text}\\
\tcbox[tcbox width=minimum left]{minimum left}\qquad
\tcbox[tcbox width=minimum left]{minimum left with long text}\\
\tcbox[tcbox width=minimum right]{minimum right}\qquad
\tcbox[tcbox width=minimum right]{minimum right with long text}
\end{exdispExample}
\end{docTcbKey}


%\subsection{Skins}
%There are additional option keys which change the appearance of a |tcolorbox|.
%If only the core package is used, there is only one \emph{skin} and these
%keys are meaningless.
%The library \mylib{skins} adds more skins. The appropriate option keys for skins of
%the core package are therefore described in \zvref{sec:skincorekeys} from
%page \pageref{sec:skincorekeys}.

\clearpage
\subsection{Counters, Labels, and References}

\begin{docTcbKey}{phantom}{=\meta{code}}{no default, initially unset}
The \meta{code} is put in a box at the upper left corner of the |tcolorbox|.
If the |tcolorbox| is breakable, the \meta{code} is executed for the first box of
the break sequence only. If there already was some phantom code given, the
new \meta{code} is appended.\par
The \meta{code} is intended to be used for counter stepping, labelling, and
related operations which do not produce visible text.
\begin{itemize}
\item The \meta{code} is executed before the title and box content, i.\,e., counter
  values are ensured to be increased before usage.
\item Labels are ensured to reference the correct page number.
\item The \meta{code} is executed only once even during fitting operations for
  title and box content.
\item In combination with the \refPkg{hyperref} package, the hyper anchor is set
  to the upper left corner of the |tcolorbox|, i.\,e., links inside the pdf document
  will jump to the box pleasantly.
\item Since the \meta{code} is executed inside a \TeX\ group, only global
  operations can survive this group.
\end{itemize}
Examples for the |phantom| usage are given in Section \ref{listing:exercises}
from page \pageref{listing:exercises}, e.\,g.\
Example \ref{exe:tabular_example} on page \pageref{exe:tabular_example}.
\end{docTcbKey}

\begin{docTcbKey}{nophantom}{}{no value, initially set}
Removes the phantom code if set before.
\end{docTcbKey}


\begin{docTcbKey}[][doc new=2023-02-08]{label is label}{}{no value, initially set}
Sets |\label| as labelling for |tcolorbox| (global or per box).
When \refKey{/tcb/label} and friends are used, the underlying labeling macro is |\label|.
References can be made by |\ref|, |\pageref|, and similar macros. This is the
default \LaTeX\ behaviour.
Note that this option is not getting reset by \refKey{/tcb/reset}.
\end{docTcbKey}


\begin{docTcbKey}[][doc new=2023-02-08]{label is zlabel}{}{no value, initially unset}
Sets |\zlabel| as labelling for |tcolorbox| (global or per box).
When \refKey{/tcb/label} and friends are used, the underlying labeling macro is |\zlabel|.
This needs the \refPkg{zref} package \cite{oberdiek:zref} or packages building upon \refPkg{zref} to be loaded.
References can be made by |\zref|. If the \refPkg{zref-clever} package \cite{barros:zref-clever} is loaded,
|\zcref|, |\zcpageref|, and similar macros can be applied.
Note that this option is not getting reset by \refKey{/tcb/reset}.
\begin{marker}
Recent versions of \refPkg{zref-clever} hook into the common |\label| and not
need setting |\zlabel| explicitly. \refKey{/tcb/label is zlabel} can be
omitted and applied only for optimization (for experts).\par
See \refKey{/tcb/label type} for more information.
\end{marker}
\end{docTcbKey}


\begin{docTcbKey}[][doc updated=2023-02-08]{label}{=\meta{marker}}{no default, initially unset}
The \meta{marker} is set as label text with |\label| for a reference with the |\ref| macro,
if \refKey{/tcb/label is label} is active,
or as label text with |\zlabel| for a reference with the |\zref| macro and similar,
if \refKey{/tcb/label is zlabel} is active.
Typically, this option is used for numbered boxes, see Subsection \ref{sec:numberedboxes}
from page \pageref{sec:numberedboxes}, e.\,g.\ \refKey{/tcb/new/auto counter}.
\end{docTcbKey}


\begin{docTcbKey}[][doc new=2014-11-28]{phantomlabel}{=\meta{marker}}{no default, initially unset}
Equivalent to \refKey{/tcb/label} for an \emph{unnumbered} box.
A |\phantomsection| from the package \refPkg{hyperref} \cite{rahtz:hyperref} is used to set a correct
hyperlink target.
This is not needed and should typically not be used for a numbered box.
\end{docTcbKey}

\clearpage

\begin{docTcbKey}{step}{=\meta{counter}}{no default, initially unset}
Shortcut for \refKey{/tcb/phantom}|={\refstepcounter{#1}}|. The given \meta{counter} is
increased and ready for labelling. This option is not needed when
using the convenient automated numbering introduced with version 2.40,
see Subsection \ref{sec:numberedboxes}
from page \pageref{sec:numberedboxes}.
\end{docTcbKey}

\begin{docTcbKey}{step and label}{=\marg{counter}\marg{marker}}{no default, initially unset}
Shortcut for using \refKey{/tcb/step} and \refKey{/tcb/label}. This option is not needed when
using the convenient automated numbering introduced with version 2.40,
see Subsection \ref{sec:numberedboxes}
from page \pageref{sec:numberedboxes}.
\end{docTcbKey}


\begin{docTcbKey}[][doc updated=2024-10-22]{label type}{=\meta{type}}{no default, initially unset}
This option has a different meaning based upon the chosen labeling mechanism.
\begin{itemize}
\item If \refKey{/tcb/label is label} is active:\par
  The option key \refKey{/tcb/label type} can be used in conjunction with the \refPkg{cleveref} package
  \cite{cubitt:2018a} or the \refPkg{zref-clever} package \cite{barros:zref-clever}
  which have to be loaded separately.
  \meta{type} has to be a cross-reference type \emph{known} to |cleveref|
  like |theorem|, |algorithm|, |result|, etc. References made with |cleveref|
  will use this type. For an example, see \zvref{theo:meanvaluetheorem}.
  Recent versions of \refPkg{zref-clever} also work with \refKey{/tcb/label is label}
  and \refKey{/tcb/label type} can be used as described below for \refKey{/tcb/label is zlabel}.
\item If \refKey{/tcb/label is zlabel} is active:\par
  The option key \refKey{/tcb/label type} can be used only in conjunction with the
  \refPkg{zref-clever} package \cite{barros:zref-clever}
  which has to be loaded separately.
  \meta{type} is passed to the |zref-clever| option |reftype|, thus it has to
  be a cross-reference type \emph{known} to |zref-clever|.
  This \meta{type} is bound to an automatic |tcolorbox| counter which has to be set up using
  \refKey{/tcb/new/auto counter},
  \refKey{/tcb/new/use counter from},
  \refKey{/tcb/new/use counter}, or
  \refKey{/tcb/new/use counter*}.
\end{itemize}
\end{docTcbKey}


\begin{docTcbKey}[][doc updated=2023-02-09]{no label type}{}{no value, initially set}
Removes a \refKey{/tcb/label type}, if set before.
\end{docTcbKey}


\clearpage
\begin{docTcbKey}[][doc new and updated={2016-06-22}{2023-02-08}]{nameref}{=\meta{text}}{no default, initially unset}
If the \refPkg{nameref} package is loaded, the given \meta{text} is used for
corresponding |\nameref| macros.
If the \refPkg[zref]{zref-titleref} package is loaded, the given \meta{text} is used to
set the reference text for |\ztitleref|.
Typically, the \meta{text} will be chosen
to be identical or nearly identical to the one for \refKey{/tcb/title}.

\inputpreamblelisting{A}

\begin{dispExample}
%\RequirePackage{zref-clever,zref-titleref}
\begin{pabox}[label={mynamelabel},nameref={Title or anything else}]{Title text}
This is a tcolorbox.
\end{pabox}
This box is automatically numbered with \ref{mynamelabel} on page
\pageref{mynamelabel}.

The box is titled \enquote{\nameref{mynamelabel}},
alternatively \enquote{\ztitleref{mynamelabel}},
or \enquote{\zcref[noname,ref=title]{mynamelabel}}.
\end{dispExample}

\begin{marker}
\refKey{/tcb/nameref} is used automatically inside \refCom{newtcbtheorem}.
\end{marker}

\end{docTcbKey}


\begin{docTcbKey}[][doc new=2022-07-18]{short title}{=\meta{text}}{style, no default}
Sets both \refKey{/tcb/list entry} and \refKey{/tcb/nameref} to \meta{text}.
The most likely use case is to set a \emph{short title} for the
\flqq list of tcolorbox(es)\frqq\ and for box referencing.
\end{docTcbKey}


\clearpage
\begin{docTcbKey}[][doc new=2017-02-03]{hypertarget}{=\meta{marker}}{no default, initially unset}
A |\hypertarget| from the package \refPkg{hyperref} \cite{rahtz:hyperref} is used to
create an internal link of an anchor \meta{marker}.
This \meta{marker} can be referenced by |\hyperlink| or
\refKey{/tcb/hyperlink}.
  \begin{dispExample*}{sbs,lefthand ratio=0.7}
% \usepackage{hyperref}%
\begin{tcolorbox}[enhanced,
  colback=red!10,colframe=red!50!black,
  hypertarget=hypertwinA,
  hyperlink=hypertwinB,
  title=Box A]
Click me to jump to Box B.
\end{tcolorbox}
  \end{dispExample*}
\end{docTcbKey}



\begin{docTcbKey}[][doc new=2017-02-10]{bookmark}{=\meta{text}}{no default, initially unset}
  Sets a PDF bookmark with the given \meta{text}, if the package \refPkg{bookmark} \cite{oberdiek:bookmark}
  is loaded. This bookmark is set with an automated destination (the current box)
  and is set one level below the current bookmark level.
  \begin{dispExample*}{sbs,lefthand ratio=0.7}
% \usepackage{bookmark}%
\begin{tcolorbox}[colback=blue!10,colframe=blue!50!black,
  bookmark=Example for using a bookmark,
  title=Example for using a bookmark]
Open the bookmark view of the previewer
to see the bookmark.
\end{tcolorbox}
  \end{dispExample*}
\end{docTcbKey}


\begin{docTcbKey}[][doc new=2017-02-10]{bookmark*}{=\marg{options}\marg{text}}{no default, initially unset}
  Identical to \refKey{/tcb/bookmark}, but additional \meta{options}
  from the package \refPkg{bookmark} \cite{oberdiek:bookmark} can be given.
  \begin{dispExample*}{sbs,lefthand ratio=0.7}
% \usepackage{bookmark}%
\begin{tcolorbox}[colback=red!10,colframe=red!50!black,
  bookmark*={color=red,italic,bold}%
            {Another bookmark example},
  title=Red and bold bookmark]
Open the bookmark view of the previewer
to see the bookmark.
\end{tcolorbox}
  \end{dispExample*}
\end{docTcbKey}



\begin{docTcbKey}[][doc new=2018-07-26]{index}{=\meta{entry}}{no default, initially unset}
  Adds an index \meta{entry} for the box. This is a shortcut for
  setting |\index|\marg{entry} to \refKey{/tcb/phantom}.
\end{docTcbKey}


\begin{docTcbKey}[][doc new=2018-07-26]{index*}{=\marg{name}\marg{entry}}{no default, initially unset}
  Adds an \meta{entry} to an index with a specific \meta{name}.
  This is a shortcut for
  setting |\index|\oarg{name}\marg{entry} to \refKey{/tcb/phantom}.
  An index extension package like \refPkg{imakeidx} has to be loaded to use
  this option key.
\end{docTcbKey}








\clearpage
\subsection{Even and Odd Pages}

\begin{marker}
Also see
\refKey{/tcb/toggle left and right} and
\refKey{/tcb/toggle enlargement}
for further even/odd options.
\end{marker}

\begin{docTcbKey}[][doc updated=2015-11-13]{check odd page}{\colOpt{=true\textbar false}}{default |true|, initially |false|}
If set to |true|, a precise even/odd page testing for the current box
is applied. This is done by using labels. If a box moves to another page,
the document has to be compiled twice for the correct settings.
If set to |false|, even/odd page tests may give wrong results for the first box
of a page.

\refKey{/tcb/toggle left and right},
\refKey{/tcb/toggle enlargement}, and
\refKey{/tcb/if odd page}
automatically set |check odd page|, but for
\refCom{tcbifoddpage} this option has to be set explicitly.
\end{docTcbKey}

\enlargethispage*{1cm}
\begin{docTcbKey}[][doc new=2015-11-13]{if odd page}{=\marg{odd options}\marg{even options}}{style, no default}
If the current box is on an odd page, the \meta{odd options} are applied.
On an even page, the \meta{even options} are applied.
\refKey{/tcb/check odd page} is automatically set for precise even/odd page testing.

\begin{dispExample}
\begin{tcolorbox}[if odd page={colback=yellow!50}{colback=red!50}]
  This box is colored in yellow on an odd page
  and is colored in red on an even page.
\end{tcolorbox}
\end{dispExample}

\begin{marker}
If a box is \refKey{/tcb/breakable}, using \refKey{/tcb/if odd page}
only acts upon the \emph{first} box. If the setting should be
repeated for every partial box of the break sequence, the option should be
packed into \refKey{/tcb/extras}. In this case, \refKey{/tcb/check odd page}
has to be set explicitly! Also see \refKey{/tcb/if odd page*}.
\end{marker}
\end{docTcbKey}


\begin{docTcbKey}[][doc new=2016-11-18]{if odd page or oneside}{=\marg{odd options}\marg{even options}}{style, no default}
  For onesided documents, the \meta{odd options} are applied always.
  For twosided documents, this style is identical to \refKey{/tcb/if odd page}.
\end{docTcbKey}


\clearpage
\begin{docTcbKey}[][doc new=2015-11-13]{if odd page*}{=\marg{odd options}\marg{even options}}{style, no default}
\begin{marker}
This option needs the \mylib{breakable} library, see \zvref{sec:breakable}.
\end{marker}
For breakable boxes, if the current partial box is on an odd page, the \meta{odd options} are applied.
On an even page, the \meta{even options} are applied.
\refKey{/tcb/check odd page} is automatically set for precise even/odd page testing.

In contrast to \refKey{/tcb/if odd page}, \refKey{/tcb/if odd page*} is used
on \emph{every} partial box of a break sequences and not only on the
\emph{first} box. Another difference is that \refKey{/tcb/if odd page*}
is applied quite \emph{late} during option processing, while
\refKey{/tcb/if odd page} is applied immediately.

\refKey{/tcb/if odd page*} is implemented as \refKey{/tcb/if odd page}
packed into \refKey{/tcb/extras}.

\begin{dispExample}
% \tcbuselibrary{breakable}
\begin{tcolorbox}[breakable,if odd page*={colback=yellow!50}{colback=red!50}]
  This breakable box is colored in yellow on an odd page
  and is colored in red on an even page. For every partial box, the
  test is repeated, i.\,e., this would give a yellow, red, yellow, red, \ldots\
  sequence for a long content.
\end{tcolorbox}
\end{dispExample}
\end{docTcbKey}


\begin{docTcbKey}[][doc new=2016-11-18]{if odd page or oneside*}{=\marg{odd options}\marg{even options}}{style, no default}
  For onesided documents, the \meta{odd options} are applied always.
  For twosided documents, this style is identical to \refKey{/tcb/if odd page*}.
\end{docTcbKey}


\clearpage
\begin{docCommand}[doc new=2015-11-13]{tcbifoddpage}{\marg{odd code}\marg{even code}}
If the current box is on an odd page, the \meta{odd code} is executed.
On an even page, the \meta{even code} is executed.
For precise even/odd page testing, the \refKey{/tcb/check odd page} has to be
set manually inside the box options.

The macro \refCom{tcbifoddpage} can be used inside underlay, overlay, or watermark code to
test if the box is on an odd page. This will work also for boxes in a break sequence.

The macro can also be used inside the box \textbf{content text}. For unbreakable boxes,
the correct page test is applied.
But for \refKey{/tcb/breakable} boxes, \refCom{tcbifoddpage}
will always give the result for the page of the \emph{first} box inside
the box \textbf{content text}. If needed, the methods from the packages
\refPkg{changepage} or \refPkg{ifoddpage} could be used here.
%To mention it again, for overlays, watermarks, etc, \refCom{tcbifoddpage} gives
%the correct page test.

\begin{dispExample}
\tcbset{colframe=blue!75!black,colback=white,fonttitle=\bfseries}

\begin{tcolorbox}[enhanced,check odd page,
  title={Example for a box on an \tcbifoddpage{odd}{even} page},
  watermark text={\tcbifoddpage{Odd}{Even} page!}]
\lipsum[1]
\end{tcolorbox}
\end{dispExample}
\end{docCommand}


\begin{docCommand}[doc new=2016-11-18]{tcbifoddpageoroneside}{\marg{odd code}\marg{even code}}
  For onesided documents, the \meta{odd code} is executed always.
  For twosided documents, this macro is identical to \refCom{tcbifoddpage}.
\end{docCommand}

\clearpage
\begin{docCommand}[doc new=2015-11-13]{thetcolorboxnumber}{}
This is a unique identifier (arabic number) for a tcolorbox. It is locally
defined inside boxes and has no meaning outside. It is used for
precise even/odd page testing, but may also be valuable for elaborate user
code.

\begin{dispExample}
\begin{tcolorbox}[colback=yellow!5,title=Box \thetcolorboxnumber]
  This box is \thetcolorboxnumber.
  \tcbox[on line,size=fbox]{This box is \thetcolorboxnumber} and
  \tcbox[on line,size=fbox]{this box is \thetcolorboxnumber}.
  This box is \thetcolorboxnumber.
\end{tcolorbox}
\end{dispExample}
\end{docCommand}


\begin{docCommand}[doc new=2015-11-13]{thetcolorboxpage}{}
This macro contains the expanded arabic page number of the current tcolorbox.
It is locally defined inside boxes and has no meaning outside.
It is precise only, if \refKey{/tcb/check odd page} was set.

\begin{dispExample}
\begin{tcolorbox}[colback=yellow!5,check odd page,
    title=Box on page~\thetcolorboxpage]
  This box is located on page~\thetcolorboxpage.
\end{tcolorbox}
\end{dispExample}
\end{docCommand}





\clearpage
\subsection{Externalization}
\begin{marker}
See \zvref{sec:external} for the \mylib{external} library of |tcolorbox|.
\end{marker}

If the \emph{externalization} library |external| of the \refPkg{tikz} package is used,
a |tcolorbox| could trigger the externalization process which will arise
a compilation error.

To avoid this, there are two possible strategies:
\begin{itemize}
\item Ensure, that |\tikzexternaldisable| is set before a |tcolorbox| is used.
  If you typically use the pattern |\tikzexternalenable| \textit{some picture} |\tikzexternaldisable|,
  there is nothing to care about.
\item If \emph{externalization} is enabled globally, use \refKey{/tcb/shield externalize} to
  shield any |tcolorbox|. The preamble code could look like this:
\begin{dispListing}
\usetikzlibrary{external}
\tikzexternalize
\tcbset{shield externalize}
\end{dispListing}
\end{itemize}

\begin{docTcbKey}{shield externalize}{\colOpt{=true\textbar false}}{default |true|, initially |false|}
If set to |true|, the drawing part of the |tcolorbox| is not being externalized
which is a good thing at the current state of art. Nevertheless, if the
|tcolorbox| contains a |tikzpicture|, this picture is still externalized.
Pictures drawn with help of \refKey{/tcb/tikz upper} or alike are \emph{not}
externalized.
\end{docTcbKey}

\begin{marker}
If a |tcolorbox| is used inside a node of an encircling |tikzpicture| which is externalized,
do \emph{not} use |\tikzexternaldisable| in front of the |tcolorbox|.
\refKey{/tcb/shield externalize} is deactivated automatically inside a |tikzpicture|.
\end{marker}

\begin{marker}
\refKey{/tcb/shield externalize} is applied for every following |tcolorbox|
inside the current \TeX\ group and is not affected by \refKey{/tcb/reset}.
\end{marker}

\begin{docTcbKey}{external}{=\meta{file name}}{no default, initially unset}
  Convenience option which calls |\tikzsetnextfilename|\marg{file name}. Typically,
  it may be used inside the option list of a |tcolorbox| to set the
  externalization \meta{file name} for the first |tikzpicture| which is discovered
  \emph{inside} the box content.
  The package \refPkg{tikz} \cite{tantau:tikz_and_pgf} or the library \mylib{skins} has to be loaded to use this option.
  Additionally, |\usetikzlibrary{external}| has to be used.
\end{docTcbKey}

\begin{docTcbKey}{remake}{\colOpt{=true\textbar false}}{default |true|, initially |false|}
  Convenience option which calls |/tikz/external/remake next|. Typically,
  it may be used inside the option list of a |tcolorbox| to force the remake
  of the first |tikzpicture| which is discovered \emph{inside} the box content.
  The package \refPkg{tikz} \cite{tantau:tikz_and_pgf} or the library \mylib{skins} has to be loaded to use this option.
  Additionally, |\usetikzlibrary{external}| has to be used.
\end{docTcbKey}


\clearpage
\subsection{Miscellaneous}
\begin{docTcbKey}{reset}{}{no value, initially set}
Sets (nearly) all |tcolorbox| settings (including loaded libraries) back to their default values
\emph{plus} any settings given by \refCom{tcbsetforeverylayer}.
\refKey{/tcb/savedelimiter}, \refKey{/tcb/capture}, and
\refKey{/tcb/shield externalize} keep their values.
Also, all raster values (see \zvref{sec:raster}) are not reset.

This option is useful for boxes in boxes where the inner box should not inherit
the settings of the outer box.
Note that for boxes inside boxes the |reset| is done automatically, if the
standard settings of the package are used (v2.40 and above), see
Section \ref{subsec:everybox} from page \pageref{subsec:everybox}.
%See \refCom{tcbhighmath} for an example.
\end{docTcbKey}




\begin{docTcbKey}{code}{=\meta{code}}{no default, initially unset}
  The given \meta{code} is executed immediately. This option is useful
  to place some arbitrary code into an option list.
\begin{exdispExample}{code}
\tcbset{colback=red!5!white,colframe=red!75!black,
  code={Useless at this spot but functional.},
  fonttitle=\bfseries}

\begin{tcolorbox}[code={\newcommand{\mycommand}{\textit{working}}},
  title=My \mycommand\ title]
This is a \textbf{tcolorbox}.
\end{tcolorbox}
\end{exdispExample}
\end{docTcbKey}

\clearpage

\begin{docTcbKeys}[doc description = {no default}, doc new = 2022-05-25 ]
  {
    {
      doc name        = IfBlankTF,
      doc parameter   = {=\marg{token list}\marg{true options}\marg{false options}},
    },
    {
      doc name        = IfBlankT,
      doc parameter   = {=\marg{token list}\marg{true options}},
    },
    {
      doc name        = IfBlankF,
      doc parameter   = {=\marg{token list}\marg{false options}},
    }
  }
  Wraps the |\tl_if_blank:n(TF)| command(s) of \refPkg[l3kernel]{expl3} for option setting.
  If the \meta{token list} consists only of blank spaces or is entirely empty, the \meta{true options} are set.
  Otherwise, the \meta{false options} are set.
\begin{dispExample}
\newtcolorbox{mybox}[1]{ IfBlankF={#1}{title=`#1'} }

\begin{mybox}{My title}
This is a tcolorbox.
\end{mybox}

\begin{mybox}{    }
This is a tcolorbox.
\end{mybox}
\end{dispExample}
\end{docTcbKeys}

\medskip

\begin{docTcbKeys}[doc description = {no default}, doc new = 2022-05-25 ]
  {
    {
      doc name        = IfEmptyTF,
      doc parameter   = {=\marg{token list}\marg{true options}\marg{false options}},
    },
    {
      doc name        = IfEmptyT,
      doc parameter   = {=\marg{token list}\marg{true options}},
    },
    {
      doc name        = IfEmptyF,
      doc parameter   = {=\marg{token list}\marg{false options}},
    }
  }
  Wraps the |\tl_if_empty:n(TF)| command(s) of \refPkg[l3kernel]{expl3} for option setting.
  If the \meta{token list} is entirely empty, the \meta{true options} are set.
  Otherwise, the \meta{false options} are set.
\begin{dispExample}
\newtcolorbox{mybox}[1]{ IfEmptyTF={#1}{colframe=red}{title=`#1'} }

\begin{mybox}{My title}
This is a tcolorbox.
\end{mybox}

\begin{mybox}{}
This is a tcolorbox.
\end{mybox}
\end{dispExample}
\end{docTcbKeys}


\clearpage

\begin{docTcbKeys}[doc description = {no default}]
  {
    {
      doc name        = IfNoValueTF,
      doc parameter   = {=\marg{argument}\marg{true options}\marg{false options}},
    },
    {
      doc name        = IfNoValueT,
      doc parameter   = {=\marg{argument}\marg{true options}},
      doc new         = 2020-09-16,
    },
    {
      doc name        = IfNoValueF,
      doc parameter   = {=\marg{argument}\marg{false options}},
      doc new         = 2020-09-16,
    }
  }
  Wraps the |\IfNoValue(TF)| command(s) for option setting, see \cite{latexproject:usrguide}.
  If the \meta{argument} has no value, the \meta{true options} are set.
  Otherwise, the \meta{false options} are set.
\begin{dispExample}
\DeclareTColorBox{mybox}{ o }{colframe=red!75!black,
  IfNoValueTF={#1}{colback=red!5!white}{enhanced,interior style image=#1}}

\begin{mybox}
This is a tcolorbox.
\end{mybox}

\begin{mybox}[goldshade.png]
This is a tcolorbox.
\end{mybox}
\end{dispExample}
\end{docTcbKeys}

\medskip

\begin{docTcbKeys}[doc description = {no default}]
  {
    {
      doc name        = IfValueTF,
      doc parameter   = {=\marg{argument}\marg{true options}\marg{false options}},
    },
    {
      doc name        = IfValueT,
      doc parameter   = {=\marg{argument}\marg{true options}},
      doc new         = 2020-09-16,
    },
    {
      doc name        = IfValueF,
      doc parameter   = {=\marg{argument}\marg{false options}},
      doc new         = 2020-09-16,
    }
  }
  Wraps the |\IfValue(TF)| command(s) for option setting, see \cite{latexproject:usrguide}.
  If the \meta{argument} has a value, the \meta{true options} are set.
  Otherwise, the \meta{false options} are set.
\begin{dispExample}
\DeclareTColorBox{mybox}{ o }{colframe=red!75!black,colback=red!5!white,
  IfValueT={#1}{title={\flqq #1\frqq},fonttitle=\bfseries}}

\begin{mybox}
This is a tcolorbox.
\end{mybox}

\begin{mybox}[My title]
This is a tcolorbox.
\end{mybox}
\end{dispExample}
\end{docTcbKeys}

%\medskip
\clearpage

\begin{docTcbKeys}[doc description = {no default}]
  {
    {
      doc name        = IfBooleanTF,
      doc parameter   = {=\marg{argument}\marg{true options}\marg{false options}},
    },
    {
      doc name        = IfBooleanT,
      doc parameter   = {=\marg{argument}\marg{true options}},
      doc new         = 2020-09-16,
    },
    {
      doc name        = IfBooleanF,
      doc parameter   = {=\marg{argument}\marg{false options}},
      doc new         = 2020-09-16,
    }
  }
  Wraps the |\IfBoolean(TF)| command(s) for option setting, see \cite{latexproject:usrguide}.
  If the \meta{argument} is |\BooleanTrue|, the \meta{true options} are set.
  If the \meta{argument} is |\BooleanFalse|, the \meta{false options} are set.

\begin{dispExample}
\DeclareTColorBox{mybox}{ s }{colframe=red!75!black,
  IfBooleanTF={#1}{colback=yellow!50!red}{colback=red!5!white}}

\begin{mybox}
This is a tcolorbox.
\end{mybox}

\begin{mybox}*
This is a tcolorbox.
\end{mybox}
\end{dispExample}
\end{docTcbKeys}




\clearpage
\begin{docTcbKey}[][doc new=2016-10-21]{void}{}{no value, initially unset}
  Annihilates the current |tcolorbox| as far as possible.
  Basically, this comments out the whole |tcolorbox| by using a key.
  If the option list of the current |tcolorbox| contains arbitrary code with global
  impact (like counter settings), these actions are not undone automatically.
  Nevertheless, the effects of \refKey{/tcb/phantom}, \refKey{/tcb/step},
  \refKey{/tcb/new/auto counter}, etc., are removed by \refKey{/tcb/void}.

\begin{exdispExample}{void}
A%
  \begin{tcolorbox}[
      title=This box is completely removed by the following key,
      void
    ]
  This is a \textbf{tcolorbox}.
  \end{tcolorbox}
B
\end{exdispExample}

\begin{marker}
  This option key cannot be applied for every situation.
  For example, if several box environments with the same environment name
  are nested, for the outer environment \refKey{/tcb/void} cannot be used,
  since the end of the inner environment will be misinterpreted as
  end of the outer environment. Also, \refKey{/tcb/void} cannot be used
  for environments wrapped with \refCom{tcolorboxenvironment}.
\end{marker}
\end{docTcbKey}


\begin{docTcbKey}[][doc new=2019-03-01]{nirvana}{}{no value, initially unset}
  The contents of the current |tcolorbox| are processed including counter
  settings, but the box is just not drawn.
  Therefore, \refKey{/tcb/nirvana} is less radical than \refKey{/tcb/void}
  and several box environments can be nested without problems.

\begin{exdispExample}{nirvana}
A%
  \begin{tcolorbox}[
      title=This box is completely removed by the following key,
      nirvana
    ]
  This is a \textbf{tcolorbox}.
    \begin{tcolorbox}
    Nested Box
    \end{tcolorbox}
  \end{tcolorbox}%
B
\end{exdispExample}
\end{docTcbKey}
