% !TeX root = tcolorbox.tex
% include file of tcolorbox.tex (manual of the LaTeX package tcolorbox)
\clearpage
\section{Technical Overview and Customization}\label{sec:technical}%
\tcbset{external/prefix=external/technical_}%
This section provides a technical overview of the skin concept of |tcolorbox|.
For most applications of |tcolorbox|, one will not need to know the bells and
whistles described herein.
You may proceed to \zvref{sec:skins} where the customization options
for most users are documented.

The following explanations also cover options and settings from the \mylib{skins} library,
see  \zvref{sec:skins}.

\subsection{Skins and Drawing Engines}\label{sec:skincorekeys}
From a technical point of view, a \emph{skin} is a style definition for the
appearance of a |tcolorbox|. The core package provides some additional
option keys for skins but only two skins called \refSkin{standard}
and \refSkin{standard jigsaw}.
The \mylib{skins} library adds several more skins. To change to a skin, only one
option from the core package has to be set.

\begin{docTcbKey}{skin}{=\meta{name}}{style, no default, initially \texttt{standard}}
  Sets the current skin to \meta{name}. This is a style definition which sets all the following
  keys, i.\,e., for many use cases there is nothing more to do.
\begin{dispExample}
\tcbset{colback=Salmon!50!white,colframe=FireBrick!75!black,
  width=(\linewidth-8mm)/2,before=,after=\hfill,equal height group=ske}

\begin{tcolorbox}[adjusted title=My title]
  This is my content.
\end{tcolorbox}
\begin{tcolorbox}[beamer,adjusted title=My title]
  This is my content.
\end{tcolorbox}
\end{dispExample}
\end{docTcbKey}

\begin{docTcbKey}{skin first}{=\meta{name}}{style, no default, initially \texttt{standard}}
  If the box is set to be \refKey{/tcb/breakable} and \emph{is} broken actually,
  then the skin for the \emph{first} part of the break sequence
  is set to \meta{name}, see Subsection \ref{subsec:breaksequence} on page \pageref{subsec:breaksequence}.
  Typically, this key is set by a \refKey{/tcb/skin}.
\end{docTcbKey}


\begin{docTcbKey}{skin middle}{=\meta{name}}{style, no default, initially \texttt{standard}}
  If the box is set to be \refKey{/tcb/breakable} and \emph{is} broken actually,
  then the skin for the \emph{middle} parts (if any) of the break sequence
  is set to \meta{name}, see Subsection \ref{subsec:breaksequence} on page \pageref{subsec:breaksequence}.
  Typically, this key is set by a \refKey{/tcb/skin}.
\end{docTcbKey}


\begin{docTcbKey}{skin last}{=\meta{name}}{style, no default, initially \texttt{standard}}
  If the box is set to be \refKey{/tcb/breakable} and \emph{is} broken actually,
  then the skin for the \emph{last} part of the break sequence
  is set to \meta{name}, see Subsection \ref{subsec:breaksequence} on page \pageref{subsec:breaksequence}.
  Typically, this key is set by a \refKey{/tcb/skin}.
\end{docTcbKey}


\clearpage

The skin of a |tcolorbox| is drawn by up to four \emph{engines}.
Afterwards, the text content is drawn which is not part of a skin.
The four steps are:
\begin{enumerate}
\item The \emph{frame} of the box, drawn by \refKey{/tcb/frame engine}.
\item The \emph{interior} of the box. The interior of a box with title is
  drawn differently from a box without title.
  \refKey{/tcb/interior titled engine} or \refKey{/tcb/interior engine}
  is used to draw the interior.
\item The \emph{segmentation} (line) of the box, if there is a lower part;
  drawn by \refKey{/tcb/segmentation engine}.
\item The \emph{title area} of the box, if there is a title and
  \refKey{/tcb/title filled} is set to |true|; drawn
  by \refKey{/tcb/title engine}.
\end{enumerate}

\begin{docTcbKey}{frame engine}{=\meta{name}}{no default, initially \texttt{standard}}
  Sets the \emph{frame} drawing engine for a box to \meta{name}.
  Typically, this key is set by a \refKey{/tcb/skin}.
  Feasible values for \meta{name} are:
  \begin{itemize}
  \item\docValue{standard}: the original code from the core package,
  \item\docValue{path}: a \tikzname\ path which is controlled by \refKey{/tcb/frame style},
  \item\docValue{pathjigsaw}: a \tikzname\ path which is controlled by \refKey{/tcb/frame style},
  \item\docValue{pathfirst}: a \tikzname\ path which is controlled by \refKey{/tcb/frame style},
  \item\docValue{pathfirstjigsaw}: a \tikzname\ path which is controlled by \refKey{/tcb/frame style},
  \item\docValue{pathmiddle}: a \tikzname\ path which is controlled by \refKey{/tcb/frame style},
  \item\docValue{pathmiddlejigsaw}: a \tikzname\ path which is controlled by \refKey{/tcb/frame style},
  \item\docValue{pathlast}: a \tikzname\ path which is controlled by \refKey{/tcb/frame style},
  \item\docValue{pathlastjigsaw}: a \tikzname\ path which is controlled by \refKey{/tcb/frame style},
  \item\docValue{freelance}: deprecated.
  \item\docValue{spartan}: a quite spartan code.
  \item\docValue{empty}: draw nothing.
\end{itemize}
\end{docTcbKey}

\begin{docTcbKey}{interior titled engine}{=\meta{name}}{no default, initially \texttt{standard}}
  Sets the \emph{interior} drawing engine for a titled box to \meta{name}.
  Typically, this key is set by a \refKey{/tcb/skin}.
  Feasible values for \meta{name} are:
  \begin{itemize}
  \item\docValue{standard}: the original code from the core package,
  \item\docValue{path}: a \tikzname\ path which is controlled by \refKey{/tcb/interior style},
  \item\docValue{pathfirst}: a \tikzname\ path which is controlled by \refKey{/tcb/interior style},
  \item\docValue{pathmiddle}: a \tikzname\ path which is controlled by \refKey{/tcb/interior style},
  \item\docValue{pathlast}: a \tikzname\ path which is controlled by \refKey{/tcb/interior style},
  \item\docValue{freelance}: deprecated.
  \item\docValue{spartan}: a quite spartan code.
  \item\docValue{empty}: draw nothing.
  \end{itemize}
\end{docTcbKey}

\clearpage
\begin{docTcbKey}{interior engine}{=\meta{name}}{no default, initially \texttt{standard}}
  Sets the \emph{interior} drawing engine for an untitled box to \meta{name}.
  Typically, this key is set by a \refKey{/tcb/skin}.
  Feasible values for \meta{name} are:
  \begin{itemize}
  \item\docValue{standard}: the original code from the core package,
  \item\docValue{path}: a \tikzname\ path which is controlled by \refKey{/tcb/interior style},
  \item\docValue{pathfirst}: a \tikzname\ path which is controlled by \refKey{/tcb/interior style},
  \item\docValue{pathmiddle}: a \tikzname\ path which is controlled by \refKey{/tcb/interior style},
  \item\docValue{pathlast}: a \tikzname\ path which is controlled by \refKey{/tcb/interior style},
  \item\docValue{freelance}: deprecated.
  \item\docValue{spartan}: a quite spartan code.
  \item\docValue{empty}: draw nothing.
  \end{itemize}
\end{docTcbKey}

\begin{docTcbKey}{segmentation engine}{=\meta{name}}{no default, initially \texttt{standard}}
  Sets the \emph{segmentation} (line) drawing engine for a box to \meta{name}.
  Typically, this key is set by a \refKey{/tcb/skin}.
  Feasible values for \meta{name} are:
  \begin{itemize}
  \item\docValue{standard}: the original code from the core package,
  \item\docValue{path}: a \tikzname\ path which is controlled by \refKey{/tcb/segmentation style},
  \item\docValue{freelance}: deprecated.
  \item\docValue{spartan}: a quite spartan code.
  \item\docValue{empty}: draw nothing.
  \end{itemize}
\end{docTcbKey}

\begin{docTcbKey}{title engine}{=\meta{name}}{no default, initially \texttt{standard}}
  Sets the \emph{title area} drawing engine for a titled box to \meta{name}.
  Typically, this key is set by a \refKey{/tcb/skin}.
  Feasible values for \meta{name} are:
  \begin{itemize}
  \item\docValue{standard}: the original code from the core package,
  \item\docValue{path}: a \tikzname\ path which is controlled by \refKey{/tcb/title style},
  \item\docValue{pathfirst}: a \tikzname\ path which is controlled by \refKey{/tcb/title style},
  \item\docValue{pathmiddle}: a \tikzname\ path which is controlled by \refKey{/tcb/title style},
  \item\docValue{pathlast}: a \tikzname\ path which is controlled by \refKey{/tcb/title style},
  \item\docValue{freelance}: deprecated.
  \item\docValue{spartan}: a quite spartan code.
  \item\docValue{empty}: draw nothing.
  \end{itemize}
\end{docTcbKey}

\begin{marker}
After an engine is set to an initializing value, the resulting graphical
code can be changed using code option keys, see \zvref{subsec:addcodeoptions}.
\end{marker}

\clearpage
\begin{docTcbKey}{geometry nodes}{\colOpt{=true\textbar false}}{default |true|, initially |false|}
  If set to |true|, up to four \tikzname\ nodes are defined for a |tcolorbox|
  which are named |frame|, |interior|, |segmentation|, and |title|. These nodes
  describe the boundaries of the equally named parts of a |tcolorbox|.
  They are used by most engines based on \tikzname.
  Typically, this key is set automatically by a \refKey{/tcb/skin}.
\begin{dispExample}
\tcbset{colback=Salmon!50!white,colframe=FireBrick!75!black,
  width=(\linewidth-8mm)/2,before=,after=\hfill,equal height group=geon}

\begin{tcolorbox}[adjusted title=The title]
  The upper part.\tcblower The lower part.
\end{tcolorbox}
\begin{tcolorbox}[enhanced,adjusted title=The title,
  frame code={\path[draw=red,fill=red!25]
      (frame.south west) rectangle (frame.north east);},
  interior titled code={\path[draw=blue,fill=blue!25]
      (interior.south west) rectangle (interior.north east);},
  segmentation code={\path[draw=green,fill=green!25]
      (segmentation.south west) rectangle (segmentation.north east);},
  title code={\path[draw=black,fill=brown!75!black]
      (title.south west) rectangle (title.north east);}]
  The upper part.\tcblower The lower part.
\end{tcolorbox}
\end{dispExample}
\end{docTcbKey}






\clearpage
\subsection{Code Option Keys}\label{subsec:addcodeoptions}
The following code options are applicable for all skins.
%which
%use engines of type |freelance|.
%Especially, the skin \refSkin{freelance} supports \emph{all} of them,
%\refSkin{standard} and \refSkin{enhanced} \emph{none} of them.
The used \meta{graphical code} can be any \refPkg{pgf} code. For all skins
with exception of \refSkin{standard}
and \refSkin{standard jigsaw}, the \meta{graphical code} can also
be any \refPkg{tikz} code.


\begin{docTcbKey}{frame code}{\colOpt{=\meta{graphical code}}}{code, default from |standard|}
  The given \meta{graphical code} is used for drawing the \emph{frame} of the box.
  %This option is available only if the \refKey{/tcb/frame engine} is set to |freelance|.
\begin{dispExample}
\tcbset{colback=red!5!white,colframe=red!75!black}

\begin{tcolorbox}[enhanced,frame code={
  \foreach \n in {north east,north west,south east,south west}
  {\path [fill=red!75!black] (interior.\n) circle (3mm); } }]
This is a \textbf{tcolorbox}.
\tcblower
This is the lower part.
\end{tcolorbox}
\end{dispExample}
\end{docTcbKey}


\begin{docTcbKey}{frame empty}{}{style, no value}
  This is a shortcut for setting  \refKey{/tcb/frame code} to empty.
  This option removes the drawing of the frame.
  Alternatively, use \refKey{/tcb/frame hidden}.
\end{docTcbKey}


\begin{docTcbKey}{interior titled code}{\colOpt{=\meta{graphical code}}}{code, default from |standard|}
  The given \meta{graphical code} is used
  for drawing the \emph{interior} of the box, if the box comes with a title.
  %This option is available only if the \refKey{/tcb/interior titled engine} is set to |freelance|.
\begin{dispExample}
\tcbset{colback=red!5!white,colframe=red!75!black,fonttitle=\bfseries}

\begin{tcolorbox}[enhanced,title=My title,interior titled code={
  \path[draw=red!5!white,line width=5mm,line cap=round]
    ([xshift=3mm,yshift=-3mm]interior.north west)
    --([xshift=-3mm,yshift=3mm]interior.south east)
    ([xshift=3mm,yshift=3mm]interior.south west)
    --([xshift=-3mm,yshift=-3mm]interior.north east);}]
This is a \textbf{tcolorbox}.
\tcblower
This is the lower part.
\end{tcolorbox}
\end{dispExample}
\end{docTcbKey}


\begin{docTcbKey}{interior titled empty}{}{style, no value}
  This is a shortcut for setting  \refKey{/tcb/interior titled code} to empty.
  This option removes the drawing of the untitled interior.
  Alternatively, use \refKey{/tcb/interior hidden}.
\end{docTcbKey}



\clearpage
\begin{docTcbKey}{interior code}{\colOpt{=\meta{graphical code}}}{code, default from |standard|}
  The given \meta{graphical code} is used
  for drawing the \emph{interior} of the box, if the box is without a title.%\\
  %This option is available only if the \refKey{/tcb/interior engine} is set to |freelance|.
\begin{dispExample}
\tcbset{colback=red!5!white,colframe=red!75!black}

\begin{tcolorbox}[enhanced,interior code={
  \path[draw=red!5!white,line width=5mm,line cap=round]
    ([xshift=3mm,yshift=-3mm]interior.north west)
    --([xshift=-3mm,yshift=3mm]interior.south east)
    ([xshift=3mm,yshift=3mm]interior.south west)
    --([xshift=-3mm,yshift=-3mm]interior.north east);}]
This is a \textbf{tcolorbox}.
\tcblower
This is the lower part.
\end{tcolorbox}
\end{dispExample}
\end{docTcbKey}

\begin{docTcbKey}{interior empty}{}{style, no value}
  This is a shortcut for setting  \refKey{/tcb/interior code} to empty.
  This option removes the drawing of the interior.
  Alternatively, use \refKey{/tcb/interior hidden}.
\end{docTcbKey}


\begin{docTcbKey}{segmentation code}{\colOpt{=\meta{graphical code}}}{code, default from |standard|}
  The given \meta{graphical code} is used for drawing the
  \emph{segmentation} area of the box.%\\
  %This option is available only if the \refKey{/tcb/segmentation engine} is set to |freelance|.
\begin{dispExample}
\tcbset{colback=red!5!white,colframe=red!75!black,fonttitle=\bfseries}

\begin{tcolorbox}[enhanced,title=My title,segmentation code={
  \path[top color=red!5!white,bottom color=red!5!white,middle color=blue]
  (segmentation.south west) rectangle (segmentation.north east);}]
This is a \textbf{tcolorbox}.
\tcblower
This is the lower part.
\end{tcolorbox}
\end{dispExample}
\end{docTcbKey}


\begin{docTcbKey}{segmentation empty}{}{style, no value}
  This is a shortcut for setting  \refKey{/tcb/segmentation code} to empty.
  This option removes the drawing of the segmentation line.
  Alternatively, use \refKey{/tcb/segmentation hidden}.
\end{docTcbKey}



\clearpage
\begin{docTcbKey}{title code}{\colOpt{=\meta{graphical code}}}{code, default from |standard|}
  The given \meta{graphical code} is used for drawing the
  \emph{title} area of the box.
  %\\
  %This option is available only if the \refKey{/tcb/title engine} is set to |freelance|.
\begin{dispExample}
\tcbset{colback=red!5!white,colframe=red!75!black,fonttitle=\bfseries,
  coltitle=black}

\begin{tcolorbox}[enhanced,title=My title,title code={
  \path[draw=yellow,solid,decorate,line width=2mm,
    decoration={coil,aspect=0,segment length=10.1mm}]
    ([xshift=1mm]title.west) -- ([xshift=-1mm]title.east);}]
This is a \textbf{tcolorbox}.
\tcblower
This is the lower part.
\end{tcolorbox}
\end{dispExample}
\end{docTcbKey}


\begin{docTcbKey}{title empty}{}{style, no value}
  This is a shortcut for setting  \refKey{/tcb/title code} to empty.
  This option removes the drawing of the title area.
  Alternatively, use \refKey{/tcb/title hidden}.
\end{docTcbKey}

\clearpage
\subsection{Subskins}\label{subsec:subskins}
A subskin is a new \refKey{/tcb/skin} based on an existing skin which
is extended or changed.

\begin{marker}
Never use geometry settings or bounding box options inside a subskin definition!
If one skin is replaced by another skin, the overall bounding box should stay
constant. Especially, if a skin is used for a breakable box, unpredictable
and unpleasant results could arise otherwise. If you want to change the
geometry also, use an additional style. See the skin \refSkin{beamer} and
the style \refKey{/tcb/beamer} as pattern.
\end{marker}

\begin{docCommand}{tcbsubskin}{\marg{name}\marg{base skin}\marg{options}}
  Creates a new skin \meta{name} which inherits all properties of an
  existing \meta{base skin} plus the given \meta{options}.
  The new skin \meta{name} can be used as value for the keys
  \refKey{/tcb/skin},
  \refKey{/tcb/skin first},
  \refKey{/tcb/skin middle}, and
  \refKey{/tcb/skin last}.
  As \meta{base skin}, one can take \refSkin{standard}, \refSkin{empty},
  \refSkin{enhanced}, or any skin from the \mylib{skins} library,
  see \zvref{sec:skins}.

\begin{dispExample}
% \tcbuselibrary{skins}
\tcbsubskin{mycross}{empty}{frame code={%
  \draw[red,line width=5pt] (frame.south west)--(frame.north east);
  \draw[red,line width=5pt] (frame.north west)--(frame.south east);},
  skin first=mycross,skin middle=mycross,skin last=mycross }

\begin{tcolorbox}[skin=mycross]
  This is my content.
\end{tcolorbox}
\end{dispExample}

\end{docCommand}


\begin{docTcbKey}{skin first is subskin of}{=\marg{base skin}\marg{options}}{no default, initially unset}
  Creates a new unnamed skin which inherits all properties of an
  existing \meta{base skin} plus the given \meta{options}.
  This skin is set as \refKey{/tcb/skin first}.\\
  See a detailed example on page~\pageref{freeboxexample}.
\end{docTcbKey}


\begin{docTcbKey}{skin middle is subskin of}{=\marg{base skin}\marg{options}}{no default, initially unset}
  Creates a new unnamed skin which inherits all properties of an
  existing \meta{base skin} plus the given \meta{options}.
  This skin is set as \refKey{/tcb/skin middle}.\\
  See a detailed example on page~\pageref{freeboxexample}.
\end{docTcbKey}


\begin{docTcbKey}{skin last is subskin of}{=\marg{base skin}\marg{options}}{no default, initially unset}
  Creates a new unnamed skin which inherits all properties of an
  existing \meta{base skin} plus the given \meta{options}.
  This skin is set as \refKey{/tcb/skin last}.\\
  See a detailed example on page~\pageref{freeboxexample}.
\end{docTcbKey}


\clearpage
\subsection{Drawing Scheme}\label{subsec:tcolorboxdrawing}

\bgroup

Depending on the complexity of a |tcolorbox| definition, the resulting box
is drawn in a more or less complex series of steps.

To document and demonstrate these drawing steps, we consider the
following box definition:

\begin{dispListing}
\newtcolorbox{testbox}[1][]{enhanced,title=Test Box,
  boxrule=1mm,titlerule=0.5mm,colframe=blue!50!black,
  interior style={top color=blue!20!green!50!white,bottom color=blue!20!yellow!50!white},
  colbacktitle=blue!50!green!90!white,segmentation style={solid},
  fonttitle=\bfseries,drop fuzzy shadow,borderline={0.3mm}{0.35mm}{yellow!50!white},
  underlay={\path[fill image opacity=0.15,fill image scale=0.9,
    fill stretch picture={\draw[blue,line width=2mm] circle (1);}]
    (interior.south west) rectangle (interior.north east);},
  watermark text={Watermark},watermark color={green!20!white},
  finish={\begin{tcbclipframe}
    \path[bottom color=black,top color=black!50!white,opacity=0.1]
      (frame.south west) -- (frame.south east) -- (frame.north east) -- cycle;
    \path[top color=white,bottom color=black!50!white,opacity=0.1]
      (frame.south west) -- (frame.north east) -- (frame.north west) -- cycle;
    \end{tcbclipframe}},#1}
\end{dispListing}
\tcbusetemp

For this definition, we get the maximal number of drawing steps:

\newtcolorbox{itembox}{oversize,enhanced,frame empty,segmentation empty,colback=white,middle=2mm,boxsep=0pt,
  fuzzy halo=0.5mm with blue!15!white,
  before lower=\raggedright\begin{itemize},after lower=\end{itemize}}

\tcbset{finishcomment/.style={
  finish={\node[fill=white,draw=black!50!white,rounded corners] at (frame.center) {#1};}
}}


\begin{itembox}
\begin{testbox}[frame empty,interior titled empty,title empty,segmentation empty,
  no shadow,no borderline,no underlay,no overlay,no finish,opacityupper=0,opacitylower=0,opacitytitle=0,
  drop fuzzy shadow=red!75!white,
  finishcomment={1.~shadow}]
\lipsum[2]
\tcblower
Lower part
\end{testbox}
\tcblower
\item Section~\ref{subsec:shadows} on page \pageref{subsec:shadows}.
\end{itembox}

\begin{itembox}
\begin{testbox}[interior titled empty,title empty,segmentation empty,
  no borderline,no underlay,no overlay,no finish,opacityupper=0,opacitylower=0,opacitytitle=0,
  colframe=red!75!white,
  finishcomment={2.~frame}]
\lipsum[2]
\tcblower
Lower part
\end{testbox}
\tcblower
\item\refKey{/tcb/colframe}, \refKey{/tcb/opacityframe}
\item\refKey{/tcb/frame code}
\item\refKey{/tcb/frame style}
\end{itembox}


\begin{itembox}
\begin{testbox}[title empty,segmentation empty,
  no borderline,no underlay,no overlay,no finish,opacityupper=0,opacitylower=0,opacitytitle=0,
  colback=red!75!white,
  finishcomment={3.~interior}]
\lipsum[2]
\tcblower
Lower part
\end{testbox}
\tcblower
\item\refKey{/tcb/colback}, \refKey{/tcb/opacityback}
\item\refKey{/tcb/interior code}, \refKey{/tcb/interior titled code}
\item\refKey{/tcb/interior style}
\end{itembox}


\begin{itembox}
\begin{testbox}[segmentation empty,
  no borderline,no underlay,no overlay,no finish,opacityupper=0,opacitylower=0,opacitytitle=0,
  colbacktitle=red!75!white,
  finishcomment={4.~title area}]
\lipsum[2]
\tcblower
Lower part
\end{testbox}
\tcblower
\item\refKey{/tcb/colbacktitle}, \refKey{/tcb/opacitybacktitle}
\item\refKey{/tcb/title code}
\item\refKey{/tcb/title style}
\end{itembox}


\begin{itembox}
\begin{testbox}[
  no borderline,no underlay,no overlay,no finish,opacityupper=0,opacitylower=0,opacitytitle=0,
  segmentation style={solid,red!75!white},
  finishcomment={5.~segmentation}]
\lipsum[2]
\tcblower
Lower part
\end{testbox}
\tcblower
\item\refKey{/tcb/lower separated}
\item\refKey{/tcb/segmentation code}
\item\refKey{/tcb/segmentation style}
\end{itembox}


\begin{itembox}
\begin{testbox}[
  no borderline,no underlay,no overlay,no finish,opacityupper=0,opacitylower=0,opacitytitle=0,
  borderline={0.3mm}{0.35mm}{red!75!white},
  finishcomment={6.~border line}]
\lipsum[2]
\tcblower
Lower part
\end{testbox}
\tcblower
\item\zvref{subsec:borderline}
\end{itembox}


\begin{itembox}
\begin{testbox}[
  no underlay,no overlay,no finish,opacityupper=0,opacitylower=0,opacitytitle=0,
  underlay={\path[opacity=0.15,fill image scale=0.9,
    fill stretch picture={\draw[red,line width=2mm] circle (1);}]
    (interior.south west) rectangle (interior.north east);},
  finishcomment={7.~underlay}]
\lipsum[2]
\tcblower
Lower part
\end{testbox}
\tcblower
\item\zvref{subsec:skinboxedtitle}
\item\zvref{subsec:skinunderlay}
\end{itembox}


\begin{itembox}
\begin{testbox}[
  no finish,opacityupper=0,opacitylower=0,opacitytitle=0,
  watermark color={red!75!white},
  finishcomment={8.~overlay}]
\lipsum[2]
\tcblower
Lower part
\end{testbox}
\tcblower
\item\zvref{subsec:overlays}
\item\zvref{subsec:watermarks}
\end{itembox}


\begin{itembox}
\begin{testbox}[
  no finish,
  colupper=red!75!white,coltitle=red!75!white,collower=red!75!white,
  finishcomment={9.~text content}]
\lipsum[2]
\tcblower
Lower part
\end{testbox}
\tcblower
\item\refKey{/tcb/colupper}, \refKey{/tcb/collower}, \refKey{/tcb/coltitle}
\item\refKey{/tcb/fontupper}, \refKey{/tcb/fontlower}, \refKey{/tcb/fonttitle}
\item\refKey{/tcb/opacityupper}, \refKey{/tcb/opacitylower}, \refKey{/tcb/opacitytitle}
\end{itembox}


\begin{itembox}
\begin{testbox}[
  finishcomment={10.~finish}]
\lipsum[2]
\tcblower
Lower part
\end{testbox}
\tcblower
\item\zvref{subsec:skinfinish}
\end{itembox}

All together, the box is drawn:
\begin{dispExample}
% \usepackage{lipsum}
\begin{testbox}
\lipsum[2]
\tcblower
Lower part
\end{testbox}
\end{dispExample}

\egroup


\clearpage
\subsection{Color Names}\label{subsec:tech_colornames}
Color settings for a |tcolorbox| are saved into named colors which may be
used inside a box, e.\,g.\ for an overlay.
\tcbdocmarginnote{\tcbdocnew{2019-01-18}}
These color names are
\begin{itemize}
\item\docColor{tcbcolframe} set by \refKey{/tcb/colframe} (frame color)
\item\docColor{tcbcolback} set by \refKey{/tcb/colback} (background color)
\item\docColor{tcbcolbacktitle} set by \refKey{/tcb/colbacktitle} (background color of the title)
\item\docColor{tcbcolbacklower} set by \refKey{/tcb/colbacklower} (skin dependent background color
  of the lower part; needs \mylib{skins} to be loaded)
\item\docColor{tcbcolupper} set by \refKey{/tcb/colupper} (text color upper part)
\item\docColor{tcbcollower} set by \refKey{/tcb/collower} (text color lower part)
\item\docColor{tcbcoltitle} set by \refKey{/tcb/coltitle} (text color title)
\end{itemize}

\medskip

\begin{dispExample}
% \tcbuselibrary{skins}
\begin{tcolorbox}[title=Color names,
    colframe=blue!50!black,colback=blue!5,
    colbacktitle=blue!50,colupper=red!35!black]
  \foreach \name in {tcbcolframe,tcbcolback,tcbcolbacktitle,tcbcolbacklower,
    tcbcolupper,tcbcollower,tcbcoltitle}
  {\tikz\path[draw,fill=\name]
    (0,0) rectangle node[right=4mm,font=\ttfamily]{\name} (0.8,0.8);\par}
\end{tcolorbox}
\end{dispExample}



\clearpage
\subsection{Useful Properties}\label{subsec:tech_properties}

The following macros describe certain \textit{properties} which may be
used for the drawing scheme, see \zvref{subsec:tcolorboxdrawing}.
Sometimes, they are even available inside the box content. All of them are
considered to be \textit{read-only} and should never be redefined by the user.

\begin{docCommand}[doc new=2016-02-16]{tcbheightspace}{}
  If the height of a |tcolorbox| is not the natural height, the space
  difference between the forced and the natural size is hold by
  \refCom{tcbheightspace}.
  This macro is not usable inside the box content, but for skins or
  inside \refKey{/tcb/underlay}, \refKey{/tcb/overlay}, etc.

  If such a space information is needed inside the box content, see
  \refKey{/tcb/space to} instead.
\begin{dispExample}
% \tcbuselibrary{skins}
\newtcolorbox{testbox}[2][]{enhanced,size=fbox,
  colframe=blue!75!black,colback=white,height=#2,
  underlay={\node[above,inner sep=3pt] at (interior.south){%
    \includegraphics[width=\tcbtextwidth,height=\tcbheightspace-3pt]{goldshade.png}};
    },
  #1}
\begin{testbox}{3cm}
  This is my box. The space is filled with a picture.
\end{testbox}
\begin{testbox}{2cm}
  This is my box. The space is filled with a picture.
\end{testbox}
\end{dispExample}
\end{docCommand}

\enlargethispage*{1cm}

\begin{docCommand}[doc new=2016-02-16]{tcbtextwidth}{}
  This property describes the box content width.
  \begin{itemize}
  \item  If there also is a lower part, it describes the width of the upper part.
  \item For \refKey{/tcb/sidebyside} boxes, it describes the combined text
    width plus segmentation.
  \item This property can be used inside the box content text with exception
    of \refKey{/tcb/fit} boxes.
  \item \refCom{tcbtextwidth} can be used for all box types for skins or
    inside \refKey{/tcb/underlay}, \refKey{/tcb/overlay}, etc.
  \end{itemize}
\begin{dispExample}
\begin{tcolorbox}[colframe=blue!75!black]
  Inside a box: \tcbtextwidth\ (=\the\linewidth).
\end{tcolorbox}
\end{dispExample}
\end{docCommand}

\pagebreak
\begin{docCommand}[doc new=2016-02-16]{tcbtextheight}{}
  This property describes the designated box content height. If the
  box is larger than the natural height, the actual content will be smaller
  than \refCom{tcbtextheight}.
  \begin{itemize}
  \item For boxes with a fixed \refKey{/tcb/height}, this property can
    be used inside the box content text. For other boxes, it denotes |0pt|
    inside the box content.
  \item \refCom{tcbtextheight} can be used for all box types for skins or
    inside \refKey{/tcb/underlay}, \refKey{/tcb/overlay}, etc.
  \end{itemize}

\begin{dispExample}
% \tcbuselibrary{skins}
\begin{tcolorbox}[enhanced,colframe=blue!75!black,
    underlay={\node[left,red] at (frame.east) {Here: \tcbtextheight};}]
  Inside a box with natural height: \tcbtextheight.
\end{tcolorbox}
\begin{tcolorbox}[enhanced,colframe=blue!75!black,height=1cm,
    underlay={\node[left,red] at (frame.east) {Here: \tcbtextheight};}]
  Inside a box with fixed height: \tcbtextheight.
\end{tcolorbox}
\end{dispExample}
\end{docCommand}


\begin{docCommand}[doc new=2017-04-25]{tcbsegmentstate}{}
  This macro contains |0|, |1|, or |2|. It is set for every unbroken box and every broken partial box
  with the following meaning:
  \begin{itemize}
    \item\docValue{0}: The current (partial) box contains only an upper part.
    \item\docValue{1}: The current (partial) box contains an upper and a lower part.
      The segmentation node can be used for positioning.
    \item\docValue{2}: The current (partial) box contains only a lower part.
      This can only be true for parts of breakable boxes.
  \end{itemize}
  Skins like \refSkin{bicolor} use this property to paint the (partial) boxes.
\begin{dispExample}
% \tcbuselibrary{skins,raster}
\begin{tcbraster}[raster equal height,enhanced,
    watermark text=\tcbsegmentstate]
  \begin{tcolorbox}Upper part\end{tcolorbox}
  \begin{tcolorbox}Upper part\tcblower Lower part\end{tcolorbox}
\end{tcbraster}
\end{dispExample}
\end{docCommand}
