% !TeX root = tcolorbox.tex
% include file of tcolorbox.tex (manual of the LaTeX package tcolorbox)
\clearpage
\section{Library \mylib{skins} - Catalog of Skins}\label{sec:skincatalog}%
\tcbset{external/prefix=external/skincatalog_}%
The \mylib{skins} library provides a catalog of skins to choose from which
is documented in the following. The \mylib{skins} library has to be loaded
by a package option or inside the preamble by:
\begin{dispListing}
\tcbuselibrary{skins}
\end{dispListing}

See \zvref{sec:skins} for the documentation of all other options of the \mylib{skins} library.

\begin{itemize}
\item In principle, a skin is applied by choosing a value for
  \refKey{/tcb/skin}, e.\,g.\ \docValue*{enhanced}.
  Since the parts of a breakable box should look different,
  there are individual skins for breakable boxes, also see \zvref{subsec:breaksequence}.
  Skins for breakable boxes derived from a base skin are called a skin family
  in the following.
\item Instead of setting values for \refKey{/tcb/skin}, equally named options
  can be used which are shortcuts and which sometimes also change some
  geometry or style settings. These are the intended options for normal users.
  Typically, one of the following options is sufficient to select a skin:
  \begin{itemize}
  \item \refKey{/tcb/standard}
  \item \refKey{/tcb/standard jigsaw}
  \item \refKey{/tcb/enhanced}
  \item \refKey{/tcb/enhanced jigsaw}
  \item \refKey{/tcb/enhanced standard}
  \item \refKey{/tcb/enhanced standard jigsaw}
  \item \refKey{/tcb/bicolor}
  \item \refKey{/tcb/tile}
  \item \refKey{/tcb/beamer}
  \item \refKey{/tcb/widget}
  \item \refKey{/tcb/empty}
  \item \refKey{/tcb/spartan}
  \item \refKey{/tcb/draft}
  \end{itemize}
  Additionally, there are some special applications:
  \begin{itemize}
  \item \refKey{/tcb/marker}
  \item \refKey{/tcb/blank}
  \item \refKey{/tcb/blanker}
  \item \refKey{/tcb/blankest}
  \end{itemize}
\end{itemize}



\clearpage

The auxiliary macro \docAuxCommand{skinExampleSet} is used for the
following examples to display skin applications. Note that
\docAuxCommand{skinExampleSet} is not part of the package, but is
defined just for this documentation.

\begin{dispListing}
\NewDocumentCommand{\skinExampleSet}{m}{%
  \begin{tcbraster}[raster equal height,raster columns=3,
      colback=LightGreen,colframe=DarkGreen,colbacktitle=LimeGreen!75!DarkGreen,
      #1,
      left=1mm,right=1mm,top=1mm,bottom=1mm,middle=1mm,
      sidebyside gap=4mm]
    \begin{tcolorbox}
      This is my content.
    \end{tcolorbox}
    \begin{tcolorbox}
      This is my content.
      \tcblower
      More content.
    \end{tcolorbox}
    \begin{tcolorbox}[sidebyside]
      My content.
      \tcblower
      More content.
    \end{tcolorbox}
    \begin{tcolorbox}[adjusted title=My title]
      This is my content.
    \end{tcolorbox}
    \begin{tcolorbox}[adjusted title=My title]
      This is my content.
      \tcblower
      More content.
    \end{tcolorbox}
    \begin{tcolorbox}[adjusted title=My title,sidebyside]
      My content.
      \tcblower
      More content.
    \end{tcolorbox}
  \end{tcbraster}
}
\end{dispListing}
\tcbusetemp


\clearpage
\tcbset{skintable/.style={colframe=red!50!yellow!50!black,
  colback=red!50!yellow!5!white,coltitle=red!50!yellow!3!white,
  fonttitle=\bfseries,before=\par\smallskip,
  title=Environment and engines for the skin \enquote{\texttt{#1}}}}

\subsection{Skin Family \enquote{standard}}\label{subsec:skinstandard}
\begin{marker}Note that the option keys \refKey{/tcb/frame style},
  \refKey{/tcb/interior style},
  \refKey{/tcb/segmentation style}, and
  \refKey{/tcb/title style} are not be applicable to the standard skin.
  Also, watermarks (see Subsection \ref{subsec:watermarks})
  are not usable with the standard skin.
\end{marker}

\begin{docSkin}{standard}
  This is the standard skin from the core package. All drawing engines
  are set to type |standard|. The drawing is based on \refPkg{pgf} commands and
  does not need the \refPkg{tikz} package.
\begin{tcolorbox}[skintable=standard]
  \begin{tabbing}
    \refKey{/tcb/interior titled engine}: \=\kill
    \refKey{/tcb/frame engine}:           \> |standard|\\
    \refKey{/tcb/interior titled engine}: \> |standard|\\ 
    \refKey{/tcb/interior engine}:        \> |standard|\\
    \refKey{/tcb/segmentation engine}:    \> |standard|\\
    \refKey{/tcb/title engine}:           \> |standard|
  \end{tabbing}
\end{tcolorbox}
\end{docSkin}

\begin{docTcbKey}{standard}{}{style, no value}
  This is an abbreviation for setting |skin=standard|.
\end{docTcbKey}

\begin{dispExample}
\skinExampleSet{standard}
\end{dispExample}

\clearpage

\begin{docSkin}{standard jigsaw}
  This is the standard jigsaw skin from the core package. It differs from
  the skin \refSkin{standard} by its frame engine, see \zvref{subsec:skinjigsaw}.
\begin{tcolorbox}[skintable=standard jigsaw]
  \begin{tabbing}
    \refKey{/tcb/interior titled engine}: \=\kill
    \refKey{/tcb/frame engine}:           \> |standardjigsaw|\\
    \refKey{/tcb/interior titled engine}: \> |standard|\\ 
    \refKey{/tcb/interior engine}:        \> |standard|\\
    \refKey{/tcb/segmentation engine}:    \> |standard|\\
    \refKey{/tcb/title engine}:           \> |standard|
  \end{tabbing}
\end{tcolorbox}
\end{docSkin}

\begin{docTcbKey}{standard jigsaw}{}{style, no value}
  This is an abbreviation for setting |skin=standard jigsaw|.
\end{docTcbKey}

\begin{dispExample*}{segmentation style={preaction={fill=white},pattern=checkerboard,pattern color=gray!40}}
\skinExampleSet{standard jigsaw,
  opacityframe=0.5,opacityback=0.5,opacitybacktitle=0.5,
}
\end{dispExample*}


\clearpage
\subsection{Skin Family \enquote{enhanced}}
\begin{marker}
If you like the standard appearance of a |tcolorbox| but you want to
have some \enquote{enhanced} features, the |enhanced| skin is what you are looking for.
\end{marker}

\begin{docSkin}{enhanced}
  This skin translates the drawing commands of the core package into \tikzname\ path commands. Therefore, it allows all \tikzname\ high level options for
  these paths and has more flexibility compared to the \refSkin{standard} skin.
  You pay for this with some prolonged compilation time.
  The \tikzname\ path options can
  be given with the option keys
  \refKey{/tcb/frame style},
  \refKey{/tcb/interior style},
  \refKey{/tcb/segmentation style}, and
  \refKey{/tcb/title style}.
\begin{tcolorbox}[skintable=enhanced]
  \begin{tabbing}
    \refKey{/tcb/interior titled engine}: \=\kill
    \refKey{/tcb/frame engine}:           \> |path|\\
    \refKey{/tcb/interior titled engine}: \> |path|\\ 
    \refKey{/tcb/interior engine}:        \> |path|\\
    \refKey{/tcb/segmentation engine}:    \> |path|\\
    \refKey{/tcb/title engine}:           \> |path|
  \end{tabbing}
\end{tcolorbox}
\end{docSkin}


\begin{docTcbKey}{enhanced}{}{style, no value}
  This is an abbreviation for setting |skin=enhanced|.
\end{docTcbKey}

\begin{dispExample}
\skinExampleSet{enhanced}
\end{dispExample}

\clearpage

\begin{dispExample}
% \usetikzlibrary{shadings}         % preamble
\tcbset{skin=enhanced,fonttitle=\bfseries,
  frame style={upper left=blue,upper right=red,lower left=yellow,lower right=green},
  interior style={white,opacity=0.5},
  segmentation style={black,solid,opacity=0.2,line width=1pt}}

\begin{tcolorbox}[title=Nice box in rainbow colors]
  With the \enquote{enhanced} skin, it is quite easy to produce fancy looking effects.
  \tcblower
  Note that this is still a \texttt{tcolorbox}.
\end{tcolorbox}
\end{dispExample}


\begin{dispExample}
% \usetikzlibrary{decorations.pathmorphing} % preamble
\tcbset{skin=enhanced,fonttitle=\bfseries,boxrule=1mm,
  frame style={draw=FireBrick,fill=Salmon},drop fuzzy shadow,
  interior style={draw=FireBrick,top color=Salmon!10,bottom color=Salmon!20},
  segmentation style={draw=FireBrick,solid,decorate,
        decoration={coil,aspect=0,segment length=10.1mm}}}

\begin{tcblisting}{title=A listing box with shadow and some specials}
Of course, skins can be used for listings also.
\begin{equation}
  \int\limits_1^2 \frac{1}{x}~dx = \ln(2).
\end{equation}
\end{tcblisting}
\end{dispExample}


\clearpage


\begin{docTcbKey}{enhanced standard}{}{style, no value}
  This style sets the \refSkin{enhanced} skin of an unbroken box
  for all boxes, including unbreakable
  boxes and every box of a \emph{break sequence}. Thus, it is another variation
  of the \refSkin{standard} skin, see \zvref{subsec:breaksequence}.
\end{docTcbKey}


\begin{docTcbKey}{blank}{}{style, initially unset}
  This style relies on the skin \refSkin{enhanced}. All drawing operations
  are hidden and all margins are set to |0pt|. See \refKey{/tcb/blanker}
  for switching off the drawing engines.
\begin{dispExample}
\begin{tcolorbox}[blank,watermark text=A blank box]
\lipsum[1]
\end{tcolorbox}
\end{dispExample}
\end{docTcbKey}

\clearpage
\begin{docCommand}{tcbline}{}
  Sometimes, a line is only a line. With \refCom{tcblower} you separate
  the box content into two functional units. |\tcbline| draws only a line
  which looks like the segmentation line between upper and lower part.
  Furthermore, you can use |\tcbline| more than just once.
  |\tcbline| always uses the |path| drawing engine. Therefore,
  the \refKey{/tcb/segmentation style} can be applied.

\begin{dispExample}
\tcbset{enhanced,colframe=blue!50!black,colback=white}

\begin{tcolorbox}[colupper=red!50!black,collower=green!50!black]
\lipsum[1]
\tcbline
\lipsum[2]
\tcblower
\lipsum[3]
\tcbline
\lipsum[4]
\end{tcolorbox}
\end{dispExample}
\end{docCommand}

\begin{docCommand}{tcbline*}{}
  Equivalent to \refCom{tcbline}, but in a breakable box, \refCom{tcbline*}
  is removed if at a page/box break. Also, it is removed at the end
  of a box.
\end{docCommand}

\clearpage
\begin{docSkin}{enhancedfirst}
This is a flavor of \refSkin{enhanced} which is used as a \emph{first} part
in a break sequence for \refSkin{enhanced}.
Nevertheless, this skin can be applied independently.
\begin{tcolorbox}[skintable=enhancedfirst]
  \begin{tabbing}
    \refKey{/tcb/interior titled engine}: \=\kill
    \refKey{/tcb/frame engine}:           \> |pathfirst|\\
    \refKey{/tcb/interior titled engine}: \> |pathfirst|\\ 
    \refKey{/tcb/interior engine}:        \> |pathfirst|\\
    \refKey{/tcb/segmentation engine}:    \> |path|\\
    \refKey{/tcb/title engine}:           \> |pathfirst|
  \end{tabbing}
\end{tcolorbox}
\end{docSkin}


\begin{dispExample}
\skinExampleSet{skin=enhancedfirst}
\end{dispExample}

\medskip

%\clearpage
\begin{docSkin}{enhancedmiddle}
This is a flavor of \refSkin{enhanced} which is used as a \emph{middle} part
in a break sequence for \refSkin{enhanced}.
Nevertheless, this skin can be applied independently.
\begin{tcolorbox}[skintable=enhancedmiddle]
  \begin{tabbing}
    \refKey{/tcb/interior titled engine}: \=\kill
    \refKey{/tcb/frame engine}:           \> |pathmiddle|\\
    \refKey{/tcb/interior titled engine}: \> |pathmiddle|\\ 
    \refKey{/tcb/interior engine}:        \> |pathmiddle|\\
    \refKey{/tcb/segmentation engine}:    \> |path|\\
    \refKey{/tcb/title engine}:           \> |pathmiddle|
  \end{tabbing}
\end{tcolorbox}
\end{docSkin}


\begin{dispExample}
\skinExampleSet{skin=enhancedmiddle}
\end{dispExample}


\clearpage
\begin{docSkin}{enhancedlast}
This is a flavor of \refSkin{enhanced} which is used as a \emph{last} part
in a break sequence for \refSkin{enhanced}.
Nevertheless, this skin can be applied independently.
\begin{tcolorbox}[skintable=enhancedlast]
  \begin{tabbing}
    \refKey{/tcb/interior titled engine}: \=\kill
    \refKey{/tcb/frame engine}:           \> |pathlast|\\
    \refKey{/tcb/interior titled engine}: \> |pathlast|\\ 
    \refKey{/tcb/interior engine}:        \> |pathlast|\\
    \refKey{/tcb/segmentation engine}:    \> |path|\\
    \refKey{/tcb/title engine}:           \> |pathlast|
  \end{tabbing}
\end{tcolorbox}
\end{docSkin}

\begin{dispExample}
\skinExampleSet{skin=enhancedlast}
\end{dispExample}

\clearpage

\begin{docSkin}{enhanced jigsaw}
  This is the jigsaw variant of skin \refSkin{enhanced}.
  It differs by its frame engine, see \zvref{subsec:skinjigsaw}.
\begin{tcolorbox}[skintable=enhanced jigsaw]
  \begin{tabbing}
    \refKey{/tcb/interior titled engine}: \=\kill
    \refKey{/tcb/frame engine}:           \> |pathjigsaw|\\
    \refKey{/tcb/interior titled engine}: \> |path|\\ 
    \refKey{/tcb/interior engine}:        \> |path|\\
    \refKey{/tcb/segmentation engine}:    \> |path|\\
    \refKey{/tcb/title engine}:           \> |path|
  \end{tabbing}
\end{tcolorbox}
\end{docSkin}

\begin{docTcbKey}{enhanced jigsaw}{}{style, no value}
  This is an abbreviation for setting |skin=enhanced jigsaw|.
\end{docTcbKey}


\begin{dispExample*}{segmentation style={preaction={fill=white},pattern=checkerboard,pattern color=gray!40}}
\skinExampleSet{enhanced jigsaw,
  opacityframe=0.5,opacityback=0.5,opacitybacktitle=0.5,
}
\end{dispExample*}


\begin{docTcbKey}[][doc new=2017-07-01]{enhanced standard jigsaw}{}{style, no value}
  This style sets the \refSkin{enhanced jigsaw} skin of an unbroken box
  for all boxes, including unbreakable
  boxes and every box of a \emph{break sequence}. Thus, it is another variation
  of the \refSkin{standard} skin, see \zvref{subsec:breaksequence}.
\end{docTcbKey}


\clearpage
\begin{docSkin}{enhancedfirst jigsaw}
  This is the jigsaw variant of skin \refSkin{enhancedfirst}.
  It differs by its frame engine, see \zvref{subsec:skinjigsaw}.
\begin{tcolorbox}[skintable=enhancedfirst jigsaw]
  \begin{tabbing}
    \refKey{/tcb/interior titled engine}: \=\kill
    \refKey{/tcb/frame engine}:           \> |pathfirstjigsaw|\\
    \refKey{/tcb/interior titled engine}: \> |pathfirst|\\ 
    \refKey{/tcb/interior engine}:        \> |pathfirst|\\
    \refKey{/tcb/segmentation engine}:    \> |path|\\
    \refKey{/tcb/title engine}:           \> |pathfirst|
  \end{tabbing}
\end{tcolorbox}
\end{docSkin}


\begin{dispExample*}{segmentation style={preaction={fill=white},pattern=checkerboard,pattern color=gray!40}}
\skinExampleSet{skin=enhancedfirst jigsaw,
  opacityframe=0.5,opacityback=0.5,opacitybacktitle=0.5,
}
\end{dispExample*}


\clearpage
\begin{docSkin}{enhancedmiddle jigsaw}
  This is the jigsaw variant of skin \refSkin{enhancedmiddle}.
  It differs by its frame engine, see \zvref{subsec:skinjigsaw}.
\begin{tcolorbox}[skintable=enhancedmiddle jigsaw]
  \begin{tabbing}
    \refKey{/tcb/interior titled engine}: \=\kill
    \refKey{/tcb/frame engine}:           \> |pathmiddlejigsaw|\\
    \refKey{/tcb/interior titled engine}: \> |pathmiddle|\\ 
    \refKey{/tcb/interior engine}:        \> |pathmiddle|\\
    \refKey{/tcb/segmentation engine}:    \> |path|\\
    \refKey{/tcb/title engine}:           \> |pathmiddle|
  \end{tabbing}
\end{tcolorbox}
\end{docSkin}


\begin{dispExample*}{segmentation style={preaction={fill=white},pattern=checkerboard,pattern color=gray!40}}
\skinExampleSet{skin=enhancedmiddle jigsaw,
  opacityframe=0.5,opacityback=0.5,opacitybacktitle=0.5,
}
\end{dispExample*}


\begin{docTcbKey}{marker}{}{style, no value}
  This styles relies on the skin \refSkin{enhancedmiddle jigsaw}. It is
  intended to be used as an optical marker like a highlighter pen.
\begin{dispExample}
\begin{tcolorbox}[marker]
\lipsum[2]
\end{tcolorbox}
\end{dispExample}
\end{docTcbKey}

\clearpage

\begin{dispListing*}{before upper={This examples demonstrates the creation of several
  \emph{text marker} environments based on \refSkin{enhancedmiddle}.\par\medskip}}
\tcbset{textmarker/.style={%
    skin=enhancedmiddle jigsaw,breakable,parbox=false,
    boxrule=0mm,leftrule=5mm,rightrule=5mm,boxsep=0mm,arc=0mm,outer arc=0mm,
    left=3mm,right=3mm,top=1mm,bottom=1mm,toptitle=1mm,bottomtitle=1mm,oversize}}

\newtcolorbox{yellow}{textmarker,colback=yellow!5!white,colframe=yellow}
\newtcolorbox{orange}{textmarker,colback=DarkOrange!5!white,
                        colframe=DarkOrange!75!yellow}
\newtcolorbox{red}{textmarker,colback=red!5!white,colframe=red}
\newtcolorbox{blue}{textmarker,colback=DeepSkyBlue!5!white,colframe=DeepSkyBlue}
\newtcolorbox{green}{textmarker,colback=Chartreuse!5!white,colframe=Chartreuse}
\newtcolorbox{rainbow}{textmarker,interior hidden,
  frame style={top color=blue,bottom color=red,middle color=green}}

\begin{yellow}
  \lipsum[1-3]
\end{yellow}

\begin{orange}
  \lipsum[4]
\end{orange}

\begin{red}
  \lipsum[5]
\end{red}

\begin{green}
  \lipsum[6]
\end{green}

\begin{blue}
  \lipsum[7]
\end{blue}

\begin{rainbow}
  \lipsum[8]
\end{rainbow}
\end{dispListing*}
{\tcbusetemp}


\clearpage
\begin{docSkin}{enhancedlast jigsaw}
  This is the jigsaw variant of skin \refSkin{enhancedlast}.
  It differs by its frame engine, see \zvref{subsec:skinjigsaw}.
\begin{tcolorbox}[skintable=enhancedlast]
  \begin{tabbing}
    \refKey{/tcb/interior titled engine}: \=\kill
    \refKey{/tcb/frame engine}:           \> |pathlastjigsaw|\\
    \refKey{/tcb/interior titled engine}: \> |pathlast|\\ 
    \refKey{/tcb/interior engine}:        \> |pathlast|\\
    \refKey{/tcb/segmentation engine}:    \> |path|\\
    \refKey{/tcb/title engine}:           \> |pathlast|
  \end{tabbing}
\end{tcolorbox}
\end{docSkin}


\begin{dispExample*}{segmentation style={preaction={fill=white},pattern=checkerboard,pattern color=gray!40}}
\skinExampleSet{skin=enhancedlast jigsaw,
  opacityframe=0.5,opacityback=0.5,opacitybacktitle=0.5,
}
\end{dispExample*}



\clearpage
\subsection{Skin Family \enquote{bicolor}}
\begin{docSkin}{bicolor}
  This skin is quite similar to the \refSkin{standard} and \refSkin{enhanced} skin.
  But instead of a segmentation line, the optional lower part of the box is filled with a
  different color or drawn with a different style.
\begin{tcolorbox}[skintable=bicolor]
  \begin{tabbing}
    \refKey{/tcb/interior titled engine}: \=\kill
    \refKey{/tcb/frame engine}:           \> |path|\\
    \refKey{/tcb/interior titled engine}: \> \emph{special}\\ 
    \refKey{/tcb/interior engine}:        \> \emph{special}\\
    \refKey{/tcb/segmentation engine}:    \> \emph{special}\\
    \refKey{/tcb/title engine}:           \> |path|
  \end{tabbing}
\end{tcolorbox}
  \begin{itemize}
  \item The most basic usage of this skin is to set the background color of
    the lower part by \refKey{/tcb/colbacklower} and all other options like for
    the \refSkin{standard} skin.
\begin{dispExample}
\begin{tcolorbox}[skin=bicolor,title=The title,
    colframe=FireBrick!75!black,colback=Salmon!50!white,colbacklower=Salmon]
  The upper part.
  \tcblower
  The lower part.
\end{tcolorbox}
\end{dispExample}
  \item The more advanced usage of this skin is to apply the \refKey{/tcb/frame style}
    and the \refKey{/tcb/interior style} like for
    the \refSkin{enhanced} skin. Also, the \refKey{/tcb/segmentation style} can be
    used, but it is applied to the whole lower part.
\begin{dispExample}
\begin{tcolorbox}[skin=bicolor,title=The title,
    frame style={top color=FireBrick,
                 bottom color=FireBrick!15!white,draw=black},
    interior style={left color=Salmon,right color=Salmon!50!white},
    segmentation style={right color=Salmon,left color=Salmon!50!white}]
  The upper part.
  \tcblower
  The lower part.
\end{tcolorbox}
\end{dispExample}
  \end{itemize}
\end{docSkin}

\clearpage

\begin{docTcbKey}{bicolor}{}{style, no value}
  This is an abbreviation for setting |skin=bicolor|.
\end{docTcbKey}


\begin{dispExample}
\skinExampleSet{bicolor,
  colbacklower=LimeGreen!75!LightGreen,
}
\end{dispExample}

\clearpage


\begin{marker}
  The following options \refKey{/tcb/colbacklower} and \refKey{/tcb/opacitybacklower}
  are executed before \refKey{/tcb/segmentation style}, i.\,e.,
  \refKey{/tcb/segmentation style} overrules them.
\end{marker}

\begin{docTcbKey}{colbacklower}{=\meta{color}}{no default, initially \texttt{black!15!white}}
  Sets the background \meta{color} of the lower part. It depends on the skin,
  if this value is used.
\end{docTcbKey}

\begin{dispExample}
\tcbset{gitexample/.style={listing and comment,comment={#1},
  skin=bicolor,boxrule=1mm,fonttitle=\bfseries,coltitle=black,
  frame style={draw=black,left color=Gold,right color=Goldenrod!50!Gold},
  colback=black,colbacklower=Goldenrod!75!Gold,
  colupper=white,collower=black,
  listing options={language={bash},aboveskip=0pt,belowskip=0pt,nolol,
  basicstyle=\ttfamily\bfseries,extendedchars=true}}}

\begin{tcblisting}{title={Snapshot of the staging area},
  gitexample={The option `-a' automatically stages all tracked and modified
              files before the commit.\par
              This can be combined with the message option `-m'
              as seen in the third line.}}
git commit
git commit -a
git commit -am 'changes to my example'
\end{tcblisting}
\end{dispExample}

\smallskip

\begin{docTcbKey}[][doc new=2021-05-21]{opacitybacklower}{=\meta{fraction}}{no default, initially \texttt{1.0}}
  Sets the background opacity of the lower part to the given \meta{fraction}.
  It depends on the skin, if this value is used.
\end{docTcbKey}

\begin{dispExample}
\begin{tcolorbox}[bicolor,
  frame style={preaction={fill=blue!50!black},
    pattern=checkerboard,pattern color=blue!50!gray},
  fonttitle=\bfseries, overlaplower=0mm,
  colback=blue!10, colbacklower=white, opacitybacklower=0.65,
  title={Example for a semilucent lower part}]
This is the upper part.
\tcblower
And that is the lower part.
\end{tcolorbox}
\end{dispExample}

\clearpage

\begin{docTcbKey}[][doc new=2022-01-11]{overlaplower}{=\meta{length}}{no default, initially \texttt{0.1mm}}
  The backgrounds of the lower parts for the skin families \enquote{bicolor}, \enquote{tile}, and \enquote{beamer}
  are drawn differently than the backgrounds of the upper parts. If the distance between these
  backgrounds of upper and lower parts is |0pt|, some previewers show the
  frame color as thin line between upper and lower part. To avoid this glitch,
  the lower part is drawn with an overlap of \meta{length} over the upper part.\par
  This value can be adapted for special applications. For example,
  semilucent lower parts better use |0pt|, see \refKey{/tcb/opacitybacklower}.
  Also see \refCom{tcboverlaplower} for using a larger value.
\end{docTcbKey}


\begin{docCommand}[doc new=2022-01-11]{tcboverlaplower}{}
  Macro which contains the length value set by \refKey{/tcb/overlaplower}.
  May be used for fine positioning at the segmentation between upper and
  lower part and should be seen \emph{read-only}.

\begin{dispExample}
\begin{tcolorbox}[bicolor, sharp corners,
    colframe=blue!50!black, colback=blue!10, colbacklower=red!10,
    top=5mm, bottom=2mm, middle=3.5mm, overlaplower=1.5mm,
    underlay={
      \node[minimum width=1cm,minimum height=0.5cm,outer sep=auto,
        anchor=north east,fill=white] at (interior.north east)
        {\itshape\small upper};
      \node[minimum width=1cm,minimum height=0.5cm,outer sep=auto,
        anchor=north east,fill=white]
        at ([yshift=\tcboverlaplower]segmentation.east)
        {\itshape\small lower};
    }
  ]
This is the upper part.
\tcblower
And that is the lower part.
\end{tcolorbox}
\end{dispExample}

\end{docCommand}



\clearpage

\begin{docSkin}{bicolorfirst}
This is a flavor of \refSkin{bicolor} which is used as a \emph{first} part
in a break sequence for \refSkin{bicolor}.
Nevertheless, this skin can be applied independently.
\begin{tcolorbox}[skintable=bicolorfirst]
  \begin{tabbing}
    \refKey{/tcb/interior titled engine}: \=\kill
    \refKey{/tcb/frame engine}:           \> |pathfirst|\\
    \refKey{/tcb/interior titled engine}: \> \emph{special}\\ 
    \refKey{/tcb/interior engine}:        \> \emph{special}\\
    \refKey{/tcb/segmentation engine}:    \> \emph{special}\\
    \refKey{/tcb/title engine}:           \> |pathfirst|
  \end{tabbing}
\end{tcolorbox}
\end{docSkin}

\begin{dispExample}
\skinExampleSet{skin=bicolorfirst,
  colbacklower=LimeGreen!75!LightGreen,
}
\end{dispExample}


\clearpage

\begin{docSkin}{bicolormiddle}
This is a flavor of \refSkin{bicolor} which is used as a \emph{middle} part
in a break sequence for \refSkin{bicolor}.
Nevertheless, this skin can be applied independently.
\begin{tcolorbox}[skintable=bicolormiddle]
  \begin{tabbing}
    \refKey{/tcb/interior titled engine}: \=\kill
    \refKey{/tcb/frame engine}:           \> |pathmiddle|\\
    \refKey{/tcb/interior titled engine}: \> \emph{special}\\ 
    \refKey{/tcb/interior engine}:        \> \emph{special}\\
    \refKey{/tcb/segmentation engine}:    \> \emph{special}\\
    \refKey{/tcb/title engine}:           \> |pathmiddle|
  \end{tabbing}
\end{tcolorbox}
\end{docSkin}


\begin{dispExample}
\skinExampleSet{skin=bicolormiddle,
  colbacklower=LimeGreen!75!LightGreen,
}
\end{dispExample}


\clearpage
\begin{docSkin}{bicolorlast}
This is a flavor of \refSkin{bicolor} which is used as a \emph{last} part
in a break sequence for \refSkin{bicolor}.
Nevertheless, this skin can be applied independently.
\begin{tcolorbox}[skintable=bicolorlast]
  \begin{tabbing}
    \refKey{/tcb/interior titled engine}: \=\kill
    \refKey{/tcb/frame engine}:           \> |pathlast|\\
    \refKey{/tcb/interior titled engine}: \> \emph{special}\\ 
    \refKey{/tcb/interior engine}:        \> \emph{special}\\
    \refKey{/tcb/segmentation engine}:    \> \emph{special}\\
    \refKey{/tcb/title engine}:           \> |pathlast|
  \end{tabbing}
\end{tcolorbox}
\end{docSkin}


\begin{dispExample}
\skinExampleSet{skin=bicolorlast,
  colbacklower=LimeGreen!75!LightGreen,
}
\end{dispExample}


\clearpage

\begin{docSkin}[doc new=2021-05-21]{bicolor jigsaw}
  This is the jigsaw variant of skin \refSkin{bicolor}.
  It differs by its frame engine, see \zvref{subsec:skinjigsaw}.
\begin{tcolorbox}[skintable=bicolor jigsaw]
  \begin{tabbing}
    \refKey{/tcb/interior titled engine}: \=\kill
    \refKey{/tcb/frame engine}:           \> |pathjigsaw|\\
    \refKey{/tcb/interior titled engine}: \> \emph{special}\\ 
    \refKey{/tcb/interior engine}:        \> \emph{special}\\
    \refKey{/tcb/segmentation engine}:    \> \emph{special}\\
    \refKey{/tcb/title engine}:           \> |path|
  \end{tabbing}
\end{tcolorbox}
\end{docSkin}

\begin{docTcbKey}{bicolor jigsaw}{}{style, no value}
  This is an abbreviation for setting |skin=enhanced jigsaw|.
\end{docTcbKey}


\begin{dispExample*}{segmentation style={preaction={fill=white},pattern=checkerboard,pattern color=gray!40}}
\skinExampleSet{bicolor jigsaw,
  colbacklower=LimeGreen!75!LightGreen,
  opacityframe=0.5,opacityback=0.5,opacitybacktitle=0.5,
  opacitybacklower=0.5,
}
\end{dispExample*}


\clearpage


\begin{docSkin}[doc new=2021-05-21]{bicolorfirst jigsaw}
  This is the jigsaw variant of skin \refSkin{bicolorfirst}.
  It differs by its frame engine, see \zvref{subsec:skinjigsaw}.
\begin{tcolorbox}[skintable=bicolorfirst jigsaw]
  \begin{tabbing}
    \refKey{/tcb/interior titled engine}: \=\kill
    \refKey{/tcb/frame engine}:           \> |pathfirstjigsaw|\\
    \refKey{/tcb/interior titled engine}: \> \emph{special}\\ 
    \refKey{/tcb/interior engine}:        \> \emph{special}\\
    \refKey{/tcb/segmentation engine}:    \> \emph{special}\\
    \refKey{/tcb/title engine}:           \> |pathfirst|
  \end{tabbing}
\end{tcolorbox}
\end{docSkin}

\begin{dispExample*}{segmentation style={preaction={fill=white},pattern=checkerboard,pattern color=gray!40}}
\skinExampleSet{skin=bicolorfirst jigsaw,
  colbacklower=LimeGreen!75!LightGreen,
  opacityframe=0.5,opacityback=0.5,opacitybacktitle=0.5,
  opacitybacklower=0.5,
}
\end{dispExample*}



\clearpage

\begin{docSkin}[doc new=2021-05-21]{bicolormiddle jigsaw}
  This is the jigsaw variant of skin \refSkin{bicolormiddle}.
  It differs by its frame engine, see \zvref{subsec:skinjigsaw}.
\begin{tcolorbox}[skintable=bicolormiddle jigsaw]
  \begin{tabbing}
    \refKey{/tcb/interior titled engine}: \=\kill
    \refKey{/tcb/frame engine}:           \> |pathmiddlejigsaw|\\
    \refKey{/tcb/interior titled engine}: \> \emph{special}\\ 
    \refKey{/tcb/interior engine}:        \> \emph{special}\\
    \refKey{/tcb/segmentation engine}:    \> \emph{special}\\
    \refKey{/tcb/title engine}:           \> |pathmiddle|
  \end{tabbing}
\end{tcolorbox}
\end{docSkin}


\begin{dispExample*}{segmentation style={preaction={fill=white},pattern=checkerboard,pattern color=gray!40}}
\skinExampleSet{skin=bicolormiddle jigsaw,
  colbacklower=LimeGreen!75!LightGreen,
  opacityframe=0.5,opacityback=0.5,opacitybacktitle=0.5,
  opacitybacklower=0.5,
}
\end{dispExample*}


\clearpage
\begin{docSkin}[doc new=2021-05-21]{bicolorlast jigsaw}
  This is the jigsaw variant of skin \refSkin{bicolorlast}.
  It differs by its frame engine, see \zvref{subsec:skinjigsaw}.
\begin{tcolorbox}[skintable=bicolorlast jigsaw]
  \begin{tabbing}
    \refKey{/tcb/interior titled engine}: \=\kill
    \refKey{/tcb/frame engine}:           \> |pathlastjigsaw|\\
    \refKey{/tcb/interior titled engine}: \> \emph{special}\\ 
    \refKey{/tcb/interior engine}:        \> \emph{special}\\
    \refKey{/tcb/segmentation engine}:    \> \emph{special}\\
    \refKey{/tcb/title engine}:           \> |pathlast|
  \end{tabbing}
\end{tcolorbox}
\end{docSkin}


\begin{dispExample*}{segmentation style={preaction={fill=white},pattern=checkerboard,pattern color=gray!40}}
\skinExampleSet{skin=bicolorlast jigsaw,
  colbacklower=LimeGreen!75!LightGreen,
  opacityframe=0.5,opacityback=0.5,opacitybacktitle=0.5,
  opacitybacklower=0.5,
}
\end{dispExample*}



\clearpage
\subsection{Skin Family \enquote{tile}}
\begin{docSkin}[doc new=2016-02-25]{tile}
  This skin is a variant of skin \refSkin{bicolor}. Especially, the
  optional lower part of the box is colored by \refKey{/tcb/colbacklower}.
  The main difference to \refSkin{bicolor} is that \refSkin{tile} has no
  frame.
\begin{tcolorbox}[skintable=tile]
  \begin{tabbing}
    \refKey{/tcb/interior titled engine}: \=\kill
    \refKey{/tcb/frame engine}:           \> |empty|\\
    \refKey{/tcb/interior titled engine}: \> \emph{special}\\ 
    \refKey{/tcb/interior engine}:        \> \emph{special}\\
    \refKey{/tcb/segmentation engine}:    \> \emph{special}\\
    \refKey{/tcb/title engine}:           \> |path|
  \end{tabbing}
\end{tcolorbox}
\end{docSkin}

\begin{docTcbKey}[][doc new=2016-02-25]{tile}{}{style, initially\\
  |skin=tile,boxrule=0pt,sharp corners,title filled,fonttitle=\textbackslash bfseries|
}
  This key applies |skin=tile| and in addition changes the geometry and some style options.
\end{docTcbKey}


\begin{dispExample}
\skinExampleSet{tile,
  colbacklower=LimeGreen!75!LightGreen,
}
\end{dispExample}


\clearpage
\begin{docSkin}[doc new=2016-02-25]{tilefirst}
This is a flavor of \refSkin{tile} which is used as a \emph{first} part
in a break sequence for \refSkin{tile}.
Nevertheless, this skin can be applied independently.
\begin{tcolorbox}[skintable=tilefirst]
  \begin{tabbing}
    \refKey{/tcb/interior titled engine}: \=\kill
    \refKey{/tcb/frame engine}:           \> |empty|\\
    \refKey{/tcb/interior titled engine}: \> \emph{special}\\ 
    \refKey{/tcb/interior engine}:        \> \emph{special}\\
    \refKey{/tcb/segmentation engine}:    \> \emph{special}\\
    \refKey{/tcb/title engine}:           \> |pathfirst|
  \end{tabbing}
\end{tcolorbox}
\end{docSkin}

\begin{dispExample}
\skinExampleSet{skin=tilefirst,
  colbacklower=LimeGreen!75!LightGreen,
  boxrule=0pt,
}
\end{dispExample}


\clearpage
\begin{docSkin}[doc new=2016-02-25]{tilemiddle}
This is a flavor of \refSkin{tile} which is used as a \emph{middle} part
in a break sequence for \refSkin{tile}.
Nevertheless, this skin can be applied independently.
\begin{tcolorbox}[skintable=tilemiddle]
  \begin{tabbing}
    \refKey{/tcb/interior titled engine}: \=\kill
    \refKey{/tcb/frame engine}:           \> |empty|\\
    \refKey{/tcb/interior titled engine}: \> \emph{special}\\ 
    \refKey{/tcb/interior engine}:        \> \emph{special}\\
    \refKey{/tcb/segmentation engine}:    \> \emph{special}\\
    \refKey{/tcb/title engine}:           \> |pathmiddle|
  \end{tabbing}
\end{tcolorbox}
\end{docSkin}


\begin{dispExample}
\skinExampleSet{skin=tilemiddle,
  colbacklower=LimeGreen!75!LightGreen,
  boxrule=0pt,
}
\end{dispExample}


\clearpage
\begin{docSkin}[doc new=2016-02-25]{tilelast}
This is a flavor of \refSkin{tile} which is used as a \emph{last} part
in a break sequence for \refSkin{tile}.
Nevertheless, this skin can be applied independently.
\begin{tcolorbox}[skintable=tilelast]
  \begin{tabbing}
    \refKey{/tcb/interior titled engine}: \=\kill
    \refKey{/tcb/frame engine}:           \> |empty|\\
    \refKey{/tcb/interior titled engine}: \> \emph{special}\\ 
    \refKey{/tcb/interior engine}:        \> \emph{special}\\
    \refKey{/tcb/segmentation engine}:    \> \emph{special}\\
    \refKey{/tcb/title engine}:           \> |pathlast|
  \end{tabbing}
\end{tcolorbox}
\end{docSkin}


\begin{dispExample}
\skinExampleSet{skin=tilelast,
  colbacklower=LimeGreen!75!LightGreen,
  boxrule=0pt,
}
\end{dispExample}



\clearpage
\subsection{Skin Family \enquote{beamer}}

\begin{docSkin}{beamer}
  This skin resembles boxes known from the \refPkg{beamer} class and therefore is
  called \enquote{beamer}. It uses the normal colors from the core package but shades
  them a little bit.\par
  While the motivation for this skin is to use \enquote{beamerish} looking
  boxes apart from the \refPkg{beamer} class, one may want to use |tcolorbox|es
  as blocks inside |beamer| slides. For this purpose see the package \refPkg{beamertheme-tcolorbox}
  which provides a |beamer| inner theme reproducing standard |beamer| blocks using |tcolorbox|es.
\begin{tcolorbox}[skintable=beamer]
  \begin{tabbing}
    \refKey{/tcb/interior titled engine}: \=\kill
    \refKey{/tcb/frame engine}:           \> |path|\\
    \refKey{/tcb/interior titled engine}: \> \emph{special}\\ 
    \refKey{/tcb/interior engine}:        \> \emph{special}\\
    \refKey{/tcb/segmentation engine}:    \> \emph{special}\\
    \refKey{/tcb/title engine}:           \> |path|
  \end{tabbing}
\end{tcolorbox}
\end{docSkin}



\begin{docTcbKey}{beamer}{}{style, initially\\
  |skin=beamer,boxrule=0mm,titlerule=1mm,toptitle=0.5mm,|\\
  |arc=2mm,fonttitle=\textbackslash bfseries,drop fuzzy shadow|
}
  This key applies |skin=beamer| and in addition changes the geometry and some style options.
\end{docTcbKey}



\begin{dispExample}
\skinExampleSet{beamer,title filled=false}
\end{dispExample}



\begin{dispExample}
\begin{tcolorbox}[beamer,colback=Salmon!50!white,colframe=FireBrick!75!black,
  adjusted title=A colored box with the \enquote{beamer} skin]
This box looks like a box provided by the \texttt{beamer} class.
\end{tcolorbox}
\end{dispExample}


\begin{dispExample}
\begin{tcolorbox}[beamer,colframe=blue,colback=black,
  watermark graphics=lichtspiel.jpg,
  coltext=white,watermark opacity=0.75,watermark stretch=1.0,
  title=Beamer Box with background picture]
\lipsum[1]
\end{tcolorbox}
\end{dispExample}


\begin{dispExample}
\newtcolorbox{myblock}[2][]{%
  beamer,breakable,colback=LightBlue,colframe=DarkBlue,#1,title={#2}}%

\begin{myblock}{Beamerish \texttt{block}: \texttt{myblock}}
\lipsum[1]
\end{myblock}
\end{dispExample}


\clearpage
\begin{docSkin}{beamerfirst}
This is a flavor of \refSkin{beamer} which is used as a \emph{first} part
in a break sequence for \refSkin{beamer}.
Nevertheless, this skin can be applied independently.
\begin{tcolorbox}[skintable=beamerfirst]
  \begin{tabbing}
    \refKey{/tcb/interior titled engine}: \=\kill
    \refKey{/tcb/frame engine}:           \> |pathfirst|\\
    \refKey{/tcb/interior titled engine}: \> \emph{special}\\ 
    \refKey{/tcb/interior engine}:        \> \emph{special}\\
    \refKey{/tcb/segmentation engine}:    \> \emph{special}\\
    \refKey{/tcb/title engine}:           \> |pathfirst|
  \end{tabbing}
\end{tcolorbox}
\end{docSkin}


\begin{dispExample}
\skinExampleSet{beamer,title filled=false,skin=beamerfirst}
\end{dispExample}


\medskip

\begin{docSkin}{beamermiddle}
This is a flavor of \refSkin{beamer} which is used as a \emph{middle} part
in a break sequence for \refSkin{beamer}.
Nevertheless, this skin can be applied independently.
\begin{tcolorbox}[skintable=beamermiddle]
  \begin{tabbing}
    \refKey{/tcb/interior titled engine}: \=\kill
    \refKey{/tcb/frame engine}:           \> |pathmiddle|\\
    \refKey{/tcb/interior titled engine}: \> \emph{special}\\ 
    \refKey{/tcb/interior engine}:        \> \emph{special}\\
    \refKey{/tcb/segmentation engine}:    \> \emph{special}\\
    \refKey{/tcb/title engine}:           \> |pathmiddle|
  \end{tabbing}
\end{tcolorbox}
\end{docSkin}


\begin{dispExample}
\skinExampleSet{beamer,title filled=false,skin=beamermiddle}
\end{dispExample}


\clearpage
\begin{docSkin}{beamerlast}
This is a flavor of \refSkin{beamer} which is used as a \emph{last} part
in a break sequence for \refSkin{beamer}.
Nevertheless, this skin can be applied independently.
\begin{tcolorbox}[skintable=beamerlast]
  \begin{tabbing}
    \refKey{/tcb/interior titled engine}: \=\kill
    \refKey{/tcb/frame engine}:           \> |pathlast|\\
    \refKey{/tcb/interior titled engine}: \> \emph{special}\\ 
    \refKey{/tcb/interior engine}:        \> \emph{special}\\
    \refKey{/tcb/segmentation engine}:    \> \emph{special}\\
    \refKey{/tcb/title engine}:           \> |pathlast|
  \end{tabbing}
\end{tcolorbox}
\end{docSkin}

\begin{dispExample}
\skinExampleSet{beamer,title filled=false,skin=beamerlast}
\end{dispExample}



\clearpage
\subsection{Skin Family \enquote{widget}}
\begin{docSkin}{widget}
  This skin uses the normal colors from the core package but shades
  them a little bit.
  The appearance of the skin can be controlled by \refKey{/tcb/frame style},
  \refKey{/tcb/interior style}, and \refKey{/tcb/segmentation style},
  if needed.
\begin{tcolorbox}[skintable=widget]
  \begin{tabbing}
    \refKey{/tcb/interior titled engine}: \=\kill
    \refKey{/tcb/frame engine}:           \> |path|\\
    \refKey{/tcb/interior titled engine}: \> |path|\\ 
    \refKey{/tcb/interior engine}:        \> |path|\\
    \refKey{/tcb/segmentation engine}:    \> \emph{special}\\
    \refKey{/tcb/title engine}:           \> \emph{special}
  \end{tabbing}
\end{tcolorbox}
\end{docSkin}


\begin{docTcbKey}[][doc updated={2020-09-23}]{widget}{}{style, initially\\
  |skin=widget,arc=0.5mm,fonttitle=\bfseries,titlerule=0mm|
}
  This key applies |skin=widget| and in addition changes the geometry and some style options.
\end{docTcbKey}


\begin{dispExample}
\skinExampleSet{widget}
\end{dispExample}


\begin{dispExample}
\begin{tcolorbox}[widget,colback=Salmon!50!white,colframe=FireBrick!75!black,
  adjusted title=A colored box with the \enquote{widget} skin]
This is my content.
\end{tcolorbox}
\end{dispExample}

\clearpage

\begin{docSkin}{widgetfirst}
This is a flavor of \refSkin{widget} which is used as a \emph{first} part
in a break sequence for \refSkin{widget}.
Nevertheless, this skin can be applied independently.
\begin{tcolorbox}[skintable=widgetfirst]
  \begin{tabbing}
    \refKey{/tcb/interior titled engine}: \=\kill
    \refKey{/tcb/frame engine}:           \> |pathfirst|\\
    \refKey{/tcb/interior titled engine}: \> |pathfirst|\\ 
    \refKey{/tcb/interior engine}:        \> |pathfirst|\\
    \refKey{/tcb/segmentation engine}:    \> \emph{special}\\
    \refKey{/tcb/title engine}:           \> \emph{special}
  \end{tabbing}
\end{tcolorbox}
\end{docSkin}


\begin{dispExample}
\skinExampleSet{widget,skin=widgetfirst}
\end{dispExample}

\medskip

\begin{docSkin}{widgetmiddle}
This is a flavor of \refSkin{widget} which is used as a \emph{middle} part
in a break sequence for \refSkin{widget}.
Nevertheless, this skin can be applied independently.
\begin{tcolorbox}[skintable=widgetmiddle]
  \begin{tabbing}
    \refKey{/tcb/interior titled engine}: \=\kill
    \refKey{/tcb/frame engine}:           \> |pathmiddle|\\
    \refKey{/tcb/interior titled engine}: \> |pathmiddle|\\ 
    \refKey{/tcb/interior engine}:        \> |pathmiddle|\\
    \refKey{/tcb/segmentation engine}:    \> \emph{special}\\
    \refKey{/tcb/title engine}:           \> \emph{special}
  \end{tabbing}
\end{tcolorbox}
\end{docSkin}

\begin{dispExample}
\skinExampleSet{widget,skin=widgetmiddle}
\end{dispExample}


\clearpage
\begin{docSkin}{widgetlast}
This is a flavor of \refSkin{widget} which is used as a \emph{last} part
in a break sequence for \refSkin{widget}.
Nevertheless, this skin can be applied independently.
\begin{tcolorbox}[skintable=widgetlast]
  \begin{tabbing}
    \refKey{/tcb/interior titled engine}: \=\kill
    \refKey{/tcb/frame engine}:           \> |pathlast|\\
    \refKey{/tcb/interior titled engine}: \> |pathlast|\\ 
    \refKey{/tcb/interior engine}:        \> |pathlast|\\
    \refKey{/tcb/segmentation engine}:    \> \emph{special}\\
    \refKey{/tcb/title engine}:           \> \emph{special}
  \end{tabbing}
\end{tcolorbox}
\end{docSkin}


\begin{dispExample}
\skinExampleSet{widget,skin=widgetlast}
\end{dispExample}


\clearpage
\subsection{Skin Family \enquote{empty}}

\begin{docSkin}{empty}
  This skin sets all engines to |empty|, i.\,e., nothing is drawn at all.
  Therefore, this skin is a good starting point to create a complete
  new style by yourself.
\begin{tcolorbox}[skintable=empty]
  \begin{tabbing}
    \refKey{/tcb/interior titled engine}: \=\kill
    \refKey{/tcb/frame engine}:           \> |empty|\\
    \refKey{/tcb/interior titled engine}: \> |empty|\\ 
    \refKey{/tcb/interior engine}:        \> |empty|\\
    \refKey{/tcb/segmentation engine}:    \> |empty|\\
    \refKey{/tcb/title engine}:           \> |empty|
  \end{tabbing}
\end{tcolorbox}
\end{docSkin}


\begin{marker}
  Note that the text colors stay unchanged when a skin is applied.
  Since the standard title color
  is white, the title of a box with skin \docValue*{empty} becomes
  invisible, if not set to another color by \refKey{/tcb/coltitle}.
\end{marker}


\begin{docTcbKey}{empty}{}{style, no value}
  This is an abbreviation for setting |skin=empty|.
\end{docTcbKey}


\begin{dispExample}
\skinExampleSet{empty,
  coltitle=Navy,borderline={2pt}{0pt}{black!10!white},
}
\end{dispExample}


\clearpage
\begin{docTcbKey}{blanker}{}{style, initially unset}
  This style relies on the skin \refSkin{empty}. All engines
  are set to empty and all margins are set to |0pt|.
  In contrast to \refKey{/tcb/blank}, the graphical paths are
  not constructed with exception of the geometry nodes.
\begin{dispExample}
\begin{tcolorbox}[blanker,watermark text=A blank box]
\lipsum[1]
\end{tcolorbox}
\end{dispExample}

\begin{dispExample}
% \tcbuselibrary{fitting}
\newtcboxfit{\mybox}[1]{blanker,width=4cm,height=7cm,top=4pt,
  watermark text=#1}

\begin{tabular}{|c|c|c|}\hline
A & B & C\\\hline
\mybox{A}{\lipsum[1]} & \mybox{B}{\lipsum[2]} & \mybox{C}{\lipsum[3]}\\\hline
\end{tabular}
\end{dispExample}
\end{docTcbKey}



\clearpage
\begin{docTcbKey}{blankest}{}{style, initially unset}
  This style extends \refKey{/tcb/blanker}.
  All engines are set to empty and all margins are set to |0pt|.
  In contrast to \refKey{/tcb/blanker}, also title, shadow, underlay,
  overlay, finish and borderline are removed.

\begin{dispExample}
% \tcbuselibrary{raster}
\begin{tcbraster}[raster columns=3,raster equal height,
    title=Box \thetcbrasternum,
    enhanced,size=small,colframe=red!50!black,colback=red!10!white,
    coltitle=yellow!85!black,
    drop fuzzy shadow,watermark text={Box \thetcbrasternum},
    borderline={.25mm}{-0.5mm}{green!40!black},
    finish={\begin{tcbclipframe}\draw[blue,opacity=0.1,line width=1cm]
      (frame.south west) -- (frame.north east);\end{tcbclipframe}},
    ]
  \begin{tcolorbox}\lipsum[4]\end{tcolorbox}
  \begin{tcolorbox}[blanker]\lipsum[4]\end{tcolorbox}
  \begin{tcolorbox}[blankest]\lipsum[4]\end{tcolorbox}
\end{tcbraster}
\end{dispExample}
\end{docTcbKey}


\clearpage
\begin{docSkin}{emptyfirst}
This is a flavor of \refSkin{empty} which is used as a \emph{first} part
in a break sequence for \refSkin{empty}.
Nevertheless, this skin can be applied independently.
\begin{tcolorbox}[skintable=emptyfirst]
  \begin{tabbing}
    \refKey{/tcb/interior titled engine}: \=\kill
    \refKey{/tcb/frame engine}:           \> |empty|\\
    \refKey{/tcb/interior titled engine}: \> |empty|\\ 
    \refKey{/tcb/interior engine}:        \> |empty|\\
    \refKey{/tcb/segmentation engine}:    \> |empty|\\
    \refKey{/tcb/title engine}:           \> |empty|
  \end{tabbing}
\end{tcolorbox}
\end{docSkin}


\begin{dispExample}
\skinExampleSet{skin=emptyfirst,
  coltitle=Navy,borderline={2pt}{0pt}{black!10!white},
}
\end{dispExample}


\clearpage

\begin{docSkin}{emptymiddle}
This is a flavor of \refSkin{empty} which is used as a \emph{middle} part
in a break sequence for \refSkin{empty}.
Nevertheless, this skin can be applied independently.
\begin{tcolorbox}[skintable=emptymiddle]
  \begin{tabbing}
    \refKey{/tcb/interior titled engine}: \=\kill
    \refKey{/tcb/frame engine}:           \> |empty|\\
    \refKey{/tcb/interior titled engine}: \> |empty|\\ 
    \refKey{/tcb/interior engine}:        \> |empty|\\
    \refKey{/tcb/segmentation engine}:    \> |empty|\\
    \refKey{/tcb/title engine}:           \> |empty|
  \end{tabbing}
\end{tcolorbox}
\end{docSkin}


\begin{dispExample}
\skinExampleSet{skin=emptymiddle,
  coltitle=Navy,borderline={2pt}{0pt}{black!10!white},
}
\end{dispExample}


\clearpage
\begin{docSkin}{emptylast}
This is a flavor of \refSkin{empty} which is used as a \emph{last} part
in a break sequence for \refSkin{empty}.
Nevertheless, this skin can be applied independently.
\begin{tcolorbox}[skintable=emptylast]
  \begin{tabbing}
    \refKey{/tcb/interior titled engine}: \=\kill
    \refKey{/tcb/frame engine}:           \> |empty|\\
    \refKey{/tcb/interior titled engine}: \> |empty|\\ 
    \refKey{/tcb/interior engine}:        \> |empty|\\
    \refKey{/tcb/segmentation engine}:    \> |empty|\\
    \refKey{/tcb/title engine}:           \> |empty|
  \end{tabbing}
\end{tcolorbox}
\end{docSkin}

\begin{dispExample}
\skinExampleSet{skin=emptylast,
  coltitle=Navy,borderline={2pt}{0pt}{black!10!white},
}
\end{dispExample}

\clearpage
\begin{dispListing*}{breakable,phantomlabel=freeboxexample,before upper={This example demonstrates
a breakable customized box. Here, we define an environment |freebox|.
The first application of |freebox| produces an unbroken |tcolorbox|.
The box is drawn by the code given by \refKey{/tcb/frame code}
and \refKey{/tcb/interior code}.\par
The second application of |freebox| is broken into several parts which
are drawn by the codes given by
\refKey{/tcb/skin first is subskin of},
\refKey{/tcb/skin middle is subskin of}, and
\refKey{/tcb/skin last is subskin of}.
\par\bigskip
}}
% Preamble:
%\usepackage{lipsum}
%\tcbuselibrary{skins,breakable}
\tikzset{coltria/.style={fill=red!15!white}}

\newtcolorbox{freebox}[1][]{empty,
  breakable,height fixed for=first and middle,
  leftrule=5mm,left=2mm,
  frame style={fill,top color=red!75!black,bottom color=red!75!black,middle color=red},
  colback=yellow!50!white,
  watermark color=red!50!yellow!75!white,
  watermark text on=unbroken is unbroken box,
  watermark text on=first is first part,
  watermark text on=middle is middle part,
  watermark text on=last is last part,
  % code for unbroken boxes:
  frame code={\path[tcb fill frame] (frame.south west)--(frame.north west)
    --([xshift=-5mm]frame.north east)--([yshift=-5mm]frame.north east)
    --([yshift=5mm]frame.south east)--([xshift=-5mm]frame.south east)--cycle; },
  interior code={\path[tcb fill interior] (interior.south west)--(interior.north west)
    --([xshift=-4.8mm]interior.north east)--([yshift=-4.8mm]interior.north east)
    --([yshift=4.8mm]interior.south east)--([xshift=-4.8mm]interior.south east)
    --cycle; },
  % code for the first part of a break sequence:
  skin first is subskin of={emptyfirst}{%
    frame code={\path[tcb fill frame] (frame.south west)--(frame.north west)
      --([xshift=-5mm]frame.north east)--([yshift=-5mm]frame.north east)
      --(frame.south east)--cycle;
      \path[coltria] ([xshift=2.5mm,yshift=1mm]frame.south west) -- +(120:2mm)
      -- +(60:2mm)-- cycle; },
    interior code={\path[tcb fill interior] (interior.south west|-frame.south)
      --(interior.north west)--([xshift=-4.8mm]interior.north east)
      --([yshift=-4.8mm]interior.north east)--(interior.south east|-frame.south)
      --cycle; },
  },%
  % code for the middle part of a break sequence:
  skin middle is subskin of={emptymiddle}{%
    frame code={\path[tcb fill frame] (frame.south west)--(frame.north west)
      --(frame.north east)--(frame.south east)--cycle;
      \path[coltria] ([xshift=2.5mm,yshift=-1mm]frame.north west) -- +(240:2mm)
        -- +(300:2mm) -- cycle;
      \path[coltria] ([xshift=2.5mm,yshift=1mm]frame.south west) -- +(120:2mm)
        -- +(60:2mm) -- cycle;
      },
    interior code={\path[tcb fill interior] (interior.south west|-frame.south)
      --(interior.north west|-frame.north)--(interior.north east|-frame.north)
      --(interior.south east|-frame.south)--cycle; },
    },
  % code for the last part of a break sequence:
  skin last is subskin of={emptylast}{%
    frame code={\path[tcb fill frame] (frame.south west)--(frame.north west)
      --(frame.north east)--([yshift=5mm]frame.south east)
      --([xshift=-5mm]frame.south east)--cycle;
      \path[coltria] ([xshift=2.5mm,yshift=-1mm]frame.north west) -- +(240:2mm)
      -- +(300:2mm) -- cycle;
      },
    interior code={\path[tcb fill interior] (interior.south west)
      --(interior.north west|-frame.north)--(interior.north east|-frame.north)
      --([yshift=4.8mm]interior.south east)--([xshift=-4.8mm]interior.south east)
      --cycle; },
    },
  #1}

\begin{freebox}
\lipsum[1]
\end{freebox}

\begin{freebox}
\lipsum[1-12]
\end{freebox}
\end{dispListing*}
{\tcbusetemp}


\clearpage

\subsection{Skin \enquote{spartan}}\label{subsec:spartan}

\begin{docSkin}{spartan}
  This skin is quite \ldots\ spartan.
  It supports no rounded corners, no overlays, no shadows, no borderlines,
  and no finishes. The only exception are underlays.
  One cannot do very fancy things with this skin, but it compiles very fast.
  Therefore, the |spartan| skin is
  used for the draft mode, see \zvref{subsec:draftmode}.
  Nevertheless, it can be used as a normal skin.

\begin{tcolorbox}[skintable=spartan]
  \begin{tabbing}
    \refKey{/tcb/interior titled engine}: \=\kill
    \refKey{/tcb/frame engine}:           \> |spartan|\\
    \refKey{/tcb/interior titled engine}: \> |spartan|\\ 
    \refKey{/tcb/interior engine}:        \> |spartan|\\
    \refKey{/tcb/segmentation engine}:    \> |spartan|\\
    \refKey{/tcb/title engine}:           \> |spartan|
  \end{tabbing}
\end{tcolorbox}
\end{docSkin}


\begin{docTcbKey}{spartan}{}{style, no value}
  This is an abbreviation for setting |skin=spartan|.
\end{docTcbKey}


\begin{dispExample}
\skinExampleSet{spartan}
\end{dispExample}


\clearpage

\subsection{Skin \enquote{draft}}\label{subsec:draft}

\begin{docSkin}{draft}
  This skin is intended to be used while drafting new geometric settings
  for a |tcolorbox|.
\begin{tcolorbox}[skintable=draft]
  \begin{tabbing}
    \refKey{/tcb/interior titled engine}: \=\kill
    \refKey{/tcb/frame engine}:           \> \emph{special}\\
    \refKey{/tcb/interior titled engine}: \> \emph{special}\\ 
    \refKey{/tcb/interior engine}:        \> \emph{special}\\
    \refKey{/tcb/segmentation engine}:    \> |path|\\
    \refKey{/tcb/title engine}:           \> |path|
  \end{tabbing}
\end{tcolorbox}
\end{docSkin}

\begin{docTcbKey}{draft}{}{style, no value}
  This is an abbreviation for setting |skin=draft|.
\end{docTcbKey}


\begin{dispExample}
\skinExampleSet{draft}
\end{dispExample}



\begin{dispExample}
\vspace*{3mm}
\begin{tcolorbox}[draft,title=A colored box with the \enquote{draft} skin]
\lipsum[1-3]
\tcblower
\lipsum[4-6]
\end{tcolorbox}
\end{dispExample}



\clearpage
\subsection{Skin Family \enquote{freelance}}
\begin{marker}
This skin family \enquote{freelance} is deprecated with |tcolorbox| 3.00.
It is not longer needed, because
\refKey{/tcb/frame code},
\refKey{/tcb/interior code},
\refKey{/tcb/interior titled code}, and
\refKey{/tcb/title code}
can be applied to every skin now. In this sense, everything has become
\emph{freelance} now.\par
For users of \refKey{/tcb/freelance}: Old code should continue to work. There may be
exceptions for breakable freelance boxes under certain circumstances.
For new code, use \refKey{/tcb/empty} or \refKey{/tcb/enhanced} where
you would have used \refKey{/tcb/freelance} before.
\end{marker}

\begin{docSkin}{freelance}
  This skin gives full freedom for the appearance of the |tcolorbox|.
  All drawing engines are set to type |freelance|; they use the \refPkg{tikz}
  package and compute the \refKey{/tcb/geometry nodes}.
  %This skin is useful for boxes which should differ much from the normal
  %appearance. Note that this difference has to be programmed by the user.
  %The drawing code can be given
  %with the following option keys. As default value, the code from the |standard|
  %skin is set.
\begin{tcolorbox}[skintable=freelance]
  \begin{tabbing}
    \refKey{/tcb/interior titled engine}: \=\kill
    \refKey{/tcb/frame engine}:           \> |freelance|\\
    \refKey{/tcb/interior titled engine}: \> |freelance|\\ 
    \refKey{/tcb/interior engine}:        \> |freelance|\\
    \refKey{/tcb/segmentation engine}:    \> |freelance|\\
    \refKey{/tcb/title engine}:           \> |freelance|
  \end{tabbing}
\end{tcolorbox}
\end{docSkin}

\begin{docTcbKey}{freelance}{}{style, no value}
  This is an abbreviation for setting |skin=freelance|.
\end{docTcbKey}

\begin{docSkin}{freelancefirst}
  This skin equals \refSkin{freelance} with exception of the break sequence,
  see \zvref{subsec:breaksequence}.
  %It is used as first part of the
  %break sequence of \refSkin{freelance}. \refKey{/tcb/extend freelancefirst}
  %can be used to customize this part.
\end{docSkin}

\begin{docSkin}{freelancemiddle}
  This skin equals \refSkin{freelance} with exception of the break sequence,
  see \zvref{subsec:breaksequence}.
  %It is used as middle part of the
  %break sequence of \refSkin{freelance}. \refKey{/tcb/extend freelancemiddle}
  %can be used to customize this part.
\end{docSkin}

\begin{docSkin}{freelancelast}
  This skin equals \refSkin{freelance} with exception of the break sequence,
  see \zvref{subsec:breaksequence}.
  %It is used as last part of the
  %break sequence of \refSkin{freelance}. \refKey{/tcb/extend freelancelast}
  %can be used to customize this part.
\end{docSkin}


\begin{docTcbKey}{extend freelance}{=\meta{options}}{no default, initially empty}
The \meta{options} are added to the skin definition of \refSkin{freelance}.
\end{docTcbKey}

\begin{docTcbKey}{extend freelancefirst}{=\meta{options}}{no default, initially empty}
The \meta{options} are added to the skin definition of \refSkin{freelancefirst} which
is used as first part of the break sequence of \refSkin{freelance}.
See \refKey{/tcb/skin first is subskin of} for a substitute of this key.
\end{docTcbKey}

\begin{docTcbKey}{extend freelancemiddle}{=\meta{options}}{no default, initially empty}
The \meta{options} are added to the skin definition of \refSkin{freelancemiddle} which
is used as middle part of the break sequence of \refSkin{freelance}.
See \refKey{/tcb/skin middle is subskin of} for a substitute of this key.
\end{docTcbKey}

\enlargethispage*{1cm}

\begin{docTcbKey}{extend freelancelast}{=\meta{options}}{no default, initially empty}
The \meta{options} are added to the skin definition of \refSkin{freelancelast} which
is used as last part of the break sequence of \refSkin{freelance}.
See \refKey{/tcb/skin last is subskin of} for a substitute of this key.
\end{docTcbKey}



