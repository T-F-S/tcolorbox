%% The LaTeX package tcolorbox - version 6.1.0 (2023/09/26)
%% tcolorbox-example.tex: an example for tcolorbox
%%
%% -------------------------------------------------------------------------------------------
%% Copyright (c) 2006-2023 by Prof. Dr. Dr. Thomas F. Sturm <thomas dot sturm at unibw dot de>
%% -------------------------------------------------------------------------------------------
%%
%% This work may be distributed and/or modified under the
%% conditions of the LaTeX Project Public License, either version 1.3
%% of this license or (at your option) any later version.
%% The latest version of this license is in
%%   http://www.latex-project.org/lppl.txt
%% and version 1.3 or later is part of all distributions of LaTeX
%% version 2005/12/01 or later.
%%
%% This work has the LPPL maintenance status `author-maintained'.
%%
%% This work consists of all files listed in README
%%
% arara: pdflatex: {  }
% arara: pdflatex: { synctex: yes }
%
\documentclass{article}
\usepackage{tikz,lipsum,lmodern}
\usepackage[most]{tcolorbox}

\begin{document}

%----------------------------------------------------------
\section{Colored boxes}

\begin{tcolorbox}[colback=red!5!white,colframe=red!75!black]
  My box.
\end{tcolorbox}

\begin{tcolorbox}[colback=blue!5!white,colframe=blue!75!black,title=My title]
  My box with my title.
\end{tcolorbox}

\begin{tcolorbox}[colback=green!5!white,colframe=green!75!black]
  Upper part of my box.
  \tcblower
  Lower part of my box.
\end{tcolorbox}

\begin{tcolorbox}[colback=yellow!5!white,colframe=yellow!50!black,
  colbacktitle=yellow!75!black,title=My title]
  I can do this also with a title.
  \tcblower
  Lower part of my box.
\end{tcolorbox}

\begin{tcolorbox}[colback=yellow!10!white,colframe=red!75!black,lowerbox=invisible,
  savelowerto=\jobname_ex.tex]
  Now, we play hide and seek. Where is the lower part?
  \tcblower
  I'm invisible until you find me.
\end{tcolorbox}

\begin{tcolorbox}[colback=yellow!10!white,colframe=red!75!black,title=Here I am]
  \input{\jobname_ex.tex}
\end{tcolorbox}


\begin{tcolorbox}[enhanced,sharp corners=uphill,
    colback=blue!50!white,colframe=blue!25!black,coltext=yellow,
    fontupper=\Large\bfseries,arc=6mm,boxrule=2mm,boxsep=5mm,
    borderline={0.3mm}{0.3mm}{white}]
  Funny settings.
\end{tcolorbox}


\begin{tcolorbox}[enhanced,frame style image=blueshade.png,
  opacityback=0.75,opacitybacktitle=0.25,
  colback=blue!5!white,colframe=blue!75!black,
  title=My title]
  This box is filled with an external image.\par
  Title and interior are made partly transparent to show the image.
\end{tcolorbox}


\begin{tcolorbox}[enhanced,attach boxed title to top center={yshift=-3mm,yshifttext=-1mm},
  colback=blue!5!white,colframe=blue!75!black,colbacktitle=red!80!black,
  title=My title,fonttitle=\bfseries,
  boxed title style={size=small,colframe=red!50!black} ]
  This box uses a \textit{boxed title}. The box of the title can
  be formatted independently from the main box.
\end{tcolorbox}


\clearpage
%----------------------------------------------------------
\section{\LaTeX-Examples}

\begin{tcblisting}{colback=red!5!white,colframe=red!75!black}
This is a \LaTeX\ example:
\begin{equation}
\sum\limits_{i=1}^n i = \frac{n(n+1)}{2}.
\end{equation}
\end{tcblisting}


\begin{tcblisting}{colback=red!5!white,colframe=red!75!black,listing side text,
  title=Side by side,fonttitle=\bfseries}
This is a \LaTeX\ example:
\begin{equation}
\sum\limits_{i=1}^n i = \frac{n(n+1)}{2}.
\end{equation}
\end{tcblisting}


%----------------------------------------------------------
\section{Theorems}

\newtcbtheorem[auto counter,number within=section]{theo}%
  {Theorem}{fonttitle=\bfseries\upshape, fontupper=\slshape,
     arc=0mm, colback=blue!5!white,colframe=blue!75!black}{theorem}

\begin{theo}{Summation of Numbers}{summation}
  For all natural number $n$ it holds:
  \begin{equation}
  \tcbhighmath{\sum\limits_{i=1}^n i = \frac{n(n+1)}{2}.}
  \end{equation}
\end{theo}

We have given Theorem \ref{theorem:summation} on page \pageref{theorem:summation}.

\newtcbtheorem[use counter from=theo]{antheo}%
  {Theorem}{theorem style=change,oversize,enlarge top by=1mm,enlarge bottom by=1mm,
    enhanced jigsaw,interior hidden,fuzzy halo=1mm with green,
     fonttitle=\bfseries\upshape,fontupper=\slshape,
     colframe=green!75!black,coltitle=green!50!blue!75!black}{antheorem}

\begin{antheo}{Summation of Numbers}{summation}
  For all natural number $n$ it holds:
  \begin{equation}
  \tcbhighmath{\sum\limits_{i=1}^n i = \frac{n(n+1)}{2}.}
  \end{equation}
\end{antheo}

%----------------------------------------------------------
\section{Watermarks}

\begin{tcolorbox}[enhanced,watermark graphics=Basilica_5.png,
  watermark opacity=0.3,watermark zoom=0.9,
  colback=green!5!white,colframe=green!75!black,
  fonttitle=\bfseries, title=Box with a watermark picture]
  Here, you see my nice box with a picture as a watermark.
  This picture is automatically resized to fit the dimensions
  of my box. Instead of a picture, some text could be used or
  arbitrary graphical code. See the documentation for more options.
\end{tcolorbox}

%----------------------------------------------------------
\section{Boxes in boxes}
\begin{tcolorbox}[colback=yellow!10!white,colframe=yellow!50!black,
  every box/.style={fonttitle=\bfseries},title=Box]
  \begin{tcolorbox}[enhanced,colback=red!10!white,colframe=red!50!black,
    colbacktitle=red!85!black,
    title=Box inside box,drop fuzzy shadow]
    \begin{tcolorbox}[beamer,colframe=blue!50!black,title=Box inside box inside box]
      And now for something completely different: Boxes!\par\medskip
      \newtcbox{\mybox}[1][]{nobeforeafter,tcbox raise base,colframe=green!50!black,colback=green!10!white,
        sharp corners,top=1pt,bottom=1pt,before upper=\strut,#1}
      \mybox[rounded corners=west]{This} \mybox{is} \mybox{another} \mybox[rounded corners=east]{box.}
    \end{tcolorbox}
  \end{tcolorbox}
\end{tcolorbox}



%----------------------------------------------------------
\section{Breakable Boxes}
\begin{tcolorbox}[enhanced jigsaw,breakable,pad at break*=1mm,
  colback=blue!5!white,colframe=blue!75!black,title=Breakable box,
  watermark color=white,watermark text=\Roman{tcbbreakpart}]
  \lipsum[1-12]
\end{tcolorbox}

%----------------------------------------------------------
\clearpage
\section{Fit Boxes}

\begin{tcolorbox}[enhanced,fit to height=10cm,
  colback=green!25!black!10!white,colframe=green!75!black,title=Fit box (10cm),
  drop fuzzy shadow,watermark color=white,watermark text=Fit]
  \lipsum[1-4]
\end{tcolorbox}

\begin{tcolorbox}[enhanced,fit to height=5cm,
  colback=green!25!black!10!white,colframe=green!75!black,title=Fit box (5cm),
  drop fuzzy shadow,watermark color=white,watermark text=Fit]
  \lipsum[1-4]
\end{tcolorbox}

%----------------------------------------------------------
\begin{tcolorbox}[tile,size=fbox,boxsep=2mm,boxrule=0pt,
    colback=blue!20!black,colbacktitle=yellow!40!black,
    phantom={\thispagestyle{empty}},
    title=\section{Rasters of Boxes},spread=-1cm]
  \begin{tcbitemize}[raster columns=3,raster rows=4,raster equal skip=2mm,
      raster height=\tcbtextheight,
      title={Box~\thetcbrasternum},fonttitle=\bfseries,
      enhanced,colframe=blue!50,colback=blue!5]
    \tcbitem A
    \tcbitem[flip title={interior hidden}] B
    \tcbitem[colbacktitle=blue!40!black] C
    \tcbitem[colbacktitle=blue!40!black,flip title] D
    \tcbitem[beamer] E
    \tcbitem[tile,colbacktitle=red!40!black] F
    \tcbitem[tile,colbacktitle=red!40!black,flip title={sharp corners}] G
    \tcbitem[blankest]
      \begin{tcbitemize}[raster columns=2,raster rows=2,raster height=\tcbtextheight,
          tile,halign=center,valign=center,fontupper=\bfseries\Huge,before upper=\strut,
          colback=green!20]
        \tcbitem a
        \tcbitem b
        \tcbitem c
        \tcbitem d
      \end{tcbitemize}
    \tcbitem[widget] I
    \tcbitem[frame style image=goldshade,opacityback=0.5,colback=white] J
    \tcbitem[frame style image=blueshade,opacityback=0.5,colback=white,
      center title,flip title={frame hidden,opacityback=0.25,colback=black}] K
    \tcbitem[octogon arc,arc is angular,halign=center,valign=center,
      detach title,watermark text={\tcbtitletext}] L
  \end{tcbitemize}
\end{tcolorbox}



\end{document}

