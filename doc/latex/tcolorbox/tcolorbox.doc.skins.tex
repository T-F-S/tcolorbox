% !TeX root = tcolorbox.tex
% include file of tcolorbox.tex (manual of the LaTeX package tcolorbox)
\clearpage
\section{Library \mylib{skins}}\label{sec:skins}%
\tcbset{external/prefix=external/skins_}%
The library is loaded by a package option or inside the preamble by:
\begin{dispListing}
\tcbuselibrary{skins}
\end{dispListing}
This also loads the package \refPkg[tikzfill]{tikzfill.image}.

In the following, general settings and options of the library are
documented.
The actual catalog of skins is found in \zvref{sec:skincatalog}.

\subsection{Style Option Keys}\label{subsec:addstyleoptions}
The following style options are applicable for all skins which
use engines of type |path|, |pathfirst|, |pathmiddle|, or |pathlast|.
Especially, the skin \refSkin{enhanced} supports \emph{all} of them
and \refSkin{standard} \emph{none}.

\begin{docTcbKey}{frame style}{=\meta{\texttt{\upshape tikz} keys}}{style, no default}
  The \meta{\texttt{\upshape tikz} keys} are used inside the \tikzname\ path command
  for drawing the \emph{frame} of the box.\\
  This option is available if the \refKey{/tcb/frame engine} is set to
  |path|, |pathfirst|, |pathmiddle|, or |pathlast|.
  It is \emph{not} available for |standard|.
\begin{exdispExample*}{frame_style}{sbs,lefthand ratio=0.66}
\tcbset{colback=red!5!white,fonttitle=\bfseries}

\begin{tcolorbox}[enhanced,title=My title,
  frame style={left color=red!75!black,
               right color=blue!75!black}]
This is a \textbf{tcolorbox}.
\tcblower
This is the lower part.
\end{tcolorbox}
\end{exdispExample*}
\end{docTcbKey}


\begin{docTcbKey}{frame style image}{=\meta{file name}}{no default, initially unset}
  Fills the frame with an external image referenced by \meta{file name}.
\begin{exdispExample*}{frame_style_image}{sbs,lefthand ratio=0.66}
\tcbset{colback=red!5!white,fonttitle=\bfseries}

\begin{tcolorbox}[enhanced,title=My title,
  frame style image=blueshade.png]
This is a \textbf{tcolorbox}.
\tcblower
This is the lower part.
\end{tcolorbox}
\end{exdispExample*}
\end{docTcbKey}

\clearpage
\begin{docTcbKey}{frame style tile}{=\marg{graphics options}\marg{file name}}{no default, initially unset}
  Fills the frame with a tile pattern based on an external image referenced by \meta{file name}.
  The \meta{graphics options} are given to the underlying \docAuxCommand*{includegraphics} command.
\begin{exdispExample*}{frame_style_tile}{sbs,lefthand ratio=0.66}
\tcbset{colback=red!5!white,coltitle=red!30!black,
  opacityback=0.75,fonttitle=\bfseries}

\begin{tcolorbox}[enhanced,title=My title,
  frame style tile={width=1cm}{pink_marble.png}]
This is a \textbf{tcolorbox}.
\tcblower
This is the lower part.
\end{tcolorbox}
\end{exdispExample*}
\end{docTcbKey}


\begin{docTcbKey}{frame hidden}{}{style, no value}
  This is a shortcut for |frame style={draw=none,fill=none}|.
  Depending on the skin, this option switches off the drawing of the
  frame.
  Alternatively, use \refKey{/tcb/frame empty}.
\begin{exdispExample*}{frame_hidden}{sbs,lefthand ratio=0.66}
\tcbset{colback=red!5!white,colframe=red!75!black,
  fonttitle=\bfseries,coltitle=black}

\begin{tcolorbox}[enhanced,title=My title,
  frame hidden]
This is a \textbf{tcolorbox}.
\tcblower
This is the lower part.
\end{tcolorbox}
\end{exdispExample*}
\end{docTcbKey}


\begin{docTcbKey}{interior style}{=\meta{\texttt{\upshape tikz} keys}}{style, no default}
  The \meta{\texttt{\upshape tikz} keys} are used inside the \tikzname\ path command
  for drawing the \emph{interior} of the box. They are used for the titled
  and for the untitled version as well.\\
  This option is available if the \refKey{/tcb/interior titled engine}
  or \refKey{/tcb/interior engine} is set to
  |path|, |pathfirst|, |pathmiddle|, or |pathlast|.
  It is \emph{not} available for |standard|.
\begin{exdispExample*}{interior_style}{sbs,lefthand ratio=0.66}
\tcbset{colframe=red!75!black,fonttitle=\bfseries}

\begin{tcolorbox}[enhanced,title=My title,
  interior style={left color=red!20!white,
                  right color=yellow!50!white}]
This is a \textbf{tcolorbox}.
\tcblower
This is the lower part.
\end{tcolorbox}
\end{exdispExample*}
\end{docTcbKey}

\clearpage
\begin{docTcbKey}{interior style image}{=\meta{file name}}{no default, initially unset}
  Fills the interior with an external image referenced by \meta{file name}.
\begin{exdispExample*}{interior_style_image}{sbs,lefthand ratio=0.66}
\tcbset{colframe=red!75!black,fonttitle=\bfseries}

\begin{tcolorbox}[enhanced,title=My title,
  interior style image=goldshade.png]
This is a \textbf{tcolorbox}.
\tcblower
This is the lower part.
\end{tcolorbox}
\end{exdispExample*}
\end{docTcbKey}


\begin{docTcbKey}{interior style tile}{=\marg{graphics options}\marg{file name}}{no default, initially unset}
  Fills the interior with a tile pattern based on an external image referenced by \meta{file name}.
  The \meta{graphics options} are given to the underlying \docAuxCommand*{includegraphics} command.
\begin{exdispExample*}{interior_style_tile}{sbs,lefthand ratio=0.66}
\tcbset{colframe=red!75!black,fonttitle=\bfseries}

\begin{tcolorbox}[enhanced,title=My title,
  interior style tile={width=2cm}{crinklepaper.png}]
This is a \textbf{tcolorbox}.
\tcblower
This is the lower part.
\end{tcolorbox}
\end{exdispExample*}
\end{docTcbKey}


\begin{docTcbKey}{interior hidden}{}{style, no value}
  This is a shortcut for |interior style={draw=none,fill=none}|.
  Depending on the skin, this option switches off the drawing of the
  interior.
  Alternatively, use \refKey{/tcb/interior empty} and/or \refKey{/tcb/interior titled empty}.
\begin{exdispExample*}{interior_hidden}{sbs,lefthand ratio=0.66}
\tcbset{frame style={top color=red!20!white,
  bottom color=red!20!white!75!black},
  fonttitle=\bfseries,coltitle=black}

\begin{tcolorbox}[enhanced,title=My title,
  interior hidden]
This is a \textbf{tcolorbox}.
\tcblower
This is the lower part.
\end{tcolorbox}
\end{exdispExample*}
\end{docTcbKey}

\clearpage
\begin{docTcbKey}{segmentation style}{=\meta{\texttt{\upshape tikz} keys}}{style, no default}
  The \meta{\texttt{\upshape tikz} keys} are used inside the \tikzname\ path command
  for drawing the \emph{segmentation} line of the box.\\
  This option is available if the \refKey{/tcb/segmentation engine}
  is set to |path|.
  It is \emph{not} available for |standard|.
\begin{exdispExample*}{segmentation_style}{sbs,lefthand ratio=0.66}
\tcbset{colback=red!5!white,colframe=red!75!black,
  fonttitle=\bfseries}

\begin{tcolorbox}[enhanced,title=My title,
  segmentation style={double=white,draw=blue,
                  double distance=1pt,solid}]
This is a \textbf{tcolorbox}.
\tcblower
This is the lower part.
\end{tcolorbox}
\end{exdispExample*}
\end{docTcbKey}


\begin{docTcbKey}{segmentation hidden}{}{style, no value}
  This is a shortcut for |segmentation style={draw=none,fill=none}|.
  Depending on the skin, this option switches off the drawing of the
  segmentation line. See also \refKey{/tcb/lower separated} which
  has the same effect for most skins.
  Alternatively, use \refKey{/tcb/segmentation empty}.
\begin{exdispExample*}{segmentation_hidden}{sbs,lefthand ratio=0.66}
\tcbset{colback=red!5!white,colframe=red!75!black,
  fonttitle=\bfseries}

\begin{tcolorbox}[title=My title,
  enhanced,segmentation hidden]
This is a \textbf{tcolorbox}.
\tcblower
This is the lower part.
\end{tcolorbox}
\end{exdispExample*}
\end{docTcbKey}


\begin{docTcbKey}{title style}{=\meta{\texttt{\upshape tikz} keys}}{style, no default}
  The \meta{\texttt{\upshape tikz} keys} are used inside the \tikzname\ path command
  for drawing the \emph{title area} of the box.\\
  This option is available if the \refKey{/tcb/title engine} is set to
  |path|, |pathfirst|, |pathmiddle|, or |pathlast|.
  It is \emph{not} available for |standard|.
\begin{exdispExample*}{title_style}{sbs,lefthand ratio=0.66}
\tcbset{colback=red!5!white,colframe=red!75!black,
  coltitle=blue!50!black,fonttitle=\bfseries}

\begin{tcolorbox}[enhanced,title=My title,
  title style={left color=blue!15!yellow,
               right color=red!85!black}]
This is a \textbf{tcolorbox}.
\tcblower
This is the lower part.
\end{tcolorbox}
\end{exdispExample*}
\end{docTcbKey}

\clearpage
\begin{docTcbKey}{title style image}{=\meta{file name}}{no default, initially unset}
  Fills the title area with an external image referenced by \meta{file name}.
\begin{exdispExample*}{title_style_image}{sbs,lefthand ratio=0.66}
\tcbset{colback=blue!5!white,colframe=blue!75!black,
  fonttitle=\bfseries}

\begin{tcolorbox}[enhanced,title=My title,
  title style image=blueshade.png]
This is a \textbf{tcolorbox}.
\tcblower
This is the lower part.
\end{tcolorbox}
\end{exdispExample*}
\end{docTcbKey}


\begin{docTcbKey}{title style tile}{=\marg{graphics options}\marg{file name}}{no default, initially unset}
  Fills the title area with a tile pattern based on an external image referenced by \meta{file name}.
  The \meta{graphics options} are given to the underlying \docAuxCommand*{includegraphics} command.
\begin{exdispExample*}{title_style_tile}{sbs,lefthand ratio=0.66}
\tcbset{colback=red!5!white,colframe=red!75!black,
  coltitle=blue!50!black,fonttitle=\bfseries}

\begin{tcolorbox}[enhanced,title=My title,
  title style tile={width=1cm}{pink_marble.png}]
This is a \textbf{tcolorbox}.
\tcblower
This is the lower part.
\end{tcolorbox}
\end{exdispExample*}
\end{docTcbKey}


\begin{docTcbKey}{title hidden}{}{style, no value}
  This is a shortcut for |title style={draw=none,fill=none}|.
  Depending on the skin, this option switches off the drawing of the
  title background. See also \refKey{/tcb/title filled} for a similar effect.
  Alternatively, use \refKey{/tcb/title empty}.
\begin{exdispExample*}{title_hidden}{sbs,lefthand ratio=0.66}
\tcbset{colback=red!5!white,colframe=red!75!black,
  fonttitle=\bfseries}

\begin{tcolorbox}[title=My title,
  enhanced,title hidden]
This is a \textbf{tcolorbox}.
\tcblower
This is the lower part.
\end{tcolorbox}
\end{exdispExample*}
\end{docTcbKey}


\clearpage


\begin{docTcbKey}[][doc new=2015-01-14]{titlerule style}{=\meta{\texttt{\upshape tikz} keys}}{style, no default}
  The \meta{\texttt{\upshape tikz} keys} are used to draw a title rule,
  i.\,e., a rule below the optional title. The width of the rule is controlled
  by \refKey{/tcb/titlerule}. It may be set directly to a smaller width
  to create mixed effects with the standard rule.
  This option is implemented as an \refKey{/tcb/underlay}. Thus, it is not
  available for \refSkin{standard} and \refSkin{standard jigsaw}, but for
  all other skins, e.\,g.\ \refSkin{enhanced}.
  As an underlay, this option can be used multiple times and is removed
  by \refKey{/tcb/no underlay}.
\begin{exdispExample*}{titlerule_style_1}{sbs,lefthand ratio=0.66}
\begin{tcolorbox}[enhanced,
  colback=red!5!white,colframe=red!75!black,
  colbacktitle=red!50!yellow,fonttitle=\bfseries,
  title=My title,
  titlerule=1mm,
  titlerule style=yellow  ]
This is a \textbf{tcolorbox}.
\end{tcolorbox}
\end{exdispExample*}

\begin{exdispExample*}{titlerule_style_2}{sbs,lefthand ratio=0.66}
\begin{tcolorbox}[enhanced,
  colback=red!5!white,colframe=red!75!black,
  colbacktitle=red!50!yellow,fonttitle=\bfseries,
  title=My title,
  titlerule=1mm,
  titlerule style={yellow,line width=0.5mm}  ]
This is a \textbf{tcolorbox}.
\end{tcolorbox}
\end{exdispExample*}

\begin{exdispExample*}{titlerule_style_3}{sbs,lefthand ratio=0.66}
\begin{tcolorbox}[enhanced,
  colback=red!10!white,colframe=red!75!black,
  colbacktitle=red!50!yellow,fonttitle=\bfseries,
  frame hidden,
  title=My title,
  boxrule=0pt,titlerule=1mm,
  titlerule style=red!50!black  ]
This is a \textbf{tcolorbox}.
\end{tcolorbox}
\end{exdispExample*}

\begin{exdispExample*}{titlerule_style_4}{sbs,lefthand ratio=0.66}
%\usetikzlibrary{arrows.meta}
\begin{tcolorbox}[empty,
  coltitle=red!75!black,fonttitle=\bfseries,
  borderline horizontal={0.5mm}{0pt}{red!50!white},
  title=My title,
  titlerule style={red,
    arrows = {Hooks[arc=270]-Hooks[arc=270]}} ]
This is a \textbf{tcolorbox}.
\end{tcolorbox}
\end{exdispExample*}
\end{docTcbKey}

\clearpage

The combined \tikzname\ style applied to frame, interior, and title
background can used by authors in customizing code.

\begin{docTikzKey}{tcb fill frame}{}{style, no value}
This is a \tikzname\ style which is finally applied to the \emph{frame}
of the box.

\begin{exdispExample*}{tcb_fill_frame}{sbs,lefthand ratio=0.66}
% \tcbuselibrary{hooks}
\tcbset{enhanced,colback=red!5!white,
  colframe=red!75!black,fonttitle=\bfseries,
  frame code app={\path[tcb fill frame]
    ([yshift=-2mm]frame.north)
        circle (8mm); }  }

\begin{tcolorbox}[title=My title]
This is a \textbf{tcolorbox}.
\tcblower
This is the lower part.
\end{tcolorbox}
\end{exdispExample*}
\end{docTikzKey}


\begin{docTikzKey}{tcb fill interior}{}{style, no value}
This is a \tikzname\ style which is finally applied to the \emph{interior}
of the box.

\begin{exdispExample*}{tcb_fill_interior}{sbs,lefthand ratio=0.66}
% \tcbuselibrary{hooks}
\tcbset{enhanced,colback=red!5!white,
  colframe=red!75!black,fonttitle=\bfseries,
  interior titled code app={\path[tcb fill interior]
      ([yshift=-0.1pt]interior.north east)
    --([yshift=3pt]interior.north)
    --([yshift=-0.1pt]interior.north west)
    --cycle;}  }

\begin{tcolorbox}[title=My title]
This is a \textbf{tcolorbox}.
\tcblower
This is the lower part.
\end{tcolorbox}
\end{exdispExample*}
\end{docTikzKey}


\begin{docTikzKey}{tcb fill title}{}{style, no value}
This is a \tikzname\ style which is finally applied to the \emph{title area}
of the box.

\begin{exdispExample*}{tcb_fill_title}{sbs,lefthand ratio=0.66}
% \tcbuselibrary{hooks}
\tcbset{enhanced,colback=red!5!white,
  colframe=red!75!black,fonttitle=\bfseries,
  colbacktitle=blue!75!black,
  title code app={\path[tcb fill title]
    (title) circle (5mm); }  }

\begin{tcolorbox}[title=My title]
This is a \textbf{tcolorbox}.
\tcblower
This is the lower part.
\end{tcolorbox}
\end{exdispExample*}
\end{docTikzKey}

\clearpage

\begin{docTikzKey}[][doc new=2025-07-07]{tcb fill lower bicolor}{}{style, no value}
This is a \tikzname\ style which is finally applied to the
\emph{interior lower part} of the box, when the \refSkin{bicolor} skin is used.

\begin{exdispExample*}{tcb_fill_lower_bicolor}{sbs,lefthand ratio=0.66}
% \tcbuselibrary{hooks}
% \usetikzlibrary{decorations.pathmorphing}
\tcbset{bicolor,colback=red!5!white,
  colbacklower=blue!15!white,
  colframe=red!75!black,fonttitle=\bfseries,
  segmentation code app={
    \path[tcb fill lower bicolor,
          decoration={zigzag,amplitude=1mm}]
      decorate {
          ([yshift=-0.1pt]segmentation.east)
        --([yshift=-0.1pt]segmentation.west) }
      --cycle;}  }

\begin{tcolorbox}[title=My title]
This is a \textbf{tcolorbox}.
\tcblower
This is the lower part.
\end{tcolorbox}
\end{exdispExample*}
\end{docTikzKey}



\clearpage
\subsection{Boxed Title Option Keys}\label{subsec:skinboxedtitle}

\subsubsection{Boxed Title Placement}
The following options place the title text into an own \refCom{tcbox}.
This boxed title can be customized independently from the main box using
\refKey{/tcb/boxed title style}.
The placement can be influenced by \meta{boxtitle options}.

\begin{docTcbKey}{attach boxed title to top left}{\colOpt{=\marg{boxtitle options}}}{style, default empty}
The title is boxed with a \refCom{tcbox} and attached to
the top left corner of the main box.
\begin{exdispExample*}{attach_boxed_title_to_top_left}{sbs,lefthand ratio=0.66}
\begin{tcolorbox}[enhanced,title=My title,
  attach boxed title to top left]
  This is a \textbf{tcolorbox}.
\end{tcolorbox}
\end{exdispExample*}
\end{docTcbKey}

\begin{docTcbKey}[][doc new=2021-07-30]{attach boxed title to top text left}{\colOpt{=\marg{boxtitle options}}}{style, default empty}
The title is boxed with a \refCom{tcbox} and attached to
the top left corner of the main box
and shifted to match title text and box text.
\begin{exdispExample*}{attach_boxed_title_to_top_text_left}{sbs,lefthand ratio=0.66}
\begin{tcolorbox}[enhanced,title=My title,
  attach boxed title to top text left]
  This is a \textbf{tcolorbox}.
\end{tcolorbox}
\end{exdispExample*}
\end{docTcbKey}

\begin{docTcbKey}{attach boxed title to top center}{\colOpt{=\marg{boxtitle options}}}{style, default empty}
The title is boxed with a \refCom{tcbox} and attached to
the top of the main box.
\begin{exdispExample*}{attach_boxed_title_to_top_center}{sbs,lefthand ratio=0.66}
\begin{tcolorbox}[enhanced,title=My title,
  attach boxed title to top center]
  This is a \textbf{tcolorbox}.
\end{tcolorbox}
\end{exdispExample*}
\end{docTcbKey}

\begin{docTcbKey}[][doc new=2021-07-30]{attach boxed title to top text right}{\colOpt{=\marg{boxtitle options}}}{style, default empty}
The title is boxed with a \refCom{tcbox} and attached to
the top right corner of the main box
and shifted to match title text and box text.
\begin{exdispExample*}{attach_boxed_title_to_top_text_right}{sbs,lefthand ratio=0.66}
\begin{tcolorbox}[enhanced,title=My title,
  halign=right,
  attach boxed title to top text right]
  This is a \textbf{tcolorbox}.
\end{tcolorbox}
\end{exdispExample*}
\end{docTcbKey}


\begin{docTcbKey}{attach boxed title to top right}{\colOpt{=\marg{boxtitle options}}}{style, default empty}
The title is boxed with a \refCom{tcbox} and attached to
the top right corner of the main box.
\begin{exdispExample*}{attach_boxed_title_to_top_right}{sbs,lefthand ratio=0.66}
\begin{tcolorbox}[enhanced,title=My title,
  attach boxed title to top right]
  This is a \textbf{tcolorbox}.
\end{tcolorbox}
\end{exdispExample*}
\end{docTcbKey}

\clearpage

\begin{docTcbKey}{attach boxed title to bottom left}{\colOpt{=\marg{boxtitle options}}}{style, default empty}
The title is boxed with a \refCom{tcbox} and attached to
the bottom left corner of the main box.
\begin{exdispExample*}{attach_boxed_title_to_bottom_left}{sbs,lefthand ratio=0.66}
\begin{tcolorbox}[enhanced,title=My title,
  attach boxed title to bottom left]
  This is a \textbf{tcolorbox}.
\end{tcolorbox}
\end{exdispExample*}
\end{docTcbKey}


\begin{docTcbKey}[][doc new=2021-07-30]{attach boxed title to bottom text left}{\colOpt{=\marg{boxtitle options}}}{style, default empty}
The title is boxed with a \refCom{tcbox} and attached to
the bottom left corner of the main box
and shifted to match title text and box text.
Note that this matches the \emph{upper} part, even, if there is a \emph{lower} part.
\begin{exdispExample*}{attach_boxed_title_to_bottom_text_left}{sbs,lefthand ratio=0.66}
\begin{tcolorbox}[enhanced,title=My title,
  attach boxed title to bottom text left]
  This is a \textbf{tcolorbox}.
\end{tcolorbox}
\end{exdispExample*}
\end{docTcbKey}


\begin{docTcbKey}{attach boxed title to bottom center}{\colOpt{=\marg{boxtitle options}}}{style, default empty}
The title is boxed with a \refCom{tcbox} and attached to
the bottom of the main box.
\begin{exdispExample*}{attach_boxed_title_to_bottom_center}{sbs,lefthand ratio=0.66}
\begin{tcolorbox}[enhanced,title=My title,
  attach boxed title to bottom center]
  This is a \textbf{tcolorbox}.
\end{tcolorbox}
\end{exdispExample*}
\end{docTcbKey}


\begin{docTcbKey}[][doc new=2021-07-30]{attach boxed title to bottom text right}{\colOpt{=\marg{boxtitle options}}}{style, default empty}
The title is boxed with a \refCom{tcbox} and attached to
the bottom right corner of the main box
and shifted to match title text and box text.
Note that this matches the \emph{upper} part, even, if there is a \emph{lower} part.
\begin{exdispExample*}{attach_boxed_title_to_bottom_text_right}{sbs,lefthand ratio=0.66}
\begin{tcolorbox}[enhanced,title=My title,
  halign=right,
  attach boxed title to bottom text right]
  This is a \textbf{tcolorbox}.
\end{tcolorbox}
\end{exdispExample*}
\end{docTcbKey}


\begin{docTcbKey}{attach boxed title to bottom right}{\colOpt{=\marg{boxtitle options}}}{style, default empty}
The title is boxed with a \refCom{tcbox} and attached to
the bottom right corner of the main box.
\begin{exdispExample*}{attach_boxed_title_to_bottom_right}{sbs,lefthand ratio=0.66}
\begin{tcolorbox}[enhanced,title=My title,
  attach boxed title to bottom right]
  This is a \textbf{tcolorbox}.
\end{tcolorbox}
\end{exdispExample*}
\end{docTcbKey}


\clearpage
\begin{docTcbKey}[][doc new=2016-02-26]{attach boxed title to top}{\colOpt{=\marg{boxtitle options}}}{style, default empty}
  This is a convenient style to mimic a standard title.
  It uses \refKey{/tcb/attach boxed title to top center},
  \refKey{/tcb/minipage boxed title}, and sizes the boxed title to match
  the base box.
\begin{dispExample*}{sbs,lefthand ratio=0.66}
\begin{tcolorbox}[enhanced,title=My title,
  attach boxed title to top,
  boxed title style={colframe=red}]
  This is a \textbf{tcolorbox}.
\end{tcolorbox}
\end{dispExample*}
\end{docTcbKey}

\begin{docTcbKey}[][doc new=2016-02-26]{attach boxed title to top*}{\colOpt{=\marg{boxtitle options}}}{style, default empty}
  In contrast to \refKey{/tcb/attach boxed title to top}, this style
  uses smaller left and right rules to avoid previewer glitches.
  Typically, one would not use different colors for the frame as in the
  example below.
\begin{dispExample*}{sbs,lefthand ratio=0.66}
\begin{tcolorbox}[enhanced,title=My title,
  attach boxed title to top*,
  boxed title style={colframe=red}]
  This is a \textbf{tcolorbox}.
\end{tcolorbox}
\end{dispExample*}
\end{docTcbKey}

\begin{docTcbKey}[][doc new=2016-02-26]{attach boxed title to bottom}{\colOpt{=\marg{boxtitle options}}}{style, default empty}
  This is a convenient style to produce a standard-like title at the bottom
  of the box.
  It uses \refKey{/tcb/attach boxed title to bottom center},
  \refKey{/tcb/minipage boxed title}, and sizes the boxed title to match
  the base box.
\begin{dispExample*}{sbs,lefthand ratio=0.66}
\begin{tcolorbox}[enhanced,title=My title,
  attach boxed title to bottom,
  boxed title style={colframe=red}]
  This is a \textbf{tcolorbox}.
\end{tcolorbox}
\end{dispExample*}
\end{docTcbKey}

\begin{docTcbKey}[][doc new=2016-02-26]{attach boxed title to bottom*}{\colOpt{=\marg{boxtitle options}}}{style, default empty}
  In contrast to \refKey{/tcb/attach boxed title to top}, this style
  uses smaller left and right rules to avoid previewer glitches.
\begin{dispExample*}{sbs,lefthand ratio=0.66}
\begin{tcolorbox}[enhanced,title=My title,
  attach boxed title to bottom*]
  This is a \textbf{tcolorbox}.
\end{tcolorbox}
\end{dispExample*}
\end{docTcbKey}

\begin{docTcbKey}[][doc new=2016-02-26]{flip title}{\colOpt{=\marg{options}}}{style, default empty}
  This style combines \refKey{/tcb/attach boxed title to bottom*}
  with \refKey{/tcb/boxed title style}. The \meta{options} are given to
  \refKey{/tcb/boxed title style}.
\begin{dispExample*}{sbs,lefthand ratio=0.66}
\begin{tcolorbox}[tile,flip title={sharp corners},
  title=My title,colback=red!10,
  colbacktitle=red!75!black]
  This is a \textbf{tcolorbox}.
\end{tcolorbox}
\end{dispExample*}
\end{docTcbKey}

\clearpage

\subsubsection{Options for the Boxed Title Placement}
The \meta{boxtitle options} of the keys described above are shift values.
The dimensions of the boxed title are stored into two macros
\docAuxCommand{tcboxedtitleheight} and
\docAuxCommand{tcboxedtitlewidth}. These macros can be used inside the
following \meta{boxtitle options}:

\begin{boxTcbKey}{xshift}{=\meta{length}}{no default, initially |0pt|}
The boxed title is shifted by \meta{length} in the horizontal direction.
\begin{exdispExample*}{xshift}{sbs,lefthand ratio=0.66}
\begin{tcolorbox}[enhanced,title=My title,
  attach boxed title to top left={xshift=-2mm},
  boxed title style={size=small,colback=blue}]
  This is a \textbf{tcolorbox}.
\end{tcolorbox}
\end{exdispExample*}
\end{boxTcbKey}

\begin{boxTcbKey}{yshift}{=\meta{length}}{no default, initially |0pt|}
The boxed title is shifted by \meta{length} in the vertical direction.
\begin{exdispExample*}{yshift}{sbs,lefthand ratio=0.66}
\begin{tcolorbox}[enhanced,title=My title,
  attach boxed title to top center=
    {yshift=-\tcboxedtitleheight/2},
  boxed title style={size=small,colback=blue}]
  This is a \textbf{tcolorbox}.
\end{tcolorbox}
\end{exdispExample*}
\end{boxTcbKey}

\begin{boxTcbKey}{yshifttext}{=\meta{length}}{no default, initially |0pt|}
The text inside the main box is shifted by \meta{length} to give room for e.\,g.\ a sunken title.
\begin{exdispExample*}{yshifttext}{sbs,lefthand ratio=0.66}
\begin{tcolorbox}[enhanced,title=My title,
  attach boxed title to top center=
    {yshift=-3mm,yshifttext=-1mm},
  boxed title style={size=small,colback=blue}]
  This is a \textbf{tcolorbox}.
\end{tcolorbox}
\end{exdispExample*}
\end{boxTcbKey}

\begin{boxTcbKey}{yshift*}{=\meta{length}}{no default, initially |0pt|}
Sets \refKey{/tcb/boxtitle/yshift} and \refKey{/tcb/boxtitle/yshifttext}
the same time.\\
\refKey{/tcb/boxtitle/yshifttext} is only set if necessary.
\begin{exdispExample*}{yshiftstar}{sbs,lefthand ratio=0.66}
\begin{tcolorbox}[enhanced,title=My title,
  attach boxed title to top center={yshift*=-3mm},
  boxed title style={size=small,colback=blue}]
  This is a \textbf{tcolorbox}.
\end{tcolorbox}
\end{exdispExample*}
\end{boxTcbKey}

\begin{marker}
The bounding box of the resulting total |tcolorbox| is adapted automatically to the
\emph{vertical} dimensions of the boxed title. Possible horizontal enlargements
are \emph{not} automatically computed.
\end{marker}


\begin{exdispExample*}{boxed_title_example_1}{sbs,lefthand ratio=0.66}
\begin{tcolorbox}[enhanced,title=My title,
  attach boxed title to top left=
    {xshift=-2mm,yshift=-2mm},
  boxed title style={size=small,colback=blue},
  show bounding box]
  This is a \textbf{tcolorbox}.
\end{tcolorbox}
\end{exdispExample*}


\clearpage
\subsubsection{Options for the Boxed Title Box}
\begin{marker}
  The boxed title options are implemented as an underlay, see \zvref{subsec:skinunderlay}.
  Therefore, a boxed title is not drawn, if a skin does not support underlays
  like \refSkin{standard}. Still, the room for the boxed
  titles gets reserved in these cases.
\end{marker}

\begin{marker}
  A \tikzname\ node |title| is produced by a boxed title which can be used
  inside \refKey{/tcb/frame code}, \refKey{/tcb/interior code},
  underlays, overlays, and finishes.
\end{marker}

\begin{marker}
  A boxed title is almost always the first underlay. The only exceptions are
  underlays defined by \refKey{/tcb/underlay boxed title} which are drawn
  before. Additionally, underlays defined by \refKey{/tcb/underlay boxed title}
  are only drawn, if a boxed title is actually set. They are ignored, if
  there is no boxed title.
\end{marker}



\begin{docTcbKey}[][doc new=2016-02-26]{boxed title size}{=\meta{size}}{no default, initially |title|}
  This setting defines the basic size for the title box. Further settings
  can be applied using \refKey{/tcb/boxed title style}.
  Feasible values for \meta{size} are:
  \begin{itemize}
  \item\docValue{title}: Sets the size according to \refKey{/tcb/size}|=|\docValue{title}.
  \item\docValue{standard}: No size setting. Typically, this is identical to
    \refKey{/tcb/size}|=|\docValue{normal}.
  \item\docValue{copy}: The size values for a title of the base box are copied
    for the title box.
  \end{itemize}

\begin{dispExample}
% \tcbuselibrary{raster}
\begin{tcbraster}[raster columns=3,enhanced,boxrule=0.4pt,
    title=My title,attach boxed title to top center]
  \begin{tcolorbox}[boxed title size=title]
    This is a \textbf{tcolorbox}.
  \end{tcolorbox}
  \begin{tcolorbox}[boxed title size=standard]
    This is a \textbf{tcolorbox}.
  \end{tcolorbox}
  \begin{tcolorbox}[boxed title size=copy]
    This is a \textbf{tcolorbox}.
  \end{tcolorbox}
\end{tcbraster}
\end{dispExample}

\end{docTcbKey}


\clearpage
\begin{docTcbKey}[][doc updated=2016-02-26]{boxed title style}{=\meta{options}}{style, initially empty}
By default, a boxed title is dimensioned with \refKey{/tcb/size}|=|\docValue*{title}
and inherits the \refKey{/tcb/skin} and \refKey{/tcb/colframe} of the main box.
Also, the \refKey{/tcb/colback} is inherited from the main \refKey{/tcb/colbacktitle}.
Font and color of the title text are set as usual.
All other \meta{options} are set by the \refKey{/tcb/boxed title style} key.
Since a boxed title is set by \refCom{tcbox}, all |tcolorbox| options are
applicable here. If \refKey{/tcb/boxed title style} is used several times,
the \meta{options} are appended.

\begin{exdispExample*}{boxed_title_style_1}{sbs,lefthand ratio=0.66}
\begin{tcolorbox}[enhanced,title=My title,
  fonttitle=\bfseries,coltitle=green!25!black,
  attach boxed title to top center=
    {yshift=-2mm,yshifttext=-1mm},
  boxed title style={colframe=green!75!black,
    colback=yellow!50!green}]
  This is a \textbf{tcolorbox}.
\end{tcolorbox}
\end{exdispExample*}


\begin{exdispExample*}{boxed_title_style_2}{sbs,lefthand ratio=0.66}
\begin{tcolorbox}[enhanced,title=My title,
  colframe=red!50!black,colback=red!10!white,
  arc=1mm,colbacktitle=red!10!white,
  fonttitle=\bfseries,coltitle=red!50!black,
  attach boxed title to top text left=
    {yshift=-0.50mm},
  boxed title style={skin=enhancedfirst jigsaw,
    size=small,arc=1mm,bottom=-1mm,
    interior style={fill=none,
      top color=red!30!white,
      bottom color=red!20!white}}]
  This is a \textbf{tcolorbox}.
\end{tcolorbox}
\end{exdispExample*}

\begin{exdispExample}{boxed_title_style_3}
\begin{tcolorbox}[enhanced,title=My title,
  colframe=blue!50!black,colback=blue!10!white,colbacktitle=blue!5!yellow!10!white,
  fonttitle=\bfseries,coltitle=black,attach boxed title to top center=
    {yshift=-0.25mm-\tcboxedtitleheight/2,yshifttext=2mm-\tcboxedtitleheight/2},
  boxed title style={boxrule=0.5mm,
    frame code={ \path[tcb fill frame] ([xshift=-4mm]frame.west)
    -- (frame.north west) -- (frame.north east) -- ([xshift=4mm]frame.east)
    -- (frame.south east) -- (frame.south west) -- cycle; },
    interior code={ \path[tcb fill interior] ([xshift=-2mm]interior.west)
    -- (interior.north west) -- (interior.north east)
    -- ([xshift=2mm]interior.east) -- (interior.south east) -- (interior.south west)
    -- cycle;}  }]
  \lipsum[2]
\end{tcolorbox}
\end{exdispExample}


\begin{exdispExample}{boxed_title_style_4}
% \usepackage{varwidth}
\newtcolorbox{mybox}[2][]{enhanced,skin=enhancedlast jigsaw,
  attach boxed title to top left={xshift=-4mm,yshift=-0.5mm},
  fonttitle=\bfseries\sffamily,varwidth boxed title=0.7\linewidth,
  colbacktitle=blue!45!white,colframe=red!50!black,
  interior style={top color=blue!10!white,bottom color=red!10!white},
  boxed title style={empty,arc=0pt,outer arc=0pt,boxrule=0pt},
  underlay boxed title={
    \fill[blue!45!white] (title.north west) -- (title.north east)
      -- +(\tcboxedtitleheight-1mm,-\tcboxedtitleheight+1mm)
      -- ([xshift=4mm,yshift=0.5mm]frame.north east) -- +(0mm,-1mm)
      -- (title.south west) -- cycle;
    \fill[blue!45!white!50!black] ([yshift=-0.5mm]frame.north west)
      -- +(-0.4,0) -- +(0,-0.3) -- cycle;
    \fill[blue!45!white!50!black] ([yshift=-0.5mm]frame.north east)
      -- +(0,-0.3) -- +(0.4,0) -- cycle;  },
  title={#2},#1}

\begin{mybox}{My title}
  \lipsum[2]
\end{mybox}
\end{exdispExample}


\begin{exdispExample}{boxed_title_style_5}
% \usepackage{varwidth}
\newtcolorbox{mybox}[2][]{enhanced,
  attach boxed title to top left={xshift=1cm,yshift=-2mm},
  fonttitle=\bfseries,varwidth boxed title=0.7\linewidth,
  colbacktitle=green!45!white,coltitle=green!10!black,colframe=green!50!black,
  interior style={top color=yellow!10!white,bottom color=green!10!white},
  boxed title style={boxrule=0.75mm,colframe=white,
    borderline={0.1mm}{0mm}{green!50!black},
    borderline={0.1mm}{0.75mm}{green!50!black},
    interior style={top color=green!10!white,bottom color=green!10!white,
      middle color=green!50!white},
    drop fuzzy shadow},
  title={#2},#1}

\begin{mybox}{My title}
  \lipsum[2]
\end{mybox}
\end{exdispExample}


\begin{exdispExample}{boxed_title_style_6}
\newtcolorbox{flipbox}[2][]{
  enhanced,colframe=blue!50!black,colback=yellow!5,fonttitle=\bfseries,
  flip title={interior hidden},title={#2},#1}

\begin{flipbox}{My title}
\lipsum[2]
\end{flipbox}
\end{exdispExample}


\begin{exdispExample}{boxed_title_style_7}
% \usepackage{varwidth}
\newtcolorbox{mybox}[2][]{skin=enhancedlast jigsaw,interior hidden,
  boxsep=0pt,top=0pt,colframe=red,coltitle=red!50!black,
  fonttitle=\bfseries\sffamily,
  attach boxed title to bottom center,
  boxed title style={empty,boxrule=0.5mm},
  varwidth boxed title=0.5\linewidth,
  underlay boxed title={
    \draw[white,line width=0.5mm]
      ([xshift=0.3mm-\tcboxedtitleheight*2,yshift=0.3mm]title.north west)
      --([xshift=-0.3mm+\tcboxedtitleheight*2,yshift=0.3mm]title.north east);
    \path[draw=red,top color=white,bottom color=red!50!white,line width=0.5mm]
    ([xshift=0.25mm-\tcboxedtitleheight*2,yshift=0.25mm]title.north west)
    cos +(\tcboxedtitleheight,-\tcboxedtitleheight/2)
    sin +(\tcboxedtitleheight,-\tcboxedtitleheight/2)
    -- ([xshift=0.25mm,yshift=0.25mm]title.south west)
    -- ([yshift=0.25mm]title.south east)
    cos +(\tcboxedtitleheight,\tcboxedtitleheight/2)
    sin +(\tcboxedtitleheight,\tcboxedtitleheight/2); },
  title={#2},#1}

\begin{mybox}{My title}
  \lipsum[2]
\end{mybox}
\end{exdispExample}


\begin{exdispExample}{boxed_title_style_8}
% \usepackage{varwidth}
\newtcolorbox{mybox}[2][]{enhanced,
  before skip=2mm,after skip=2mm,
  colback=black!5,colframe=black!50,boxrule=0.2mm,
  attach boxed title to top left={xshift=1cm,yshift*=1mm-\tcboxedtitleheight},
  varwidth boxed title*=-3cm,
  boxed title style={frame code={
    \path[fill=tcbcolback!30!black]
      ([yshift=-1mm,xshift=-1mm]frame.north west)
        arc[start angle=0,end angle=180,radius=1mm]
      ([yshift=-1mm,xshift=1mm]frame.north east)
        arc[start angle=180,end angle=0,radius=1mm];
    \path[left color=tcbcolback!60!black,right color=tcbcolback!60!black,
      middle color=tcbcolback!80!black]
      ([xshift=-2mm]frame.north west) -- ([xshift=2mm]frame.north east)
      [rounded corners=1mm]-- ([xshift=1mm,yshift=-1mm]frame.north east)
      -- (frame.south east) -- (frame.south west)
      -- ([xshift=-1mm,yshift=-1mm]frame.north west)
      [sharp corners]-- cycle;
    },interior engine=empty,
  },
  fonttitle=\bfseries,
  title={#2},#1}

\begin{mybox}[colbacktitle=green]{My title}
\lipsum[2]
\end{mybox}
\begin{mybox}[colbacktitle=red]{My title}
\lipsum[3]
\end{mybox}
\end{exdispExample}
\end{docTcbKey}



\begin{docTcbKey}[][doc new=2016-02-26]{no boxed title style}{}{style, initially set}
  Removes all options which were set by \refKey{/tcb/boxed title style}.
\end{docTcbKey}


\clearpage
\begin{docTcbKey}{hbox boxed title}{}{no value, initially set}
The title text content is captured with a horizontal box.
Especially, there are no linebreak possible.
\begin{exdispExample*}{hbox_boxed_title}{sbs,lefthand ratio=0.66}
\newtcolorbox{mybox}[1]{hbox boxed title,
  enhanced,attach boxed title to top center=
    {yshift=-3mm,yshifttext=-1mm},
  boxed title style={size=small,colback=red},
  title={#1}}

\begin{mybox}{Short title}
  This is a \textbf{tcolorbox}.
\end{mybox}\bigskip

\begin{mybox}{This title is not really very short}
  This is a \textbf{tcolorbox}.
\end{mybox}
\end{exdispExample*}
\end{docTcbKey}


\begin{docTcbKey}{minipage boxed title}{\colOpt{=\meta{length}}}{initially unset}
The title text content is captured with a minipage with a width of \meta{length}.
By default, the resulting boxed title is somewhat smaller than the main box.
\begin{exdispExample*}{minipage_boxed_title}{sbs,lefthand ratio=0.66}
\newtcolorbox{mybox}[1]{minipage boxed title,
  enhanced,attach boxed title to top center=
    {yshift=-3mm,yshifttext=-1mm},
  boxed title style={size=small,colback=red},
  center title,title={#1}}

\begin{mybox}{Short title}
  This is a \textbf{tcolorbox}.
\end{mybox}\bigskip

\begin{mybox}{This title is not really very short}
  This is a \textbf{tcolorbox}.
\end{mybox}
\end{exdispExample*}
\end{docTcbKey}


\begin{docTcbKey}{minipage boxed title*}{\colOpt{=\meta{length}}}{initially unset}
The title text content is captured with a minipage with a width of main box width plus \meta{length}.
By default, the resulting boxed title is somewhat smaller than the main box.
\begin{exdispExample*}{minipage_boxed_title_star}{sbs,lefthand ratio=0.66}
\newtcolorbox{mybox}[1]{minipage boxed title*=-2cm,
  enhanced,attach boxed title to top center=
    {yshift=-3mm,yshifttext=-1mm},
  boxed title style={size=small,colback=red},
  center title,title={#1}}

\begin{mybox}{Short title}
  This is a \textbf{tcolorbox}.
\end{mybox}\bigskip

\begin{mybox}{This title is not really very short}
  This is a \textbf{tcolorbox}.
\end{mybox}
\end{exdispExample*}
\end{docTcbKey}


\clearpage
\begin{docTcbKey}{tikznode boxed title}{=\meta{options}}{initially unset}
The title text content is captured with a \tikzname\ node with given \tikzname\ \meta{options}.
The text is centered by default
\begin{exdispExample*}{tikznode_boxed_title}{sbs,lefthand ratio=0.66}
\newtcolorbox{mybox}[1]{tikznode boxed title,
  enhanced,attach boxed title to top center=
    {yshift=-3mm,yshifttext=-1mm},
  boxed title style={size=small,colback=red},
  title={#1}}

\begin{mybox}{Short title}
  This is a \textbf{tcolorbox}.
\end{mybox}\bigskip

\begin{mybox}{This title\\is not really\\very short}
  This is a \textbf{tcolorbox}.
\end{mybox}
\end{exdispExample*}
\end{docTcbKey}


\begin{docTcbKey}{varwidth boxed title}{\colOpt{=\meta{length}}}{initially unset}
The title text content is captured with a |varwidth| environment with a width of \meta{length}.
This style needs the \refPkg{varwidth} package \cite{arseneau:2011a} to be loaded manually.
By default, the resulting boxed title is somewhat smaller than the main box.
\begin{exdispExample*}{varwidth_boxed_title}{sbs,lefthand ratio=0.66}
% \usepackage{varwidth}
\newtcolorbox{mybox}[1]{varwidth boxed title,
  enhanced,attach boxed title to top center=
    {yshift=-3mm,yshifttext=-1mm},
  boxed title style={size=small,colback=red},
  center title,title={#1}}

\begin{mybox}{Short title}
  This is a \textbf{tcolorbox}.
\end{mybox}\bigskip

\begin{mybox}{This title is not really very short}
  This is a \textbf{tcolorbox}.
\end{mybox}
\end{exdispExample*}
\end{docTcbKey}


\begin{docTcbKey}{varwidth boxed title*}{\colOpt{=\meta{length}}}{initially unset}
The title text content is captured with a |varwidth| environment with a width of main box width plus \meta{length}.
This style needs the \refPkg{varwidth} package \cite{arseneau:2011a} to be loaded manually.
By default, the resulting boxed title is somewhat smaller than the main box.
\begin{exdispExample*}{varwidth_boxed_title_star}{sbs,lefthand ratio=0.66}
% \usepackage{varwidth}
\newtcolorbox{mybox}[1]{varwidth boxed title*=-2cm,
  enhanced,attach boxed title to top center=
    {yshift=-3mm,yshifttext=-1mm},
  boxed title style={size=small,colback=red},
  center title,title={#1}}

\begin{mybox}{Short title}
  This is a \textbf{tcolorbox}.
\end{mybox}\bigskip

\begin{mybox}{This title is not really very short}
  This is a \textbf{tcolorbox}.
\end{mybox}
\end{exdispExample*}
\end{docTcbKey}


\clearpage
\subsection{Watermark Option Keys}\label{subsec:watermarks}
The skin \refSkin{standard} does not support these watermarks,
but all other skins, e.\,g.\ \refSkin{enhanced}.

\begin{marker}
The watermark options rely on the more general overlay options described in
Section \ref{subsec:overlays} from page \pageref{subsec:overlays}.
Therefore, \emph{watermarks} and \emph{overlays} cannot be used mixed.
But a mixture is possible with the \mylib{hooks} library, see Section \ref{sec:hooks}.
\end{marker}


\begin{docTcbKey}{watermark text}{=\meta{text}}{no default, initially unset}
  Writes some \meta{text} in the center of the interior region of a |tcolorbox|.
  This \meta{text} is written \emph{after} the
  frame and interior are drawn and \emph{before} the text content is drawn.
  It is zoomed or stretched according the values of
  \refKey{/tcb/watermark zoom} or \refKey{/tcb/watermark stretch}.
\begin{dispExample}
\tcbset{colback=red!5!white,colframe=red!75!black,fonttitle=\bfseries}

\begin{tcolorbox}[enhanced,title=My title,watermark text=My Watermark]
\lipsum[1]
\tcblower
\lipsum[2]
\end{tcolorbox}
\end{dispExample}
\end{docTcbKey}

\enlargethispage*{1cm}

\begin{docTcbKey}{watermark text on}{=\meta{part} is \meta{text}}{no default, initially unset}
  This option writes some \meta{text} in the center of the interior region of a |tcolorbox|
  as described for \refKey{/tcb/watermark text}.
  But this is done only for boxes named \meta{part} of a break sequence, see
  \refKey{/tcb/breakable}.\\ 
  Feasible values for \meta{part} are:
  \begin{itemize}
  \item\docValue{broken}: all broken box parts,
  \item\docValue{unbroken}: unbroken boxes only,
  \item\docValue{first}: first parts of a break sequence,
  \item\docValue{middle}: middle parts of a break sequence,
  \item\docValue{last}: last parts of a break sequence,
  \item\docValue{unbroken and first}: unbroken boxes and first parts of a break sequence,
  \item\docValue{middle and last}: middle and last parts of a break sequence.
  \item\docValue{first and middle}: first and middle parts of a break sequence.
  \end{itemize}
\end{docTcbKey}


\clearpage


\begin{docTcbKey}{watermark graphics}{=\meta{file name}}{no default, initially unset}
  Draws an external picture referenced by \meta{file name}
  in the center of the interior region of a |tcolorbox|.
  The picture is drawn \emph{after} the
  frame and interior are drawn and \emph{before} the text content is drawn.
  It is zoomed or stretched according the values of
  \refKey{/tcb/watermark zoom} or \refKey{/tcb/watermark stretch}.
\begin{dispExample}
\tcbset{colback=red!5!white,colframe=red!75!black,fonttitle=\bfseries}

\begin{tcolorbox}[enhanced,title=My title,watermark graphics=Basilica_5.png,
  watermark opacity=0.15]
\lipsum[1-2]
\tcblower
This example uses a public domain picture from\\
\url{http://commons.wikimedia.org/wiki/File:Basilica_5.png}
\end{tcolorbox}
\end{dispExample}
\end{docTcbKey}


\begin{docTcbKey}{watermark graphics on}{=\meta{part} is \meta{file name}}{no default, initially unset}
  This option draws a picture referenced by \meta{file name} in the center of the interior region of a |tcolorbox|
  as described for \refKey{/tcb/watermark graphics}.
  But this is done only for boxes named \meta{part} of a break sequence, see
  \refKey{/tcb/breakable}.\\ 
  Feasible values for \meta{part} are:
  \begin{itemize}
  \item\docValue{broken}: all broken box parts,
  \item\docValue{unbroken}: unbroken boxes only,
  \item\docValue{first}: first parts of a break sequence,
  \item\docValue{middle}: middle parts of a break sequence,
  \item\docValue{last}: last parts of a break sequence,
  \item\docValue{unbroken and first}: unbroken boxes and first parts of a break sequence,
  \item\docValue{middle and last}: middle and last parts of a break sequence.
  \end{itemize}
\end{docTcbKey}



\clearpage
\begin{docTcbKey}{watermark tikz}{=\meta{graphical code}}{no default, initially unset}
  Draws the given \tikzname\ \meta{graphical code}
  in the center of the interior region of a |tcolorbox|.
  The code is executed \emph{after} the
  frame and interior are drawn and \emph{before} the text content is drawn.
  The result is zoomed or stretched according the values of
  \refKey{/tcb/watermark zoom} or \refKey{/tcb/watermark stretch}.
\begin{dispExample}
\tcbset{colback=red!5!white,colframe=red!75!black,fonttitle=\bfseries}

\begin{tcolorbox}[enhanced,title=My title,
  watermark tikz={\draw[line width=2mm] circle (1cm)
    node{\fontfamily{ptm}\fontseries{b}\fontsize{20mm}{20mm}\selectfont ?};}]
\lipsum[1]
\tcblower
\lipsum[2]
\end{tcolorbox}
\end{dispExample}
\end{docTcbKey}



\begin{docTcbKey}{watermark tikz on}{=\meta{part} is \meta{graphical code}}{no default, initially unset}
  This option draws the given \tikzname\ \meta{graphical code} in the center of the interior region of a |tcolorbox|
  as described for \refKey{/tcb/watermark tikz}.
  But this is done only for boxes named \meta{part} of a break sequence, see
  \refKey{/tcb/breakable}.\\ 
  Feasible values for \meta{part} are:
  \begin{itemize}
  \item\docValue{broken}: all broken box parts,
  \item\docValue{unbroken}: unbroken boxes only,
  \item\docValue{first}: first parts of a break sequence,
  \item\docValue{middle}: middle parts of a break sequence,
  \item\docValue{last}: last parts of a break sequence,
  \item\docValue{unbroken and first}: unbroken boxes and first parts of a break sequence,
  \item\docValue{middle and last}: middle and last parts of a break sequence.
  \end{itemize}
\end{docTcbKey}


\begin{docTcbKey}{no watermark}{}{style, no default, initially set}
  Removes the watermark if set before. This is an alias for
  \refKey{/tcb/no overlay}.
\end{docTcbKey}


\clearpage
\begin{docTcbKey}{watermark opacity}{=\meta{fraction}}{no default, initially |1.00|}
  Sets the opacity value $\in[0,1]$ for a watermark.
\begin{dispExample}
\tcbset{enhanced,colback=red!5!white,colframe=red!75!black,fonttitle=\bfseries,
  watermark text=Watermark,nobeforeafter,width=(\linewidth-2mm)/2}

\begin{tcolorbox}[title=Opacity 1.00,watermark opacity=1.00]
\lipsum[2]
\end{tcolorbox}\hfill%
\begin{tcolorbox}[title=Opacity 0.50,watermark opacity=0.50]
\lipsum[2]
\end{tcolorbox}%
\end{dispExample}
\end{docTcbKey}

\enlargethispage*{1cm}

\begin{docTcbKey}[][doc updated=2022-07-21]{watermark zoom}{=\meta{fraction}}{default |1|, initially |0.75|}
  Sets the zoom value for a watermark. The zoom respects the aspect ratio.
  The value $1.0$ means to fill the whole box until the watermark touches the frame.
\begin{dispExample}
\tcbset{enhanced,colback=red!5!white,colframe=red!75!black,fonttitle=\bfseries,
  watermark text=Watermark,nobeforeafter,width=(\linewidth-2mm)/2}

\begin{tcolorbox}[title=Zoom 1.0,watermark zoom=1.0]
\lipsum[2]
\end{tcolorbox}\hfill%
\begin{tcolorbox}[title=Zoom 0.5,watermark zoom=0.5]
\lipsum[2]
\end{tcolorbox}%
\end{dispExample}
\end{docTcbKey}

\clearpage

\begin{docTcbKey}[][doc updated=2022-07-21]{watermark shrink}{=\meta{fraction}}{default |1|, initially unset}
  Identically to \refKey{/tcb/watermark zoom}, but the watermark
  never gets enlarged. Thus, the watermark keeps its original size or is shrunk.
\end{docTcbKey}


\begin{docTcbKey}[][doc updated=2022-07-21]{watermark overzoom}{=\meta{fraction}}{default |1|, initially unset}
  Sets the overzoom value for a watermark. The overzoom respects the aspect ratio.
  The value $1.0$ means to fill the whole box until the watermark touches
  all four sides of the frame.
\begin{dispExample}
\tcbset{enhanced,colback=white,colframe=blue!50!black,fonttitle=\bfseries,
  watermark opacity=0.5,
  watermark graphics=lichtspiel.jpg,nobeforeafter,width=(\linewidth-2mm)/2}

\begin{tcolorbox}[title=Zoom 1.0,watermark zoom=1.0]
\lipsum[1]
\end{tcolorbox}\hfill%
\begin{tcolorbox}[title=Overzoom 1.0,watermark overzoom=1.0]
\lipsum[1]
\end{tcolorbox}%
\end{dispExample}
\end{docTcbKey}

\begin{marker}
If a \refKey{/tcb/watermark overzoom} value of |1.0| is used in connection
with invisible top and bottom rules which still have a thickness greater than |0pt|,
the space of these invisible rules may not be covered by the watermark.
For example, this situation may occur during the breaking of \refKey{/tcb/enhanced} boxes.
To avoid this optical glitch, just set \refKey{/tcb/pad at break} to any desired value.
\end{marker}

\clearpage
\begin{docTcbKey}[][doc updated=2022-07-21]{watermark stretch}{=\meta{fraction}}{default |1|, initially unset}
  Sets the stretch value for a watermark. The stretch value is applied to width
  and height in relation to the box dimensions. It does not respect the aspect ratio.
  The value $1.0$ means to fill the whole box.
\begin{dispExample}
\tcbset{enhanced,colback=white,colframe=blue!50!black,fonttitle=\bfseries,
  watermark graphics=lichtspiel.jpg,watermark opacity=0.5,
  nobeforeafter,width=(\linewidth-2mm)/2}

\begin{tcolorbox}[title=Stretch 1.00,watermark stretch=1.00]
\lipsum[2]
\end{tcolorbox}\hfill%
\begin{tcolorbox}[title=Stretch 0.50,watermark stretch=0.50]
\lipsum[2]
\end{tcolorbox}%
\end{dispExample}
\end{docTcbKey}

\begin{docTcbKey}{watermark color}{=\meta{color}}{no default, initially mixed background and frame color}
  Sets the color for the watermark.
\begin{dispExample}
\tcbset{colback=red!5!white,colframe=red!75!black,fonttitle=\bfseries}

\begin{tcolorbox}[enhanced,title=My title,watermark text=My Watermark,
  watermark color=yellow!50!red]
\lipsum[1]
\end{tcolorbox}
\end{dispExample}
\end{docTcbKey}

\clearpage

\begin{docTcbKey}{clip watermark}{\colOpt{=true\textbar false}}{default |true|, initially |true|}
  Sets the watermark to be clipped to the interior area.
\begin{dispExample}
\tcbset{enhanced,colback=white,colframe=blue!50!white,fonttitle=\bfseries,
  watermark opacity=0.5,watermark stretch=1.00,arc=3mm,
  watermark graphics=lichtspiel.jpg}

\begin{tcolorbox}[title=Clip (default),clip watermark]
\lipsum[1]
\end{tcolorbox}

\begin{tcolorbox}[title=No clip,clip watermark=false]
\lipsum[1]
\end{tcolorbox}%
\end{dispExample}
\end{docTcbKey}

\begin{marker}
Removing the clipping should be necessary only in very rare situations. Until
version 5.1.1 (2022/06/24), theoretically, the watermark could be extended
over the boundaries of the box without limit. Newer versions restrict the
watermark to a box scaled 4 times the actual |tcolorbox|.
If you really need to extend further, you are strongly advised to implement
this by using \refKey{/tcb/overlay} or \refKey{/tcb/underlay} directly
where no restrictions apply.
Note that a watermark is just a special \refKey{/tcb/overlay}.
\end{marker}


\clearpage
\subsection{Clip Environments}\label{subsec:clipping}
The following clip environments are applicable for all skins which
use engines of type |path|, |pathfirst|, |pathmiddle|, or |pathlast|.
Especially, the skin \refSkin{enhanced} supports \emph{all} of them
and \refSkin{standard} \emph{none}. The typical area of application
is inside overlay code, see Section \ref{subsec:overlays} from
page \pageref{subsec:overlays}.


\begin{docEnvironment}{tcbclipframe}{}
Defines a \tikzname\ scope which clips to the frame area path.
\begin{dispExample}
\makeatletter
\newtcolorbox{picturebox}[2][]{%
  enhanced,frame hidden,interior hidden,fonttitle=\bfseries,
  overlay={\begin{tcbclipframe}\node at (frame)
    {\includegraphics[width=\tcb@width,height=\tcb@height]{#2}};\end{tcbclipframe}%
    \begin{tcbclipinterior}\fill[white,opacity=0.75]
    (frame.south west) rectangle (frame.north east);\end{tcbclipinterior}},#1}
\makeatother

\begin{picturebox}[title=My Picture Box]{lichtspiel.jpg}
\lipsum[1]
\end{picturebox}
\end{dispExample}
\end{docEnvironment}

\clearpage
\begin{docEnvironment}{tcbinvclipframe}{}
Defines a \tikzname\ scope which clips to the \emph{outside} of the frame area path.

\begin{dispExample*}{segmentation style={preaction={fill=white},pattern=checkerboard,pattern color=gray!40}}
\tcbset{enhanced jigsaw,fonttitle=\bfseries,opacityback=0.35,colback=blue!5!white,
  frame style={left color=red!75!black,right color=red!10!yellow}}

\begin{tikzpicture}% draw two balls
  \path[use as bounding box] (0,0.8) rectangle +(0.1,0.1);
  \shadedraw [shading=ball] (0,0) circle (1cm);
  \shadedraw [ball color=red] (3,-2.2) circle (1cm);
\end{tikzpicture}

\begin{tcolorbox}[title=A translucent box,
  overlay={\begin{tcbinvclipframe}
    \draw[red,line width=1cm] ([xshift=-2mm,yshift=2mm]frame.north west)
      --([xshift=2mm,yshift=-2mm]frame.south east);
    \draw[red,line width=1cm] ([xshift=-2mm,yshift=-2mm]frame.south west)
      --([xshift=2mm,yshift=2mm]frame.north east);
  \end{tcbinvclipframe}}]
  \lipsum[2]
\end{tcolorbox}
\end{dispExample*}
\end{docEnvironment}

\clearpage
\begin{docEnvironment}{tcbclipinterior}{}
Defines a \tikzname\ scope which clips to the interior area path.
\begin{dispExample}
\begin{tcolorbox}[enhanced,title=My Title,
  overlay={\begin{tcbclipinterior}
    \draw[red,line width=1cm] (interior.north west)--(interior.south east);
    \draw[red,line width=1cm] (interior.south west)--(interior.north east);
  \end{tcbclipinterior}}]
\lipsum[1]
\end{tcolorbox}
\end{dispExample}
\end{docEnvironment}

\begin{docEnvironment}{tcbcliptitle}{}
Defines a \tikzname\ scope which clips to the title area path.
\begin{dispExample}
\begin{tcolorbox}[enhanced,title=My Title,colframe=blue,colback=yellow!10!white,
  overlay={\begin{tcbcliptitle}\node at (title)
  {\includegraphics[width=\linewidth]{lichtspiel.jpg}};\end{tcbcliptitle}}]
\lipsum[1]
\end{tcolorbox}
\end{dispExample}
\end{docEnvironment}

\clearpage
\begin{docTcbKey}{clip title}{\colOpt{=true\textbar false}}{default |true|, initially |false|}
  Sets the title to be clipped to the title area.
\begin{dispExample}
\tcbset{enhanced,width=5cm,colframe=red!50!white,coltitle=black,
  colbacktitle=yellow!50!white}

\begin{tcolorbox}[title=\mbox{This is a title which is unbreakable and far too long}]
This is a tcolorbox.
\end{tcolorbox}

\begin{tcolorbox}[title=\mbox{This is a title which is unbreakable and far too long},
  clip title]
This is a tcolorbox.
\end{tcolorbox}
\end{dispExample}
\end{docTcbKey}


\begin{docTcbKey}{clip upper}{\colOpt{=true\textbar false}}{default |true|, initially |false|}
  Sets the upper part to be clipped to the interior area.
\begin{dispExample}
\newcommand{\mygraphics}[2][]{%
  \tcbox[enhanced,boxsep=0pt,top=0pt,bottom=0pt,left=0pt,
    right=0pt,boxrule=0.4pt,drop fuzzy shadow,clip upper,
    colback=black!75!white,toptitle=2pt,bottomtitle=2pt,nobeforeafter,
    center title,fonttitle=\small\sffamily,title=\detokenize{#2}]
  {\includegraphics[width=\the\dimexpr(\linewidth-4mm)/2\relax]{#2}}}

\mygraphics{lichtspiel.jpg}\hfill
\mygraphics{Basilica_5.png}
\end{dispExample}
\end{docTcbKey}

\clearpage
The example for \refKey{/tcb/clip upper} sizes the box according to
the dimensions of the picture. To do it the other way around, the watermark
options provide an easy solution.
\begin{dispExample}
\newcommand{\mygraphics}[2][]{%
  \tcbox[enhanced,capture=minipage,boxsep=0pt,top=0pt,bottom=0pt,left=0pt,
    right=0pt,boxrule=0.4pt,drop fuzzy shadow,nobeforeafter,
    colback=black!75!white,toptitle=2pt,bottomtitle=2pt,
    center title,fonttitle=\small\sffamily,title=\detokenize{#2},
    width=(\linewidth-4mm)/2,height=6cm,colbacktitle={black},
    watermark zoom=1.0,watermark graphics={#2}]{}}

\mygraphics{lichtspiel.jpg}\hfill
\mygraphics{Basilica_5.png}
\end{dispExample}


\begin{docTcbKey}{clip lower}{\colOpt{=true\textbar false}}{default |true|, initially |false|}
  Sets the lower part to be clipped to the interior area.
\begin{dispExample}
\tcbset{enhanced,width=5cm,colframe=red!50!black,text and listing}

\begin{tcblisting}{}
Donau\-dampf\-schiff\-fahrts\-ka\-pi\-t\"ans\-m\"ut\-zen\-fran\-sen
\end{tcblisting}

\begin{tcblisting}{clip lower}
Donau\-dampf\-schiff\-fahrts\-ka\-pi\-t\"ans\-m\"ut\-zen\-fran\-sen
\end{tcblisting}
\end{dispExample}
\end{docTcbKey}


\clearpage
\subsection{Border Line Option Keys}\label{subsec:borderline}
The skin \refSkin{standard} does not support these border lines,
but most other skins, e.\,g.\ \refSkin{enhanced}.

The borderlines are independent from the normal |tcolorbox| rules.
They may be used with or without the \refKey{/tcb/segmentation engine}.

The borderlines are stackable, i.\,e., several different border lines can be
used on the same |tcolorbox|. They are drawn \emph{after} the box frame and box
interior and \emph{before} overlays or watermarks.

\begin{marker}
Technically, the normal |tcolorbox| rules result from a \tikzname\  \emph{filling}
process. The border lines are created by a \tikzname\  \emph{drawing} process.
This can be used to apply different effects.
\end{marker}


\begin{docTcbKey}{borderline}{=\marg{width}\marg{offset}\marg{options}}{no default, initially unset}
  Adds a new borderline to the stack of border lines.
  This border line is drawn with the given \meta{width} and gets an
  \meta{offset} computed from the frame outline. A positive \meta{offset} value
  moves the borderline inside the |tcolorbox| and a negative \meta{offset} value
  moves it outside without changing the bounding box.\\
  The border line is drawn along a \tikzname\  path with the given \tikzname\  \meta{options}.
  Note that the \tikzname\  |line width| option should not be used here.\\
  The border lines adapt to the rounded corners of the |tcolorbox|. An inside
  borderline will switch to sharp corners if necessary, an outside borderline will
  always be rounded except for \refKey{/tcb/sharp corners}.
\begin{dispExample}
\begin{tcolorbox}[enhanced,title=Rounded corners,fonttitle=\bfseries,boxsep=5pt,
  arc=8pt,
  borderline={0.5pt}{0pt}{red},
  borderline={0.5pt}{5pt}{blue,dotted},
  borderline={0.5pt}{-5pt}{green} ]
This is a tcolorbox.
\end{tcolorbox}
\bigskip
\begin{tcolorbox}[enhanced,title=Sharp corners,fonttitle=\bfseries,boxsep=5pt,
  arc=8pt,sharp corners=downhill,
  borderline={0.5pt}{0pt}{red},
  borderline={0.5pt}{5pt}{blue,dotted},
  borderline={0.5pt}{-5pt}{green} ]
This is a tcolorbox.
\end{tcolorbox}
\end{dispExample}

\begin{dispExample}
% \usepackage{lipsum}
\begin{tcolorbox}[enhanced,arc=3mm,boxrule=1.5mm,boxsep=1.5mm,
  colback=yellow!20!white,
  colframe=blue,
  borderline={1mm}{1mm}{white},
  borderline={1mm}{2mm}{red} ]
  \lipsum[1]
\end{tcolorbox}
\end{dispExample}


\begin{dispExample}
% \usepackage{lipsum}
\begin{tcolorbox}[enhanced,arc=3mm,boxrule=1.5mm,
  frame hidden,colback=blue!10!white,
  borderline={1mm}{0mm}{blue,dotted} ]
  \lipsum[2]
\end{tcolorbox}
\end{dispExample}


\begin{dispExample}
% \usepackage{lipsum}
\begin{tcolorbox}[enhanced,skin=enhancedmiddle,
  frame hidden,interior hidden,top=0mm,bottom=0mm,boxsep=0mm,
  borderline={0.75mm}{0mm}{red},
  borderline={0.75mm}{0.75mm}{red!50!yellow},
  borderline={0.75mm}{1.5mm}{yellow}, ]
  \lipsum[3]
\end{tcolorbox}
\end{dispExample}

\begin{dispExample}
% \usepackage{lipsum}
\newtcolorbox{mygreenbox}[2][]{%
  enhanced,width=\linewidth-6pt,
  enlarge top by=3pt,enlarge bottom by=3pt,
  enlarge left by=3pt,enlarge right by=3pt,
  title={#2},frame hidden,boxrule=0pt,top=1mm,bottom=1mm,
  colframe=green!30!black, colbacktitle=green!50!yellow,
  coltitle=black, colback=green!25!white,
  borderline={0.5pt}{-0.5pt}{green!75!blue},
  borderline={1pt}{-3pt}{green!50!blue},#1}

\begin{mygreenbox}{My title}
  \lipsum[4]
\end{mygreenbox}
\end{dispExample}
\end{docTcbKey}


\begin{docTcbKey}{no borderline}{}{no default, initially set}
  Removes all borderlines if set before.
\end{docTcbKey}


\begin{docTcbKey}{show bounding box}{\colOpt{=\meta{color}}}{default |red|, initially unset}
  Displays the bounding box borderline of a |tcolorbox|.
  Its intended use is debugging and fine tuning.
  It should not be part of a final document.
  The optional \meta{color} is the base color for the bounding box
  borderline.
\begin{dispExample}
\tcbset{enhanced,nobeforeafter,width=4cm,fonttitle=\bfseries}

\begin{tcolorbox}[show bounding box,title=Normal]
This is a tcolorbox.
\end{tcolorbox}%
\begin{tcolorbox}[show bounding box=blue,title=Shadow,drop fuzzy shadow]
This is a tcolorbox.
\end{tcolorbox}%
\begin{tcolorbox}[show bounding box=green,title=Enlarged,drop fuzzy shadow,
  enlarge by=2mm]
This is a tcolorbox.
\end{tcolorbox}
\end{dispExample}
\end{docTcbKey}

\clearpage

\begin{marker}
The following \emph{partial} borderlines act slightly different from the
complete borderlines described before. They ignore rounded corner settings,
their length is not modified by their \meta{offset}, they ignore skin settings
but adapt to breakable boxes.
\end{marker}

\begin{docTcbKey}[][doc new=2014-10-20]{borderline north}{=\marg{width}\marg{offset}\marg{options}}{no default, initially unset}
  Adds a new borderline with the given \meta{width} to the
  north of the |tcolorbox|.
  A positive \meta{offset} value
  moves the borderline inside the |tcolorbox| and a negative \meta{offset} value
  moves it outside without changing the bounding box.
\begin{dispExample*}{sbs,lefthand ratio=0.6}
\begin{tcolorbox}[enhanced,
  borderline north={2pt}{-2pt}{red}]
  This is a \textbf{tcolorbox}.
\end{tcolorbox}
\end{dispExample*}
\end{docTcbKey}

\begin{docTcbKey}[][doc new=2014-10-20]{borderline south}{=\marg{width}\marg{offset}\marg{options}}{no default, initially unset}
  Adds a new borderline with the given \meta{width} to the
  south of the |tcolorbox|.
  A positive \meta{offset} value
  moves the borderline inside the |tcolorbox| and a negative \meta{offset} value
  moves it outside without changing the bounding box.
\begin{dispExample*}{sbs,lefthand ratio=0.6}
\begin{tcolorbox}[enhanced,
  borderline south={2pt}{-2pt}{red}]
  This is a \textbf{tcolorbox}.
\end{tcolorbox}
\end{dispExample*}
\end{docTcbKey}

\begin{docTcbKey}[][doc new=2014-10-20]{borderline east}{=\marg{width}\marg{offset}\marg{options}}{no default, initially unset}
  Adds a new borderline with the given \meta{width} to the
  east of the |tcolorbox|.
  A positive \meta{offset} value
  moves the borderline inside the |tcolorbox| and a negative \meta{offset} value
  moves it outside without changing the bounding box.
\begin{dispExample*}{sbs,lefthand ratio=0.6}
\begin{tcolorbox}[enhanced,
  borderline east={2pt}{-2pt}{red}]
  This is a \textbf{tcolorbox}.
\end{tcolorbox}
\end{dispExample*}
\end{docTcbKey}

\begin{docTcbKey}[][doc new=2014-10-20]{borderline west}{=\marg{width}\marg{offset}\marg{options}}{no default, initially unset}
  Adds a new borderline with the given \meta{width} to the
  west of the |tcolorbox|.
  A positive \meta{offset} value
  moves the borderline inside the |tcolorbox| and a negative \meta{offset} value
  moves it outside without changing the bounding box.
\begin{dispExample*}{sbs,lefthand ratio=0.6}
\begin{tcolorbox}[enhanced,
  borderline west={2pt}{-2pt}{red}]
  This is a \textbf{tcolorbox}.
\end{tcolorbox}
\end{dispExample*}
\end{docTcbKey}

\clearpage
\begin{docTcbKey}[][doc new=2014-10-20]{borderline horizontal}{=\marg{width}\marg{offset}\marg{options}}{no default, initially unset}
  Adds a new borderline with the given \meta{width} to the
  north and south of the |tcolorbox|.
  A positive \meta{offset} value
  moves the borderlines inside the |tcolorbox| and a negative \meta{offset} value
  moves them outside without changing the bounding box.
\begin{dispExample*}{sbs,lefthand ratio=0.6}
\begin{tcolorbox}[blanker,top=3mm,bottom=3mm,
   borderline horizontal={2pt}{0pt}{red}]
  This is a \textbf{tcolorbox}.
\end{tcolorbox}
\end{dispExample*}
\end{docTcbKey}


\begin{docTcbKey}[][doc new=2014-10-20]{borderline vertical}{=\marg{width}\marg{offset}\marg{options}}{no default, initially unset}
  Adds a new borderline with the given \meta{width} to the
  east and west of the |tcolorbox|.
  A positive \meta{offset} value
  moves the borderlines inside the |tcolorbox| and a negative \meta{offset} value
  moves them outside without changing the bounding box.
\begin{dispExample*}{sbs,lefthand ratio=0.6}
\begin{tcolorbox}[blanker,left=3mm,right=3mm,
   borderline vertical={2pt}{0pt}{red}]
  This is a \textbf{tcolorbox}.\\
  My second line.
\end{tcolorbox}
\end{dispExample*}
\end{docTcbKey}

\begin{dispExample}
\begin{tcolorbox}[enhanced,colback=yellow!10!white,boxrule=0pt,frame hidden,
  borderline north={1mm}{-2mm}{red},
  borderline south={1mm}{-2mm}{blue},
  borderline west={1mm}{-2mm}{green},
  borderline east={1mm}{-2mm}{yellow}]
\lipsum[1]
\end{tcolorbox}
\end{dispExample}

\clearpage
\subsection{Shadow Option Keys}\label{subsec:shadows}
The skin \refSkin{standard} does not support these shadows,
but most other skins, e.\,g.\ \refSkin{enhanced}.

The shadows are stackable, i.\,e., several different shadows can be
used on the same |tcolorbox|. They are drawn \emph{before} the box frame is drawn.

\begin{docTcbKey}{no shadow}{}{no default}
  Removes all shadows if set before.
\end{docTcbKey}

\subsubsection{Common Shadows and Halos}

\begin{docTcbKey}{drop shadow}{\colOpt{=\meta{color}}}{style, default |black!50!white|}
  Adds a new shadow with standard dimensions to the stack of shadows.
  Optionally, the \meta{color} for the shadow can be changed.
\begin{dispExample*}{sbs,lefthand ratio=0.6}
\tcbset{enhanced,colback=red!5!white,
  colframe=red!75!black,fonttitle=\bfseries}

\begin{tcolorbox}[drop shadow]
This is a tcolorbox.
\end{tcolorbox}\par\bigskip
\begin{tcolorbox}[title=Another shadow,
  drop shadow=blue]
This is a tcolorbox.
\end{tcolorbox}
\end{dispExample*}
\end{docTcbKey}


\begin{docTcbKey}{drop fuzzy shadow}{\colOpt{=\meta{color}}}{style, default |black!50!white|}
  Adds a new fuzzy shadow with standard dimensions to the stack of shadows.
  Optionally, the \meta{color} for the shadow can be changed.
\begin{dispExample*}{sbs,lefthand ratio=0.6}
\tcbset{enhanced,colback=red!5!white,
  colframe=red!75!black,fonttitle=\bfseries}

\begin{tcolorbox}[drop fuzzy shadow]
This is a tcolorbox.
\end{tcolorbox}\par\bigskip
\begin{tcolorbox}[title=Another shadow,
  drop fuzzy shadow=blue]
This is a tcolorbox.
\end{tcolorbox}
\end{dispExample*}
\end{docTcbKey}


\begin{docTcbKey}{drop midday shadow}{\colOpt{=\meta{color}}}{style, default |black!50!white|}
  Adds a new shadow with standard dimensions to the stack of shadows.
  Optionally, the \meta{color} for the shadow can be changed.
\begin{dispExample*}{sbs,lefthand ratio=0.6}
\tcbset{enhanced,colback=red!5!white,
  colframe=red!75!black,fonttitle=\bfseries}

\begin{tcolorbox}[drop midday shadow]
This is a tcolorbox.
\end{tcolorbox}\par\bigskip
\begin{tcolorbox}[title=Another shadow,
  drop midday shadow=blue]
This is a tcolorbox.
\end{tcolorbox}
\end{dispExample*}
\end{docTcbKey}

%\enlargethispage*{2cm}
\begin{docTcbKey}{drop fuzzy midday shadow}{\colOpt{=\meta{color}}}{style, default |black!50!white|}
  Adds a new fuzzy shadow with standard dimensions to the stack of shadows.
  Optionally, the \meta{color} for the shadow can be changed.
\begin{dispExample*}{sbs,lefthand ratio=0.6}
\tcbset{enhanced,colback=red!5!white,
  colframe=red!75!black,fonttitle=\bfseries}

\begin{tcolorbox}[drop fuzzy midday shadow]
This is a tcolorbox.
\end{tcolorbox}\par\bigskip
\begin{tcolorbox}[title=Another shadow,
  drop fuzzy midday shadow=blue]
This is a tcolorbox.
\end{tcolorbox}
\end{dispExample*}
\end{docTcbKey}


\begin{docTcbKey}{halo}{\colOpt{=\meta{size} \texttt{with} \meta{color}}}{style, default |0.9mm with yellow|}
  Adds a new halo shadow with the given \meta{color}
  which overlaps the colorbox an all sides by \meta{size}.
\begin{dispExample*}{sbs,lefthand ratio=0.6}
\tcbset{enhanced,colback=red!5!white,
  colframe=red!75!black,fonttitle=\bfseries}

\begin{tcolorbox}[title=My own halo,halo]
This is a tcolorbox.
\end{tcolorbox}
\par\bigskip\bigskip
\begin{tcolorbox}[title=Another halo,
  halo=2mm with green]
This is a tcolorbox.
\end{tcolorbox}
\end{dispExample*}
\end{docTcbKey}

\enlargethispage*{2cm}
\begin{docTcbKey}{fuzzy halo}{\colOpt{=\meta{size} \texttt{with} \meta{color}}}{style, default |0.9mm with yellow|}
  Adds a new fuzzy halo shadow with the given \meta{color}
  which overlaps the colorbox an all sides by \meta{size} plus |0.48mm|.
\begin{dispExample*}{sbs,lefthand ratio=0.6}
\tcbset{enhanced,colback=red!5!white,
  colframe=red!75!black,fonttitle=\bfseries}

\begin{tcolorbox}[title=My own halo,fuzzy halo]
This is a tcolorbox.
\end{tcolorbox}
\par\bigskip\bigskip
\begin{tcolorbox}[title=Another halo,
  fuzzy halo=2mm with green]
This is a tcolorbox.
\end{tcolorbox}
\end{dispExample*}

\begin{dispExample}
\begin{tcolorbox}[blank,enhanced jigsaw,boxsep=2pt,arc=2pt,
  fuzzy halo=2mm with red!50!white,
  fuzzy halo=1mm with white]
\lipsum[1]
\end{tcolorbox}
\end{dispExample}
\end{docTcbKey}


\clearpage
For all following shadows, the optionally given \meta{color} for the shadow can be changed
equivalent to the preceding examples.

\begin{docTcbKey}{drop shadow southeast}{\colOpt{=\meta{color}}}{style, default |black!50!white|}
  Adds a new shadow with standard dimensions to the stack of shadows.
  This shadow is identical to \refKey{/tcb/drop shadow}.
\begin{dispExample*}{sbs,lefthand ratio=0.7}
\begin{tcolorbox}[drop shadow southeast,
  enhanced,colback=red!5!white,colframe=red!75!black]
  This is a tcolorbox.
\end{tcolorbox}
\end{dispExample*}
\end{docTcbKey}%

\begin{docTcbKey}{drop shadow south}{\colOpt{=\meta{color}}}{style, default |black!50!white|}
  Adds a new shadow with standard dimensions to the stack of shadows.
  This shadow is identical to \refKey{/tcb/drop midday shadow}.
\begin{dispExample*}{sbs,lefthand ratio=0.7}
\begin{tcolorbox}[drop shadow south,
  enhanced,colback=red!5!white,colframe=red!75!black]
  This is a tcolorbox.
\end{tcolorbox}
\end{dispExample*}
\end{docTcbKey}%

\begin{docTcbKey}{drop shadow southwest}{\colOpt{=\meta{color}}}{style, default |black!50!white|}
  Adds a new shadow with standard dimensions to the stack of shadows.
\begin{dispExample*}{sbs,lefthand ratio=0.7}
\begin{tcolorbox}[drop shadow southwest,
  enhanced,colback=red!5!white,colframe=red!75!black]
  This is a tcolorbox.
\end{tcolorbox}
\end{dispExample*}
\end{docTcbKey}%

\begin{docTcbKey}{drop shadow west}{\colOpt{=\meta{color}}}{style, default |black!50!white|}
  Adds a new shadow with standard dimensions to the stack of shadows.
\begin{dispExample*}{sbs,lefthand ratio=0.7}
\begin{tcolorbox}[drop shadow west,
  enhanced,colback=red!5!white,colframe=red!75!black]
  This is a tcolorbox.
\end{tcolorbox}
\end{dispExample*}
\end{docTcbKey}%

\begin{docTcbKey}{drop shadow northwest}{\colOpt{=\meta{color}}}{style, default |black!50!white|}
  Adds a new shadow with standard dimensions to the stack of shadows.
\begin{dispExample*}{sbs,lefthand ratio=0.7}
\begin{tcolorbox}[drop shadow northwest,
  enhanced,colback=red!5!white,colframe=red!75!black]
  This is a tcolorbox.
\end{tcolorbox}
\end{dispExample*}
\end{docTcbKey}%

\begin{docTcbKey}{drop shadow north}{\colOpt{=\meta{color}}}{style, default |black!50!white|}
  Adds a new shadow with standard dimensions to the stack of shadows.
\begin{dispExample*}{sbs,lefthand ratio=0.7}
\begin{tcolorbox}[drop shadow north,
  enhanced,colback=red!5!white,colframe=red!75!black]
  This is a tcolorbox.
\end{tcolorbox}
\end{dispExample*}
\end{docTcbKey}%

\clearpage
\begin{docTcbKey}{drop shadow northeast}{\colOpt{=\meta{color}}}{style, default |black!50!white|}
  Adds a new shadow with standard dimensions to the stack of shadows.
\begin{dispExample*}{sbs,lefthand ratio=0.7}
\begin{tcolorbox}[drop shadow northeast,
  enhanced,colback=red!5!white,colframe=red!75!black]
  This is a tcolorbox.
\end{tcolorbox}
\end{dispExample*}
\end{docTcbKey}%

\begin{docTcbKey}{drop shadow east}{\colOpt{=\meta{color}}}{style, default |black!50!white|}
  Adds a new shadow with standard dimensions to the stack of shadows.
\begin{dispExample*}{sbs,lefthand ratio=0.7}
\begin{tcolorbox}[drop shadow east,
  enhanced,colback=red!5!white,colframe=red!75!black]
  This is a tcolorbox.
\end{tcolorbox}
\end{dispExample*}
\end{docTcbKey}%


\begin{docTcbKey}{drop fuzzy shadow southeast}{\colOpt{=\meta{color}}}{style, default |black!50!white|}
  Adds a new fuzzy shadow with standard dimensions to the stack of shadows.
  This shadow is identical to \refKey{/tcb/drop fuzzy shadow}.
\begin{dispExample*}{sbs,lefthand ratio=0.7}
\begin{tcolorbox}[drop fuzzy shadow southeast,
  enhanced,colback=red!5!white,colframe=red!75!black]
  This is a tcolorbox.
\end{tcolorbox}
\end{dispExample*}
\end{docTcbKey}%

\begin{docTcbKey}{drop fuzzy shadow south}{\colOpt{=\meta{color}}}{style, default |black!50!white|}
  Adds a new fuzzy shadow with standard dimensions to the stack of shadows.
  This shadow is identical to \refKey{/tcb/drop fuzzy midday shadow}.
\begin{dispExample*}{sbs,lefthand ratio=0.7}
\begin{tcolorbox}[drop fuzzy shadow south,
  enhanced,colback=red!5!white,colframe=red!75!black]
  This is a tcolorbox.
\end{tcolorbox}
\end{dispExample*}
\end{docTcbKey}%

\begin{docTcbKey}{drop fuzzy shadow southwest}{\colOpt{=\meta{color}}}{style, default |black!50!white|}
  Adds a new fuzzy shadow with standard dimensions to the stack of shadows.
\begin{dispExample*}{sbs,lefthand ratio=0.7}
\begin{tcolorbox}[drop fuzzy shadow southwest,
  enhanced,colback=red!5!white,colframe=red!75!black]
  This is a tcolorbox.
\end{tcolorbox}
\end{dispExample*}
\end{docTcbKey}%

\begin{docTcbKey}{drop fuzzy shadow west}{\colOpt{=\meta{color}}}{style, default |black!50!white|}
  Adds a new fuzzy shadow with standard dimensions to the stack of shadows.
\begin{dispExample*}{sbs,lefthand ratio=0.7}
\begin{tcolorbox}[drop fuzzy shadow west,
  enhanced,colback=red!5!white,colframe=red!75!black]
  This is a tcolorbox.
\end{tcolorbox}
\end{dispExample*}
\end{docTcbKey}%

\clearpage
\begin{docTcbKey}{drop fuzzy shadow northwest}{\colOpt{=\meta{color}}}{style, default |black!50!white|}
  Adds a new fuzzy shadow with standard dimensions to the stack of shadows.
\begin{dispExample*}{sbs,lefthand ratio=0.7}
\begin{tcolorbox}[drop fuzzy shadow northwest,
  enhanced,colback=red!5!white,colframe=red!75!black]
  This is a tcolorbox.
\end{tcolorbox}
\end{dispExample*}
\end{docTcbKey}%

\begin{docTcbKey}{drop fuzzy shadow north}{\colOpt{=\meta{color}}}{style, default |black!50!white|}
  Adds a new fuzzy shadow with standard dimensions to the stack of shadows.
\begin{dispExample*}{sbs,lefthand ratio=0.7}
\begin{tcolorbox}[drop fuzzy shadow north,
  enhanced,colback=red!5!white,colframe=red!75!black]
  This is a tcolorbox.
\end{tcolorbox}
\end{dispExample*}
\end{docTcbKey}%

\begin{docTcbKey}{drop fuzzy shadow northeast}{\colOpt{=\meta{color}}}{style, default |black!50!white|}
  Adds a new fuzzy shadow with standard dimensions to the stack of shadows.
\begin{dispExample*}{sbs,lefthand ratio=0.7}
\begin{tcolorbox}[drop fuzzy shadow northeast,
  enhanced,colback=red!5!white,colframe=red!75!black]
  This is a tcolorbox.
\end{tcolorbox}
\end{dispExample*}
\end{docTcbKey}%

\begin{docTcbKey}{drop fuzzy shadow east}{\colOpt{=\meta{color}}}{style, default |black!50!white|}
  Adds a new fuzzy shadow with standard dimensions to the stack of shadows.
\begin{dispExample*}{sbs,lefthand ratio=0.7}
\begin{tcolorbox}[drop fuzzy shadow east,
  enhanced,colback=red!5!white,colframe=red!75!black]
  This is a tcolorbox.
\end{tcolorbox}
\end{dispExample*}
\end{docTcbKey}%


\clearpage
\subsubsection{Lifted Shadows}

\begin{docTcbKey}{drop lifted shadow}{\colOpt{=\meta{color}}}{style, default |black!50!white|}
  Adds a new lifted shadow with standard dimensions to the stack of shadows.
  Optionally, the \meta{color} for the shadow can be changed.
\begin{dispExample*}{sbs,lefthand ratio=0.6}
\tcbset{enhanced,colback=red!5!white,
  boxrule=0.4pt,sharp corners,
  colframe=red!75!black,fonttitle=\bfseries}

\begin{tcolorbox}[drop lifted shadow]
This is a tcolorbox.
\end{tcolorbox}\par\bigskip
\begin{tcolorbox}[title=Another shadow,
  drop lifted shadow=blue]
This is a tcolorbox.
\end{tcolorbox}
\end{dispExample*}
\end{docTcbKey}


\begin{docTcbKey}{drop small lifted shadow}{\colOpt{=\meta{color}}}{style, default |black!50!white|}
  Adds a new small lifted shadow with standard dimensions to the stack of shadows.
  Optionally, the \meta{color} for the shadow can be changed.
\begin{dispExample*}{sbs,lefthand ratio=0.6}
\tcbset{enhanced,colback=red!5!white,
  boxrule=0.4pt,sharp corners,
  colframe=red!75!black,fonttitle=\bfseries}

\tcbox[drop small lifted shadow,size=fbox]
  {This is a tcolorbox.}
\par\bigskip
\begin{tcolorbox}[title=Another shadow,
  drop small lifted shadow=black]
This is a tcolorbox.
\end{tcolorbox}
\end{dispExample*}
\end{docTcbKey}


\begin{docTcbKey}{drop large lifted shadow}{\colOpt{=\meta{color}}}{style, default |black!50!white|}
  Adds a new large lifted shadow with standard dimensions to the stack of shadows.
  Optionally, the \meta{color} for the shadow can be changed.
\begin{dispExample*}{sbs,lefthand ratio=0.6}
\tcbset{enhanced,colback=red!5!white,
  colframe=red!75!black,fonttitle=\bfseries}

\begin{tcolorbox}[drop large lifted shadow]
This is a tcolorbox.
\end{tcolorbox}\par\bigskip
\begin{tcolorbox}[title=Another shadow,
  drop large lifted shadow=blue]
This is a tcolorbox.
\end{tcolorbox}
\end{dispExample*}
\end{docTcbKey}


\clearpage

\subsubsection{Generic Shadows}
\begin{docTcbKey}{shadow}{=\marg{xshift}\marg{yshift}\marg{offset}\marg{options}}{no default}
  Adds a new shadow to the stack of shadows.
  This shadow follows the outline of the |tcolorbox| but is shifted by
  \meta{xshift} and \meta{yshift}. The \meta{offset} value is a distance value
  from the frame outline.  A positive \meta{offset} value shrinks the shadow
  and a negative \meta{offset} value enlarges the shadow.
  The shadow is filled along a \tikzname\  path with the given \tikzname\  \meta{options}.\\
  The shadows adapt to the rounded corners of the |tcolorbox|. A shrunken shadow
  will switch to sharp corners if necessary, an enlarged shadow may become
  more rounded depending on several factors. But \refKey{/tcb/sharp corners}
  have sharp shadows.
  \begin{marker}
  Shadows are not considered for the bounding box computation by default.
  Large shadows may be overlapped by the following content. But, the
  bounding box can be adapted if necessary.
  \end{marker}

\begin{dispExample*}{sbs,lefthand ratio=0.6}
\tcbset{enhanced,colback=red!5!white,
  colframe=red!75!black,fonttitle=\bfseries}

\begin{tcolorbox}[title=My own shadow,
  shadow={2mm}{-1mm}{0mm}{black!50!white}]
This is a tcolorbox.
\end{tcolorbox}
\par\bigskip
\begin{tcolorbox}[title=Another shadow,
  shadow={-1mm}{-2mm}{0mm}{fill=blue,
    opacity=0.5}]
This is a tcolorbox.
\end{tcolorbox}
\par\bigskip
\begin{tcolorbox}[title=Double shadow,
  shadow={-1.5mm}{-1.5mm}{0mm}{fill=blue,
    opacity=0.25},
  shadow={1.5mm}{-1.5mm}{0mm}{fill=red,
    opacity=0.25}]
This is a tcolorbox.
\end{tcolorbox}
\par\bigskip
\begin{tcolorbox}[title=Far shadow,
  shadow={5.5mm}{-3.5mm}{2mm}{fill=black,
    opacity=0.25}]
This is a tcolorbox.
\end{tcolorbox}
\par\bigskip\bigskip
\begin{tcolorbox}[title=Halo shadow,
  shadow={0mm}{0mm}{-1.5mm}%
     {fill=yellow!75!red,opacity=0.5}]
This is a tcolorbox.
\end{tcolorbox}
\end{dispExample*}
\end{docTcbKey}

\clearpage
\begin{docTcbKey}{fuzzy shadow}{=\marg{xshift}\marg{yshift}\marg{offset}\marg{step}\marg{options}}{no default}
  Adds a new fuzzy shadow to the stack of shadows. Actually, this option
  adds several shadows which appear like a shadow with a fuzzy border.
  This fuzzy shadow follows the outline of the |tcolorbox| but is shifted by
  \meta{xshift} and \meta{yshift}. The \meta{offset} value is a distance value
  from the frame outline.  A positive \meta{offset} value shrinks the shadow
  and a negative \meta{offset} value enlarges the shadow.
  The \marg{step} value describes a shrink
  offset used for the combination of the partial shadows.
  The shadow is filled along a \tikzname\  path with the given \tikzname\  \meta{options} but
  any |opacity| value will be ignored.
\begin{dispExample*}{sbs,lefthand ratio=0.6}
\tcbset{enhanced,colback=red!5!white,
  colframe=red!75!black,fonttitle=\bfseries}

\begin{tcolorbox}[title=My own shadow,
  fuzzy shadow={2mm}{-1mm}{0mm}{0.1mm}%
               {black!50!white}]
This is a tcolorbox.
\end{tcolorbox}
\par\bigskip
\begin{tcolorbox}[title=Another shadow,
  fuzzy shadow={-1mm}{-2mm}{0mm}{0.2mm}%
               {fill=blue}]
This is a tcolorbox.
\end{tcolorbox}
\par\bigskip
\begin{tcolorbox}[title=Double shadow,
  fuzzy shadow={-1.5mm}{-1.5mm}{0mm}{0.1mm}%
               {blue},
  fuzzy shadow={1.5mm}{-1.5mm}{0mm}{0.1mm}%
               {red}]
This is a tcolorbox.
\end{tcolorbox}
\par\bigskip
\begin{tcolorbox}[title=Far shadow,
  fuzzy shadow={5.5mm}{-3.5mm}{0mm}{0.3mm}%
               {black}]
This is a tcolorbox.
\end{tcolorbox}
\par\bigskip\bigskip
\begin{tcolorbox}[title=Glow shadow,
  fuzzy shadow={0mm}{0mm}{-1.5mm}{0.15mm}%
               {yellow!75!red}]
This is a tcolorbox.
\end{tcolorbox}
\end{dispExample*}

\begin{dispExample}
\newtcolorbox{mybox}[1][]{enhanced,
  fuzzy shadow={1.0mm}{-1.0mm}{0.12mm}{0mm}{blue!50!white},
  fuzzy shadow={-1.0mm}{-1.0mm}{0.12mm}{0mm}{red!50!white},
  fuzzy shadow={-1.0mm}{1.0mm}{0.12mm}{0mm}{green!50!white},
  fuzzy shadow={1.0mm}{1.0mm}{0.12mm}{0mm}{yellow!50!white},#1
}

\begin{mybox}[title=A multi shadow box]
This is a tcolorbox.
\end{mybox}
\end{dispExample}
\end{docTcbKey}


\clearpage
\begin{docTcbKey}[][doc new=2015-05-05]{smart shadow arc}{\colOpt{=true\textbar false}}{default |true|,
  initially |true|}
If set to |true|, the shadow drawing algorithm tries to do a somewhat
smart calculation of the arc for the shadow. The result is pleasing for typical boxes
with rounded corners, but gives strange results for circular boxes.

\begin{dispExample}
\tcbset{enhanced,nobeforeafter,colback=red!5!white,
  colframe=red!75!black,width=3cm,square,halign=center,valign=center
  }

\begin{tcolorbox}[drop shadow]
Smart shadow arc (arguably better than normal)
\end{tcolorbox}
\hfill
\begin{tcolorbox}[smart shadow arc=false, drop shadow]
Normal shadow arc
\end{tcolorbox}
\hfill
\begin{tcolorbox}[circular arc, drop shadow]
Smart shadow arc (worse than normal)
\end{tcolorbox}
\hfill
\begin{tcolorbox}[circular arc, smart shadow arc=false, drop shadow]
Normal shadow arc
\end{tcolorbox}
\end{dispExample}
\end{docTcbKey}


\begin{docTcbKey}{lifted shadow}{=\marg{xshift}\marg{yshift}\marg{bend}\marg{step}\marg{options}}{no default}
  Adds a new lifted shadow to the stack of shadows. Actually, this option
  adds several shadows which appear like a shadow with a fuzzy border.
  This lifted shadow follows the outline of the |tcolorbox| but is shifted by
  \meta{xshift} and \meta{yshift} on the lower left corner and by
  $-$\meta{xshift} and \meta{yshift} on the lower right corner.
  Additionally, there is a \meta{bend} in the middle.
  The \marg{step} value describes a shrink
  offset used for the combination of the partial shadows.
  The shadow is filled along a \tikzname\  path with the given \tikzname\  \meta{options} but
  any |opacity| value will be ignored.
\begin{dispExample*}{sbs,lefthand ratio=0.6}
\tcbset{enhanced,colback=red!5!white,
  boxrule=0.1pt,
  colframe=red!75!black,fonttitle=\bfseries}

\begin{tcolorbox}[title=My own shadow,
  lifted shadow={1mm}{-2mm}{3mm}{0.1mm}%
               {black!50!white}]
This is a tcolorbox.
\end{tcolorbox}
\end{dispExample*}
\end{docTcbKey}

\clearpage
\subsubsection{\tikzname\ Shadows}
Alternative to the package shadow options described before, shadows from
the \flqq Shadows Library\frqq\ of \tikzname\ can be used.
Such shadows can be added directly to the frame path using \refKey{/tcb/frame style}.

\begin{exdispExample*}{tikz_shadow_1}{sbs,lefthand ratio=0.7}
% \usetikzlibrary{shadows}
\begin{tcolorbox}[enhanced,
  colback=red!5!white,colframe=red!75!black,
  frame style={drop shadow} ]
  This is a tcolorbox.
\end{tcolorbox}
\end{exdispExample*}

\begin{exdispExample*}{tikz_shadow_2}{sbs,lefthand ratio=0.7}
% \usetikzlibrary{shadows}
\begin{tcolorbox}[enhanced,height=3cm,
  colback=red!5!white,colframe=red!75!black,
  halign=center,valign=center,
  frame style={circular drop shadow} ]
  This is a tcolorbox.
\end{tcolorbox}
\end{exdispExample*}

\begin{exdispExample*}{tikz_shadow_3}{sbs,lefthand ratio=0.7}
% \usetikzlibrary{shadows}
\begin{tcolorbox}[enhanced,width=2.5cm,
  square,circular arc,
  halign=center,valign=center,
  colback=red!5!white,colframe=red!75!black,
  frame style={circular glow={fill=red}} ]
  tcolorbox
\end{tcolorbox}
\end{exdispExample*}


\clearpage
\subsection{\tikzname\  Picture Option Keys}\label{subsec:tikzpicture}

\begin{docTcbKey}{tikz}{=\meta{tikz option list}}{no default, initially empty}
  Adds the given \meta{tikz option list} to the main |tikzpicture| environment
  used to draw the color box, see \cite{tantau:tikz_and_pgf}. If this option is
  applied a second time, the new \meta{tikz option list} is appended to the
  current option list.
  Note that, if \tikzname\ coordinate transformation options are used,
  |transform shape| may also be needed to transform all box elements.
\begin{dispExample*}{sbs,lefthand ratio=0.66,
  segmentation style={preaction={fill=white},pattern=checkerboard,pattern color=gray!40}}
\tcbset{enhanced,colback=red!5!white,
  colframe=red!75!black,fonttitle=\bfseries}

\begin{tcolorbox}[title=Transparent box,
  tikz={opacity=0.5,transparency group}]
This is a tcolorbox.
\end{tcolorbox}
\end{dispExample*}

\begin{dispExample*}{sbs,lefthand ratio=0.66}
\tcbset{enhanced,colback=red!5!white,
  colframe=red!75!black,fonttitle=\bfseries,
  fontupper=\bfseries\Huge,
  halign title=center,halign=center}

\begin{tcolorbox}[title=Rotated box,
  tikz={rotate=30}]
Sold!
\end{tcolorbox}
\end{dispExample*}

\end{docTcbKey}


\begin{docTcbKey}{tikz reset}{}{initially set}
  Removes all options given by \refKey{/tcb/tikz}.
\end{docTcbKey}


\begin{docTcbKey}{at begin tikz}{=\meta{tikz code}}{no default, initially empty}
  The given \meta{tikz code} is executed at the beginning of the |tikzpicture| environment
  after the \tikzname\  option |execute at begin picture| was applied.
  If this option is applied a second time, the new \meta{tikz code} is appended to the current code.
\end{docTcbKey}


\begin{docTcbKey}{at begin tikz reset}{}{initially set}
  Removes all code given by \refKey{/tcb/at begin tikz}.
\end{docTcbKey}


\begin{docTcbKey}{at end tikz}{=\meta{tikz code}}{no default, initially empty}
  The given \meta{tikz code} is executed at the ending of the |tikzpicture| environment
  before the \tikzname\  option |execute at end picture| was applied.
  If this option is applied a second time, the new \meta{tikz code} is appended to the current code.
\end{docTcbKey}


\begin{docTcbKey}{at end tikz reset}{}{initially set}
  Removes all code given by \refKey{/tcb/at end tikz}.
\end{docTcbKey}


\clearpage
\begin{docTcbKey}{rotate}{=\meta{angle}}{no default, initially unset}
  Rotates the |tcolorbox| by the given \meta{angle}.
  This also applys |transform shape|.
  Note that this is a \tikzname\  coordinate transformation i.\,e., not all graphical elements like shadings
  will really be rotated.
\begin{dispExample*}{sbs,lefthand ratio=0.66}
\tcbset{enhanced,colback=red!5!white,
  colframe=red!75!black,fonttitle=\bfseries}

\begin{tcolorbox}[title=Rotated box,rotate=30]
This is a tcolorbox.
\end{tcolorbox}
\end{dispExample*}
\end{docTcbKey}

\begin{docTcbKey}{scale}{=\meta{fraction}}{no default, initially unset}
  Scales the |tcolorbox| by the given \meta{fraction}.
  This also applys |transform shape|.
  Note that this is a \tikzname\  coordinate transformation i.\,e., not all graphical elements like line widths
  will really be scaled.
\begin{dispExample*}{sbs,lefthand ratio=0.66}
\tcbset{enhanced,colback=red!5!white,
  colframe=red!75!black,fonttitle=\bfseries}

\begin{tcolorbox}[title=Scaled box,scale=0.5]
This is a tcolorbox.
\end{tcolorbox}
\begin{tcolorbox}[title=Scaled box,scale=1.25]
This is a tcolorbox.
\end{tcolorbox}
\end{dispExample*}
\end{docTcbKey}


\begin{docTcbKey}{remember}{}{style, initially unset}
  Shortcut for |tikz={remember picture}|. This allows one to reference nodes
  in other \tikzname\  pictures.
\begin{dispExample}
\begin{tcolorbox}[enhanced,remember,colback=red!5!white,colframe=red!75!black,
  fonttitle=\bfseries,title=The four corners of a paper,
  overlay={\draw[red!50!white,line width=1mm,opacity=0.5,shorten >=3mm]
    (frame.north west) edge[->] (current page.north west)
    (frame.north east) edge[->] (current page.north east)
    (frame.south west) edge[->] (current page.south west)
    (frame.south east) edge[->] (current page.south east);}]
This is a tcolorbox.
\end{tcolorbox}
\end{dispExample}
\end{docTcbKey}

\clearpage
\tcbinterruptdraftmode%
\begin{docTcbKey}{remember as}{=\meta{name}}{style, no default, initially unset}
  The |frame| node will be remembered by the given \meta{name} to be referenced
  in other \tikzname\  pictures.
\begin{dispExample}
% \usepackage{lipsum}
\newtcolorbox{mybox}[1][]{enhanced,colframe=blue!75!black,colback=blue!10!white,
  fonttitle=\bfseries,#1}

\begin{mybox}[title=First Box,nobeforeafter,width=\linewidth/4,remember as=one]
This is a test.
\end{mybox}
\hfill
\begin{mybox}[title=Second Box,nobeforeafter,width=\linewidth/4,remember as=two]
This is a test.
\end{mybox}
\hfill
\begin{mybox}[title=Third Box,nobeforeafter,width=\linewidth/4,remember as=three]
This is a test.
\end{mybox}

\lipsum[2]

\begin{mybox}[title=Fourth Box,remember as=four]
This is a test.
\end{mybox}

\begin{tikzpicture}[overlay,remember picture,line width=1mm,draw=red!75!black]
  \draw[->] (one.east) to[bend right] node[above] {A} (two.west);
  \draw[->] (two.east) to[bend left] node[above] {B} (three.west);
  \draw[->] (three.east) to[bend left=90] node[right] {C} (four.east);
  \draw[->] (four.west) to[bend left=90] node[left] {D} (one.west);
\end{tikzpicture}
\end{dispExample}
\end{docTcbKey}
\tcbcontinuedraftmode%


\clearpage
\subsection{Underlay Option Keys}\label{subsec:skinunderlay}

Underlays are quite similar to overlays described in \zvref{subsec:overlays}.
Underlays are drawn \emph{after} the frame and interior are
drawn and \emph{before} overlays and the text content is drawn; see
\zvref{subsec:tcolorboxdrawing} for the general drawing scheme.

The differences between underlays and overlays are:
\begin{itemize}
\item Underlays are not applicable for the skins
  \refSkin{standard} and
  \refSkin{standard jigsaw},
  whereas overlays are applicable also for these skins.
  The skin \refSkin{spartan} supports underlays but no overlays.
  \begin{marker}
  If an underlay is used with the \refSkin{standard} skin, it is silently ignored.
  \end{marker}
\item Underlays are stackable, i.\,e., several different underlays can be
  used on the same |tcolorbox|. Overlays are not stackable by default (but with
  some help of the library \mylib{hooks}).
\item Boxed titles are implemented with underlays (\zvref{subsec:skinboxedtitle}),
  watermarks are implemented with overlays (\zvref{subsec:watermarks}).
\end{itemize}


\begin{docTcbKey}{underlay}{=\meta{graphical code}}{no default, initially unset}
  Adds \meta{graphical code} to the box drawing process. This \meta{graphical code}
  is drawn \emph{after} the frame and interior and \emph{before} the text content.
\begin{dispExample}
\newtcolorbox{mybox}[1][]{enhanced,colback=red!5!white,
  colbacktitle=red!85!black!50!white,
  colframe=red!75!black,fonttitle=\bfseries,watermark color=yellow!50!white,
  underlay={\begin{tcbclipinterior}
    \draw[red!40!white,line width=1cm] (interior.south west)--(interior.north east);
    \end{tcbclipinterior}},
  attach boxed title to top center={yshift=-2mm},#1}

\begin{mybox}[title=My box,watermark text=My Watermark]
\lipsum[2]
\end{mybox}
\end{dispExample}
\end{docTcbKey}


\begin{docTcbKey}{no underlay}{}{style, no default, initially set}
  Removes the underlay if set before.
\end{docTcbKey}

\clearpage
\begin{docTcbKey}{underlay broken}{=\meta{graphical code}}{no default, initially unset}
  If the box is set to be \refKey{/tcb/breakable} and \emph{is} broken actually,
  then the \meta{graphical code} is added to the box drawing process.
  \refKey{/tcb/underlay} overwrites this key.
\end{docTcbKey}

\enlargethispage*{1cm}

\begin{docTcbKey}{underlay unbroken}{=\meta{graphical code}}{no default, initially unset}
  If the box is set to be \refKey{/tcb/breakable} but \emph{is not} broken actually
  or if the box is set to be \refKey{/tcb/unbreakable},
  then the \meta{graphical code} is added to the box drawing process.
  \refKey{/tcb/underlay} overwrites this key.
\end{docTcbKey}

\begin{docTcbKey}{no underlay unbroken}{}{style, no default, initially set}
  Removes the unbroken underlay if set before.
\end{docTcbKey}

\begin{docTcbKey}{underlay first}{=\meta{graphical code}}{no default, initially unset}
  If the box is set to be \refKey{/tcb/breakable} and \emph{is} broken actually,
  then the \meta{graphical code} is added to the box drawing process for
  the \emph{first} part of the break sequence.
  \refKey{/tcb/underlay} overwrites this key.
\end{docTcbKey}

\begin{docTcbKey}{no underlay first}{}{style, no default, initially set}
  Removes the first underlay if set before.
\end{docTcbKey}

\begin{docTcbKey}{underlay middle}{=\meta{graphical code}}{no default, initially unset}
  If the box is set to be \refKey{/tcb/breakable} and \emph{is} broken actually,
  then the \meta{graphical code} is added to the box drawing process for
  the \emph{middle} parts (if any) of the break sequence.
  \refKey{/tcb/underlay} overwrites this key.
\end{docTcbKey}

\begin{docTcbKey}{no underlay middle}{}{style, no default, initially set}
  Removes the middle underlay if set before.
\end{docTcbKey}

\begin{docTcbKey}{underlay last}{=\meta{graphical code}}{no default, initially unset}
  If the box is set to be \refKey{/tcb/breakable} and \emph{is} broken actually,
  then the \meta{graphical code} is added to the box drawing process for
  the \emph{last} part of the break sequence.
  \refKey{/tcb/underlay} overwrites this key.
\end{docTcbKey}

\begin{docTcbKey}{no underlay last}{}{style, no default, initially set}
  Removes the last underlay if set before.
\end{docTcbKey}

\begin{docTcbKey}{underlay boxed title}{=\meta{graphical code}}{no default, initially unset}
  If the box has a \emph{boxed title}, see \zvref{subsec:skinboxedtitle},
  then the \meta{graphical code} is added to the box drawing process
  \emph{before} the boxed title is drawn.
\end{docTcbKey}

\begin{docTcbKey}{no underlay boxed title}{}{style, no default, initially set}
  Removes the boxed title underlay if set before.
\end{docTcbKey}

\begin{docTcbKey}{underlay unbroken and first}{=\meta{graphical code}}{no default, initially unset}
  This is an abbreviation for setting
  \refKey{/tcb/underlay unbroken} and
  \refKey{/tcb/underlay first} together.
  \refKey{/tcb/underlay} overwrites this key.
\end{docTcbKey}

\begin{docTcbKey}{underlay middle and last}{=\meta{graphical code}}{no default, initially unset}
  This is an abbreviation for setting
  \refKey{/tcb/underlay middle} and
  \refKey{/tcb/underlay last} together.
  \refKey{/tcb/underlay} overwrites this key.
\end{docTcbKey}

\begin{docTcbKey}{underlay unbroken and last}{=\meta{graphical code}}{no default, initially unset}
  This is an abbreviation for setting
  \refKey{/tcb/underlay unbroken} and
  \refKey{/tcb/underlay last} together.
  \refKey{/tcb/underlay} overwrites this key.
\end{docTcbKey}

\begin{docTcbKey}[][doc new=2014-09-19]{underlay first and middle}{=\meta{graphical code}}{no default, initially unset}
  This is an abbreviation for setting
  \refKey{/tcb/underlay first} and
  \refKey{/tcb/underlay middle} together.
  \refKey{/tcb/underlay} overwrites this key.
\end{docTcbKey}


\clearpage
\subsection{Finish Option Keys}\label{subsec:skinfinish}

Finishes are quite similar to underlays described in \zvref{subsec:skinunderlay}
and overlays described in \zvref{subsec:overlays}.
Finishes are drawn \emph{after} the text content is drawn; see
\zvref{subsec:tcolorboxdrawing} for the general drawing scheme.
Therefore, a finish will reduce the readability of the text content.

Finishes are intended for special effects like highlights or glosses or text over text.

\begin{itemize}
\item Finishes are only applicable for the skins
  \refSkin{enhanced},
  \refSkin{empty},
  \refSkin{freelance},
  \refSkin{bicolor},
  \refSkin{beamer}, and
  \refSkin{widget}.
  \begin{marker}
  If a finish is used with the \refSkin{standard} skin, it is silently ignored.
  \end{marker}
\item Finishes are stackable, i.\,e., several different finishes can be
  used on the same |tcolorbox|.
\end{itemize}

\enlargethispage*{2cm}
\begin{docTcbKey}{finish}{=\meta{graphical code}}{no default, initially unset}
  Adds \meta{graphical code} to the box drawing process. This \meta{graphical code}
  is drawn \emph{after} the text content.
\begin{dispExample}
\newtcolorbox{mybox}[1][]{enhanced,colback=red!5!white,
  colbacktitle=red!85!black!50!white,colframe=red!75!black,fonttitle=\bfseries,
  finish={\begin{tcbclipframe}
    \path[bottom color=black,top color=black!50!white,opacity=0.1]
      (frame.south west) -- (frame.south east) -- (frame.north east) -- cycle;
    \path[top color=white,bottom color=black!50!white,opacity=0.1]
      (frame.south west) -- (frame.north east) -- (frame.north west) -- cycle;
    \end{tcbclipframe}},#1}

\begin{mybox}[title=My box]
\lipsum[2]
\end{mybox}
\end{dispExample}
\begin{dispExample}
\newtcolorbox{mybox}[1][]{enhanced,colback=red!5!white,
  colbacktitle=red!85!black!50!white,colframe=red!75!black,fonttitle=\bfseries,
  finish={\node[draw,fill=white,fill opacity=0.85,inner sep=5mm,
    rounded corners] at (frame.center) {\Huge\bfseries Finish!};},#1}

\begin{mybox}[title=My box]
\lipsum[2]
\end{mybox}
\end{dispExample}
\end{docTcbKey}

\clearpage
\begin{docTcbKey}{no finish}{}{style, no default, initially set}
  Removes the finish if set before.
\end{docTcbKey}


\begin{docTcbKey}{finish broken}{=\meta{graphical code}}{no default, initially unset}
  If the box is set to be \refKey{/tcb/breakable} and \emph{is} broken actually,
  then the \meta{graphical code} is added to the box drawing process.
  \refKey{/tcb/finish} overwrites this key.
\end{docTcbKey}

\begin{docTcbKey}{finish unbroken}{=\meta{graphical code}}{no default, initially unset}
  If the box is set to be \refKey{/tcb/breakable} but \emph{is not} broken actually
  or if the box is set to be \refKey{/tcb/unbreakable},
  then the \meta{graphical code} is added to the box drawing process.
  \refKey{/tcb/finish} overwrites this key.
\end{docTcbKey}

\begin{docTcbKey}{no finish unbroken}{}{style, no default, initially set}
  Removes the unbroken finish if set before.
\end{docTcbKey}

\begin{docTcbKey}{finish first}{=\meta{graphical code}}{no default, initially unset}
  If the box is set to be \refKey{/tcb/breakable} and \emph{is} broken actually,
  then the \meta{graphical code} is added to the box drawing process for
  the \emph{first} part of the break sequence.
  \refKey{/tcb/finish} overwrites this key.
\end{docTcbKey}

\begin{docTcbKey}{no finish first}{}{style, no default, initially set}
  Removes the first finish if set before.
\end{docTcbKey}

\begin{docTcbKey}{finish middle}{=\meta{graphical code}}{no default, initially unset}
  If the box is set to be \refKey{/tcb/breakable} and \emph{is} broken actually,
  then the \meta{graphical code} is added to the box drawing process for
  the \emph{middle} parts (if any) of the break sequence.
  \refKey{/tcb/finish} overwrites this key.
\end{docTcbKey}

\begin{docTcbKey}{no finish middle}{}{style, no default, initially set}
  Removes the middle finish if set before.
\end{docTcbKey}

\begin{docTcbKey}{finish last}{=\meta{graphical code}}{no default, initially unset}
  If the box is set to be \refKey{/tcb/breakable} and \emph{is} broken actually,
  then the \meta{graphical code} is added to the box drawing process for
  the \emph{last} part of the break sequence.
  \refKey{/tcb/finish} overwrites this key.
\end{docTcbKey}

\begin{docTcbKey}{no finish last}{}{style, no default, initially set}
  Removes the last finish if set before.
\end{docTcbKey}

\begin{docTcbKey}{finish unbroken and first}{=\meta{graphical code}}{no default, initially unset}
  This is an abbreviation for setting
  \refKey{/tcb/finish unbroken} and
  \refKey{/tcb/finish first} together.
  \refKey{/tcb/finish} overwrites this key.
\end{docTcbKey}

\begin{docTcbKey}{finish middle and last}{=\meta{graphical code}}{no default, initially unset}
  This is an abbreviation for setting
  \refKey{/tcb/finish middle} and
  \refKey{/tcb/finish last} together.
  \refKey{/tcb/finish} overwrites this key.
\end{docTcbKey}

\begin{docTcbKey}{finish unbroken and last}{=\meta{graphical code}}{no default, initially unset}
  This is an abbreviation for setting
  \refKey{/tcb/finish unbroken} and
  \refKey{/tcb/finish last} together.
  \refKey{/tcb/finish} overwrites this key.
\end{docTcbKey}


\begin{docTcbKey}[][doc new=2014-09-19]{finish first and middle}{=\meta{graphical code}}{no default, initially unset}
  This is an abbreviation for setting
  \refKey{/tcb/finish first} and
  \refKey{/tcb/finish middle} together.
  \refKey{/tcb/finish} overwrites this key.
\end{docTcbKey}

\clearpage
\subsection{Hyper Option Keys}\label{subsec:hyper}
All options of this section need the package \refPkg{hyperref} \cite{rahtz:hyperref}
to be loaded separately. All these options are implemented as
\refKey{/tcb/finish} and can be disabled by \refKey{/tcb/no finish}.

\begin{marker}
If the package \refPkg{hyperref} \cite{rahtz:hyperref} is not loaded or if
the \refSkin{standard} skin is used, all hyper option are silently ignored.
\end{marker}

\begin{docTcbKey}[][doc new=2017-02-03]{hyperref}{=\meta{marker}}{no default, initially unset}
  The whole \textit{frame} of a |tcolorbox| is made an active hyperlink for a
  \meta{marker} which was given by |\label| or \refKey{/tcb/label} or \refKey{/tcb/phantomlabel}.
  Such, the |tcolorbox| is made a clickable button (depending on the previewer).
  \begin{dispExample*}{sbs,lefthand ratio=0.7}
% \section{Library skins}\label{sec:skins}%
\begin{tcolorbox}[beamer,colback=red!50,
  hyperref=sec:skins]
Jump to the heading of Section~\ref*{sec:skins}.
\end{tcolorbox}
  \end{dispExample*}
\end{docTcbKey}

\begin{docTcbKey}[][doc new=2017-02-03]{hyperref interior}{=\meta{marker}}{no default, initially unset}
  Identical to \refKey{/tcb/hyperref}, but only the \textit{interior} of a
  |tcolorbox| is made a hyperlink (without frame and title).
\end{docTcbKey}

\begin{docTcbKey}[][doc new=2017-02-03]{hyperref title}{=\meta{marker}}{no default, initially unset}
  Identical to \refKey{/tcb/hyperref}, but only the \textit{title} of a
  |tcolorbox| is made a hyperlink.
\end{docTcbKey}

\begin{docTcbKey}[][doc new=2017-02-03]{hyperref node}{=\marg{marker}\marg{node}}{no default, initially unset}
  Identical to \refKey{/tcb/hyperref}, but only the given \tikzname\ \meta{node}
  is made a hyperlink. This \meta{node} may be |frame|, |interior|, |title|, or
  any other named node used for drawing the |tcolorbox|.
  The \meta{node} may be defined inside
  \refKey{/tcb/underlay}, \refKey{/tcb/overlay} or \refKey{/tcb/finish}.
  If the later is used, define the node \emph{before} \refKey{/tcb/hyperref node}
  is applied.
  \begin{dispExample*}{sbs,lefthand ratio=0.7}
% \section{Library skins}\label{sec:skins}%
\begin{tcolorbox}[enhanced,colback=yellow!10,
  underlay={\node[red,fill=red!30,inner sep=3mm]
   (click) at (frame.center) {X};},
  hyperref node={sec:skins}{click}]
Jump to the heading of Section~\ref*{sec:skins}
(X marks the click point).
\end{tcolorbox}
  \end{dispExample*}
\end{docTcbKey}

\begin{docTcbKey}[][doc new=2017-02-03]{hyperlink}{=\meta{marker}}{no default, initially unset}
  The whole \textit{frame} of a |tcolorbox| is made an active hyperlink for a
  \meta{marker} which was given by |\hypertarget| or \refKey{/tcb/hypertarget}.
  Such, the |tcolorbox| is made a clickable button (depending on the previewer).
  \begin{dispExample*}{sbs,lefthand ratio=0.7}
% \usepackage{hyperref}%
\begin{tcolorbox}[enhanced,
  colback=blue!10,colframe=blue!50!black,
  hypertarget=hypertwinB,
  hyperlink=hypertwinA,
  title=Box B]
Click me to jump to Box A.
\end{tcolorbox}
  \end{dispExample*}
\end{docTcbKey}

\clearpage
\begin{docTcbKey}[][doc new=2017-02-03]{hyperlink interior}{=\meta{marker}}{no default, initially unset}
  Identical to \refKey{/tcb/hyperlink}, but only the \textit{interior} of a
  |tcolorbox| is made a hyperlink (without frame and title).
\end{docTcbKey}

\begin{docTcbKey}[][doc new=2017-02-03]{hyperlink title}{=\meta{marker}}{no default, initially unset}
  Identical to \refKey{/tcb/hyperlink}, but only the \textit{title} of a
  |tcolorbox| is made a hyperlink.
\end{docTcbKey}

\begin{docTcbKey}[][doc new=2017-02-03]{hyperlink node}{=\marg{marker}\marg{node}}{no default, initially unset}
  Identical to \refKey{/tcb/hyperlink}, but only the given \tikzname\ \meta{node}
  is made a hyperlink. This \meta{node} may be |frame|, |interior|, |title|, or
  any other named node used for drawing the |tcolorbox|.
  The \meta{node} may be defined inside
  \refKey{/tcb/underlay}, \refKey{/tcb/overlay} or \refKey{/tcb/finish}.
  If the later is used, define the node \emph{before} \refKey{/tcb/hyperlink node}
  is applied.
\end{docTcbKey}

\begin{docTcbKey}[][doc new=2017-02-03]{hyperurl}{=\meta{url}}{no default, initially unset}
  The whole \textit{frame} of a |tcolorbox| is made an active hyperlink for an
  \meta{url} in the same manner as using |\href| or |\url|.
  Such, the |tcolorbox| is made a clickable button (depending on the previewer).
  \begin{dispExample*}{sbs,lefthand ratio=0.7}
\begin{tcolorbox}[enhanced,colback=red!50,
  hyperurl=https://www.ctan.org/pkg/tcolorbox]
View CTAN with a browser.
\end{tcolorbox}
  \end{dispExample*}
\end{docTcbKey}

\begin{docTcbKey}[][doc new=2017-02-03]{hyperurl interior}{=\meta{url}}{no default, initially unset}
  Identical to \refKey{/tcb/hyperurl}, but only the \textit{interior} of a
  |tcolorbox| is made a hyperlink (without frame and title).
\end{docTcbKey}

\begin{docTcbKey}[][doc new=2017-02-03]{hyperurl title}{=\meta{url}}{no default, initially unset}
  Identical to \refKey{/tcb/hyperurl}, but only the \textit{title} of a
  |tcolorbox| is made a hyperlink.
\end{docTcbKey}

\begin{docTcbKey}[][doc new=2017-02-03]{hyperurl node}{=\marg{url}\marg{node}}{no default, initially unset}
  Identical to \refKey{/tcb/hyperurl}, but only the given \tikzname\ \meta{node}
  is made a hyperlink. This \meta{node} may be |frame|, |interior|, |title|, or
  any other named node used for drawing the |tcolorbox|.
  The \meta{node} may be defined inside
  \refKey{/tcb/underlay}, \refKey{/tcb/overlay} or \refKey{/tcb/finish}.
  If the later is used, define the node \emph{before} \refKey{/tcb/hyperurl node}
  is applied.
\end{docTcbKey}


\begin{docTcbKey}[][doc new=2017-02-03]{hyperurl*}{=\marg{options}\marg{url}}{no default, initially unset}
  Identical to \refKey{/tcb/hyperurl}, but additional
  \refPkg{hyperref} \cite{rahtz:hyperref} \meta{options} are applied.
  \begin{dispExample*}{sbs,lefthand ratio=0.7}
\begin{tcolorbox}[enhanced,colback=green!50,
  hyperurl*={page=3,pdfnewwindow=true}%
            {tcolorbox-example.pdf}]
Open example file on Page~3.
\end{tcolorbox}
  \end{dispExample*}
\end{docTcbKey}

\begin{docTcbKey}[][doc new=2017-02-03]{hyperurl* interior}{=\marg{options}\marg{url}}{no default, initially unset}
  Identical to \refKey{/tcb/hyperurl interior}, but additional |hyperref| \cite{rahtz:hyperref}
  \meta{options} are applied.
\end{docTcbKey}

\begin{docTcbKey}[][doc new=2017-02-03]{hyperurl* title}{=\marg{options}\marg{url}}{no default, initially unset}
  Identical to \refKey{/tcb/hyperurl title}, but additional |hyperref| \cite{rahtz:hyperref}
  \meta{options} are applied.
\end{docTcbKey}

\enlargethispage*{1cm}

\begin{docTcbKey}[][doc new=2017-02-03]{hyperurl* node}{=\marg{options}\marg{url}\marg{node}}{no default, initially unset}
  Identical to \refKey{/tcb/hyperurl node}, but additional |hyperref| \cite{rahtz:hyperref}
  \meta{options} are applied.
\end{docTcbKey}



\clearpage
\subsection{Jigsaw Skin Variants}\label{subsec:skinjigsaw}
As described in \zvref{sec:skincorekeys}, a |tcolorbox| is drawn by up to
four \emph{engines}. Typically, the \emph{frame} engine fills the complete box area
with color and the other engines fill certain areas with other colors.
Finally, only the area which you see as \emph{frame} of the box will display
the frame color. For most applications, this is a good approach.

For certain boxes, a more delicate procedure is needed. E.\,g., if the box should
be translucent, an already painted area cannot be made unpainted. Therefore,
more elaborate frame engines saw holes into the frame where the interior area and
optionally the title area will be painted.
The resulting skins are called \emph{jigsaw} skins. For \refSkin{standard},
\refSkin{enhanced}, and \refSkin{bicolor}, there are variants called \refSkin{standard jigsaw},
\refSkin{enhanced jigsaw}, and \refSkin{bicolor jigsaw}.


\begin{dispExample*}{segmentation style={preaction={fill=white},pattern=checkerboard,pattern color=gray!40}}
\newcommand{\ballexample}{\begin{tikzpicture}
  \path[use as bounding box] (0,0.8) rectangle +(0.1,0.1);
  \shadedraw [shading=ball] (0,0) circle (1cm);
  \shadedraw [ball color=red] (3,-2.2) circle (1cm);
\end{tikzpicture}}

\tcbset{enhanced,colback=blue!5!white,
  frame style={left color=red!75!black,right color=red!10!yellow},
  fonttitle=\bfseries }

\ballexample

\begin{tcolorbox}[title=A normal box]
  \lipsum[2]
\end{tcolorbox}

\ballexample

\begin{tcolorbox}[title=A translucent jigsaw box,
  enhanced jigsaw,opacityback=0.35]
  \lipsum[2]
\end{tcolorbox}
\end{dispExample*}


\begin{dispExample*}{segmentation style={preaction={fill=white},pattern=checkerboard,pattern color=gray!40}}
\tcbset{enhanced,colback=red!10!white,coltitle=black,
  frame style={left color=red!75!black,right color=red!10!yellow},
  fonttitle=\bfseries,interior hidden,title hidden}

\begin{tcolorbox}[title=A normal box with hidden interior and title]
  This is a tcolorbox.
\end{tcolorbox}

\begin{tcolorbox}[enhanced jigsaw,
  title=A jigsaw box with hidden interior and title]
  This is a tcolorbox.
\end{tcolorbox}
\end{dispExample*}


\begin{dispExample*}{segmentation style={preaction={fill=white},pattern=checkerboard,pattern color=gray!40}}
\newtcolorbox{mybox}{skin=enhancedmiddle jigsaw,leftrule=5mm,rightrule=5mm,
  boxsep=0mm,top=0mm,bottom=0mm,
  frame style={top color=blue,bottom color=red},interior hidden}

\begin{mybox}
  \lipsum[2]
\end{mybox}
\end{dispExample*}


\clearpage
\subsection{Draft Mode}\label{subsec:draftmode}
To reduce the compilation time while drafting a document, the \emph{draft mode}
can be applied. Basically, it changes all skins to \refSkin{spartan} and
sets the \refKey{/tcb/fit algorithm} to |squeeze|. Especially,
when fuzzy shadows are used, the speedup will be considerable high.

\begin{marker}
It is strongly recommended that the draft mode is \emph{not} used for the final document.
Use \refSkin{spartan} directly, if you want to stay with it. The draft mode
implementation may change in future.
\end{marker}

\begin{marker}
Normally, switching to the draft mode should not alter the geometry of
your document. Since overlays are deactivated, any code placed there
(e.\,g.\ counter changes) is not executed anymore! Also, \refKey{/tcb/remember as}
will not have any effect. You may exclude critical code with
\refCom{tcbinterruptdraftmode} / \refCom{tcbcontinuedraftmode}
from converting to draft mode.
\end{marker}


\begin{docCommand}{tcbstartdraftmode}{}
  Any following |tcolorbox| code is put into \emph{draft mode}. All skin
  settings are overruled with \refSkin{spartan}. Overlays, watermarks,
  shadows, borderlines, and rounded corners are deactivated for all |tcolorbox|
  layers.
\end{docCommand}

\begin{docCommand}{tcbstopdraftmode}{}
  The \emph{draft mode} is deactivated for the following code.
\end{docCommand}

\begin{docCommand}{tcbinterruptdraftmode}{}
  If the compilation is in \emph{draft mode}, the \emph{draft mode} is deactivated
  until a following \refCom{tcbcontinuedraftmode} is detected.\par
  If the compilation is not in \emph{draft mode}, nothing happens and a following
  \refCom{tcbcontinuedraftmode} will not start the \emph{draft mode}.
  \begin{marker}
  The pair |\tcbinterruptdraftmode| and |\tcbcontinuedraftmode| cannot
  be used nested.
  \end{marker}
\end{docCommand}

\begin{docCommand}{tcbcontinuedraftmode}{}
  Continues the \emph{draft mode} which was suspended by a preceding
  \refCom{tcbinterruptdraftmode}. Nothing happens, if there was no draft
  mode before \refCom{tcbinterruptdraftmode}.
  \begin{marker}
  Code, which is place between \refCom{tcbinterruptdraftmode} and
  \refCom{tcbcontinuedraftmode} is shielded from \emph{draft mode}.
  \end{marker}
\end{docCommand}


\clearpage

\begin{docTcbKey}{draftmode}{\colOpt{=true\textbar false}}{default |true|, initially |false|}
  If set to |true|, the \emph{draft mode} is started.
  If set to |false|, the \emph{draft mode} is stopped.

\begin{dispExample*}{}
\newtcolorbox{mybeamer}[2][]{beamer,colback=Salmon!50!white,
  colframe=FireBrick!75!black,adjusted title={#2},#1}

\begin{mybeamer}{Beamer box}
This box looks like a box provided by the \texttt{beamer} class.
\end{mybeamer}\par\medskip
\begin{mybeamer}[draftmode]{Beamer box}
This box looks like a box provided by the \texttt{beamer} class.
\end{mybeamer}
\end{dispExample*}
\end{docTcbKey}


